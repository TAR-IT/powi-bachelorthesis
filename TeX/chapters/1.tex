%%%%%%%%%%%%%%
% Einleitung %
%%%%%%%%%%%%%%

% TODO: Leave as is. (14.03.2025)

\section{Einleitung}

Mit der zunehmenden Digitalisierung und Automatisierung der Arbeitswelt erleben viele 
Länder tiefgreifende strukturelle Veränderungen ihrer Arbeitsmärkte. Eine zentrale Rolle 
spielen dabei \ac{ICT}, deren Einsatz weltweit zu erheblichen Effizienzsteigerungen und 
Innovationsprozessen führt \parencite[vgl.][S. 43-45]{oecd2019measuring}.

Während technologische Fortschritte typischerweise die Nachfrage nach hochqualifizierten 
Arbeitskräften erhöhen, bleibt die Rolle von geringqualifizierten Arbeitskräften im 
technologischen Wandel unklar. Besonders die Differenzierung zwischen „Skills“ und „Tasks“ 
spielt eine entscheidende Rolle bei der Analyse dieser Veränderungen, da technologische 
Entwicklungen bestimmte Aufgaben automatisieren oder auslagern können, während sie 
gleichzeitig die Anforderungen an die verbleibenden Tätigkeiten verändern 
\parencite[vgl.][S. 1045]{acemoglu2011skills}. Die zunehmende Digitalisierung und 
Automatisierung geht mit einer Polarisierung des Arbeitsmarktes einher. Während der Anteil 
hochqualifizierter Tätigkeiten wächst, steigt gleichzeitig die Beschäftigung in 
geringqualifizierten, niedrig entlohnten Berufen und mittlere Qualifikationsniveaus 
geraten unter Druck  \parencite[vgl.][S. 1070]{acemoglu2011skills}. In der Forschung wird 
daher diskutiert, inwiefern der technologische Fortschritt zu einer Polarisierung des 
Arbeitsmarktes beiträgt, indem er die Nachfrage nach hochqualifizierten Arbeitskräften 
erhöht, während gleichzeitig Tätigkeiten von geringer Qualifizierten durch Automatisierung 
ersetzt werden \parencite[vgl.][S. 2–4]{balsmeier2019isthis}.

Die Auswirkungen von \ac{ICT}-Investitionen auf den Arbeitsmarkt sind nicht nur ökonomisch, 
sondern auch sozial von Bedeutung. Technologischer Fortschritt führt dazu, dass manche 
Aufgaben zunehmend automatisierbar werden, wodurch insbesondere geringqualifizierte 
Arbeitskräfte einem höheren Substitutionsrisiko ausgesetzt sind, während hochqualifizierte 
Fachkräfte tendenziell von diesen Entwicklungen profitieren 
\parencite[vgl.][S. 14–15]{frey2013thefuture}. Der technologische Wandel führt nicht nur zu 
Arbeitsplatzverlusten durch Automatisierung, sondern schafft auch neue Berufsfelder, die 
insbesondere auf die Zusammenarbeit zwischen menschlichen kognitiven Fähigkeiten und digitalen 
Technologien angewiesen sind \parencite[vgl.][Kap. 12]{brynjolfsson2014thesecond}. Welche 
Auswirkungen diese Prozesse letzlich wirklich auf die Verteilung von Arbeitsplätzen haben und 
ob sie aktiv zur Polarisierung des Arbeitsmarktes beitragen, bleibt eine zentrale Frage der 
aktuellen Forschung.

Ziel dieser Arbeit ist es daher, den Einfluss von \ac{ICT}-Investitionen auf die 
Arbeitslosigkeit in verschiedenen Bildungsniveaus zu untersuchen. Es wird angenommen, 
dass technologischer Fortschritt die Nachfrage nach hochqualifizierten Arbeitskräften 
steigert, während gleichzeitig die Beschäftigungsmöglichkeiten für geringqualifizierte 
Personen durch Automatisierung reduzieren werden. Dies kann zu einer Polarisierung des 
Arbeitsmarktes führen \parencite[vgl.][S. 1045]{acemoglu2011skills}. Diese Analyse soll zur 
Debatte über die Auswirkungen der Digitalisierung auf den Arbeitsmarkt beitragen und empirisch 
untersuchen, in welchem Maße technologische Investitionen mit der Arbeitslosenquote nach 
Bildungsniveau korrelieren.

Aus der zuvor dargelegten Argumentation ergibt sich die zentrale Forschungsfrage 
dieser Arbeit:

\begin{quote} 
    \textbf{„Wie beeinflussen nationale Investitionen in Informations- und 
    Kommunikationstechnologien die Arbeitslosenquoten verschiedener Bildungsniveaus 
    in Wohlfahrtsstaaten?“}
\end{quote}

Die Analyse der Auswirkungen von Digitalisierung und \ac{ICT}-Investitionen auf die 
Beschäftigungsstruktur in Ländern der \ac{OECD} ist von hoher Relevanz, da sie Aufschluss 
über die Anpassungsfähigkeit verschiedener Wirtschaftssysteme an technologische 
Umbrüche gibt. Zudem bietet sie eine Grundlage für politische Entscheidungen im 
Bereich Arbeitsmarktregulierung und Bildungsinvestitionen. Angesichts der zunehmenden 
Bedeutung digitaler Technologien für wirtschaftliches Wachstum und soziale 
Gerechtigkeit ist es essenziell, die damit verbundenen Herausforderungen und Chancen 
für verschiedene gesellschaftliche Gruppen besser zu verstehen.
