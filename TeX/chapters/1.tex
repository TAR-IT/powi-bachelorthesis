\section{Einleitung}

Mit der zunehmenden Digitalisierung und Automatisierung der Arbeitswelt erleben viele 
Länder tiefgreifende strukturelle Veränderungen ihrer Arbeitsmärkte. Eine zentrale Rolle 
spielen dabei \ac{ICT}, deren Einsatz weltweit zu erheblichen Effizienzsteigerungen und 
Innovationsprozessen führt \parencite[vgl.][Kap. 6]{brynjolfsson2014thesecond}.

Während technologische Fortschritte die Nachfrage nach hochqualifizierten Arbeitskräften 
erhöhen können, bleibt unklar, welche Auswirkungen diese Entwicklung auf 
geringqualifizierte Arbeitskräfte hat und inwiefern sie sich messen lässt 
\parencite[vgl.][S. 1045]{acemoglu2011skills}. Einerseits eröffnen die gestiegenen 
Anforderungen an technologische Kompetenzen neue Chancen für hochqualifizierte 
Fachkräfte \parencite[vgl.][S. 1070]{acemoglu2011skills}, andererseits wächst die 
Befürchtung, dass rasante Investitionen in \ac{ICT} insbesondere für geringqualifizierte 
Personen zum Verlust von Arbeitsplätzen führen, da viele ihrer Tätigkeiten routinisiert 
und damit leichter durch Maschinen ersetzbar sind. In der Forschung wird daher diskutiert, 
ob der technologische Fortschritt die Kluft zwischen den Bildungsniveaus weiter vertieft 
und die Arbeitslosigkeit unter geringer Qualifizierten verstärkt 
\parencite[vgl.][S. 2–4]{balsmeier2019isthis}.

Die Frage nach dem Zusammenhang zwischen \ac{ICT}-Investitionen und Arbeitslosigkeit 
nach Bildungsniveau ist nicht nur ökonomisch, sondern auch sozial von Bedeutung. 
Investitionen in digitale Technologien können Polarisierungseffekte hervorrufen, 
indem sie hochqualifizierte Arbeitskräfte begünstigen, während geringer Qualifizierte 
durch Automatisierung verdrängt werden \parencite[vgl.][S. 14–15]{frey2013thefuture}. 
Gleichzeitig entstehen durch den technologischen Wandel neue Berufsfelder, die 
innovative Fähigkeiten erfordern \parencite[vgl.][Kap. 12]{brynjolfsson2014thesecond}. 
Welche Auswirkungen diese Prozesse auf die Verteilung von Arbeitsplätzen haben und 
ob sie wirklich aktiv zur Polarisierung des Arbeitsmarktes beitragen, bleibt eine 
zentrale Frage der aktuellen Forschung \parencite[vgl.][S. 2–4]{balsmeier2019isthis}.

Ziel dieser Arbeit ist es daher, den Einfluss von \ac{ICT}-Investitionen auf die 
Arbeitslosigkeit in verschiedenen Bildungsgruppen zu untersuchen. Es wird angenommen, 
dass hohe Investitionen in digitale Technologien die Arbeitslosigkeit unter 
Hochqualifizierten senken, während sie bei geringer Qualifizierten steigen könnte 
\parencite[vgl.][S. 1045]{acemoglu2011skills}. Diese Analyse soll zur Debatte über die 
Auswirkungen der Digitalisierung auf den Arbeitsmarkt beitragen und empirisch untersuchen, 
in welchem Maße technologische Investitionen mit der Arbeitslosenquote nach 
Bildungsniveau korrelieren.

Aus der zuvor dargelegten Argumentation ergibt sich die zentrale Forschungsfrage 
dieser Arbeit:

\begin{quote} 
    \textbf{„Wie beeinflussen nationale Investitionen in Informations- und 
    Kommunikationstechnologien die Arbeitslosenquoten verschiedener Bildungsniveaus 
    in Wohlfahrtsstaaten?“}
\end{quote}

Die Analyse der Auswirkungen von Digitalisierung und \ac{ICT}-Investitionen auf die 
Beschäftigungsstruktur in Ländern der \ac{OECD} ist von hoher Relevanz, da sie Aufschluss 
über die Anpassungsfähigkeit verschiedener Wirtschaftssysteme an technologische 
Umbrüche gibt. Zudem bietet sie eine Grundlage für politische Entscheidungen im 
Bereich Arbeitsmarktregulierung und Bildungsinvestitionen. Angesichts der zunehmenden 
Bedeutung digitaler Technologien für wirtschaftliches Wachstum und soziale 
Gerechtigkeit ist es essenziell, die damit verbundenen Herausforderungen und Chancen 
für verschiedene gesellschaftliche Gruppen besser zu verstehen.
