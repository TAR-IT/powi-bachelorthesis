%%%%%%%%%%%%%%%%%%%%%%%%
% Forschungsgegenstand %
%%%%%%%%%%%%%%%%%%%%%%%%

% TODO: Leave as is. (14.03.2025)

\section{Forschungsgegenstand}

Der Forschungsgegenstand dieser Arbeit umfasst die Untersuchung der Auswirkungen von 
Investitionen in \ac{ICT} auf den Arbeitsmarkt in \ac{OECD}-Ländern. Im Zentrum steht 
dabei die Frage, wie sich diese Investitionen auf die Beschäftigungslage, insbesondere 
die Arbeitslosenquote, in verschiedenen Bildungsgruppen auswirken. Dabei werden sowohl 
die nationalen Investitionen in digitale Technologien als auch die Auswirkungen auf 
Arbeitsmärkte und Beschäftigungsstrukturen betrachtet.

%%%%%%%%%%%%%%%%%%%%%%%%%%%%%%%%%%%%%%%%%%%%%%%%%%%%%%%%%%%
% Forschungsgegenstand: Digitalisierung und Industrie 4.0 %
%%%%%%%%%%%%%%%%%%%%%%%%%%%%%%%%%%%%%%%%%%%%%%%%%%%%%%%%%%%

% TODO: Leave as is. (14.03.2025)

\subsection{Digitalisierung und Industrie 4.0}

Der Begriff der Digitalisierung beschreibt den zunehmenden Einsatz digitaler 
Technologienzur Automatisierung, Optimierung und Schaffung neuer Wertschöpfungspotenziale
\parencite[vgl.][S. 6]{brennen2016theinternational}. Im wirtschaftlichen Kontext wird 
dies mit der vierten industriellen Revolution (Industrie 4.0) in Verbindung gebracht, 
die durch die Integration von \ac{ICT}, \ac{AI}, Big Data, Cloud Computing und 
cyber-physischen Systemen gekennzeichnet ist 
\parencite[vgl.][S. 13–14]{kagermann2013recommendations}. Diese Entwicklungen führen zu 
einer zunehmenden Automatisierung von Produktionsprozessen und ermöglichen eine 
vernetzte Wertschöpfung entlang der gesamten Lieferkette.

Während Unternehmen durch den Einsatz digitaler Technologien Effizienzsteigerungen 
erzielen können, ergeben sich für den Arbeitsmarkt erhebliche Herausforderungen. 
Hochqualifizierte Arbeitsplätze entstehen in den Bereichen Softwareentwicklung, 
Datenanalyse und Automatisierungstechnik, während Routineaufgaben in der industriellen 
Fertigung, im Transportwesen und in administrativen Berufen zunehmend automatisiert 
werden \parencite[vgl.][S. 36–37]{frey2013thefuture}. Dies kann zur Verdrängung mittlerer
Qualifikationsniveaus führen, was in der Literatur als \textit{Jobpolarisierung} 
beschrieben wird.

Die Digitalisierung bringt zudem tiefgreifende Veränderungen in der Arbeitsorganisation
mit sich. Neue Arbeitsformen wie Plattformarbeit, Remote Work und flexible
Arbeitszeitmodelle gewinnen an Bedeutung \parencite[vgl.][S. 58–60]{schwab2016thefourth}.
Dies erfordert nicht nur neue Kompetenzen, sondern stellt auch Arbeitnehmer*innen vor
Herausforderungen in Bezug auf Arbeitsplatzsicherheit, Datenschutz und IT-Sicherheit.

%%%%%%%%%%%%%%%%%%%%%%%%%%%%%%%%%%%%%%%%%%%
% Forschungsgegenstand: ICT-Investitionen %
%%%%%%%%%%%%%%%%%%%%%%%%%%%%%%%%%%%%%%%%%%%

% TODO: Leave as is. (14.03.2025)

\subsection{ICT-Investitionen}

Investitionen in \ac{ICT} umfassen materielle Ressourcen wie digitale Infrastrukturen
(z. B. Glasfasernetze, Rechenzentren) und immaterielle Ressourcen wie Softwarelösungen,
Cloud-Dienste und Plattformtechnologien \parencite[vgl.][S. 122]{oecd2019measuring}. 
Sie spielen eine zentrale Rolle für die digitale Transformation und haben weitreichende
Implikationen für Wirtschaftswachstum und Arbeitsmärkte.

Der verstärkte Einsatz digitaler Technologien führt zu Produktivitätssteigerungen und
zur Entwicklung neuer Geschäftsmodelle. Fortschritte in \ac{AI} und Big Data erlauben
eine effizientere Ressourcennutzung, optimierte Entscheidungsprozesse und die
Automatisierung komplexer Abläufe \parencite[vgl.][S. 122]{oecd2019measuring}. Dies
kann die Wettbewerbsfähigkeit von Unternehmen erhöhen, bringt jedoch gleichzeitig
Veränderungen in der Beschäftigungsstruktur mit sich.

Die Auswirkungen von \ac{ICT}-Investitionen auf den Arbeitsmarkt sind ambivalent.
Einerseits entstehen neue, zukunftsorientierte Berufe, andererseits werden zahlreiche
Routineaufgaben automatisiert, was insbesondere Arbeitsplätze mit mittlerem
Qualifikationsniveau gefährdet \parencite[vgl.][S. 36–40]{frey2013thefuture}. Diese
Entwicklung verstärkt bestehende Ungleichheiten auf dem Arbeitsmarkt und führt zu
Herausforderungen für Weiterbildungs- und Umschulungsmaßnahmen.

Ein weiterer Aspekt von \ac{ICT}-Investitionen ist ihre Rolle in der Verringerung
regionaler Disparitäten. Während digitale Infrastrukturen in urbanen Zentren stark
vorangetrieben werden, bleiben ländliche Regionen oft zurück, was den Zugang zu
Technologien und digitalen Arbeitsmärkten erschwert 
\parencite[vgl.][S. 98–107]{oecd2019measuring}. Die politischen Rahmenbedingungen 
können hier eine entscheidende Rolle spielen, um digitale Ungleichheiten zu vermeiden.

%%%%%%%%%%%%%%%%%%%%%%%%%%%%%%%%%%%%%%%%%%%%%%%%%%%%%%%%%
% Forschungsgegenstand: Arbeitsmarkt und Bildungsniveau %
%%%%%%%%%%%%%%%%%%%%%%%%%%%%%%%%%%%%%%%%%%%%%%%%%%%%%%%%%

% TODO: Leave as is. (14.03.2025)

\subsection{Arbeitsmarkt und Bildungsniveau}

Der Arbeitsmarkt wird maßgeblich durch technologische Entwicklungen beeinflusst,
wobei insbesondere die Digitalisierung bestehende Strukturen verändert. In dieser
Arbeit liegt der Fokus auf der Arbeitslosigkeit nach Bildungsniveau, das üblicherweise
in niedrig, mittel und hoch eingeteilt wird. Diese Differenzierung ermöglicht eine 
gezielte Analyse der Betroffenheit unterschiedlicher Gruppen.

Die Auswirkungen der Digitalisierung auf den Arbeitsmarkt werden häufig mit dem
Konzept der \textit{Jobpolarisierung} beschrieben 
\parencite[vgl.][S. 12]{autor2015whyare}. Hochqualifizierte Fachkräfte profitieren von 
der steigenden Nachfrage nach digitalen Kompetenzen, während Arbeitsplätze mit 
mittleren Qualifikationsanforderungen verstärkt unter Automatisierungsdruck geraten und 
geringqualifizierte sind besonders anfällig 
für Arbeitsplatzverdrängung, da ihre Tätigkeiten häufig leichter durch technologische 
Systeme ersetzt werden können \parencite[vgl.][S. 10]{acemoglu2002technical}.
Um den negativen Effekten entgegenzuwirken, werden gezielte bildungs- und
arbeitsmarktpolitische Maßnahmen als notwendig erachtet. Besonders
Weiterbildungsprogramme für digitale Kompetenzen sind zentral, um den
Strukturwandel am Arbeitsmarkt abzufedern 
\parencite[vgl.][Kap. 13]{brynjolfsson2014thesecond}. Länder mit einem gut ausgebauten 
Weiterbildungssystem könnten die negativen Folgen der Arbeitsmarktpolarisierung besser 
kompensieren. Diese Entwicklung verdeutlicht die Notwendigkeit einer gezielten 
politischen Steuerung, um sozial-ökonomische Risiken zu minimieren.
