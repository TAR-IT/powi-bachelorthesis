% TODO: - (Stand 07.03.2025)

\section{Forschungsstand}

Die Auswirkungen von Digitalisierung und \ac{ICT}-Investitionen auf den Arbeitsmarkt sind
ein zentrales Thema der arbeitsmarktökonomischen Forschung. Während einige Studien den
Fokus auf die technologische Verdrängung bestimmter Berufsgruppen legen, untersuchen
andere, inwieweit institutionelle Rahmenbedingungen wie Wohlfahrtsstaaten die Effekte von
Digitalisierung abmildern oder verstärken. In diesem Kapitel werden zunächst die
allgemeinen Auswirkungen der Digitalisierung auf Arbeitsmärkte analysiert, bevor der Fokus
auf die Rolle von \ac{ICT}-Investitionen und die Unterschiede zwischen verschiedenen
Wohlfahrtsstaaten gelegt wird. Abschließend werden bestehende Forschungslücken aufgezeigt,
die eine weiterführende Analyse notwendig machen.

%%%%%%%%%%%%%%%%%%%%%%%%%%%%%%%%%%%%%%%%%%%%%%%%%%%%%%
% Auswirkungen der Digitalisierung auf Arbeitsmärkte %
%%%%%%%%%%%%%%%%%%%%%%%%%%%%%%%%%%%%%%%%%%%%%%%%%%%%%%

\subsection{Auswirkungen der Digitalisierung auf Arbeitsmärkte}

Die Digitalisierung und insbesondere Investitionen in \ac{ICT} haben die Arbeitsmärkte
weltweit grundlegend verändert. Empirische Studien zeigen, dass diese Entwicklungen die
Beschäftigungsstrukturen in verschiedenen Bildungsgruppen unterschiedlich beeinflussen
\parencite[vgl.][S. 7]{autor2013thegrowth}. Die Automatisierung betrifft besonders
routinisierbare und standardisierbare Tätigkeiten 
\parencite[vgl.][S. 20]{frey2013thefuture}. Diese Entwicklungen führen zu einer 
Polarisierung des Arbeitsmarktes: Hochqualifizierte profitieren von einer steigenden 
Nachfrage nach digitalen Kompetenzen, während Arbeitsplätze mit mittlerem 
Qualifikationsniveau zunehmend unter Automatisierungsdruck geraten
\parencite[vgl.][S. 2509]{goos2014explaining}.

Autor, Levy und Murnane (2003) zeigen, dass Tätigkeiten mit hohem Anteil an routinemäßigen
kognitiven und manuellen Aufgaben besonders anfällig für Automatisierung sind. Die Theorie
des \ac{RBTC} besagt, dass insbesondere klar definierte, sich wiederholende Tätigkeiten 
durch Maschinen ersetzt werden können \parencite[vgl.][S. 1281]{autor2003theskill}. Frey 
und Osborne (2017) erweiterten diese Analyse und schätzten, dass bis zu 47\% der 
Arbeitsplätze in den USA potenziell automatisierbar sind, wobei Berufe mit niedrigem 
Qualifikationsniveau besonders betroffen sind \parencite[vgl.][S. 254]{frey2013thefuture}. 
Diese Erkenntnisse  werfen die Frage auf, inwiefern sich diese Tendenz auf andere Länder 
übertragen lässt und welche Faktoren zur Abmilderung der negativen Effekte beitragen 
können.

Parallel zur Automatisierung zeigt sich eine Polarisierung der Arbeitsmärkte. Während 
mittlere Qualifikationsgruppen unter Druck geraten, profitieren insbesondere 
hochqualifizierte Beschäftigte, die über spezialisierte technologische Kenntnisse 
verfügen, von der steigenden Nachfrage nach digitalen und analytischen Fähigkeiten 
\parencite[vgl.][S. 2510]{goos2014explaining}. Diese Entwicklung führt dazu, dass gut 
ausgebildete Arbeitskräfte mit hohen Qualifikationen von der Digitalisierung profitieren, 
während gering Qualifizierte in wachsendem Maße von Arbeitsplatzverlusten betroffen sind. 
Dies verstärkt das Risiko sozialer Ungleichheit, da Beschäftigungschancen zunehmend 
ungleich verteilt sind. Diese Divergenz wird häufig als „Digital Divide“ bezeichnet, da 
sie die Kluft zwischen hoch- und niedrigqualifizierten Arbeitskräften weiter vertieft 
\parencite[vgl.][S. 10]{acemoglu2002technical}.

Die Auswirkungen der Digitalisierung variieren zudem stark nach Branche und
Wirtschaftssektor. Während einige Sektoren wie die Industrieproduktion oder der
Einzelhandel durch die Einführung automatisierter Systeme massiv verändert wurden,
profitieren wissensintensive Dienstleistungsbranchen von den neuen technologischen 
Möglichkeiten \parencite[vgl.][S. 1555]{autor2013thegrowth}. Besonders betroffen sind 
manuelle Tätigkeiten in der Fertigungsindustrie sowie administrative Büroarbeiten, die 
zunehmend durch algorithmische Prozesse ersetzt werden 
\parencite[vgl.][S. 260]{frey2013thefuture}. Gleichzeitig entstehen neue Arbeitsplätze in 
Bereichen wie IT, Datenwissenschaft, Robotik und \ac{AI}, wodurch sich auch die 
Qualifikationsanforderungen auf dem Arbeitsmarkt verändern 
\parencite[vgl.][S. 2510]{goos2014explaining}.

Zusammenfassend verändert die Digitalisierung Arbeitsmärkte auf mehreren Ebenen: 
Einerseits verstärkt sie das Risiko der Automatisierung insbesondere für Berufe mit 
mittlerem und niedrigem Qualifikationsniveau, andererseits eröffnet sie neue 
Beschäftigungsmöglichkeiten für Hochqualifizierte 
\parencite[vgl.][S. 1555]{autor2013thegrowth}. Die zunehmende Kluft zwischen verschiedenen 
Qualifikationsgruppen hat tiefgreifende Auswirkungen auf die Einkommensverteilung, soziale 
Mobilität und die Notwendigkeit gezielter arbeitsmarktpolitischer Maßnahmen 
\parencite[S. 2510]{goos2014explaining}.

%%%%%%%%%%%%%%%%%%%%%%%%%%%%%%%%%%%%%%%%%%%%%%%%%%%%
% ICT-Investitionen als Treiber der Transformation %
%%%%%%%%%%%%%%%%%%%%%%%%%%%%%%%%%%%%%%%%%%%%%%%%%%%%

\subsection{ICT-Investitionen als Treiber der Transformation}

Investitionen in \ac{ICT} gelten als zentraler Indikator für den Digitalisierungsgrad eines 
Landes und spielen eine Schlüsselrolle bei der Transformation moderner Arbeitsmärkte. Der 
verstärkte Einsatz digitaler Technologien verändert Produktions- und Geschäftsprozesse 
grundlegend und beeinflusst die Nachfrage nach Arbeitskräften in verschiedenen 
Qualifikationsgruppen. Empirische Studien zeigen, dass Unternehmen, die verstärkt in 
\ac{ICT} investieren, effizientere Abläufe entwickeln, ihre Wettbewerbsfähigkeit steigern 
und tendenziell eine höhere Nachfrage nach qualifizierten Arbeitskräften verzeichnen 
\parencite[vgl.][S. 12]{corrado2018intangible}.

Der Einfluss von \ac{ICT}-Investitionen auf den Arbeitsmarkt ist dabei vielschichtig. Laut 
der \ac{OECD} (2019) ermöglichen diese Investitionen nicht nur eine zunehmende 
Automatisierung, sondern tragen auch zur Integration globaler Wertschöpfungsketten bei 
und treiben das wirtschaftliche Wachstum voran 
\parencite[vgl.][S. 15-17]{oecd2019measuring}. Besonders in wissensintensiven Sektoren wie 
Finanzdienstleistungen, IT-gestützte Geschäftsprozesse, E-Commerce oder digitale 
Plattformarbeit entstehen neue Geschäftsmodelle, die verstärkt auf Automatisierung und 
datenbasierte Entscheidungsprozesse setzen.

Während \ac{ICT}-Investitionen neue Arbeitsplätze schaffen können, zeigen zahlreiche 
Studien, dass diese Transformation auch polarisierende Effekte mit sich bringt. 
Hochqualifizierte Arbeitskräfte profitieren von der steigenden Nachfrage nach digitalen 
und analytischen Fähigkeiten, während geringqualifizierte Beschäftigte einem zunehmenden 
Risiko der Arbeitsplatzverdrängung ausgesetzt sind 
\parencite[vgl.][Kap. 2]{brynjolfsson2014thesecond}. Besonders betroffen sind Tätigkeiten 
mit einem hohen Anteil an repetitiven, standardisierten Prozessen, die sich leicht durch 
digitale Technologien oder \ac{AI} ersetzen lassen. Dazu zählen nicht nur 
manuelle Produktionsprozesse, sondern auch administrative Tätigkeiten im Bürobereich, die 
zunehmend durch automatisierte Softwarelösungen abgelöst werden.

Die Polarisierung des Arbeitsmarktes ist eng mit der Theorie des \ac{SBTC} verbunden, 
wonach technologischer Fortschritt die Nachfrage nach hochqualifizierten Arbeitskräften 
erhöht, während mittlere Qualifikationsniveaus unter Druck geraten 
\parencite[vgl.][S. 22]{acemoglu2002technical}. Diese Entwicklung führt zu einer 
Verschiebung in der Beschäftigungsstruktur, da insbesondere wissensintensive Berufe von 
\ac{ICT}-Investitionen profitieren, während traditionelle Berufe in der industriellen 
Fertigung oder im einfachen Dienstleistungsbereich zunehmend verdrängt werden.

Gleichzeitig zeigt sich, dass \ac{ICT}-Investitionen nicht in allen Ländern und Branchen 
gleichermaßen produktivitätssteigernd wirken. Ihre Effekte hängen stark von begleitenden 
wirtschaftspolitischen Maßnahmen ab, darunter Investitionen in digitale Infrastruktur, 
die Förderung digitaler Kompetenzen und die Anpassung von Bildungsprogrammen an die 
veränderten Anforderungen des Arbeitsmarktes 
\parencite[vgl.][S. 77]{brynjolfsson2014thesecond}. Länder mit einer gezielten digitalen 
Transformationsstrategie, wie etwa Südkorea oder die skandinavischen Staaten, konnten in 
den letzten Jahrzehnten eine positive Korrelation zwischen \ac{ICT}-Investitionen und 
Wirtschaftswachstum feststellen \parencite[vgl.][S. 34]{oecd2020digital}. Empirische 
Studien zeigen, dass digitale Infrastruktur und eine strategische Förderung von digitaler 
Bildung eine zentrale Rolle dabei spielen, die wirtschaftlichen Vorteile von 
\ac{ICT}-Investitionen vollständig auszuschöpfen \parencite[vgl.][S. 360]{vu2011ict}. 
Länder mit einem schwächeren Fokus auf digitale Bildung haben größere Schwierigkeiten, 
von diesen Entwicklungen zu profitieren, da der Mangel an digitalen Kompetenzen die 
Innovationskraft und Produktivität hemmt \parencite[vgl.][S. 34]{oecd2020digital}.

Zusammenfassend zeigen \ac{ICT}-Investitionen sowohl wachstumsfördernde als auch 
polarisierende Effekte auf den Arbeitsmarkt. Während sie die Produktivität und 
Wettbewerbsfähigkeit von Unternehmen steigern und neue Beschäftigungsmöglichkeiten für 
hochqualifizierte Arbeitskräfte schaffen, verstärken sie gleichzeitig das Risiko der 
Arbeitsplatzverdrängung für geringqualifizierte Arbeitskräfte. Die Digitalisierung des 
Dienstleistungssektors, die zunehmende Automatisierung administrativer Prozesse und die 
Integration neuer Technologien in industrielle Produktionsabläufe führen dazu, dass 
traditionelle Berufsbilder zunehmend hinterfragt und an neue Anforderungen angepasst 
werden müssen. Diese Entwicklungen unterstreichen die Notwendigkeit 
arbeitsmarktpolitischer Maßnahmen, um den Wandel sozial abzufedern und die Vorteile der 
Digitalisierung möglichst breit in der Gesellschaft zu verteilen.

%%%%%%%%%%%%%%%%%%%%%%%%%%%%%%%%%%%%%%%%%%%
% Unterschiede zwischen Wohlfahrtsstaaten %
%%%%%%%%%%%%%%%%%%%%%%%%%%%%%%%%%%%%%%%%%%%

\subsection{Unterschiede zwischen Wohlfahrtsstaaten}

Empirische Studien zeigen, dass die Auswirkungen von Digitalisierung und 
\ac{ICT}-Investitionen auf Arbeitsmärkte stark von den institutionellen Rahmenbedingungen 
eines Landes abhängen. Regierungen spielen eine zentrale Rolle bei der Förderung 
digitaler Infrastruktur, der Implementierung von Bildungspolitik und der Regulierung des 
Arbeitsmarktes \parencite[vgl.][S. 1–5]{hall2001varieties}. Studien haben gezeigt, dass 
Länder mit hohen Investitionen in digitale Bildung und Infrastruktur tendenziell bessere 
Anpassungsprozesse an den technologischen Wandel durchlaufen 
\parencite[vgl.][S. 23]{oecd2020digital}.

Laut der \ac{OECD} (2019) variieren die Investitionen in \ac{ICT} erheblich zwischen 
Ländern. Skandinavische Staaten und die Niederlande investieren überdurchschnittlich in 
digitale Bildung und Innovationen, während süd- und osteuropäische Länder vergleichsweise 
niedrigere Investitionen tätigen \parencite[vgl.][S. 45]{oecd2020digital}. Diese 
Unterschiede spiegeln sich auch in der Entwicklung der Arbeitsmärkte wider.

Länder mit hoher \ac{ICT}-Investitionsquote (z. B. Schweden, Niederlande) zeigen 
niedrigere Arbeitslosenquoten unter Hochqualifizierten und profitieren von einer 
stärkeren Nachfrage nach digitalen Kompetenzen 
\parencite[vgl.][S. 78]{brynjolfsson2014thesecond}. In Ländern mit geringeren 
\ac{ICT}-Investitionen (z. B. Spanien, Ungarn) sind hingegen vor allem Beschäftigte mit 
mittlerem Qualifikationsniveau einem höheren Automatisierungsrisiko ausgesetzt 
\parencite[vgl.][S. 12]{frey2013thefuture}.

Neben Investitionen in Digitalisierung spielen staatliche Bildungs- und 
Arbeitsmarktpolitiken eine entscheidende Rolle für die Fähigkeit eines Landes, sich an 
technologische Veränderungen anzupassen. Länder mit umfassenden Umschulungs- und 
Weiterbildungsprogrammen (z. B. Dänemark, Deutschland) haben bessere Voraussetzungen, um 
durch lebenslanges Lernen den digitalen Wandel sozialverträglich zu gestalten 
\parencite[vgl.][S. 361]{vu2011ict}. Staaten mit weniger regulierten Arbeitsmärkten 
(z. B. USA, Großbritannien) haben eine schnellere, aber oft ungleichere Anpassung an 
technologische Innovationen, was zu verstärkter Arbeitsplatzpolarisierung führen kann 
\parencite[vgl.][S. 172]{goos2014explaining}.

Studien zeigen, dass das Automatisierungsrisiko je nach Land und Wirtschaftsstruktur stark 
variiert. Laut einer \ac{OECD}-Analyse von Arntz, Gregory \& Zierahn (2016) sind in süd- 
und osteuropäischen Ländern bis zu 40\% der Arbeitsplätze einem hohen 
Automatisierungsrisiko ausgesetzt, während es in skandinavischen Ländern und Deutschland 
nur etwa 20–25\% sind \parencite[vgl.][S. 12]{arntz2016therisk}. Ein entscheidender 
Faktor für diese Unterschiede ist die Wirtschaftsstruktur: Länder mit einem hohen Anteil 
wissensintensiver Dienstleistungen (z. B. Schweden, Niederlande) sind weniger von 
Automatisierung betroffen. Industrie- und produktionslastige Länder (z. B. Spanien, Polen) 
weisen hingegen höhere Risiken für Arbeitsplatzverluste durch Automatisierung auf 
\parencite[vgl.][S. 260]{frey2013thefuture}.

Zusammenfassend zeigen die Unterschiede zwischen Wohlfahrtsstaaten, dass Digitalisierung 
und \ac{ICT}-Investitionen stark von den institutionellen Rahmenbedingungen abhängen. 
Länder mit gezielter Förderung digitaler Infrastruktur und Bildung können die 
Herausforderungen des technologischen Wandels besser bewältigen, während Länder mit 
geringeren Investitionen und restriktiveren Arbeitsmärkten stärkeren Risiken durch 
Automatisierung ausgesetzt sind. Dies verdeutlicht die Bedeutung staatlicher Steuerung 
bei der Gestaltung von Transformationsprozessen im digitalen Zeitalter.

%%%%%%%%%%%%%%%%%%%%
% Forschungslücken %
%%%%%%%%%%%%%%%%%%%%

\subsection{Forschungslücken}

Obwohl zahlreiche Studien die Auswirkungen von Digitalisierung und \ac{ICT}-Investitionen 
untersuchen, bestehen weiterhin relevante Forschungslücken. Die Mehrheit der bisherigen 
Studien konzentriert sich auf die allgemeinen Effekte von Digitalisierung auf den 
Arbeitsmarkt, ohne spezifisch zwischen verschiedenen Wohlfahrtsstaatentypen zu 
unterscheiden. Es fehlen systematische Vergleiche, die institutionelle Faktoren wie 
Bildungssysteme und Arbeitsmarktregulierungen einbeziehen. Viele empirische Studien zur 
Automatisierung betrachten vorwiegend die Situation in den USA, wohingegen umfassende 
Analysen für \ac{OECD}-Länder mit unterschiedlichen Wohlfahrtsmodellen begrenzt sind. Der 
langfristige Einfluss von \ac{ICT}-Investitionen auf die Arbeitslosigkeit verschiedener 
Bildungsgruppen wurde bisher nicht ausreichend mit einer quantitativen, 
länderübergreifenden Panelanalyse untersucht.

Um diese Forschungslücken zu schließen, wird im empirischen Teil dieser Arbeit eine 
Paneldatenanalyse über \ac{OECD}-Länder durchgeführt. Dadurch sollen systematische 
Unterschiede in den Auswirkungen von Digitalisierung auf die Arbeitslosigkeit nach 
Bildungsniveaus erfasst werden.
