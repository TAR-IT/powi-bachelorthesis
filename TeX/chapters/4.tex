%%%%%%%%%%%%%%%%%%%%%%%%%%
% Theorie und Hypothesen %
%%%%%%%%%%%%%%%%%%%%%%%%%%

\section{Theorie und Hypothesen}

Der technologische Wandel und die damit einhergehende Digitalisierung haben tiefgreifende 
Auswirkungen auf die Arbeitswelt. Besonders durch den Fortschritt in der 
Automatisierung von Routineaufgaben und in computergestützten Arbeitsprozessen sind 
zahlreiche Berufe einem fundamentalen Wandel unterworfen 
\parencite[vgl.][S. 2]{frey2013thefuture}. Diese Veränderungen werden durch tiefgreifende 
Innovationsprozesse vorangetrieben, die bestehende Praktiken und Strukturen aufbrechen und 
neue Möglichkeiten schaffen.

Ökonomische Theorien wie die der „kreativen Zerstörung“ von Joseph Schumpeter 
\parencite[vgl.][S. 81–86]{schumpeter1976capitalism} bieten in Verbindung mit dem \ac{SBTC} 
\parencite[vgl.][S. 5–6]{acemoglu2002technical} wertvolle Einsichten in diesen Wandel und 
beschreiben, wie technologische Innovationen bestehende Strukturen destabilisieren und dabei 
Platz für neue schaffen.

%%%%%%%%%%%%%%%%%%%%%%%%%%%%%%%%%%%%%%%%%%%%%%%%%%%%%%%%%%%%%
% Theorie und Hypothesen: Schumpeters „kreative Zerstörung“ %
%%%%%%%%%%%%%%%%%%%%%%%%%%%%%%%%%%%%%%%%%%%%%%%%%%%%%%%%%%%%%

\subsection{Schumpeters „kreative Zerstörung“}

Schumpeter beschreibt den Kapitalismus nicht als ein statisches, sondern als ein 
dynamisches, sich ständig veränderndes System 
\parencite[vgl.][S. 82–83]{schumpeter1976capitalism}. Seine Theorie geht davon aus, dass der 
Kapitalismus durch einen kontinuierlichen Innovationsprozess geprägt ist, der ständig neue 
Produkte, Prozesse und Märkte hervorbringt. Innovationen, vorangetrieben von Unternehmer*innen, 
sind die treibende Kraft hinter diesem Prozess, der sowohl technologischen Fortschritt als 
auch die Veränderung gesellschaftlicher und wirtschaftlicher Strukturen umfasst 
\parencite[vgl.][S. 82]{schumpeter1976capitalism}. 
Gleichzeitig führen diese Innovationen nicht nur zur Entstehung neuer Märkte und 
Unternehmen, sondern zerstören auch bestehende Strukturen. Schumpeter bezeichnet diesen 
Prozess als „schöpferische Zerstörung“ 
\parencite[vgl.][S. 83]{schumpeter1976capitalism}.

Dieser Prozess der „schöpferischen Zerstörung“ ist nicht nur unvermeidbar, sondern laut 
Schumpeter auch notwendig, um Raum für Innovation und langfristiges Wirtschaftswachstum zu 
schaffen \parencite[vgl.][S. 83]{schumpeter1976capitalism}. Der Kapitalismus lebt von dieser 
kontinuierlichen Erneuerung, wobei neue Technologien und Geschäftsmodelle ältere verdrängen. 
Die Digitalisierung und insbesondere Investitionen in \ac{ICT} sind moderne Beispiele für 
diesen Prozess. Neue technologische Entwicklungen, wie Cloud Computing, \ac{AI} und 
Automatisierung haben dazu geführt, dass bestehende Arbeitsmethoden und Geschäftsprozesse 
obsolet werden, was zu einer Umstrukturierung ganzer Branchen führt 
\parencite[vgl.][S. 14–15]{frey2013thefuture}.

Im Zusammenhang mit der Digitalisierung wird die „schöpferische Zerstörung“ zu einem 
zentralen Konzept. Während durch neue Technologien neue Arbeitsplätze entstehen, 
verschwinden gleichzeitig traditionelle Tätigkeiten. Schumpeter zeigt, dass dieser 
Prozess kurzfristig zu Marktunruhe und Arbeitsplatzverlusten führen kann, aber langfristig 
eine Voraussetzung für wirtschaftliche Erneuerung und Wachstum ist 
\parencite[vgl.][S. 151–154]{schumpeter1976capitalism}.

% TODO: Re-read. (14.03.2025)

Ein entscheidender Faktor bei der „schöpferischen Zerstörung“ ist die Fähigkeit von 
Unternehmen und Arbeitskräften, sich an neue technologische Anforderungen anzupassen. Diese 
Anpassungsfähigkeit erfordert nicht nur den Einsatz neuer Technologien, sondern auch eine 
schnelle Entwicklung neuer Fähigkeiten und Kenntnisse bei den Arbeitskräften. Schumpeter 
betont, dass nicht alle gleichermaßen von den neuen Entwicklungen profitieren. Der Übergang 
zu neuen Technologien bringt oft schmerzhafte Anpassungsprozesse mit sich, die besonders 
für weniger anpassungsfähige Arbeitskräfte herausfordernd sind 
\parencite[vgl.][S. 110–111]{schumpeter1976capitalism}.

Im Kontext der Digitalisierung gewinnen daher nicht nur Investitionen in technologische 
Infrastruktur, sondern auch in die Bildung und Ausbildung von Arbeitskräften zunehmend an 
Bedeutung. Schumpeter selbst betont, dass Innovation und die Anpassung an neue Technologien 
nicht nur von den Unternehmern, sondern auch von der gesamten Gesellschaft getragen werden 
müssen \parencite[vgl.][S. 132]{schumpeter1976capitalism}. Es geht darum, die Arbeitskräfte 
mit den notwendigen Fähigkeiten auszustatten, damit sie den Wandel aktiv mitgestalten 
können. 

Studien zeigen, dass Länder, die verstärkt in digitale Kompetenzen investieren, besser auf 
die technologischen Herausforderungen reagieren können 
\parencite[S. 15–17]{oecd2019measuring}. Investitionen in \ac{ICT} durch Unternehmen führen 
zudem zu Effizienzgewinnen und Produktivitätssteigerungen, die sich positiv auf das 
Wirtschaftswachstum und die Beschäftigung auswirken können. Schumpeter weist jedoch auch 
darauf hin, dass diese Effekte nicht automatisch und gleichmäßig auf alle Teile der 
Gesellschaft verteilt sind. Die Auswirkungen hängen stark von der Anpassungsfähigkeit der 
Arbeitskräfte und der Struktur des Bildungssystems ab \parencite[S. 48]{oecd2019measuring}.

Im Kontext von Wohlfahrtsstaaten wird Schumpeters Konzept besonders relevant. So könnten 
Institutionen, die Weiterbildung und Umschulungen fördern, den Übergang erleichtern und die 
negativen Effekte der „schöpferischen Zerstörung“ abmildern. Länder mit aktiven 
Bildungssystemen und sozialpolitischen Maßnahmen könnten besser in der Lage sein, den 
Wandel positiv zu gestalten, während weniger flexible Systeme größere Schwierigkeiten haben 
könnten, die Herausforderungen der Digitalisierung zu bewältigen 
\parencite[vgl.][S. 29–31]{espingandersen1990thethree}.

%%%%%%%%%%%%%%%%%%%%%%%%%%%%%%%%%%%%%%%%%%%%%%%%%%%%%%%%%%%%%
% Theorie und Hypothesen: Skill-biased technological change %
%%%%%%%%%%%%%%%%%%%%%%%%%%%%%%%%%%%%%%%%%%%%%%%%%%%%%%%%%%%%%

\subsection{Skill-biased technological change}

Ein zentraler Ansatz in der Debatte über die Folgen der Digitalisierung auf den 
Arbeitsmarkt ist die Theorie des \ac{SBTC}. Diese Theorie besagt, dass technologische 
Innovationen, insbesondere im Bereich der digitalen Technologien, eine 
Nachfrageverschiebung zugunsten von hochqualifizierten Arbeitskräften verursachen 
\parencite[vgl.][S. 1]{violante2008skill}. Der Grundmechanismus hinter \ac{SBTC} 
liegt in der Tatsache, dass digitale Technologien und Automatisierungsprozesse bestimmte 
Aufgaben in Unternehmen effizienter und kostengünstiger machen 
\parencite[vgl.][S. 2–3]{violante2008skill}. Dabei werden insbesondere 
Routineaufgaben, die vorher von mittleren Qualifikationsniveaus übernommen wurden, 
zunehmend durch Algorithmen, Maschinen oder Outsourcing ersetzt
\parencite[vgl.][S. 1279]{autor2003theskill}.

Diese Nachfrageverschiebung hat tiefgreifende Konsequenzen für den Arbeitsmarkt. 
Hochqualifizierte Arbeitnehmer*innen, die über die notwendigen digitalen und technischen 
Fähigkeiten verfügen, profitieren in der digitalen Transformation von der wachsenden 
Nachfrage. Sie können neue Technologien effektiv nutzen und sind daher nicht nur besser in 
der Lage, mit der „schöpferischen Zerstörung“ 
\parencite[vgl.][S. 83]{schumpeter1976capitalism} umzugehen, sondern auch von ihr zu 
profitieren - gleichzeitig steigen diesen hochqualifizierten Bereichen der \ac{ICT}, des 
Ingenieurwesens sowie in datenintensiven Berufen die Löhne, was den bereits 
bestehenden Einkommensunterschied zwischen Hoch- und Geringqualifizierten weiter verstärkt 
\parencite[vgl.][S. 2–3]{violante2008skill}.

Während einige manuelle Tätigkeiten in Bereichen wie dem Dienstleistungssektor (z. B. 
Gastronomie, Pflege) weiterhin Bestand haben, sind vor allem industrielle Produktionsprozesse 
und administrative Büroarbeiten stark von Automatisierung betroffen 
\parencite[vgl.][S. 44]{frey2013thefuture}. 
Dies führt nicht nur zu Arbeitsplatzverlusten in bestimmten Sektoren, sondern verstärkt auch 
die Polarisierung des Arbeitsmarktes, indem vor allem Berufe mit mittlerem 
Qualifikationsniveau wegfallen und der Markt sich zunehmend in hochqualifizierte (besser 
bezahlte) und geringqualifizierte (schlechter bezahlte) Jobs aufteilt 
\parencite[vgl.][S. 45]{frey2013thefuture}.

Zusammenfassend hat diese Polarisierung der Arbeitsmärkte, die durch Theorien wie den 
\ac{SBTC} und den \ac{RBTC} beschrieben werden, bedeutende soziale und wirtschaftliche 
Folgen. Der technologische Fortschritt führt dazu, dass immer mehr einfache und routinemäßige 
Aufgaben durch Algorithmen und Maschinen übernommen werden, was zu einer strukturellen 
Arbeitslosigkeitin bestimmten Qualifikationsgruppen führen kann 
\parencite[vgl.][S. 2524]{goos2014explaining}. 

%%%%%%%%%%%%%%%%%%%%%%%%%%%%%%%%%%%%%%%%%%%%%
% Theorie und Hypothesen: Wohlfahrtsstaaten %
%%%%%%%%%%%%%%%%%%%%%%%%%%%%%%%%%%%%%%%%%%%%%

\subsection{Wohlfahrtsstaaten} 

Esping-Andersen (1990) beschreibt in seiner klassischen Typologie der Wohlfahrtsstaaten, 
wie institutionelle Rahmenbedingungen die Struktur und Dynamik von Arbeitsmärkten und 
Bildungssystemen prägen. Diese Typologie unterscheidet drei grundlegende Regimetypen, die 
unterschiedliche Ansätze verfolgen, um soziale Sicherheit, Beschäftigung und 
wirtschaftliche Entwicklung zu fördern \parencite[vgl.][S. 26–29]{espingandersen1990thethree}.

\begin{enumerate}

    \item \textbf{Sozialdemokratisch} (im Folgenden „nordisch“, z. B. Schweden): Diese 
    Staaten zeichnen sich durch einen hohen Grad an De-Kommodifizierung und Universalismus 
    aus, wobei soziale Rechte nicht nur für die unteren Einkommensgruppen, sondern auch für die 
    Mittelschicht ausgebaut wurden \parencite[S. 27–28]{espingandersen1990thethree}. 
    Sozialdemokratische Wohlfahrtsstaaten streben eine umfassende soziale Absicherung an, 
    die eine Gleichheit auf hohem Niveau anstatt nur eine Mindestabsicherung gewährleistet. 
    Dies wird durch staatlich bereitgestellte Sozialleistungen und eine aktive 
    Arbeitsmarktpolitik ermöglicht, die Vollbeschäftigung und Chancengleichheit fördern 
    \parencite[vgl.][S. 27–28]{espingandersen1990thethree}. Diese Struktur könnte dazu 
    beitragen, dass nordische Staaten besser auf digitale Transformationen vorbereitet sind, 
    da sie Weiterbildung und soziale Absicherung systematisch integrieren.

    % TODO: Re-read. (14.03.2025)

    \item \textbf{Konservativ} (im Folgenden „mitteleuropäisch“, z. B. Deutschland): Diese 
    Wohlfahrtsstaaten sind durch eine starke Kopplung sozialer Rechte an den Arbeitsmarkt 
    gekennzeichnet, wobei soziale Absicherung überwiegend über Beschäftigungsstatus und 
    Beitragszahlungen erfolgt \parencite[S. 27]{espingandersen1990thethree}. Das System ist 
    historisch von korporatistischen Strukturen geprägt, die soziale Hierarchien erhalten,
    anstatt eine umfassende Gleichstellung anzustreben. Regulierte Arbeitsmärkte 
    und umfangreiche Tarifvereinbarungen sorgen für Arbeitsplatzsicherheit, können jedoch 
    auch zu struktureller Trägheit führen \parencite[vgl.][S. 27]{espingandersen1990thethree}.  
    Gleichzeitig verfügen konservative Wohlfahrtsstaaten wie Deutschland über gut ausgebaute 
    duale Ausbildungssysteme, die eine enge Verzahnung zwischen Bildung und Wirtschaft 
    ermöglichen \parencite[vgl.][S. 21–27]{hall2001varieties}. Diese Systeme stärken die 
    Wettbewerbsfähigkeit der Arbeitskräfte in technologieintensiven Sektoren, indem sie 
    praktische und technische Fähigkeiten vermitteln. Während die starke Regulierung 
    kurzfristig Schutz bietet, kann sie langfristig Anpassungen an digitale Disruptionen 
    erschweren \parencite[vgl.][S. 20–21]{hall2001varieties}. Dennoch trägt das duale 
    System zur Resilienz des Arbeitsmarktes bei, da es den Übergang in neue technologische 
    Anforderungen erleichtert \parencite[vgl.][S. 25–27]{hall2001varieties}.

    \item \textbf{Liberal} (im Folgenden „angelsächsisch“, z. B. USA): In liberalen 
    Wohlfahrtsstaaten dominieren marktorientierte Mechanismen, während staatliche 
    Sozialleistungen auf ein Minimum beschränkt bleiben und vorrangig bedarfsgeprüft sind 
    \parencite[vgl.][S. 26–27]{espingandersen1990thethree}. Sozialpolitische Maßnahmen zielen 
    weniger auf umfassende Absicherung als auf die Förderung individueller Selbstverantwortung 
    ab. Der Arbeitsmarkt ist durch geringe Regulierung und flexible Beschäftigungsmodelle 
    gekennzeichnet, was sowohl Chancen als auch Risiken mit sich bringt.  
    Hochqualifizierte Arbeitskräfte profitieren in diesen Systemen häufig von **Digitalisierung 
    und ICT-Investitionen**, da Unternehmen stark in neue Technologien investieren. Geringqualifizierte 
    Arbeitskräfte hingegen sind aufgrund schwacher Schutzmechanismen und begrenzter öffentlicher 
    Weiterbildungsangebote einem höheren Risiko der Arbeitsplatzunsicherheit ausgesetzt 
    \parencite[vgl.][S. 12–13]{goodin1999thereal}. Da der technologische Fortschritt weitgehend 
    von privaten Unternehmen vorangetrieben wird, können digitale Innovationen in diesen Ländern 
    zwar schneller implementiert werden, doch die ungleiche Verteilung der Anpassungsfähigkeit 
    führt oft zu einer stärkeren Polarisierung des Arbeitsmarktes.
    

\end{enumerate}

Die ursprüngliche Typologie von Esping-Andersen wurde von verschiedenen Forschern 
erweitert, um regionale Besonderheiten und neue Entwicklungen zu berücksichtigen. Zwei 
zusätzliche Wohlfahrtsstaatentypen sind besonders relevant:

\begin{enumerate}

    \item \textbf{Südeuropäisch} (z. B. Spanien): Diese Regime zeichnen sich durch 
    segmentierte Arbeitsmärkte, starre Arbeitsmarktregulierungen und eine schwache 
    Verknüpfung zwischen Bildungssystemen und Arbeitsmarkt aus 
    \parencite[S. 19]{ferrera1996thesouthern}. Schwächen in der Weiterbildung 
    und starke Ungleichheiten zwischen regulären und prekären Beschäftigungsverhältnissen 
    machen diese Länder anfälliger für negative Effekte der Digitalisierung 
    \parencite[S. 19–20]{ferrera1996thesouthern}. Da unbefristete Stellen oft starken 
    Kündigungsschutz genießen, führt dies zu einer Zweiteilung des Arbeitsmarktes, in dem 
    junge und geringqualifizierte Arbeitskräfte besonders stark von Arbeitslosigkeit 
    betroffen sind \parencite[S. 19–21]{ferrera1996thesouthern}.
    
    \item \textbf{Postsozialistisch} (z. B. Polen): Postsozialistische Länder befinden sich 
    in einem Übergang von zentral geplanten zu marktwirtschaftlichen Systemen. Sie sind 
    häufig durch geringe Regulierung und unzureichend entwickelte Bildungs- und 
    Weiterbildungsstrukturen gekennzeichnet \parencite[S. 88–93]{cerami2006socialpolicy}. 
    Diese Defizite erschweren die Integration von Arbeitskräften in digitale Sektoren und 
    verstärken die regionale Ungleichheit \parencite[S. 88–93]{cerami2006socialpolicy}. 
    Zudem sind postsozialistische Staaten oft durch eine hohe wirtschaftliche Dynamik 
    geprägt, doch der technologische Wandel verläuft nicht überall gleichmäßig. Während 
    große Städte und wirtschaftliche Zentren stark in digitale Technologien investieren, 
    bleiben ländliche Regionen oft zurück, was zu einer wachsenden Kluft zwischen digital 
    integrierten und traditionell geprägten Wirtschaftssektoren führt 
    \parencite[S. 90]{cerami2006socialpolicy}.
    
\end{enumerate}

Die institutionellen Rahmenbedingungen der verschiedenen Wohlfahrtsstaatentypen beeinflussen 
maßgeblich, wie Arbeitsmärkte auf technologische Veränderungen reagieren. Nordische Systeme 
mit umfassenden Bildungs- und Arbeitsmarktprogrammen können die negativen Effekte der 
Digitalisierung abmildern und die Integration sowohl hoch- als auch geringqualifizierter 
Arbeitskräfte fördern \parencite[S. 27–30]{espingandersen1990thethree}. Im Gegensatz dazu 
können starre Regulierungen in mittel- und südeuropäischen Regimen die Anpassung an 
digitale Transformationen verlangsamen \parencite[S. 155]{ferrera1996thesouthern}, während 
angelsächsische Regime oft eine stärkere Polarisierung zwischen Qualifikationsgruppen 
erleben \parencite[S. 3–5]{hall2001varieties}. Postsozialistische Wohlfahrtsstaaten 
kämpfen hingegen mit strukturellen Schwächen, die ihre Fähigkeit zur erfolgreichen 
Integration digitaler Technologien behindern \parencite[S. 88–93]{cerami2006socialpolicy}. 
Die digitale Transformation bringt somit unterschiedliche Herausforderungen für die 
jeweiligen Wohlfahrtsstaaten mit sich, deren Bewältigung maßgeblich von staatlicher 
Steuerung, Bildungs- und Arbeitsmarktpolitiken sowie Investitionen in digitale 
Infrastruktur abhängt \parencite[S. 23]{oecd2020digital}.

%%%%%%%%%%%%%%%%%%%%%%%%%%%%%%%%%%%%%%
% Theorie und Hypothesen: Hypothesen %
%%%%%%%%%%%%%%%%%%%%%%%%%%%%%%%%%%%%%%

\subsection{Hypothesen}

Basierend auf den theoretischen Überlegungen, insbesondere der Theorie der „schöpferischen 
Zerstörung“ von Schumpeter, sowie auf dem aktuellen Forschungsstand lassen sich im Rahmen 
dieser Arbeit mehrere Hypothesen formulieren, die empirisch überprüft werden sollen. Diese 
Hypothesen zielen darauf ab, die Auswirkungen der Digitalisierung und der Investitionen in 
\ac{ICT} auf den Arbeitsmarkt zu analysieren, insbesondere in Bezug auf unterschiedliche 
Bildungsniveaus. Dabei wird berücksichtigt, dass technologische Innovationen nicht nur 
bestehende Arbeitsplätze verdrängen, sondern auch neue berufliche Chancen eröffnen - 
abhängig von den Fähigkeiten und Qualifikationen der Arbeitskräfte.

\textbf{H1:} \textit{ Länder, in denen verstärkt in Informations- und Kommunikationstechnologien 
investiert wird, weisen eine geringere Arbeitslosenquote unter hochqualifizierten 
Arbeitskräften auf.}

% TODO: Re-read. (14.03.2025)

Dieser Effekt ist auf die gesteigerte Produktivität und den Bedarf an komplexen, kreativen 
Fähigkeiten zurückzuführen, die durch digitale Technologien gefördert werden 
\parencite[vgl.][S. 5–8]{acemoglu2002technical}. Im Zuge der Digitalisierung entstehen neue 
Arbeitsplätze, die spezifische technologische Fähigkeiten und Kompetenzen erfordern. 
Hochqualifizierte Arbeitskräfte, die über die notwendigen digitalen und technischen 
Fähigkeiten verfügen, sind in der Lage, diese neuen Arbeitsmärkte zu bedienen. Gleichzeitig 
können sie die mit digitalen Technologien verbundenen Produktivitätssteigerungen und 
Effizienzgewinne optimal nutzen \parencite[vgl.][Kap. 2]{brynjolfsson2014thesecond}. In 
Ländern, die verstärkt in \ac{ICT} investieren, ist zu erwarten, dass diese Investitionen 
eine verstärkte Nachfrage nach hochqualifizierten Arbeitskräften erzeugen. Da die Nachfrage 
nach qualifizierten Arbeitskräften wächst und gleichzeitig das Angebot dieser Arbeitskräfte 
nicht in gleichem Maße wächst, wird die Arbeitslosenquote unter hochqualifizierten 
Arbeitskräften tendenziell sinken. Diese Entwicklung spiegelt die These der „schöpferischen 
Zerstörung“ von Schumpeter wider, nach der durch Innovationen neue Märkte entstehen, die 
speziell hochqualifizierte Fachkräfte anziehen und somit Arbeitslosigkeit in dieser Gruppe 
verringern können \parencite[vgl.][S. 103–106]{schumpeter1976capitalism}.

\textbf{H2:} \textit{In Ländern mit hohen \ac{ICT}-Investitionen verlagert sich die Arbeitslosigkeit 
auf niedrigqualifizierte Arbeitskräfte.}

Der technologische Fortschritt im Bereich der Digitalisierung führt zu einer zunehmenden 
Automatisierung und der Nutzung von Algorithmen und Maschinen in Arbeitsprozessen, die 
früher manuelle oder einfache Aufgaben erforderten. Niedrigqualifizierte Arbeitskräfte, die 
auf diese Art von Tätigkeiten angewiesen sind, sehen sich einem höheren Risiko ausgesetzt, 
durch Maschinen ersetzt zu werden \parencite[vgl.][S. 5–10]{autor2015whyare}. Gleichzeitig 
steigt die Nachfrage nach hochqualifizierten Arbeitskräften, die in der Lage sind, mit den 
neuen Technologien zu arbeiten und sie zu steuern. In Ländern, die hohe Investitionen in 
\ac{ICT} tätigen, werden diese Trends noch verstärkt, da die Automatisierung in den 
Sektoren, die viele niedrigqualifizierte Arbeitskräfte beschäftigen, schneller 
voranschreitet \parencite[vgl.][S. 254]{frey2013thefuture}. Diese Entwicklung kann dazu 
führen, dass sich die Arbeitslosigkeit stärker auf niedrigqualifizierte Gruppen verlagert, 
während hochqualifizierte Arbeitskräfte von der Digitalisierung profitieren. Im Einklang 
mit der Theorie der \ac{SBTC} wird angenommen, dass diese Verschiebung der Arbeitslosigkeit 
von niedrigqualifizierten hin zu hochqualifizierten Arbeitskräften in Ländern mit 
intensiven \ac{ICT}-Investitionen besonders ausgeprägt ist 
\parencite[vgl.][S. 3]{acemoglu2019robots}.

\textbf{H3:} \textit{Der Typ des Wohlfahrtsstaates hat Einfluss auf die Polarisierung des 
Arbeitsmarktes. Länder mit stark entwickelten wohlfahrtsstaatlichen Systemen und flexiblen 
Arbeitsmarktstrukturen zeigen eine geringere Polarisierung auf.}

Die institutionellen Rahmenbedingungen eines Landes beeinflussen maßgeblich, wie stark die 
Polarisierung des Arbeitsmarktes durch Digitalisierung ausgeprägt ist. Nordische 
Wohlfahrtsstaaten mit robusten sozialen Sicherungssystemen und umfassenden Bildungs- und 
Weiterbildungsprogrammen können die negativen Effekte der Digitalisierung abmildern 
\parencite[vgl.][S. 27–28]{espingandersen1990thethree}. Im Gegensatz dazu fördern 
angelsächsische Arbeitsmärkte, wie sie in den USA und Großbritannien vorherrschen, häufig 
eine stärkere Spaltung zwischen hoch- und niedrigqualifizierten Arbeitskräften 
\parencite[vgl.][12–13]{goodin1999thereal}. mitteleuropäische Wohlfahrtsstaaten mit stark 
regulierten Arbeitsmärkten (z. B. Deutschland, Frankreich) bieten zwar Schutzmechanismen, 
können aber die Integration von geringqualifizierten Arbeitskräften erschweren 
\parencite[vgl.][S. 78]{hall2001varieties}. Südeuropäische Wohlfahrtsstaaten hingegen 
verstärken durch starre Regulierungen und segmentierte Arbeitsmärkte die Polarisierung 
\parencite[vgl.][S. 17–37]{ferrera1996thesouthern}.
