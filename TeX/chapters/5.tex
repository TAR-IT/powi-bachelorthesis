%%%%%%%%%%%%%%%%%%%%%%
% Daten und Methodik %
%%%%%%%%%%%%%%%%%%%%%%

% TODO: Re-read. (14.03.2025)

\section{Daten und Methodik}

%%%%%%%%%%%%%%%%%%%%%%%%%%%%%%%%%%
% Daten und Methodik: Datensätze %
%%%%%%%%%%%%%%%%%%%%%%%%%%%%%%%%%%

% TODO: Re-read. (14.03.2025)

\subsection{Datensätze}

Die vorliegenden Daten stammen aus den umfangreichen Datensätzen der \ac{OECD}, einer 
internationalen Organisation, die vergleichbare Wirtschafts- und Sozialstatistiken für ihre 
Mitgliedsländer bereitstellt. Die \ac{OECD} sammelt und veröffentlicht regelmäßig Daten zu 
wirtschaftlichen, sozialen und technologischen Entwicklungen, die es ermöglichen, langfristige 
Trends und länderspezifische Unterschiede zu analysieren \parencite{oecd2022ict}.

Für diese Untersuchung werden insbesondere die Datensätze zu \ac{ICT}-Investitionen 
\parencite{oecd2022ict} sowie zu den Arbeitslosenquoten nach Bildungsniveau 
\parencite{oecd2022unemployment} verwendet. Zusätzlich wurden weitere ökonomische und 
institutionelle Indikatoren als Kontrollvariablen integriert, um die Robustheit der Analyse 
zu erhöhen. Dazu gehören das \ac{BIP} pro Kopf \parencite{oecd2022gdp}, die 
Gewerkschaftsdichte \parencite{oecd2022tud}, der Anteil der Bevölkerung mit tertiärem 
Bildungsabschluss \parencite{oecd2022education} sowie der Grad der Regulierung des 
Arbeitnehmerschutzes \parencite{oecd2022regulation}. Zudem wird die Wohlfahrtsstaatentypologie 
nach Esping-Andersen \parencite{espingandersen1990thethree} genutzt, um institutionelle 
Unterschiede zwischen den Ländern zu erfassen. Die Wohlfahrtsstaaten-Variable wird separat 
betrachtet, da sie nichtnur eine Kontrollvariable darstellt, sondern auch eine eigene 
Hypothese in der Analyse testet. Die finalen Daten umfassen insgesamt 35 OECD- und 
ausgewählte Nicht-OECD-Länder
\footnote{
    Untersuchte Länder: Australien, Österreich, Belgien, Bulgarien, Brasilien, Kanada, 
    Kroatien, Tschechien, Dänemark, Estland, Finnland, Frankreich, Deutschland, Griechenland, 
    Ungarn, Island, Italien, Irland, Lettland, Litauen, Luxemburg, Niederlande, Neuseeland, 
    Norwegen, Polen, Portugal, Rumänien, Spanien, Schweden, Schweiz, Türkei, Slowakei, 
    Slowenien, Vereinigtes Königreich, USA.
} und decken den Zeitraum von 2005 bis 2022 ab. Nach 
der Bereinigung und Zusammenführung der relevanten Variablen verbleiben 3973 Beobachtungen 
für die Paneldatenanalyse.

Die \ac{ICT}-Investitionen messen die Bruttoanlageinvestitionen in digitale Infrastrukturen 
und Technologien \parencite{oecd2022ict}. Die Arbeitsmarktstatistiken bieten detaillierte 
Informationen über die Arbeitslosenquoten in verschiedenen Bildungsgruppen 
\parencite{oecd2022unemployment}. Durch die Ergänzung um Kontrollvariablen wie den Anteil 
tertiär gebildeter Personen und die \textit{Regulierungsstrenge des Arbeitsmarktes} wird 
sichergestellt, dass sowohl wirtschaftliche als auch institutionelle Unterschiede in den 
Ländern angemessen berücksichtigt werden. Zur Sicherstellung einer vollständigen Zeitreihe 
wurden fehlende Werte der Variablen \textit{Gewerkschaftsdichte}, 
\textit{Tertiärer Bildungsanteil} und \textit{Arbeitsmarktregulierung} mittels linearer 
Inter- und Extrapolation ergänzt. Zudem wurde die Variable \textit{\ac{BIP} pro Kopf} zur 
besseren Interpretierbarkeit durch 1000 geteilt.

%%%%%%%%%%%%%%%%%%%%%%%%%%%%%%%%%%%%%%%%%%%
% Daten und Methodik: Operationalisierung %
%%%%%%%%%%%%%%%%%%%%%%%%%%%%%%%%%%%%%%%%%%%

% TODO: Re-read. (14.03.2025)

\subsection{Operationalisierung}

Zur Beantwortung der Forschungsfrage - wie Investitionen in \ac{ICT} die Arbeitslosenquoten 
in unterschiedlichen Bildungsniveaus beeinflussen - ist eine präzise und konsistente 
Operationalisierung der zentralen Konzepte notwendig. Dies gewährleistet, dass die 
Untersuchung die beabsichtigten Zusammenhänge abbildet und die Daten sinnvoll ausgewertet 
werden können.

Die abhängige Variable dieser Untersuchung ist die \textit{Arbeitslosenquote} 
(UNEMPLOYMENT\_RATE\_PERCENT), die nach dem Bildungsniveau der Bevölkerung differenziert 
wird. Der \ac{OECD}-Datensatz unterteilt das Bildungsniveau in drei Hauptkategorien:

\begin{enumerate}

    \item \textbf{niedriges Bildungsniveau} (Low education): Personen ohne abgeschlossene 
    Schulbildung oder mit einem maximalen Hauptschulabschluss \parencite{oecd2022unemployment}.

    \item \textbf{mittleres Bildungsniveau} (Medium education): Personen mit 
    Sekundarschulabschluss oder einer abgeschlossenen Berufsausbildung 
    \parencite{oecd2022unemployment}.

    \item \textbf{hohes Bildungsniveau} (High education): Personen mit Hochschulabschluss, wie 
    einem Bachelor, Master oder Doktortitel \parencite{oecd2022unemployment}.

\end{enumerate}

Arbeitslose sind nach der Definition der \ac{OECD} Personen im erwerbsfähigen Alter, die keine 
Arbeit haben, für eine Arbeit zur Verfügung stehen und in den letzten vier Wochen konkrete 
Schritte unternommen haben, um eine Arbeit zu finden \parencite{oecd2022unemployment}. Dieser 
Indikator wird als Prozentsatz der Erwerbsbevölkerung gemessen und ist saisonbereinigt.

Die zentrale unabhängige Variable \textit{\ac{ICT}-Investitionen} (ICT\_INVEST\_SHARE\_GDP) 
misst Investitionen in digitale Infrastruktur, Software, Hardware und Technologien, die zur 
Verbesserung betrieblicher Effizienz und Produktivität beitragen \parencite{oecd2022ict}. Die 
Daten basieren auf den Definitionen des \ac{SNA08} und werden als Anteil am \ac{BIP} in 
Prozent angegeben.

Um sicherzustellen, dass der Effekt der \textit{\ac{ICT}-Investitionen} auf die 
\textit{Arbeitslosenquote} nicht durch andere Faktoren verzerrt wird, werden mehrere 
Kontrollvariablen in die Analyse aufgenommen:

\begin{itemize}
    \item \textbf{\ac{BIP} pro Kopf} (GDP\_PER\_CAPITA): Diese Variable misst den 
    wirtschaftlichen Wohlstand eines Landes in tausend US-Dollar pro Jahr und kontrolliert 
    den Entwicklungsstand eines Landes, da wirtschaftlich wohlhabendere Länder tendenziell 
    niedrigere Arbeitslosenquoten aufweisen \parencite{oecd2022gdp}.

    \item \textbf{Gewerkschaftsdichte} (PERCENT\_EMPLOYEES\_TUD): Der Anteil der in 
    Gewerkschaften organisierten Arbeitnehmer wird berücksichtigt, da Gewerkschaften eine 
    wichtige Rolle bei der Aushandlung von Arbeitsbedingungen und Arbeitsplatzsicherheit 
    spielen \parencite{oecd2022tud}. Frühere Studien zeigen, dass eine hohe Gewerkschaftsdichte 
    oft mit niedrigeren Arbeitslosenquoten für geringqualifizierte Arbeitnehmer verbunden ist, 
    da Gewerkschaften Mindestlöhne sichern und Beschäftigungsschutzmaßnahmen verstärken 
    \parencite[S. 61]{nickell1997unemployment}.

    \item \textbf{Tertiärer Bildungsanteil} (PERCENT\_TERTIARY\_EDUCATION): Diese 
    Variable gibt den Prozentsatz der Bevölkerung an, der einen tertiären Bildungsabschluss 
    besitzt. Eine höhere Bildungsbeteiligung könnte den negativen Einfluss von 
    \ac{ICT}-Investitionen auf geringqualifizierte Arbeitnehmer abschwächen, da mehr Menschen 
    für technologische Berufe qualifiziert sind \parencite{oecd2022education}.

    \item \textbf{Arbeitsmarktregulierung} (REGULATION\_STRICTNESS): Diese 
    Variable misst, wie stark der Arbeitsmarkt eines Landes reguliert ist, insbesondere im 
    Hinblick auf Kündigungsschutz und Beschäftigungsflexibilität. Striktere Regulierung kann 
    die Arbeitslosenquote erhöhen, da Unternehmen zögerlicher bei Neueinstellungen sind 
    \parencite{oecd2022regulation}.

    \item \textbf{Wohlfahrtsstaatentyp} (WELFARE\_STATE): Die Typologie nach Esping-Andersen 
    \parencite{espingandersen1990thethree} wird in die Analyse integriert, um zu untersuchen, 
    inwiefern institutionelle Unterschiede den Effekt von \textit{\ac{ICT}-Investitionen} auf 
    die \textit{Arbeitslosenquote} beeinflussen. Die Länder werden in der Analyse in fünf 
    Kategorien unterteilt
    \footnote{
        Die Zuordnung erfolgt wie folgt: 
        \textit{nordisch} (Dänemark, Schweden, Norwegen, Finnland, Island); 
        \textit{mitteleuropäisch} (Deutschland, Frankreich, Österreich, Belgien, Niederlande, 
        Luxemburg, Schweiz); 
        \textit{angelsächsisch} 
        (Vereinigte Staaten, Vereinigtes Königreich, Kanada, Australien, Neuseeland, Irland); 
        \textit{südeuropäisch} (Italien, Spanien, Portugal, Griechenland); 
        \textit{postsozialistisch} (Polen, Tschechien, Ungarn, Slowakei, Slowenien, Estland, 
        Lettland, Litauen, Rumänien, Bulgarien).
    }: \textit{nordisch}, \textit{mitteleuropäisch}, \textit{angelsächsisch}, 
    \textit{südeuropäisch} und \textit{postsozialistisch}. Diese Kontrollvariable wird 
    gesondert als eigene Hypothese getestet.

\end{itemize}

Die Kombination dieser Daten ermöglicht es, länderspezifische Unterschiede in der 
Wirtschaftskraft, den regulatorischen Rahmenbedingungen und der Bildungsstruktur zu 
kontrollieren, um Zusammenhänge zwischen \textit{\ac{ICT}-Investitionen} und 
\textit{Arbeitslosenquote} differenziert zu analysieren. Zudem erlauben die aufgenommenen 
Kontrollvariablen eine differenziertere Betrachtung institutioneller Faktoren, die den 
Arbeitsmarkt beeinflussen. 

%%%%%%%%%%%%%%%%%%%%%%%%%%%%%%%%%%%%%%%%%%%
% Daten und Methodik: Analytische Methode %
%%%%%%%%%%%%%%%%%%%%%%%%%%%%%%%%%%%%%%%%%%%

% TODO: Re-read. (14.03.2025)

\subsection{Analytische Methode}

Die Analyse dieser Arbeit basiert auf einer Paneldatenanalyse, um die Auswirkungen von 
\textit{\ac{ICT}-Investitionen} auf die \textit{Arbeitslosenquote} nach Bildungsniveau zu 
untersuchen. Die Wahl einer Paneldatenmethode ermöglicht es, sowohl individuelle Heterogenität 
zwischen Ländern als auch dynamische Entwicklungen über die Zeit zu erfassen 
\parencite{wooldridge2010econometric}. 

Für diese Analyse wurden \ac{FE}-Modelle gewählt, da sie eine robustere Schätzung der 
Zusammenhänge zwischen \textit{\ac{ICT}-Investitionen} und der \textit{Arbeitslosenquote} 
ermöglichen. Dies ist besonders relevant, da die Untersuchung auf Veränderungen innerhalb 
eines Landes über die Zeit fokussiert und länderspezifische Eigenschaften nicht als erklärende 
Variablen modelliert werden. \ac{RE}-Modelle werden aufgrund der potenziellen Korrelation 
zwischen länderspezifischen Effekten und den unabhängigen Variablen nicht verwendet, da es in 
diesem Fall zu verzerrten Schätzungen führen könnte 
\parencite[S. 251–256]{wooldridge2010econometric}. 

Darüber hinaus wird die Analyse durch Interaktionseffekte ergänzt, die es ermöglichen, 
institutionelle Unterschiede zwischen den Ländern zu berücksichtigen. Die Modelle beinhalten 
eine Variable für den \textit{Wohlfahrtsstaatentyp}, um systematische Unterschiede zwischen den 
Regimetypen in ihrer Reaktion auf \textit{\ac{ICT}-Investitionen} zu identifizieren 
\parencite{espingandersen1990thethree}. Zusätzlich werden Jahres-Fixed Effects integriert, um 
allgemeine makroökonomische Entwicklungen (z. B. Finanzkrisen oder technologische Schübe) aus 
den Modellen zu kontrollieren. Die Interaktion zwischen \textit{\ac{ICT}-Investitionen} und 
\textit{Wohlfahrtsstaatentyp} ermöglicht eine differenzierte Analyse der moderierenden Effekte 
institutioneller Rahmenbedingungen.

Durch diese Kombination aus \ac{FE}-Modellen, Interaktionseffekten und Zeitdummies bietet die
Paneldatenanalyse eine solide Grundlage für die Untersuchung der Auswirkungen von 
\textit{\ac{ICT}-Investitionen} auf die \textit{Arbeitslosenquote}. Die longitudinale Struktur 
der Daten erlaubt eine differenzierte Betrachtung, die sowohl kurzfristige als auch 
langfristige Beschäftigungseffekte berücksichtigt. 
