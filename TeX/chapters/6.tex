\section{Ergebnisse}

%%%%%%%%%%%%%%%%%%%%%%%%%%%%%%%%%%%%%%
% Ergebnisse: Deskriptive Ergebnisse %
% TODO:                              %
%%%%%%%%%%%%%%%%%%%%%%%%%%%%%%%%%%%%%%

\subsection{Deskriptive Ergebnisse}

% Deskriptive Statistik: Einleitung
Die deskriptiven Statistiken der analysierten Variablen bieten einen umfassenden Einblick in 
deren Eigenschaften und Verteilungen über die beobachteten Länder und Zeiträume. Im Folgenden 
werden die Ergebnisse detailliert beschrieben: 

% Variablenbeschreibung
% TODO: - (Stand 04.03.2025)

\begin{table}[H]
\caption{Übersicht über die Variablen}
\resizebox{\textwidth}{!}{
\centering
\begin{tabular}[t]{lrrrrrr}
\toprule
Variable & Min & Max & Mean & Median & SD & N\\
\midrule
UNEMPLOYMENT\_RATE\_PERCENT & 0.82 & 49.89 & 7.95 & 5.96 & 6.34 & 11919\\
ICT\_INVEST\_SHARE\_GDP & 0.73 & 8.69 & 2.46 & 2.25 & 0.98 & 11919\\
GDP\_PER\_CAPITA & 13.34 & 137.72 & 43.73 & 41.27 & 17.13 & 11919\\
PERCENT\_EMPLOYEES\_TUD & 4.50 & 92.20 & 28.45 & 20.40 & 20.71 & 11919\\
PERCENT\_TERTIARY\_EDUCATION & 12.87 & 59.96 & 33.65 & 34.56 & 9.27 & 11919\\
\addlinespace
REGULATION\_STRICTNESS & 0.00 & 4.88 & 2.19 & 2.26 & 0.83 & 11919\\
\bottomrule
\end{tabular}
}
\label{tab:variables_codebook}
\end{table}


% Variable "Arbeitslosenquote"
Die Variable \textit{Arbeitslosenquote} schwankt erheblich zwischen einem Minimum von 0,82\% und 
einem Maximum von 49,89\%. Der Mittelwert liegt bei 7,95\%, während der Median mit 5,96\% etwas 
niedriger ausfällt. Dies weist auf eine rechtsschiefe Verteilung hin, da einige Länder oder 
Zeitpunkte mit sehr hohen Arbeitslosenquoten als Ausreißer wirken können. Die hohe 
Standardabweichung von 6,34 deutet darauf hin, dass die Arbeitslosenquoten zwischen den Ländern 
und über die Zeit hinweg erhebliche Unterschiede aufweisen. Während einige \ac{OECD}-Länder durch 
eine geringe Arbeitslosenquote und stabile Arbeitsmärkte gekennzeichnet sind, zeigen andere 
Länder insbesondere in wirtschaftlichen Krisenzeiten oder strukturschwachen Regionen signifikant 
höhere Arbeitslosenraten. Diese Heterogenität könnte zudem mit unterschiedlichen 
Arbeitsmarktregulierungen und Bildungssystemen zusammenhängen. 

% Variable "ICT-Investitionen"
Die Variable \textit{\ac{ICT}-Investitionen} variiert zwischen einem Minimum von 0,73\% und einem 
Maximum von 8,69\%. Der Mittelwert beträgt 2,46\%, während der Median mit 2,25\% leicht darunter 
liegt. Dies deutet auf eine leicht rechtsschiefe Verteilung hin, da einige Länder besonders hohe 
Investitionen in \ac{ICT} tätigen. Die Standardabweichung von 0,98 zeigt, dass es zwischen den 
\ac{OECD}-Ländern erhebliche Unterschiede in der Intensität der \textit{\ac{ICT}-Investitionen} 
gibt. Während einige Länder konstant hohe Anteile ihrer wirtschaftlichen Ressourcen in digitale 
Technologien investieren, gibt es andere, die vergleichsweise geringe Investitionen tätigen. 
Diese Unterschiede können durch verschiedene Faktoren beeinflusst sein, darunter wirtschaftliche 
Leistungsfähigkeit, politische Strategien zur Förderung der Digitalisierung sowie strukturelle 
Unterschiede in der Entwicklung des \ac{ICT}-Sektors. 

% Variable "BIP pro Kopf"
Das \textit{\ac{BIP} pro Kopf} weist eine erhebliche Spannweite auf: Es reicht von 13,34 bis 
137,72 Tausend US-Dollar. Der Mittelwert beträgt 43,73 Tausend US-Dollar, während der Median mit 
41,27 Tausend US-Dollar nur geringfügig darunter liegt. Trotz dieser relativen Nähe deutet die 
hohe Standardabweichung von 17,13 darauf hin, dass es erhebliche Wohlstandsunterschiede zwischen 
den \ac{OECD}-Ländern gibt. Dies spricht für eine starke rechtsschiefe Verteilung, da einige 
besonders wohlhabende Länder den Durchschnittswert nach oben treiben. Diese Unterschiede sind 
insbesondere für die Interpretation der \textit{\ac{ICT}-Investitionen} relevant, da 
wohlhabendere Länder tendenziell eine höhere Kapitalausstattung und damit größere 
Investitionsmöglichkeiten in digitale Infrastruktur haben könnten. Gleichzeitig könnten 
Unterschiede im \textit{\ac{BIP} pro Kopf} Einfluss auf die Struktur des Arbeitsmarktes und 
damit auf die Verteilung der \textit{Arbeitslosenquote} nach Bildungsgrad haben. 

% Variable "Gewerkschaftsdichte"
Die Variable \textit{Gewerkschaftsdichte} zeigt eine erhebliche Varianz zwischen den Ländern. Die 
Werte reichen von einem Minimum von 4,50\% bis zu einem Maximum von 92,20\%. Der Mittelwert 
beträgt 28,45\%, während der Median mit 20,40\% darunter liegt, was darauf hindeutet, dass einige 
Länder eine besonders hohe Gewerkschaftsbindung haben, während die Mehrheit unter diesem 
Durchschnittswert bleibt. Die Standardabweichung von 20,71 verdeutlicht die große Heterogenität 
in der gewerkschaftlichen Organisation zwischen den Ländern. 

% Variable "Anteil tertiär Gebildeter"
Die Variable \textit{Tertiärer Bildungsanteil} (PERCENT\_TERTIARY\_EDUCATION) variiert zwischen 
12,87\% und 59,96\%, mit einem Mittelwert von 33,65\% und einer Standardabweichung von 9,27. 
Länder mit höheren Werten verfügen tendenziell über eine stärker wissensbasierte Wirtschaft, was 
sich positiv auf die Integration von Arbeitnehmern in technologische Sektoren auswirken kann. 
Gleichzeitig könnte eine höhere Bildungsbeteiligung dazu beitragen, die negativen Effekte der 
Digitalisierung für geringqualifizierte Arbeitskräfte abzufedern. 

% Variable "Arbeitsmarktregulierung"
Die Variable \textit{Arbeitsmarktregulierung} (REGULATION\_STRICTNESS) reicht von 0,00 bis 4,88, 
mit einem Mittelwert von 2,19 und einer Standardabweichung von 0,83. Diese Unterschiede spiegeln 
unterschiedliche Arbeitsmarktpolitiken wider: Während in einigen Ländern hohe Regulierung den 
Kündigungsschutz stärkt, kann dies gleichzeitig die Schaffung neuer Arbeitsplätze hemmen. 

% Variable "Wohlfahrtsstaatentyp"
\begin{table}[H]
\centering
\caption{Übersicht über die Verteilung der Wohlfahrtsstaatentypen}
\centering
\begin{tabular}[t]{lrr}
\toprule
Kategorie & Anzahl & Prozent\\
\midrule
Anglo-Saxon & 2382 & 19.98\\
Central European & 2943 & 24.69\\
Nordic & 2034 & 17.07\\
Other & 0 & 0.00\\
Post-socialist & 3015 & 25.30\\
\addlinespace
Southern European & 1545 & 12.96\\
\bottomrule
\end{tabular}
\end{table}


Die Verteilung der Variable \textit{Wohlfahrtsstaatentyp} zeigt, dass postsozialistische Länder 
mit 25,30\% die größte Gruppe innerhalb der Stichprobe ausmachen, gefolgt von mitteleuropäischen 
Wohlfahrtsstaaten mit 24,69\%. Nordische und angelsächsische Länder sind mit 17,07\% bzw. 19,98\% 
ebenfalls vertreten, während südeuropäische Länder mit 12,96\% den kleinsten Anteil ausmachen. 
Die Kategorie \textit{Other} ist in der vorliegenden Stichprobe nicht besetzt und dient lediglich 
als Indikator für übrige Länder, welche keiner Ausprägung der Variable zugeteilt wurden.

Diese Klassifikation ist insbesondere für die spätere Analyse der Interaktionseffekte relevant, 
da sie Aufschluss darüber geben kann, inwiefern institutionelle Rahmenbedingungen den Einfluss 
von \textit{\ac{ICT}-Investitionen} auf die \textit{Arbeitslosenquote} moderieren.

% Variablen (Zusammenfassung)
Die deskriptiven Statistiken zeigen, dass die betrachteten Variablen eine erhebliche 
Heterogenität aufweisen, die sowohl auf länderspezifische Unterschiede als auch auf strukturelle 
und wirtschaftliche Faktoren zurückgeführt werden kann. Besonders auffällig sind die Unterschiede 
in den \textit{\ac{ICT}-Investitionen}, die je nach wirtschaftlicher Leistungsfähigkeit und 
politischen Rahmenbedingungen stark variieren. Auch die \textit{Arbeitslosenquote} zeigt eine 
hohe Streuung, die möglicherweise mit den unterschiedlichen Bildungsniveaus, 
Arbeitsmarktinstitutionen und Wirtschaftsentwicklungen der Länder zusammenhängt. Die hohe Varianz 
im \textit{\ac{BIP} pro Kopf} unterstreicht die unterschiedlichen wirtschaftlichen 
Ausgangsbedingungen der Länder, was sich sowohl auf die Höhe der \textit{\ac{ICT}-Investitionen} 
als auch auf die Struktur der Arbeitsmärkte auswirken könnte. Schließlich zeigt die 
\textit{Gewerkschaftsdichte} ebenfalls starke Unterschiede zwischen den \textit{OECD}-Ländern, 
was für die Analyse der institutionellen Faktoren relevant ist, die möglicherweise als 
Moderatoren der Auswirkungen von \ac{ICT}-Investitionen auf den Arbeitsmarkt fungieren.

% Grafiken
Diese deskriptive Analyse der Variablen bildet die Grundlage für die nachfolgenden grafischen 
Darstellungen, die eine detailliertere Visualisierung der Trends und Unterschiede zwischen den 
Ländern ermöglichen. Sie bietet einen ersten Einblick auf Länderebene in die Beziehung zwischen 
\textit{\ac{ICT}-Investitionen} (gemessen als Anteil am \textit{\ac{BIP} pro Kopf}) und den 
\textit{Arbeitslosenquote}, differenziert nach den drei genannten Bildungsgruppen. Hierbei wird 
jeweils ein repräsentatives Land pro \textit{Wohlfahrtsstaatentyp} für die Analyse gewählt - 
Spanien als südeuropäischer, Polen als postsozialistischer, Schweden als nordischer und 
Deutschland als mitteleuropäischer Wohlfahrtsstaat im Zeitraum von 2005 bis 2022. 

Um die Lesbarkeit der Grafiken zu verbessern, wurden die Ländernamen in den Diagrammen 
automatisch in die deutsche Sprache übersetzt. Dies stellt sicher, dass die visuelle Darstellung 
konsistent mit der Textanalyse bleibt. Ziel ist es, vor der multivariaten Analyse bereits 
Unterschiede und Trends innerhalb der Länder und zwischen den Bildungsgruppen zu identifizieren.

% Grafik "Spanien"
\begin{figure}[htbp]
    \centering
    \caption{Überblick über \textit{\ac{ICT}-Investitionen} und \textit{Arbeitslosenquote} in 
    Spanien}
    \includegraphics[width=\textwidth]{assets/plot_spain_final.png}
    \label{fig:spain}
\end{figure}

Die Abbildung zeigt die Entwicklung der \textit{\ac{ICT}-Investitionen} als Anteil am \ac{BIP} 
pro Kopf sowie die \textit{Arbeitslosenquote} in Spanien zwischen 2005 und 2022, differenziert 
nach Bildungsniveau - Spanien steht hier repräsentativ für südeuropäische Wohlfahrtsstaaten. Am 
Beispiel Spaniens ist ein besonders markanter Anstieg der \textit{Arbeitslosenquote} während der 
Finanz- und Wirtschaftskrise von 2008 bis 2013 zu beobachten. Während die 
\textit{\ac{ICT}-Investitionen} einen insgesamt moderaten Anstieg über den gesamten Zeitraum 
hinweg zeigen, lassen sich drastische Schwankungen in der \textit{Arbeitslosenquote} 
identifizieren, insbesondere bei Personen mit niedrigem und mittlerem Bildungsniveau.

Bei Personen mit einem niedrigen Bildungsniveau zeigt sich zwischen 2005 und 2008 eine relativ 
stabile \textit{Arbeitslosenquote} von knapp unter 10\%. Ab 2008 kam es jedoch zu einem rasanten 
Anstieg, der bis 2013 einen Höchststand von über 30\% erreichte. Erst nach 2013 begann ein 
kontinuierlicher Rückgang, der sich bis 2020 auf etwa 20\% fortsetzte, bevor ein erneuter 
leichter Anstieg zu beobachten ist. Die \textit{\ac{ICT}-Investitionen} entwickelten sich 
hingegen gleichmäßiger. Sie begannen auf einem niedrigen Niveau von etwa 1,75\% des BIP, zeigten 
nach der Finanzkriese ab 2010 eine Aufwärtstendenz und stabilisierten sich nach 2015 bei etwa 
2,5\%. Der Rückgang der \textit{Arbeitslosenquote} nach 2013 verlief jedoch unabhängig von einer 
abrupten Zunahme der \textit{\ac{ICT}-Investitionen}, was darauf hindeutet, dass makroökonomische 
Faktoren (z. B. wirtschaftliche Erholung, Beschäftigungsprogramme) für die Senkung der 
Arbeitslosigkeit eine zentrale Rolle spielten.

Bei Personen mit mittlerem Bildungsniveau zeigt sich ein sehr ähnlicher Verlauf. Die 
\textit{Arbeitslosenquote} lag 2005 noch unter 8\%, stieg im Zuge der Wirtschaftskrise bis 2013 
jedoch auf über 20\% an. Erst ab 2014 begann ein deutlicher Rückgang, der sich bis 2020 auf etwa 
10\% fortsetzte. Die \textit{\ac{ICT}-Investitionen} folgten hier einem vergleichbaren Muster wie 
in der Gruppe der gering Qualifizierten, wobei ein leichter, aber kontinuierlicher Anstieg 
sichtbar ist. Dennoch ist keine direkte Korrelation zwischen dem Verlauf der 
\textit{\ac{ICT}-Investitionen} und der \textit{Arbeitslosenquote} ersichtlich, da der massive 
Anstieg und der spätere Rückgang der Arbeitslosigkeit primär durch die wirtschaftliche 
Entwicklung und nicht durch technologische Investitionen bedingt zu sein scheinen.

Bei Personen mit hohem Bildungsniveau war die \textit{Arbeitslosenquote} insgesamt niedriger, 
zeigte jedoch ebenfalls einen deutlichen Anstieg während der Wirtschaftskrise. Im Jahr 2005 lag 
sie unter 5\%, erreichte 2013 jedoch fast 15\%. Danach setzte auch hier ein Rückgang ein, und bis 
2020 fiel die \textit{Arbeitslosenquote} auf etwa 5\% zurück. Im Gegensatz zu den anderen 
Bildungsgruppen scheinen sich hier die \textit{\ac{ICT}-Investitionen} und die 
\textit{Arbeitslosenquote} teilweise gegenläufig zu entwickeln. Während die 
\textit{\ac{ICT}-Investitionen} nach 2010 eine stetige Steigerung zeigen und nach 2015 stabil auf 
etwa 2,5\% des BIP bleiben, geht die \textit{Arbeitslosenquote} in derselben Phase zurück. Dies 
könnte darauf hindeuten, dass hochqualifizierte Arbeitskräfte in Spanien stärker von der 
Digitalisierung profitieren konnten als Personen mit niedrigerem Bildungsstand.

Spanien als südeuropäischer Wohlfahrtsstaat ist durch einen stark segmentierten Arbeitsmarkt 
gekennzeichnet, der sich durch hohe Anteile an befristeten Beschäftigungsverhältnissen sowie eine 
geringere Arbeitsplatzsicherheit auszeichnet \parencite[vgl.][S. F160–F163]{bentolila2012two}. 
Dies könnte eine Erklärung für die starken Schwankungen der \textit{Arbeitslosenquote} im Zuge 
der Finanzkrise sein, da insbesondere gering und mittel Qualifizierte von Entlassungen betroffen 
waren. Die \textit{\ac{ICT}-Investitionen} scheinen langfristig zwar leicht anzusteigen, doch 
zeigt sich kein direkter Zusammenhang zwischen diesen Investitionen und der 
\textit{Arbeitslosenquote} in den jeweiligen Bildungsgruppen. Vielmehr deutet die Entwicklung 
darauf hin, dass der Arbeitsmarkt in Spanien stark konjunkturabhängig ist und die wirtschaftliche 
Erholung nach 2013 die wichtigste Triebkraft für die Reduktion der Arbeitslosigkeit war.

Zusammenfassend zeigen die Daten für Spanien eine enge Verbindung zwischen der Finanzkrise und 
den massiven Schwankungen der \textit{Arbeitslosenquote}, insbesondere bei gering und mittel 
Qualifizierten. Während \textit{\ac{ICT}-Investitionen} über den Zeitraum hinweg einen 
kontinuierlichen, aber moderaten Anstieg zeigen, sind ihre direkten Auswirkungen auf die 
Arbeitslosigkeit unklar. Es könnte jedoch sein, dass insbesondere Hochqualifizierte von den 
steigenden \textit{\ac{ICT}-Investitionen} profitieren konnten, während gering Qualifizierte eher 
von konjunkturellen Faktoren abhängig waren.

% Grafik "Polen"
\begin{figure}[htbp]
    \centering
    \caption{Überblick über \textit{\ac{ICT}-Investitionen} und \textit{Arbeitslosenquote} in 
    Polen}
    \includegraphics[width=\textwidth]{assets/plot_poland_final.png}
    \label{fig:poland}
\end{figure}

Die Abbildung zeigt die Entwicklung der \textit{\ac{ICT}-Investitionen} als Anteil am 
\ac{BIP} sowie die \textit{Arbeitslosenquote} in Polen zwischen 2005 und 2022 differenziert nach 
Bildungsniveau - Polen steht hier repräsentativ für postsozialistische Wohlfahrtsstaaten.

Auffällig ist der durchgängige Rückgang der \textit{Arbeitslosenquote} in allen Bildungsgruppen, 
während die \textit{\ac{ICT}-Investitionen} über weite Strecken konstant bleiben, beziehungsweise 
sogar ebenfalls einen Rückgang verzeichnen. Dies deutet darauf hin, dass makroökonomische oder 
arbeitsmarktpolitische Faktoren für den Rückgang der Arbeitslosigkeit maßgeblich verantwortlich 
sein könnten.

Bei Personen mit einem niedrigen Bildungsniveau lag die \textit{Arbeitslosenquote} im Jahr 2005 
bei knapp 28\%. In den darauffolgenden Jahren kam es zu einem raschen Rückgang, wobei jedoch 
zwischen 2010 und 2015 eine Stagnation mit einem kurzen Anstieg auf fast 20\% zu beobachten ist. 
Nach 2015 setzte sich der Rückgang der \textit{Arbeitslosenquote}  fort, sodass sie bis 2020 auf 
8\% fiel. Die \textit{\ac{ICT}-Investitionen} blieben über den gesamten Zeitraum hinweg 
weitgehend konstant und bewegten sich um die 1\% des BIP, mit einem leichten Rückgang zwischen 
2010 und 2015. Dies deutet darauf hin, dass  der starke Rückgang der Arbeitslosigkeit nicht 
direkt mit den \textit{\ac{ICT}-Investitionen} zusammenhängt, sondern durch andere 
wirtschaftliche Faktoren beeinflusst wurde, beispielsweise durch eine allgemeine 
wirtschaftliche Stabilisierung nach dem EU-Beitritt Polens und steigende 
Beschäftigungsmöglichkeiten in arbeitsintensiven Branchen.

Für Personen mit einem mittleren Bildungsniveau zeigt sich ein ähnliches Muster, wenn auch auf 
einem insgesamt niedrigeren Ausgangsniveau der \textit{Arbeitslosenquote}. Während diese 2005 
noch über 10\% lag, sank sie in den darauffolgenden Jahren rasch auf etwa 3\% bis 2015 und weiter 
unter 2\% bis 2020. Zwischen 2010 und 2015 ist jedoch eine leichte Erhöhung der 
\textit{Arbeitslosenquote} erkennbar, bevor der Trend weiter nach unten verlief. Der Rückgang der 
Arbeitslosigkeit erfolgt weitgehend unabhängig von der Entwicklung der 
\textit{\ac{ICT}-Investitionen}, was darauf hindeutet, dass makroökonomische Faktoren wie die 
Industrialisierung und eine steigende Nachfrage nach Arbeitskräften mit mittlerer Qualifikation 
eine bedeutendere Rolle gespielt haben könnten.

Für Personen mit einem hohen Bildungsniveau war die \textit{Arbeitslosenquote} bereits 2005 
relativ niedrig, lag aber dennoch bei etwa 6\%, was im Vergleich zu anderen europäischen Ländern 
eher hoch ist. Dies könnte auf strukturelle Faktoren des polnischen Arbeitsmarktes zurückzuführen 
sein, wie eine geringere Anzahl hochqualifizierter Beschäftigungsmöglichkeiten in den frühen 
2000er-Jahren. In den darauffolgenden Jahren fiel die \textit{Arbeitslosenquote} jedoch deutlich 
und lag bereits 2015 unter 2\%. Auffällig ist, dass die \textit{\ac{ICT}-Investitionen} in dieser 
Gruppe im Gegensatz zu den anderen Bildungsgruppen eine leichte Steigerung zeigen. In der ersten 
Hälfte des Beobachtungszeitraums bewegten sich die \textit{\ac{ICT}-Investitionen} um 1,2\% des 
BIP, während sie in den Jahren nach 2015 tendenziell anstiegen. Dies könnte darauf hindeuten, 
dass der polnische Arbeitsmarkt mit steigenden \textit{\ac{ICT}-Investitionen} zunehmend 
hochqualifizierte Beschäftigungsmöglichkeiten geschaffen hat. Dennoch bleibt die Kausalität 
unklar, da die \textit{Arbeitslosenquote} in dieser Gruppe bereits gefallen war, bevor der 
leichte Anstieg der \textit{\ac{ICT}-Investitionen} einsetzte.

Polen als postsozialistischer Wohlfahrtsstaat hat in den letzten Jahrzehnten einen tiefgreifenden 
wirtschaftlichen Wandel durchlaufen. Der EU-Beitritt im Jahr 2004 führte zu verstärkten 
ausländischen Direktinvestitionen, einer zunehmenden Integration in europäische 
Produktionsnetzwerke sowie einer generellen Modernisierung der Wirtschaft 
\parencite[vgl.][S. 186–189]{myant2013transition}. Diese Entwicklungen spiegeln sich auch in der 
Reduktion der Arbeitslosigkeit wider, die in allen Bildungsgruppen signifikant gesunken ist. 
Besonders bei Personen mit mittlerem und niedrigem Bildungsniveau könnte die Expansion von 
Industriejobs sowie der Dienstleistungssektor eine wesentliche Rolle gespielt haben.

Insgesamt zeigt die Abbildung, dass die \textit{Arbeitslosenquote} in allen Bildungsgruppen stark 
gesunken ist, während die \textit{\ac{ICT}-Investitionen} nur moderate Schwankungen aufweisen. 
Dies deutet darauf hin, dass die Haupttreiber der Beschäftigungsentwicklung in Polen eher in 
wirtschaftlichen und arbeitsmarktpolitischen Veränderungen zu suchen sind als in den direkten 
Auswirkungen von \textit{\ac{ICT}-Investitionen}. Dennoch könnte die leichte Zunahme der 
\textit{\ac{ICT}-Investitionen} im späteren Beobachtungszeitraum darauf hinweisen, dass sich der 
polnische Arbeitsmarkt allmählich in Richtung einer wissensbasierten Wirtschaft entwickelt, in 
der besonders Hochqualifizierte profitieren.

% Grafik "Schweden"
\begin{figure}[htbp]
    \centering
    \caption{Überblick über \textit{\ac{ICT}-Investitionen} und \textit{Arbeitslosenquote} in 
    Schweden} 
    \includegraphics[width=\textwidth]{assets/plot_sweden_final.png}
    \label{fig:sweden}
\end{figure}

Die Abbildung zeigt die Entwicklung der \textit{\ac{ICT}-Investitionen} als Anteil am BIP sowie 
die \textit{Arbeitslosenquote} in Schweden zwischen 2005 und 2022, differenziert nach 
Bildungsniveau - Schweden steht hier repräsentativ für nordische Wohlfahrtsstaaten. Im Gegensatz 
zu anderen Ländern ist hier eine relativ stabile Entwicklung der \textit{Arbeitslosenquote} über 
den gesamten Zeitraum zu beobachten, mit nur moderaten Schwankungen. Auffällig ist zudem, dass 
die \textit{\ac{ICT}-Investitionen} in Schweden im internationalen Vergleich auf einem 
vergleichsweise hohen Niveau liegen. Während sie in der ersten Dekade leichte Schwankungen 
zeigen, bleibt ihr Niveau ab 2010 weitgehend konstant und steigt gegen Ende des 
Betrachtungszeitraums leicht an.

Bei Personen mit einem niedrigen Bildungsniveau lag die \textit{Arbeitslosenquote} 2005 bei knapp 
5\% und zeigte bis etwa 2010 einen moderaten Anstieg. Nach 2010 stabilisierte sich die 
\textit{Arbeitslosenquote} zunächst, bevor sie ab 2015 einen erneuten Aufwärtstrend verzeichnete. 
Besonders auffällig ist der deutliche Anstieg nach 2018, der sich bis 2022 fortsetzt. Während die 
\textit{Arbeitslosenquote} für gering Qualifizierte also in den letzten Jahren gestiegen ist, 
sind die \textit{\ac{ICT}-Investitionen} im selben Zeitraum weitgehend stabil geblieben, wenn 
auch mit einer leicht positiven Tendenz. Dies könnte darauf hindeuten, dass die fortschreitende 
Digitalisierung möglicherweise die Beschäftigungsmöglichkeiten für niedrig qualifizierte 
Arbeitskräfte verschlechtert hat, indem sie bestimmte Arbeitsplätze verdrängte oder die 
Anforderungen an digitale Kompetenzen erhöhte - wahrscheinlich hängt diese Beobachtung aber eher 
mit der Corona-Pandemie zusammen.

Für Personen mit einem mittleren Bildungsniveau zeigt sich ein stabiles Muster, mit einer 
weitgehend konstanten \textit{Arbeitslosenquote} zwischen 2005 und 2018. Während die 
Arbeitslosigkeit 2005 bei unter 5\% lag, gab es bis 2015 eine leichte Abwärtsbewegung, gefolgt 
von einer weitgehenden Stabilisierung. Nach 2018 zeigt sich eine leicht steigende Tendenz der 
\textit{Arbeitslosenquote}, wenn auch weniger ausgeprägt als bei den gering Qualifizierten. Die 
\textit{\ac{ICT}-Investitionen} sind in dieser Gruppe durchgängig hoch und zeigen eine stabile 
Entwicklung mit leichten Schwankungen. Anders als bei den gering Qualifizierten ist hier keine 
klare gegenläufige Entwicklung zwischen \textit{\ac{ICT}-Investitionen} und Arbeitslosigkeit zu 
erkennen, was darauf hindeutet, dass mittlere Qualifikationen in Schweden weniger stark von den 
technologischen Veränderungen betroffen sind.

Bei Personen mit einem hohen Bildungsniveau zeigt sich über den gesamten Zeitraum hinweg eine 
extrem niedrige \textit{Arbeitslosenquote}. Bereits 2005 lag sie unter 5\% und blieb über den 
gesamten Zeitraum stabil, mit nur minimalen Schwankungen. Auffällig ist, dass die 
\textit{\ac{ICT}-Investitionen} in dieser Gruppe im internationalen Vergleich sehr hoch sind, mit 
Werten, die konstant bei 4-5\% des \ac{BIP} liegen. Die Kombination aus hoher 
\ac{ICT}-Investition und niedriger \textit{Arbeitslosenquote} deutet darauf hin, dass 
hochqualifizierte Arbeitskräfte in Schweden stark von der Digitalisierung profitieren konnten. 
Dies entspricht auch theoretischen Erwartungen, da hochqualifizierte Beschäftigte in 
wissensintensiven Branchen tätig sind, die von technologischen Innovationen profitieren.

Die stabilen \textit{\ac{ICT}-Investitionen} und die insgesamt niedrige 
\textit{Arbeitslosenquote} deuten darauf hin, dass der schwedische Arbeitsmarkt relativ 
widerstandsfähig gegenüber technologischen Veränderungen ist. Allerdings lässt sich bei niedrig 
qualifizierten Arbeitskräften ein Anstieg der Arbeitslosigkeit nach 2018 beobachten, der 
möglicherweise mit strukturellen Veränderungen auf dem Arbeitsmarkt zusammenhängt. Dies könnte 
darauf hindeuten, dass bestimmte Berufe durch die Digitalisierung zunehmend verdrängt werden oder 
dass sich die Anforderungen an digitale Kompetenzen verstärkt haben, sodass Geringqualifizierte 
Schwierigkeiten haben, sich an die veränderten Bedingungen anzupassen.

Insgesamt zeigt die Abbildung, dass Schweden ein stabiles Beschäftigungsniveau über den gesamten 
Zeitraum hinweg aufweist, wobei die \textit{\ac{ICT}-Investitionen} konstant hoch sind. Während 
Hoch- und Mittelqualifizierte weitgehend von den Entwicklungen profitieren konnten, scheint sich 
für gering Qualifizierte in den letzten Jahren eine Verschlechterung der Beschäftigungssituation 
abzuzeichnen. Dies könnte darauf hindeuten, dass Digitalisierung in hochentwickelten 
Volkswirtschaften wie Schweden zunehmend zu einer Polarisierung des Arbeitsmarktes führt, bei der 
Hochqualifizierte von den Investitionen profitieren, während gering Qualifizierte zunehmend unter 
Druck geraten.

% Grafik "Deutschland"
\begin{figure}[htbp]
    \centering
    \caption{Überblick über \textit{\ac{ICT}-Investitionen} und \textit{Arbeitslosenquote} in 
    Deutschland}
    \includegraphics[width=\textwidth]{assets/plot_germany_final.png}
    \label{fig:germany}
\end{figure}

Die Abbildung zeigt die Entwicklung der \textit{\ac{ICT}-Investitionen} als Anteil am BIP sowie 
die \textit{Arbeitslosenquote} in Deutschland zwischen 2005 und 2022, differenziert nach 
Bildungsniveau - Deutschland steht hier repräsentativ für mitteleuropäische Wohlfahrtsstaaten. 
Dabei lassen sich klare Unterschiede zwischen den drei betrachteten Gruppen - niedriges, 
mittleres und hohes Bildungsniveau - sowohl hinsichtlich des Niveaus als auch der Veränderung 
der \textit{Arbeitslosenquote} erkennen. Insgesamt zeigen sich über den gesamten Zeitraum hinweg 
deutliche Rückgänge in der \textit{Arbeitslosenquote}, während die 
\textit{\ac{ICT}-Investitionen} eine weitgehend stabile Entwicklung aufweisen.

Für Personen mit einem niedrigen Bildungsniveau zeigt sich eine besonders hohe 
\textit{Arbeitslosenquote} zu Beginn des Beobachtungszeitraums, die 2005 bei über 18\% lag. In 
den darauffolgenden Jahren kam es zu einem kontinuierlichen Rückgang, der bis 2020 Werte unter 
5\% erreichte. Diese Entwicklung spiegelt die allgemeine Verbesserung des deutschen 
Arbeitsmarktes wider, insbesondere durch wirtschaftlichen Aufschwung und Reformen im Rahmen der 
Agenda 2010. Die \textit{\ac{ICT}-Investitionen} verzeichneten zwischen 2005 und 2010 zunächst 
einen leichten Rückgang, bevor sie sich um die 1,5\% des BIP stabilisierten. Ein direkter 
Zusammenhang zwischen \textit{\ac{ICT}-Investitionen} und der sinkenden 
\textit{Arbeitslosenquote} ist nicht ersichtlich, da der Rückgang der \textit{Arbeitslosenquote} 
bereits vor der leichten Stabilisierung der Investitionen begann.

Bei Personen mit mittlerem Bildungsniveau zeigt sich ein ähnliches Muster, wenn auch auf einem 
insgesamt niedrigeren Ausgangsniveau der \textit{Arbeitslosenquote}. Während diese 2005 noch bei 
etwa 10\% lag, fiel sie bis 2020 auf rund 3\% und blieb seither weitgehend stabil. Die 
\textit{\ac{ICT}-Investitionen} zeigen eine konstante Entwicklung mit geringen Schwankungen. Auch 
hier bleibt der direkte Zusammenhang zwischen den \textit{\ac{ICT}-Investitionen} und der 
\textit{Arbeitslosenquote} unklar, da der Rückgang der Arbeitslosigkeit langfristig verläuft und 
nicht direkt mit den Investitionen korreliert.

Für Personen mit hohem Bildungsniveau zeigt sich über den gesamten Zeitraum hinweg eine sehr 
niedrige \textit{Arbeitslosenquote}. Bereits 2005 lag sie unter 5\% und sank bis 2010 auf unter 
2\%, wo sie anschließend auf diesem niedrigen Niveau stabil blieb. Im Vergleich zu den anderen 
Bildungsgruppen weist diese Gruppe somit die geringsten Schwankungen auf. Die 
\textit{\ac{ICT}-Investitionen} zeigen auch hier eine weitgehend stabile Entwicklung. Dies könnte 
darauf hindeuten, dass Hochqualifizierte vermehrt in Berufen tätig sind, die von steigenden 
\textit{\ac{ICT}-Investitionen} profitieren, jedoch bleibt auch hier die Kausalität unklar.

Deutschland als mitteleuropäischer Wohlfahrtsstaat zeichnet sich durch eine enge Verzahnung von 
Bildungssystem und Arbeitsmarkt aus \parencite[vgl.][S. 21–27]{hall2001varieties}. Insbesondere 
das duale Ausbildungssystem und gezielte arbeitsmarktpolitische Maßnahmen könnten eine Rolle beim 
Rückgang der \textit{Arbeitslosenquote} in den niedrigen und mittleren Bildungsgruppen gespielt 
haben. Die \textit{\ac{ICT}-Investitionen} zeigen über den Beobachtungszeitraum hinweg keine 
drastischen Veränderungen, was darauf hindeutet, dass technologische Entwicklungen schrittweise 
in den Arbeitsmarkt integriert wurden. Besonders für Hochqualifizierte könnte eine steigende 
Nachfrage nach digitalen Fähigkeiten eine Rolle gespielt haben, während bei den niedrigen und 
mittleren Bildungsniveaus der Arbeitsmarktrückgang vermutlich durch andere makroökonomische 
Faktoren beeinflusst wurde.

Die Abbildung verdeutlicht insgesamt, dass die \textit{Arbeitslosenquote} in allen 
Bildungsgruppen über die Jahre hinweg gesunken sind, während die \textit{\ac{ICT}-Investitionen} 
vergleichsweise stabil geblieben sind. Dies lässt darauf schließen, dass der Rückgang der 
Arbeitslosigkeit nicht direkt durch \textit{\ac{ICT}-Investitionen} getrieben wurde, sondern eher 
mit makroökonomischen Entwicklungen und strukturellen Veränderungen auf dem deutschen 
Arbeitsmarkt zusammenhängt.

% Grafik "Großbritannien"
\begin{figure}[htbp]
    \centering
    \caption{Überblick über \textit{\ac{ICT}-Investitionen} und Arbeitslosenquote in 
    Großbritannien}
    \includegraphics[width=\textwidth]{assets/plot_uk_final.png}
    \label{fig:uk}
\end{figure}

Die Abbildung zeigt die Entwicklung der \textit{\ac{ICT}-Investitionen} als Anteil am BIP sowie 
die \textit{Arbeitslosenquote} in Großbritannien zwischen 2005 und 2022, differenziert nach 
Bildungsniveau - Großbritannien steht hier repräsentativ für angelsächsische Wohlfahrtsstaaten. 
Im Vergleich zu anderen Wohlfahrtsstaatentypen weist Großbritannien eine relativ konstante 
\textit{Arbeitslosenquote} auf, die über den Zeitraum hinweg nur leichte Rückgänge zeigt. 
Auffällig ist, dass die \textit{\ac{ICT}-Investitionen} in Großbritannien zwar einen moderaten 
Anstieg aufweisen, sich jedoch auf einem relativ niedrigen Niveau bewegen.

Bei Personen mit einem niedrigen Bildungsniveau lag die \textit{Arbeitslosenquote} im Jahr 2005 
bei etwa 8\% und sank bis 2020 auf unter 4\%. Anders als in Ländern mit stärker regulierten 
Arbeitsmärkten zeigt sich hier kein abrupter Rückgang, sondern eine schrittweise Anpassung über 
den Zeitraum hinweg. Gleichzeitig zeigen die \textit{\ac{ICT}-Investitionen} einen leicht 
steigenden Trend, bleiben jedoch im Bereich von etwa 1,5\% des BIP. Ein klarer Zusammenhang 
zwischen \textit{\ac{ICT}-Investitionen} und der \textit{Arbeitslosenquote} lässt sich nicht 
unmittelbar erkennen, was darauf hindeuten könnte, dass andere arbeitsmarktpolitische oder 
wirtschaftliche Faktoren maßgeblicher für die Reduktion der Arbeitslosigkeit sind.

Bei Personen mit mittlerem Bildungsniveau zeigt sich ein ähnliches Muster. Die 
\textit{Arbeitslosenquote} lag 2005 bei etwa 6\% und fiel bis 2015 auf rund 3\%, wo sie sich 
anschließend stabilisierte. Die \textit{\ac{ICT}-Investitionen} zeigen hier eine geringe Zunahme, 
bleiben jedoch weitgehend konstant im Bereich von 1,5\% bis 2\% des BIP. Auch in dieser Gruppe 
scheint der Rückgang der \textit{Arbeitslosenquote} eher mit marktwirtschaftlichen Anpassungen 
als mit direkten Effekten der \textit{\ac{ICT}-Investitionen} zusammenzuhängen. Der relativ 
geringe Anstieg der Investitionen deutet darauf hin, dass die britische Wirtschaft zwar 
technologische Entwicklungen integriert, jedoch nicht in dem Ausmaß wie andere hochdigitalisierte 
Volkswirtschaften.

Für Personen mit hohem Bildungsniveau zeigt sich über den gesamten Zeitraum hinweg eine sehr 
niedrige \textit{Arbeitslosenquote}. Bereits 2005 lag sie unter 3\% und blieb über den gesamten 
Zeitraum weitgehend stabil, mit nur minimalen Schwankungen. Die \textit{\ac{ICT}-Investitionen} 
zeigen auch hier eine relativ konstante Entwicklung, liegen jedoch ebenfalls im Bereich von 1,5\% 
bis 2\% des BIP. Dies deutet darauf hin, dass Hochqualifizierte kaum von negativen 
Beschäftigungseffekten durch Digitalisierung betroffen sind. Vielmehr könnte der flexible 
britische Arbeitsmarkt es dieser Gruppe erleichtert haben, sich an technologische Veränderungen 
anzupassen.

Großbritannien als anglo-sächsischer Wohlfahrtsstaat zeichnet sich durch einen weniger 
regulierten Arbeitsmarkt aus, der sich durch eine hohe Flexibilität und eine geringere staatliche 
Intervention auszeichnet \parencite[vgl.][S. 21]{trabert1997entwicklung}. Diese Charakteristik 
könnte erklären, warum die \textit{Arbeitslosenquote} über den Zeitraum hinweg relativ stabil 
bleiben und gleichzeitig keine drastischen Veränderungen im Bereich der 
\textit{\ac{ICT}-Investitionen} feststellbar sind. Der moderate Rückgang der Arbeitslosigkeit 
deutet darauf hin, dass sich der britische Arbeitsmarkt schrittweise an Digitalisierung angepasst 
hat, ohne dass bestimmte Gruppen massiv benachteiligt wurden.

Zusammenfassend zeigt die Abbildung, dass sich die britische \textit{Arbeitslosenquote} über die 
Jahre hinweg in allen Bildungsgruppen verringert hat, wenn auch nicht so drastisch wie in anderen 
Ländern. Gleichzeitig bleiben die \textit{\ac{ICT}-Investitionen} auf einem relativ niedrigen 
Niveau und zeigen keine unmittelbare Korrelation mit den Veränderungen der 
\textit{Arbeitslosenquote}. Dies deutet darauf hin, dass makroökonomische Faktoren wie die 
Arbeitsmarktflexibilität und allgemeine wirtschaftliche Entwicklung eine wichtigere Rolle für die 
Beschäftigungsdynamik spielen als allein die Höhe der \textit{\ac{ICT}-Investitionen}.

%%%%%%%%%%%%%%%%%%%%%%%%%%%%%%%%%%%%%
% Ergebnisse: Multivariate Analysen %
% TODO:                             %
%%%%%%%%%%%%%%%%%%%%%%%%%%%%%%%%%%%%%

\subsection{Multivariate Analysen}

Die Zusammenfassung der Ergebnisse aus den Modellen mit Kontrollvariablen zeigt eine umfassende, 
jedoch differenzierte Analyse der Auswirkungen von \textit{\ac{ICT}-Investitionen} auf die 
\textit{Arbeitslosenquote} in den drei Bildungsgruppen („niedriges Bildungsniveau“, „mittleres 
Bildungsniveau“, „hohes Bildungsniveau“). Die Modelle liefern wichtige Hinweise auf die Bedeutung 
makroökonomischer Rahmenbedingungenund institutioneller Strukturen, während der direkte Einfluss 
von\textit{\ac{ICT}-Investitionen} signifikant, aber unterschiedlich stark ausfällt.

% Modelle ohne Interaktion
\input{assets/models_control_final.tex}

% Modell ohne Interaktion (niedriges Bildungsniveau)
Im Modell für die Gruppe mit niedrigem Bildungsniveau zeigt der geschätzte Koeffizient für 
\textit{\ac{ICT}-Investitionen} einen positiven Wert von 2,302*** (p < 0,001), was auf einen 
signifikanten Anstieg der \textit{Arbeitslosenquote} in dieser Gruppe hinweist. Dies unterstützt 
die Hypothese, dass gering Qualifizierte besonders von den negativen Effekten der Digitalisierung 
betroffen sind, da einfache Tätigkeiten stärker automatisiert werden.

Das \textit{\ac{BIP} pro Kopf} weist mit -0,194*** (p < 0,001) einen signifikanten negativen 
Einfluss auf, was darauf hindeutet, dass wirtschaftlich stärkere Länder tendenziell geringere 
\textit{Arbeitslosenquoten} für das niedrige Bildungsniveau aufweisen. Der 
\textit{Tertiärer Bildungsanteil} zeigt mit 0,606*** (p < 0,001) einen positiven Zusammenhang mit 
der \textit{Arbeitslosenquote}, was auf mögliche Verdrängungseffekte im Arbeitsmarkt hindeuten 
könnte.

Die \textit{Gewerkschaftsdichte} hat mit 0,128*** (p < 0,001) ebenfalls einen signifikant 
positiven Effekt, was darauf hindeuten könnte, dass starke Gewerkschaften zwar 
Arbeitsplatzsicherheit für bestehende Arbeitnehmer gewährleisten, aber möglicherweise den 
Marktzugang für neue gering qualifizierte Arbeitnehmer erschweren. Die 
\textit{Arbeitsmarktregulierung} zeigt mit -0,147 keinen signifikanten Effekt auf die 
\textit{Arbeitslosenquote} dieser Gruppe.

Das Modell erklärt mit einem R²-Wert von 0,304 etwa 30,4\% der Varianz in der 
\textit{Arbeitslosenquote}, wobei der adjustierte R²-Wert bei 0,295 liegt. Dies zeigt, dass 
relevante Einflussfaktoren erfasst wurden, aber weitere Determinanten für die 
Arbeitsmarktentwicklung von Geringqualifizierten berücksichtigt werden sollten.

% Modell ohne Interaktion (mittleres Bildungsniveau)
Für das mittlere Bildungsniveau zeigt sich ebenfalls ein signifikanter positiver Zusammenhang 
zwischen \textit{\ac{ICT}-Investitionen} und \textit{Arbeitslosenquote}. Der geschätzte 
Koeffizient beträgt 1,157*** (p < 0,001), was darauf hindeutet, dass höhere 
\ac{ICT}-Investitionen mit einem Anstieg der Arbeitslosigkeit in dieser Gruppe verbunden sind. 
Dies bestätigt die Annahme, dass auch mittelqualifizierte Arbeitskräfte durch technologische 
Veränderungen betroffen sein können, wenn auch weniger stark als gering Qualifizierte.

Das \textit{\ac{BIP} pro Kopf} zeigt mit -0,153*** (p < 0,001) einen signifikanten negativen 
Zusammenhang. Dies legt nahe, dass eine höhere wirtschaftliche Leistungsfähigkeit mit einer 
geringeren \textit{Arbeitslosenquote} für mittelqualifizierte Personen einhergeht, möglicherweise 
aufgrund größerer Weiterbildungsmöglichkeiten und besserer Arbeitsmarktanpassung.

Der \textit{Tertiärer Bildungsanteil} weist mit 0,282*** (p < 0,001) einen signifikanten 
positiven Effekt auf. Dies könnte darauf hinweisen, dass eine wachsende akademische Bildung den 
Wettbewerb auf dem Arbeitsmarkt verschärft und dadurch mittelqualifizierte Arbeitskräfte 
zunehmend unter Druck setzt. Die \textit{Gewerkschaftsdichte} zeigt mit 0,106*** (p < 0,001) 
ebenfalls einen signifikanten positiven Zusammenhang mit der \textit{Arbeitslosenquote}, was 
darauf hindeutet, dass Gewerkschaften möglicherweise den Kündigungsschutz stärken, aber 
gleichzeitig den Eintritt neuer Arbeitskräfte erschweren könnten.

Die \textit{Arbeitsmarktregulierung} zeigt mit -0,118* (p < 0,05) einen signifikanten negativen 
Effekt, was darauf hindeutet, dass strengere Regulierung einen stabilisierenden Einfluss auf die 
Beschäftigungssituation von Mittelqualifizierten haben könnte. 

Das Modell erklärt mit einem R²-Wert von 0,308 etwa 30,8\% der Varianz in der 
\textit{Arbeitslosenquote}, wobei der adjustierte R²-Wert bei 0,299 liegt. Dies zeigt, dass die 
erfassten Variablen relevante Einflussfaktoren für die Arbeitsmarktentwicklung dieser Gruppe 
darstellen, jedoch weitere strukturelle Mechanismen berücksichtigt werden sollten.

% Modell ohne Interaktion (hohes Bildungsniveau)
Für das hohe Bildungsniveau zeigt das Modell einen signifikanten, aber geringeren positiven 
Zusammenhang zwischen \textit{\ac{ICT}-Investitionen} und \textit{Arbeitslosenquote}. Der 
geschätzte Koeffizient beträgt 0,455*** (p < 0,001), was darauf hindeutet, dass auch 
hochqualifizierte Personen in Ländern mit höheren \ac{ICT}-Investitionen einen moderaten Anstieg 
der Arbeitslosenquote erfahren. Dies widerspricht der Erwartung, dass Hochqualifizierte weniger 
von negativen Arbeitsmarkteffekten der Digitalisierung betroffen sind. Eine mögliche Erklärung 
könnte sein, dass technologische Entwicklungen auch wissensintensive Tätigkeiten transformieren 
oder durch Automatisierung teilweise ersetzen.

Das \textit{\ac{BIP} pro Kopf} zeigt mit -0,083*** (p < 0,001) weiterhin einen signifikant 
negativen Zusammenhang, was bestätigt, dass wirtschaftlich stärkere Länder tendenziell eine 
niedrigere \textit{Arbeitslosenquote} für Hochqualifizierte aufweisen. Der 
\textit{Tertiärer Bildungsanteil} hat mit 0,140*** (p < 0,001) einen signifikant positiven 
Effekt, was darauf hindeutet, dass ein wachsender Anteil an akademisch ausgebildeten Personen den 
Wettbewerb innerhalb dieser Gruppe verstärken könnte.

Die \textit{Gewerkschaftsdichte} zeigt mit 0,025+ (p < 0,1) einen nur schwach signifikanten 
positiven Effekt. Dies könnte darauf hinweisen, dass Gewerkschaften für Hochqualifizierte eine 
geringere Rolle spielen als für andere Bildungsgruppen. Die \textit{Arbeitsmarktregulierung} 
weist mit -0,085* (p < 0,05) einen signifikant negativen Zusammenhang auf, was darauf hindeutet, 
dass striktere arbeitsrechtliche Regelungen möglicherweise stabilisierend wirken, indem sie 
Arbeitsplatzsicherheit erhöhen oder den Zugang zum Arbeitsmarkt regulieren.

Die Modellgüte zeigt mit einem R²-Wert von 0,272, dass etwa 27,2\% der Variation der 
\textit{Arbeitslosenquote} für diese Gruppe durch das Modell erklärt werden. Der adjustierte 
R²-Wert liegt bei 0,262, was darauf hindeutet, dass weitere unberücksichtigte Faktoren für die 
Beschäftigungssituation Hochqualifizierter eine Rolle spielen.

% Modell ohne Interaktion (Fazit)
Zusammenfassend zeigen die Modelle mit Kontrollvariablen, dass \textit{\ac{ICT}-Investitionen} in 
allen Bildungsgruppen einen signifikanten positiven Einfluss auf die \textit{Arbeitslosenquote} 
haben. Die geschätzten Koeffizienten sind durchweg signifikant (niedriges Bildungsniveau: 
2,302***, mittleres Bildungsniveau: 1,157***, hohes Bildungsniveau: 0,455***), was darauf 
hindeutet, dass \ac{ICT}-Investitionen in der derzeitigen Form eher mit einem Anstieg der 
Arbeitslosigkeit einhergehen. Dies bestätigt die Hypothese, dass die Digitalisierung nicht nur 
gering Qualifizierte, sondern auch Mittel- und Hochqualifizierte betrifft.

Besonders für Geringqualifizierte zeigt sich der stärkste Zusammenhang, was mit bisherigen 
Annahmen über die Verwundbarkeit dieser Gruppe auf digitalen Arbeitsmärkten übereinstimmt. 
Gleichzeitig widersprechen die Ergebnisse der Erwartung, dass Hochqualifizierte von steigenden 
\textit{\ac{ICT}-Investitionen} automatisch profitieren. Die Tatsache, dass die Arbeitslosigkeit 
auch in dieser Gruppe zunimmt, deutet darauf hin, dass technologische Innovationen zunehmend auch 
wissensintensive Tätigkeiten transformieren oder teilweise ersetzen.

Makroökonomische Faktoren, insbesondere das \textit{\ac{BIP} pro Kopf}, haben in allen Modellen 
einen signifikanten negativen Einfluss auf die \textit{Arbeitslosenquote}. Dies bestätigt, dass 
wirtschaftlich stärkere Länder tendenziell niedrigere Arbeitslosenquoten aufweisen. Striktere 
\textit{Arbeitsmarktregulierungen} zeigen in der Gruppe der Mittel- und Hochqualifizierten 
negative Effekte auf die \textit{Arbeitslosenquote}, was darauf hindeutet, dass Regulierungen 
möglicherweise stabilisierend wirken, indem sie Arbeitsplatzsicherheit erhöhen. Der 
\textit{Tertiäre Bildungsanteil} ist in allen Gruppen positiv mit der Arbeitslosenquote 
assoziiert, was darauf hindeutet, dass eine höhere Bildungsbeteiligung allein nicht ausreicht, 
um Digitalisierungseffekte auf dem Arbeitsmarkt auszugleichen.

Die Modellgüte (R²-Werte zwischen 0,272 und 0,308) zeigt, dass wesentliche strukturelle und 
institutionelle Einflussfaktoren erfasst wurden, jedoch nicht die gesamte Variation in den 
Arbeitslosenquoten erklären. Dies unterstreicht die Notwendigkeit einer differenzierteren Analyse 
durch Interaktionsmodelle, um die Mechanismen hinter den beobachteten Zusammenhängen besser zu 
verstehen.

% Modelle mit Interaktion
\input{assets/models_interaction_final.tex}

% Modell mit Interaktion (Einführung)
Die Modelle mit Interaktionseffekten liefern eine differenziertere Perspektive auf den 
Zusammenhang zwischen \textit{\ac{ICT}-Investitionen} und der \textit{Arbeitslosenquote}. Im 
Vergleich zu den Basis-Modellen ohne Interaktionen zeigen sich hier mehrere signifikante 
Zusammenhänge, insbesondere mit institutionellen Faktoren wie den \textit{Wohlfahrtsstaatentypen}.

Die Modelle mit Interaktionseffekten liefern eine differenziertere Perspektive auf den 
Zusammenhang zwischen \textit{\ac{ICT}-Investitionen} und der \textit{Arbeitslosenquote}. Im 
Vergleich zu den Basis-Modellen ohne Interaktionen zeigen sich hier mehrere signifikante 
Zusammenhänge, insbesondere mit institutionellen Faktoren wie den \textit{Wohlfahrtsstaatentypen}. 

% Modell mit Interaktion (Referenzkategorie)
Zunächst ist jedoch der Haupteffekt von \textit{\ac{ICT}-Investitionen} zu betrachten, der für 
die Referenzkategorie, die angelsächsischen Wohlfahrtsstaaten, gilt. Die Ergebnisse zeigen, dass 
höhere \textit{\ac{ICT}-Investitionen} in liberalen Staaten in allen Bildungsgruppen mit einer 
steigenden \textit{Arbeitslosenquote} assoziiert sind. Der geschätzte Effekt beträgt 4,671*** 
(p < 0,001) für das niedrige, 3,246*** (p < 0,001) für mittlere und 1,259*** (p < 0,001) für hohe 
Bildungsniveau. Diese Werte deuten darauf hin, dass der Zusammenhang zwischen Digitalisierung und 
Arbeitslosigkeit in angelsächsischen Arbeitsmärkten besonders stark ausgeprägt ist. 

Eine mögliche Erklärung hierfür ist die schwache arbeitsmarktpolitische Regulierung in 
angelsächsischen Staaten. Diese Märkte sind durch eine hohe Flexibilität, geringe staatliche 
Eingriffe und schwache gewerkschaftliche Strukturen geprägt, was bedeutet, dass Arbeitnehmer sich 
weitgehend selbst an technologische Veränderungen anpassen müssen 
\parencite[vgl.][S. 30]{hall2001varieties}. Während in anderen wohlfahrtsstaatlichen 
Modellen Umschulungsprogramme oder Arbeitsmarktregulierungen negative Beschäftigungseffekte der 
Digitalisierung abfedern können, gibt es in angelsächsischen Märkten weniger Schutzmechanismen 
für Arbeitnehmer. Dies könnte dazu führen, dass Arbeitsplatzverluste durch Automatisierung und 
Digitalisierung direkt in steigender Arbeitslosigkeit resultieren, insbesondere für gering- und 
mittelqualifizierte Arbeitnehmer.

Die Interaktionseffekte zwischen \textit{\ac{ICT}-Investitionen} und den 
\textit{Wohlfahrtsstaatentypen} liefern nun wichtige Erkenntnisse darüber, wie institutionelle 
Rahmenbedingungen diesen Zusammenhang beeinflussen. Da die liberalen Wohlfahrtsstaaten als 
Referenzkategorie dienen, sind die Interaktionseffekte relativ zu diesen Staaten zu 
interpretieren.

% Modell mit Interaktion (Interaktionen)
Für postsozialistische Wohlfahrtsstaaten zeigen sich die stärksten negativen Interaktionseffekte. 
Für das niedrige Bildungsniveau beträgt der Interaktionseffekt -5,200*** (p < 0,001), für das 
mittlere Bildungsniveau -3,579*** (p < 0,001) und für das hohe Bildungsniveau -1,415*** 
(p < 0,001). Diese signifikant negativen Werte bedeuten, dass die negativen Auswirkungen von 
\ac{ICT}-Investitionen auf die \textit{Arbeitslosenquote} in diesen Ländern geringer ausfallen 
als in den liberalen Wohlfahrtsstaaten. Da die Interaktionseffekte in ihrer absoluten Höhe sogar 
größer sind als die Haupteffekte der \ac{ICT}-Investitionen, deutet dies darauf hin, dass 
\ac{ICT}-Investitionen in postsozialistischen Staaten nicht zu einer höheren Arbeitslosigkeit 
führen, sondern möglicherweise stabilisierend wirken. Eine mögliche Erklärung hierfür ist, dass 
diese Länder gezielte wirtschaftspolitische Maßnahmen implementiert haben oder strukturelle 
Besonderheiten aufweisen, die negative Effekte der Digitalisierung auf den Arbeitsmarkt abmildern.

Für mitteleuropäische Wohlfahrtsstaaten sind die Interaktionseffekte negativ, aber nicht 
signifikant. Dies bedeutet, dass sich diese Länder nicht signifikant von der Referenzkategorie 
unterscheiden. Mit anderen Worten gibt es keine eindeutigen Belege dafür, dass 
\ac{ICT}-Investitionen in diesen Ländern systematisch andere Auswirkungen auf die 
Arbeitslosigkeit haben als in den liberalen Wohlfahrtsstaaten. Mitteleuropäische Länder wie 
Deutschland oder Frankreich verfügen über relativ stabile Arbeitsmarktstrukturen und ein starkes 
duales Bildungssystem, das möglicherweise negative Effekte abfedert. Allerdings zeigen die 
Ergebnisse keine signifikanten Vorteile dieser Struktur im Vergleich zu liberalen Arbeitsmärkten.

Für nordische Wohlfahrtsstaaten sind die Interaktionseffekte für das niedrige und mittlere 
Bildungsniveau nicht signifikant, während für das hohe Bildungsniveau ein positiver Effekt von 
0,639* (p < 0,05) beobachtet wird. Dies bedeutet, dass in nordischen Ländern Hochqualifizierte 
im Vergleich zu liberalen Wohlfahrtsstaaten tendenziell stärker von steigender Arbeitslosigkeit 
betroffen sind. Eine mögliche Erklärung könnte sein, dass nordische Staaten stark in 
Digitalisierung investieren und dadurch Arbeitsplätze in wissensintensiven Bereichen 
umstrukturieren. Dies könnte hochqualifizierte Arbeitskräfte vor neue Herausforderungen stellen, 
insbesondere wenn technologische Entwicklungen schneller voranschreiten als Bildungs- und 
Umschulungsmaßnahmen.

Für südeuropäische Wohlfahrtsstaaten zeigen sich signifikant negative Interaktionseffekte für das 
mittlere und hohe Bildungsniveau. Der Effekt für das mittlere Bildungsniveau beträgt -3,066*** 
(p < 0,001), für das hohe Bildungsniveau -2,880*** (p < 0,001). Dies deutet darauf hin, dass 
\ac{ICT}-Investitionen in diesen Ländern mit einer geringeren Arbeitslosenquote für Mittel- und 
Hochqualifizierte im Vergleich zu liberalen Wohlfahrtsstaaten verbunden sind. Dies könnte 
bedeuten, dass technologische Investitionen gezielt in qualifizierte Arbeitsplätze fließen oder 
dass strukturelle Gegebenheiten, wie Arbeitsplatzsicherheit durch Regulierungen, digitale 
Umbrüche abfedern. Gleichzeitig könnte diese geringe Dynamik jedoch auch bedeuten, dass 
technologische Transformationen langsamer verlaufen, was langfristig negative Folgen für die 
Wettbewerbsfähigkeit haben könnte.

% Modell mit Interaktion (Modelgüte)
Die erklärten Varianzen (R²-Werte) sind in den Interaktionsmodellen höher als in den 
Basis-Modellen ohne Interaktionen. Während die R²-Werte in den einfachen Modellen zwischen 
0,272 und 0,308 lagen, erreichen die Interaktionsmodelle Werte von 0,337 
(niedriges Bildungsniveau), 0,333 (mittleres Bildungsniveau) und 0,306 (hohes Bildungsniveau). 
Dies zeigt, dass institutionelle Rahmenbedingungen eine wesentliche Rolle spielen und die 
Erklärungskraft der Modelle verbessern. Trotzdem bleibt ein relevanter Teil der Variation in der 
Arbeitslosenquote unaufgeklärt, was darauf hindeutet, dass weitere Faktoren eine Rolle spielen.

% Modell mit Interaktion (Kontrollvariablen)
Das \textit{\ac{BIP} pro Kopf} bleibt weiterhin negativ und signifikant für alle Bildungsgruppen, 
zeigt jedoch einen leicht abgeschwächten Effekt im Vergleich zum Modell ohne Interaktion. Dies 
könnte darauf hindeuten, dass institutionelle Unterschiede die Rolle des 
\ac{BIP} für die Arbeitsmarktentwicklung teilweise moderieren. Die \textit{Gewerkschaftsdichte} 
bleibt für niedrig- und mittelqualifizierte Personen positiv signifikant, während sie für 
Hochqualifizierte keine signifikanten Effekte zeigt. Dies deutet darauf hin, dass Gewerkschaften 
für geringer qualifizierte Beschäftigte eine stärkere Schutzfunktion haben, während sie bei 
Hochqualifizierten eine untergeordnete Rolle spielen. Die 
\textit{Regulierungsstrenge des Arbeitsmarktes} zeigt für Hochqualifizierte weiterhin eine 
signifikant negative Wirkung (-0,091**), was darauf hindeutet, dass strengere 
Arbeitsmarktregulierungen in dieser Gruppe schützend wirken können. Für Mittelqualifizierte ist 
der Effekt hingegen nicht mehr signifikant, was darauf hindeutet, dass institutionelle 
Unterschiede den Einfluss der Regulierung auf diese Gruppe abschwächen. Der 
\textit{Anteil tertiär gebildeter Personen} zeigt ebenfalls durchweg signifikante Effekte, bleibt 
jedoch in der Richtung konsistent mit den Basis-Modellen. Die positiven Werte deuten darauf hin, 
dass eine höhere Bildungsbeteiligung weiterhin mit einer steigenden Arbeitslosenquote korreliert, 
was möglicherweise auf steigende Konkurrenz innerhalb der jeweiligen Qualifikationsgruppe 
zurückzuführen ist.

% Modell mit Interaktion (Fazit)
Insgesamt zeigen die Interaktionsmodelle, dass die Auswirkungen von \ac{ICT}-Investitionen stark 
vom Wohlfahrtsstaatentyp abhängen. In postsozialistischen Ländern scheinen Investitionen nicht 
mit einer steigenden Arbeitslosigkeit verbunden zu sein, während sich in mitteleuropäischen 
Staaten keine signifikanten Unterschiede zu liberalen Wohlfahrtsstaaten zeigen. In nordischen 
Ländern scheint es eine leicht negative Wirkung auf hochqualifizierte Arbeitskräfte zu geben, 
während in südeuropäischen Ländern \ac{ICT}-Investitionen mit einer geringeren Arbeitslosigkeit 
für Mittel- und Hochqualifizierte verbunden sind. Die insgesamt verbesserten R²-Werte zeigen, 
dass institutionelle Faktoren eine wichtige Rolle in der Erklärung der Beschäftigungseffekte der 
Digitalisierung spielen, dennoch bleibt ein relevanter Teil der Varianz unerklärt. Dies 
verdeutlicht, dass Digitalisierung und technologische Transformation nicht in allen Ländern 
dieselben Beschäftigungseffekte haben und dass wirtschaftspolitische sowie institutionelle 
Rahmenbedingungen maßgeblich beeinflussen, wie Arbeitsmärkte auf technologische Investitionen 
reagieren.
