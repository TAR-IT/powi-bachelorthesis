%%%%%%%%%%%%%%%%%%%%%%%%
% Diskussion und Fazit %
%%%%%%%%%%%%%%%%%%%%%%%%

\section{Diskussion und Fazit}

Die Ergebnisse dieser Arbeit bieten wertvolle Einblicke in die Beziehung zwischen 
\textit{\ac{ICT}-Investitionen} und der \textit{Arbeitslosenquote} in verschiedenen 
Bildungsniveaus. Sie zeigen signifikante Zusammenhänge und verdeutlichen die Rolle 
institutioneller Rahmenbedingungen für die Beschäftigungswirkungen der Digitalisierung. 
Die Untersuchung trägt zur wissenschaftlichen Debatte über die Wechselwirkungen 
zwischen technologischer Entwicklung, Arbeitsmarktstrukturen und politischen 
Institutionen bei und liefert praktische Implikationen für Politik, Unternehmen 
und Bildungssysteme.

%%%%%%%%%%%%%%%%%%%%%%%%%%%%%%%%%%%%%%%%%%%%%%%%%%%%%%%%%
% Diskussion und Fazit: Zentrale Ergebnisse der Analyse %
%%%%%%%%%%%%%%%%%%%%%%%%%%%%%%%%%%%%%%%%%%%%%%%%%%%%%%%%%

\subsection{Zentrale Ergebnisse der Analyse}

Die erste Hypothese (\textbf{H1}), dass \textit{\ac{ICT}-Investitionen} mit einer 
niedrigeren \textit{Arbeitslosenquote} unter Hochqualifizierten verbunden sind, wird 
durch die Ergebnisse nicht gestützt. Entgegen der Annahme zeigen die Modelle, dass höhere 
\textit{\ac{ICT}-Investitionen} nicht mit einer Senkung der Arbeitslosigkeit einhergehen, 
sondern auch für Hochqualifizierte mit steigenden Arbeitslosenquoten korrelieren. Dies 
widerspricht der klassischen Annahme des \ac{SBTC}, dass Hochqualifizierte grundsätzlich 
von der Digitalisierung profitieren. Eine mögliche Erklärung könnte sein, dass die 
zunehmende Automatisierung nicht nur einfache, sondern auch wissensintensive Tätigkeiten 
betrifft.

Die zweite Hypothese (\textbf{H2}), dass \textit{\ac{ICT}-Investitionen} die 
\textit{Arbeitslosenquote} unter gering Qualifizierten erhöhen, wird hingegen durch 
die Ergebnisse gestützt. Besonders in den Basis-Modellen zeigt sich für 
Geringqualifizierte der stärkste positive Effekt, was darauf hindeutet, dass einfache 
Tätigkeiten besonders stark von Automatisierung betroffen sind. Dies steht im Einklang 
mit der Theorie des \ac{SBTC} und der Polarisierungsthese, die besagen, dass Digitalisierung 
mittlere Qualifikationsniveaus verdrängt, während Hochqualifizierte davon profitieren.

Die dritte Hypothese (\textbf{H3}), dass institutionelle Faktoren wie 
Wohlfahrtsstaatentypen die negativen Effekte von \textit{\ac{ICT}-Investitionen} 
abmildern können, wird teilweise durch die Ergebnisse bestätigt. Die Interaktionsmodelle 
zeigen, dass postsozialistische und südeuropäische Wohlfahrtsstaaten in der Lage sind, die 
negativen Beschäftigungseffekte der Digitalisierung abzuschwächen, während in 
liberalen Staaten ein deutlicher Zusammenhang zwischen \textit{\ac{ICT}-Investitionen} 
und steigender \textit{Arbeitslosenquote} besteht. Dies deutet darauf hin, dass 
institutionelle Strukturen eine wesentliche Rolle für die Auswirkungen der 
Digitalisierung auf den Arbeitsmarkt spielen.

Die Kontrollvariablen liefern weitere relevante Erkenntnisse. Das 
\textit{\ac{BIP} pro Kopf} bleibt durch alle Bildungsniveaus hinweg ein signifikanter 
Faktor mit einem negativen Effekt auf die \textit{Arbeitslosenquote}. Die 
\textit{Regulierungsstrenge des Arbeitsmarktes} hat für Hochqualifizierte einen 
stabilisierenden Einfluss, während sie für Geringqualifizierte tendenziell mit einer 
höheren \textit{Arbeitslosenquote} korreliert. Der \textit{tertiäre Bildungsanteil} zeigt 
in allen Gruppen einen positiven Effekt auf die \textit{Arbeitslosenquote}, was darauf 
hindeutet, dass eine größere Bildungsbeteiligung allein nicht ausreicht, um die negativen 
Effekte der Digitalisierung auszugleichen.

%%%%%%%%%%%%%%%%%%%%%%%%%%%%%%%%%%%%%%%%%%%%%%%%%%%%%%%%%%%%%%%%%%%%%%%%%%%%%%%%
% Diskussion und Fazit: Einordnung der Ergebnisse in den theoretischen Kontext %
%%%%%%%%%%%%%%%%%%%%%%%%%%%%%%%%%%%%%%%%%%%%%%%%%%%%%%%%%%%%%%%%%%%%%%%%%%%%%%%%

\subsection{Einordnung der Ergebnisse in den theoretischen Kontext}

Die Theorie des \ac{SBTC} besagt, dass technologische Innovationen die Nachfrage nach 
hochqualifizierten Arbeitskräften steigern, während gering Qualifizierte durch 
Automatisierung verdrängt werden \parencite[vgl.][S. 4-8]{acemoglu2002technical}. Die 
Ergebnisse dieser Arbeit unterstützen diese Annahme jedoch nur teilweise. Während 
erwartet wurde, dass \ac{ICT}-Investitionen primär die Arbeitslosigkeit von gering 
Qualifizierten erhöhen (was durch die Basis-Modelle bestätigt wird), zeigen sich für 
Hochqualifizierte ebenfalls steigende Arbeitslosenquoten. Dies widerspricht der Annahme, 
dass Hochqualifizierte generell von der Digitalisierung profitieren. Vielmehr deuten die 
Ergebnisse darauf hin, dass Digitalisierung auch hochqualifizierte Tätigkeiten 
beeinflusst, sei es durch Automatisierung kognitiver Aufgaben oder durch eine steigende 
Konkurrenz unter hochqualifizierten Arbeitskräften.

Die Interaktionsmodelle legen nahe, dass institutionelle Faktoren diese Effekte 
modifizieren. Besonders in postsozialistischen und südeuropäischen Ländern fällt der 
Zusammenhang zwischen \ac{ICT}-Investitionen und Arbeitslosigkeit schwächer aus. Dies 
könnte im Einklang mit Schumpeters Konzept der „schöpferischen Zerstörung“ stehen, 
wonach wirtschaftlicher Wandel langfristig zu Wachstum führt, kurzfristig aber 
strukturelle Umbrüche verursacht \parencite[vgl.][S. 81-86]{schumpeter1976capitalism}. 
Während in hoch digitalisierten liberalen Staaten wie den USA oder Großbritannien 
Arbeitsmärkte bereits stark durch digitale Umstellungen transformiert wurden, könnte es 
in postsozialistischen Staaten Verzögerungseffekte geben, die erklären, warum dort die 
negativen Beschäftigungseffekte weniger ausgeprägt sind.

Darüber hinaus zeigen die Ergebnisse, dass nicht nur die Polarisierung der 
Arbeitsmärkte durch Digitalisierung (wie in der \ac{SBTC}-Theorie angenommen) von 
Bedeutung ist, sondern auch die Anpassungsfähigkeit der institutionellen 
Rahmenbedingungen. Die Unterschiede zwischen den Wohlfahrtsstaaten verdeutlichen, dass 
die Beschäftigungseffekte von \ac{ICT}-Investitionen stark von politischen und 
wirtschaftlichen Strukturen abhängen. In liberalen Wohlfahrtsstaaten zeigt sich ein 
durchgehend starker positiver Zusammenhang zwischen Digitalisierung und 
Arbeitslosigkeit, was mit der dortigen Deregulierung des Arbeitsmarktes und geringeren 
sozialen Sicherungssystemen zusammenhängen könnte. In postsozialistischen und 
südeuropäischen Staaten hingegen könnten wirtschaftspolitische Maßnahmen oder ein 
insgesamt stärker regulierter Arbeitsmarkt die negativen Folgen der Digitalisierung 
abschwächen.

Diese Erkenntnisse zeigen, dass die Auswirkungen von \ac{ICT}-Investitionen nicht 
isoliert betrachtet werden können. Vielmehr müssen sie in den jeweiligen 
institutionellen Kontext eingeordnet werden. Während Digitalisierung in einigen Ländern 
mit Arbeitsplatzverlusten einhergeht, kann sie in anderen Ländern stabilisierend oder 
sogar beschäftigungsfördernd wirken. Die Ergebnisse dieser Arbeit zeigen, dass eine 
differenzierte Betrachtung notwendig ist, um die Beschäftigungsfolgen der Digitalisierung 
besser zu verstehen. Dies hat auch wichtige Implikationen für die Arbeitsmarkt- und 
Bildungspolitik: Eine erfolgreiche digitale Transformation erfordert nicht nur 
Investitionen in Technologie, sondern auch begleitende arbeitsmarktpolitische Maßnahmen, 
die sicherstellen, dass Arbeitskräfte auf die veränderten Anforderungen vorbereitet sind.

Schließlich unterstreicht diese Untersuchung die Bedeutung langfristiger Analysen zur 
Digitalisierung und Beschäftigung. Während kurzfristige Effekte oft negativ erscheinen, 
zeigt die Theorie der „schöpferischen Zerstörung“, dass technologische Innovationen 
langfristig neue Beschäftigungsmöglichkeiten schaffen können 
\parencite[vgl.][S. 81-86]{schumpeter1976capitalism}. Die Ergebnisse legen nahe, dass 
Länder mit aktiven Anpassungsmechanismen - sei es durch Weiterbildung, soziale 
Sicherungssysteme oder aktive Arbeitsmarktpolitik - langfristig besser in der Lage sind, 
die Herausforderungen der Digitalisierung zu bewältigen. Diese Erkenntnisse sind für 
politische Entscheidungsträger von großer Bedeutung, da sie zeigen, dass Digitalisierung 
nicht automatisch zu Arbeitslosigkeit führen muss, sondern dass institutionelle 
Rahmenbedingungen eine entscheidende Rolle dabei spielen, wie sich technologische 
Veränderungen auf den Arbeitsmarkt auswirken.

%%%%%%%%%%%%%%%%%%%%%%%%%%%%%%%%%%%%%%%%%%%%%%%%%%%%%%%%%%%%%%%
% Diskussion und Fazit: Limitationen und zukünftige Forschung %
%%%%%%%%%%%%%%%%%%%%%%%%%%%%%%%%%%%%%%%%%%%%%%%%%%%%%%%%%%%%%%%

\subsection{Limitationen und zukünftige Forschung}

Trotz der wertvollen Erkenntnisse dieser Untersuchung sind einige Limitationen zu 
berücksichtigen. Erstens basiert die Analyse auf aggregierten \ac{OECD}-Daten für den 
Zeitraum 2005–2022, wodurch Unterschiede in der Erhebungsmethodik zwischen den Ländern 
die Ergebnisse beeinflussen könnten. Zweitens liegt der Fokus auf makroökonomischen 
Zusammenhängen, sodass individuelle Anpassungsstrategien von Arbeitnehmer*innen oder 
Unternehmen an die Digitalisierung nicht berücksichtigt werden konnten. Künftige Studien 
sollten verstärkt auf Umfragedaten oder firmenspezifische Datenquellen zurückgreifen, um 
differenziertere Erkenntnisse zu gewinnen.

Darüber hinaus besteht die Möglichkeit einer umgekehrten Kausalität („Reversed Causality“) 
zwischen \textit{\ac{ICT}-Investitionen} und der \textit{Arbeitslosenquote} 
\parencite[S. 29-33]{pearl2009causality}. Während in dieser Analyse angenommen wurde, dass 
\textit{\ac{ICT}-Investitionen} die \textit{Arbeitslosenquote} beeinflussen, könnte es 
auch sein, dass hohe Arbeitslosigkeit Regierungen oder Unternehmen zu verstärkten 
Investitionen in Digitalisierung veranlasst. Diese Hypothese könnte mit Methoden wie 
Granger-Kausalitätstests oder Instrumentalvariablen weiter geprüft werden.

%%%%%%%%%%%%%%%%%%%%%%%%%%%%%%%%%%%%%
% Diskussion und Fazit: Gesamtfazit %
%%%%%%%%%%%%%%%%%%%%%%%%%%%%%%%%%%%%%

\subsection{Gesamtfazit}

Zusammenfassend zeigen die Ergebnisse, dass \ac{ICT}-Investitionen keine universelle 
Lösung für Arbeitsmarktprobleme darstellen, sondern dass ihre Wirkungen maßgeblich von 
institutionellen Rahmenbedingungen abhängen. Während in liberalen Wohlfahrtsstaaten ein 
deutlicher positiver Zusammenhang zwischen \textit{\ac{ICT}-Investitionen} und 
\textit{Arbeitslosenquote} besteht, zeigen sich in postsozialistischen und 
südeuropäischen Ländern stabilisierende Effekte. Dies deutet darauf hin, dass 
wirtschafts- und arbeitsmarktpolitische Maßnahmen entscheidend sind, um die negativen 
Effekte der Digitalisierung abzufedern.

Die Ergebnisse bestätigen die Hypothese, dass gering Qualifizierte besonders stark von 
negativen Digitalisierungseffekten betroffen sind. Gleichzeitig zeigen sich auch für 
Hochqualifizierte unerwartet steigende Arbeitslosenquoten, was auf eine zunehmende 
Automatisierung und Umstrukturierung wissensintensiver Tätigkeiten hinweisen könnte. Die 
institutionelle Einbettung digitaler Transformationen erweist sich als zentraler Faktor: 
Während in manchen Ländern technologische Innovationen mit steigender Arbeitslosigkeit 
verbunden sind, können wohlfahrtsstaatliche Strukturen in anderen Ländern diese Effekte 
teilweise kompensieren.

Daraus ergibt sich eine klare politische Implikation: Eine erfolgreiche digitale 
Transformation erfordert nicht nur technologische Investitionen, sondern auch begleitende 
arbeitsmarktpolitische, wirtschaftliche und bildungspolitische Maßnahmen, um den 
Strukturwandel sozialverträglich zu gestalten.
