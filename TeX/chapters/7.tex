% TODO: Durchlesen (Stand 07.03.2025)

\section{Diskussion und Fazit}

Die Ergebnisse dieser Arbeit bieten wertvolle Einblicke in die Beziehung zwischen 
\textit{\ac{ICT}-Investitionen} und der \textit{Arbeitslosenquote} in verschiedenen 
Bildungsniveaus. Sie zeigen signifikante Zusammenhänge und verdeutlichen die Rolle 
institutioneller Rahmenbedingungen für die Beschäftigungswirkungen der Digitalisierung. 
Die Untersuchung trägt zur wissenschaftlichen Debatte über die Wechselwirkungen 
zwischen technologischer Entwicklung, Arbeitsmarktstrukturen und politischen 
Institutionen bei und liefert praktische Implikationen für Politik, Unternehmen 
und Bildungssysteme.

\subsection{Zentrale Ergebnisse der Analyse}

Die erste Hypothese (\textbf{H1}), dass \textit{\ac{ICT}-Investitionen} mit einer 
niedrigeren \textit{Arbeitslosenquote} unter Hochqualifizierten verbunden sind, wird 
durch die Ergebnisse nicht bestätigt. Die Basis-Modelle zeigen, dass höhere 
\textit{\ac{ICT}-Investitionen} über alle Bildungsgruppen hinweg mit steigender 
\textit{Arbeitslosenquote} korrelieren, auch für Hochqualifizierte. Dies steht im 
Widerspruch zur weit verbreiteten Annahme, dass Hochqualifizierte primär von 
technologischen Fortschritten profitieren.

Die zweite Hypothese (\textbf{H2}), dass \textit{\ac{ICT}-Investitionen} die 
\textit{Arbeitslosenquote} unter gering Qualifizierten erhöhen, wird hingegen durch 
die Ergebnisse gestützt. Besonders im Basis-Modell ohne Interaktion zeigt sich für 
Geringqualifizierte der stärkste positive Effekt, was darauf hindeutet, dass einfache 
Tätigkeiten besonders stark von Automatisierung betroffen sind.

Die dritte Hypothese (\textbf{H3}), dass institutionelle Faktoren wie 
Wohlfahrtsstaatentypen die negativen Effekte von \textit{\ac{ICT}-Investitionen} 
abmildern können, wird teilweise bestätigt. Die Interaktionsmodelle zeigen, dass 
postsozialistische und südeuropäische Wohlfahrtsstaaten in der Lage sind, die 
negativen Beschäftigungseffekte der Digitalisierung abzuschwächen, während in 
liberalen Staaten ein deutlicher Zusammenhang zwischen \textit{\ac{ICT}-Investitionen} 
und steigender \textit{Arbeitslosenquote} besteht. Dies deutet darauf hin, dass 
institutionelle Strukturen eine wesentliche Rolle für die Auswirkungen der 
Digitalisierung auf den Arbeitsmarkt spielen.

Die Kontrollvariablen liefern weitere relevante Erkenntnisse. Das 
\textit{\ac{BIP} pro Kopf} bleibt durch alle Bildungsniveaus hinweg ein signifikanter 
Faktor mit einem negativen Effekt auf die \textit{Arbeitslosenquote}. Die 
\textit{Regulierungsstrenge des Arbeitsmarktes} hat für Hochqualifizierte einen 
stabilisierenden Einfluss, während sie für Geringqualifizierte tendenziell mit einer 
höheren \textit{Arbeitslosenquote} korreliert. Der \textit{tertiäre Bildungsanteil} zeigt 
in allen Gruppen einen positiven Effekt auf die \textit{Arbeitslosenquote}, was darauf 
hindeutet, dass eine größere Bildungsbeteiligung allein nicht ausreicht, um die negativen 
Effekte der Digitalisierung auszugleichen.

\subsection{Einordnung der Ergebnisse in den theoretischen Kontext}

Die Theorie des \ac{SBTC} besagt, dass technologische Innovationen die Nachfrage nach 
hochqualifizierten Arbeitskräften steigern, während gering Qualifizierte durch 
Automatisierung verdrängt werden \parencite[vgl.][S. 7]{acemoglu2002technical}. Die 
Ergebnisse dieser Arbeit bestätigen diese Annahme nur teilweise. Während erwartet wurde, 
dass \ac{ICT}-Investitionen primär die Arbeitslosigkeit von gering Qualifizierten erhöhen 
(was durch die Basis-Modelle bestätigt wird), zeigen sich für Hochqualifizierte ebenfalls 
steigende Arbeitslosenquoten. Dies widerspricht der Annahme, dass Hochqualifizierte 
generell von der Digitalisierung profitieren.

Die Interaktionsmodelle legen nahe, dass institutionelle Faktoren diese Effekte 
modifizieren. Besonders in postsozialistischen und südeuropäischen Ländern fällt der 
Zusammenhang zwischen \ac{ICT}-Investitionen und Arbeitslosigkeit schwächer aus. Dies 
könnte im Einklang mit Schumpeters Konzept der „schöpferischen Zerstörung“ stehen, wonach 
wirtschaftlicher Wandel langfristig zu Wachstum führt, kurzfristig aber strukturelle 
Umbrüche verursacht \parencite[vgl.][S. 103-105]{schumpeter1976capitalism}. Während in 
hoch digitalisierten liberalen Staaten wie den USA oder Großbritannien Arbeitsmärkte 
bereits stark durch digitale Umstellungen transformiert wurden, könnte es in 
postsozialistischen Staaten Verzögerungseffekte geben, die erklären, warum dort die 
negativen Beschäftigungseffekte weniger ausgeprägt sind.

\subsection{Limitationen und zukünftige Forschung}

Trotz der wertvollen Erkenntnisse dieser Untersuchung sind einige Limitationen zu 
berücksichtigen. Erstens basiert die Analyse auf aggregierten \ac{OECD}-Daten für den 
Zeitraum 2005–2022, wodurch Unterschiede in der Erhebungsmethodik zwischen den Ländern 
die Ergebnisse beeinflussen könnten. Zweitens liegt der Fokus auf makroökonomischen 
Zusammenhängen, sodass individuelle Anpassungsstrategien von Arbeitnehmer*innen oder 
Unternehmen an die Digitalisierung nicht berücksichtigt werden konnten. Künftige Studien 
sollten verstärkt auf Umfragedaten oder firmenspezifische Datenquellen zurückgreifen, um 
differenziertere Erkenntnisse zu gewinnen.

Darüber hinaus besteht die Möglichkeit einer umgekehrten Kausalität zwischen 
\textit{\ac{ICT}-Investitionen} und der \textit{Arbeitslosenquote}. Während in dieser 
Analyse angenommen wurde, dass \textit{\ac{ICT}-Investitionen} die 
\textit{Arbeitslosenquote} beeinflussen, könnte es auch sein, dass hohe Arbeitslosigkeit 
Regierungen oder Unternehmen zu verstärkten Investitionen in Digitalisierung veranlasst. 
Diese Hypothese könnte mit Methoden wie Granger-Kausalitätstests oder 
Instrumentalvariablen weiter geprüft werden.

\subsection{Gesamtfazit}

Zusammenfassend zeigen die Ergebnisse, dass \ac{ICT}-Investitionen keine universelle 
Lösung für Arbeitsmarktprobleme darstellen, sondern dass ihre Wirkungen maßgeblich von 
institutionellen Rahmenbedingungen abhängen. Während in liberalen Wohlfahrtsstaaten ein 
deutlicher positiver Zusammenhang zwischen \ac{ICT}-Investitionen und Arbeitslosigkeit 
besteht, zeigen sich in postsozialistischen und südeuropäischen Ländern stabilisierende 
Effekte. Dies deutet darauf hin, dass wirtschafts- und arbeitsmarktpolitische Maßnahmen 
entscheidend sind, um die negativen Effekte der Digitalisierung abzufedern.

Die Ergebnisse bestätigen die Hypothese, dass gering Qualifizierte besonders stark von 
negativen Digitalisierungseffekten betroffen sind. Gleichzeitig zeigen sich auch für 
Hochqualifizierte unerwartet steigende Arbeitslosenquoten, was auf eine zunehmende 
Automatisierung und Umstrukturierung wissensintensiver Tätigkeiten hinweisen könnte. Die 
institutionelle Einbettung digitaler Transformationen erweist sich als zentraler Faktor: 
Während in manchen Ländern technologische Innovationen mit steigender Arbeitslosigkeit 
verbunden sind, können wohlfahrtsstaatliche Strukturen in anderen Ländern diese Effekte 
teilweise kompensieren.

Daraus ergibt sich eine klare politische Implikation: Eine erfolgreiche digitale 
Transformation erfordert nicht nur technologische Investitionen, sondern auch begleitende 
arbeitsmarktpolitische, wirtschaftliche und bildungspolitische Maßnahmen, um den 
Strukturwandel sozialverträglich zu gestalten.
