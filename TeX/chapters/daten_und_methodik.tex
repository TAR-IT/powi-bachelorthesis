% TODO: Durchlesen! Zitation!

%%%%%%%%%%%%%%
% Datensätze %
%%%%%%%%%%%%%%

\section{Daten und Methodik}

\subsection{Datensätze}

Die vorliegenden Daten stammen aus den umfangreichen Datensätzen der \ac{OECD}, einer internationalen 
Organisation, die vergleichbare Wirtschafts- und Sozialstatistiken für ihre Mitgliedsländer bereitstellt. 
Die \ac{OECD} sammelt und veröffentlicht regelmäßig Daten zu wirtschaftlichen, sozialen und 
technologischen Entwicklungen, die es ermöglichen, langfristige Trends und länderspezifische Unterschiede 
zu analysieren \parencite{oecd2022ict}.

Für diese Untersuchung werden insbesondere die Datensätze zu \ac{ICT}-Investitionen \parencite{oecd2022ict} 
sowie zu den Arbeitslosenquoten nach Bildungsniveau \parencite{oecd2022unemployment} verwendet. 
Diese Datensätze bieten eine solide Grundlage, um die Auswirkungen von Investitionen in \ac{ICT} auf 
die Arbeitslosigkeit in verschiedenen Bildungsgruppen auf internationaler Ebene zu analysieren, während 
gleichzeitig institutionelle Unterschiede berücksichtigt werden.

Die Daten umfassen insgesamt 35 OECD-Länder\footnote{Untersuchte Länder: Australien, Österreich, Belgien, 
Bulgarien, Brasilien, Kanada, Kroatien, Tschechien, Dänemark, Estland, Finnland, Frankreich, Deutschland, 
Griechenland, Ungarn, Island, Italien, Irland, Lettland, Litauen, Luxemburg, Niederlande, 
Neuseeland, Norwegen, Polen, Portugal, Rumänien, Spanien, Schweden, Schweiz, Türkei, 
Slowakei, Slowenien, Vereinigtes Königreich, USA.} und decken den Zeitraum von 2005 bis 
2022 ab. Nach Bereinigung der Daten und Zusammenführung der relevanten Variablen verbleiben 1506 Beobachtungen 
für die Paneldatenanalyse.

Die Daten zu \ac{ICT}-Investitionen messen die Bruttoanlageinvestitionen in digitale Infrastrukturen und 
Technologien \parencite{oecd2022ict}, während die Arbeitsmarktstatistiken detaillierte Informationen über 
die Arbeitslosenquoten in verschiedenen Bildungsgruppen bieten \parencite{oecd2022unemployment}. Durch die 
längsschnittliche Struktur der Daten wird es möglich, sowohl kurzfristige als auch langfristige 
Entwicklungen zu erfassen.

%%%%%%%%%%%%%%%%%%%%%%%
% Operationalisierung %
%%%%%%%%%%%%%%%%%%%%%%%

\subsection{Operationalisierung}

Zur Beantwortung der Forschungsfrage - wie Investitionen in Informations- und Kommunikationstechnologien 
die Arbeitslosenquoten in unterschiedlichen Bildungsgruppen beeinflussen - ist eine präzise und konsistente 
Operationalisierung der zentralen Konzepte notwendig. Dies gewährleistet, dass die Untersuchung die 
beabsichtigten Zusammenhänge abbildet und die Daten sinnvoll ausgewertet werden können.

Die abhängige Variable dieser Untersuchung ist die \textit{Arbeitslosenquote} (UNEMPLOYMENT\_RATE\_PERCENT), 
die nach dem Bildungsniveau der Bevölkerung differenziert wird. Der \ac{OECD}-Datensatz unterteilt das 
Bildungsniveau in drei Hauptkategorien:

\begin{enumerate}
    \item \textbf{Niedriges Bildungsniveau}: Personen ohne abgeschlossene Schulbildung oder mit einem 
    maximalen Hauptschulabschluss.
    \item \textbf{Mittleres Bildungsniveau}: Personen mit Sekundarschulabschluss oder einer abgeschlossenen 
    Berufsausbildung.
    \item \textbf{Hohes Bildungsniveau}: Personen mit Hochschulabschluss, wie einem Bachelor, Master oder 
    Doktortitel.
\end{enumerate}

Arbeitslose sind nach der Definition der \ac{OECD} Personen im erwerbsfähigen Alter, die keine Arbeit haben, 
für eine Arbeit zur Verfügung stehen und in den letzten vier Wochen konkrete Schritte unternommen haben, um 
eine Arbeit zu finden \parencite{oecd2022unemployment}. Dieser Indikator wird als Prozentsatz der 
Erwerbsbevölkerung gemessen und ist saisonbereinigt.

Die zentrale unabhängige Variable \textit{\ac{ICT}-Investitionen} (ICT\_INVEST\_SHARE\_GDP) misst Investitionen 
in digitale Infrastruktur, Software, Hardware und Technologien, die zur Verbesserung betrieblicher Effizienz 
und Produktivität beitragen \parencite{oecd2022ict}. Die Daten basieren auf den Definitionen des \ac{SNA08} 
und werden als Anteil am \ac{BIP} in Prozent angegeben.

Um sicherzustellen, dass der Effekt der \ac{ICT}-Investitionen auf die Arbeitslosenquote nicht durch andere 
Faktoren verzerrt wird, werden mehrere Kontrollvariablen in die Analyse aufgenommen:

\begin{itemize}
    \item \textit{\ac{BIP} pro Kopf} (GDP\_PER\_CAPITA): Diese Variable misst den wirtschaftlichen Wohlstand 
    eines Landes in tausend US-Dollar pro Jahr und kontrolliert den Entwicklungsstand eines Landes, da 
    wirtschaftlich wohlhabendere Länder tendenziell niedrigere Arbeitslosenquoten aufweisen 
    \parencite{oecd2022gdp}.
    \item \textit{Gewerkschaftsdichte} (PERCENT\_EMPLOYEES\_TUD): Der Anteil der in Gewerkschaften 
    organisierten Arbeitnehmer wird berücksichtigt, da Gewerkschaften eine wichtige Rolle bei der Aushandlung 
    von Arbeitsbedingungen und Arbeitsplatzsicherheit spielen \parencite{oecd2022tud}. Frühere Studien 
    zeigen, dass eine hohe Gewerkschaftsdichte oft mit niedrigeren Arbeitslosenquoten für geringqualifizierte 
    Arbeitnehmer verbunden ist, da Gewerkschaften Mindestlöhne sichern und Beschäftigungsschutzmaßnahmen 
    verstärken \parencite[S. 61]{nickell1997unemployment}.
    \item \textit{Wohlfahrtsstaatentyp} (WELFARE\_STATE): Die Wohlfahrtsstaatentypologie nach Esping-Andersen 
    (1990) \parencite{espingandersen1990thethree} wird in die Analyse integriert, um zu untersuchen, 
    inwiefern institutionelle Unterschiede den Effekt von \ac{ICT}-Investitionen auf die Arbeitslosenquote 
    beeinflussen. Die Länder werden in fünf Kategorien unterteilt: \textit{sozialdemokratisch}, 
    \textit{konservativ}, \textit{liberal}, \textit{südeuropäisch} und \textit{postsozialistisch}.
\end{itemize}

Die Kombination dieser Daten ermöglicht es, länderspezifische Unterschiede in der Wirtschaftskraft, den 
regulatorischen Rahmenbedingungen und der Bildungsstruktur zu kontrollieren, um den direkten Einfluss der 
\ac{ICT}-Investitionen präzise zu analysieren.

\subsection{Analytische Methode}

Die Analyse dieser Arbeit basiert auf einer Paneldatenanalyse, um die Auswirkungen von \ac{ICT}-Investitionen 
auf die Arbeitslosenquote nach Bildungsniveau zu untersuchen. Die Wahl einer Paneldatenmethode ermöglicht 
es, sowohl individuelle Heterogenität zwischen Ländern als auch dynamische Entwicklungen über die Zeit zu 
erfassen \parencite{wooldridge2010econometric}.

Zur Untersuchung der Effekte wurden zwei gängige Modelle der Paneldatenanalyse betrachtet: das \ac{FE}- 
und das \ac{RE}-Modell. Während das \ac{FE}-Modell explizit für zeitinvariante länderspezifische 
Eigenschaften kontrolliert \parencite[S. 251–256]{wooldridge2010econometric}, geht das \ac{RE}-Modell 
davon aus, dass länderspezifische Effekte zufällig verteilt und nicht mit den unabhängigen Variablen 
korreliert sind \parencite[S. 17–20]{baltagi2021econometric}. Falls diese Annahme verletzt ist, können 
die Schätzungen im \ac{RE}-Modell verzerrt sein.

Für diese Analyse wurde das \ac{FE}-Modell gewählt, da es eine robustere Schätzung der kausalen Effekte 
von ICT-Investitionen auf die Arbeitslosigkeit ermöglicht. Dies ist besonders relevant, da die 
Untersuchung auf Veränderungen innerhalb eines Landes über die Zeit fokussiert und länderspezifische 
Eigenschaften nicht als erklärende Variablen modelliert werden.

Darüber hinaus wird die Analyse durch Interaktionseffekte ergänzt, die es ermöglichen, institutionelle 
Unterschiede zwischen den Ländern zu berücksichtigen. Die Modelle beinhalten Dummy-Variablen für die 
Wohlfahrtsstaatentypen, um systematische Unterschiede zwischen den Regimetypen in ihrer Reaktion auf 
\ac{ICT}-Investitionen zu identifizieren \parencite{espingandersen1990thethree}. Zusätzlich werden 
Jahres-Faktor-Dummies integriert, um allgemeine makroökonomische Entwicklungen (z. B. Finanzkrisen 
oder technologische Schübe) aus den Modellen zu kontrollieren.

Durch diese Kombination aus \ac{FE}-Modellen, Interaktionseffekten und Zeitdummies bietet die 
Paneldatenanalyse eine solide Grundlage für die Untersuchung der Auswirkungen von \ac{ICT}-Investitionen 
auf die Arbeitslosigkeit. Die longitudinale Struktur der Daten erlaubt eine differenzierte Betrachtung, 
die sowohl kurzfristige als auch langfristige Beschäftigungseffekte berücksichtigt.
