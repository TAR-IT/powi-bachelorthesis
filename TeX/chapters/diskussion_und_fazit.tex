% TODO: - (Stand 04.03.2025)

\section{Diskussion und Fazit}

Die Ergebnisse dieser Arbeit bieten wertvolle Einblicke in die Beziehung zwischen 
\ac{ICT}-Investitionen und der Arbeitslosenquote in verschiedenen Bildungsgruppen. Sie 
zeigen signifikante Zusammenhänge und verdeutlichen die Rolle institutioneller 
Rahmenbedingungen für die Beschäftigungswirkungen der Digitalisierung. Die Untersuchung 
trägt zur wissenschaftlichen Debatte über die Wechselwirkungen zwischen technologischer 
Entwicklung, Arbeitsmarktstrukturen und politischen Institutionen bei und liefert 
praktische Implikationen für Politik, Unternehmen und Bildungssysteme.

\subsection{Zentrale Ergebnisse der Analyse}

Die Analyse zeigt, dass \ac{ICT}-Investitionen in den Basis-Modellen ohne Interaktion 
über alle Bildungsgruppen hinweg einen signifikant positiven Effekt auf die Arbeitslosenquote 
haben. Dies bedeutet, dass höhere \ac{ICT}-Investitionen in diesen Modellen mit höheren 
Arbeitslosenquoten korrelieren. Besonders stark ausgeprägt ist dieser Effekt für 
gering- und mittelqualifizierte Personen, bleibt aber auch für hochqualifizierte 
Arbeitskräfte signifikant. Dies widerspricht der weit verbreiteten Annahme, dass 
Digitalisierung primär positive Effekte für Hochqualifizierte hat, während gering 
Qualifizierte besonders negativ betroffen sind. 

Allerdings zeigen die Interaktionsmodelle mit Wohlfahrtsstaaten, dass dieser Effekt nicht 
universell ist, sondern stark von institutionellen Rahmenbedingungen abhängt. Während in 
liberalen Wohlfahrtsstaaten (Referenzkategorie) der Haupteffekt positiv bleibt und höhere 
\ac{ICT}-Investitionen mit steigender Arbeitslosigkeit verbunden sind, zeigt sich in anderen 
Wohlfahrtsstaatentypen ein differenziertes Bild. In postsozialistischen Wohlfahrtsstaaten 
sind die Interaktionseffekte negativ und signifikant. Dies bedeutet, dass in diesen Ländern 
\ac{ICT}-Investitionen nicht zu einer höheren Arbeitslosigkeit führen - im Vergleich zu 
liberalen Staaten können diese Investitionen hier also besser in den Arbeitsmarkt integriert 
werden. Mitteleuropäische Wohlfahrtsstaaten zeigen keine signifikanten Unterschiede zu den 
liberalen Staaten, was darauf hindeutet, dass sich die Digitalisierungseffekte in diesen Ländern 
nicht signifikant von den Effekten in liberalen Märkten unterscheiden. In südeuropäischen 
Wohlfahrtsstaaten zeigen sich für Mittel- und Hochqualifizierte negative Interaktionseffekte. 
Dies bedeutet, dass diese Gruppen dort weniger von steigender Arbeitslosigkeit durch Digitalisierung 
betroffen sind als in liberalen Staaten. Nordische Wohlfahrtsstaaten zeigen für Hochqualifizierte 
positive Interaktionseffekte, was darauf hindeutet, dass in diesen Ländern hochqualifizierte 
Arbeitskräfte möglicherweise stärker von der Digitalisierung betroffen sind als in anderen 
Wohlfahrtsstaatentypen.

Diese Ergebnisse verdeutlichen, dass die Effekte von \ac{ICT}-Investitionen keineswegs linear oder 
eindeutig negativ sind, sondern stark von den institutionellen und wirtschaftlichen Rahmenbedingungen 
abhängen. Während in liberalen Wohlfahrtsstaaten der Anstieg der \ac{ICT}-Investitionen mit steigender 
Arbeitslosigkeit in allen Bildungsgruppen verbunden ist, scheinen postsozialistische und südeuropäische 
Länder in der Lage zu sein, die negativen Beschäftigungseffekte abzumildern.

Die Kontrollvariablen zeigen ebenfalls einige wichtige Zusammenhänge. Das \ac{BIP} pro Kopf 
hat in allen Bildungsgruppen einen signifikant negativen Einfluss auf die Arbeitslosenquote. 
Dies bestätigt, dass wirtschaftlich stärkere Länder tendenziell geringere Arbeitslosenquoten 
aufweisen. Die Regulierungsstrenge des Arbeitsmarkts wirkt sich für gering Qualifizierte eher 
negativ aus (höhere Regulierung erhöht die Arbeitslosenquote), während sie für Hochqualifizierte 
eher stabilisierend wirkt. Der tertiäre Bildungsanteil zeigt in allen Gruppen einen signifikant 
negativen Einfluss auf die Arbeitslosenquote, was darauf hindeutet, dass eine höhere 
Bildungsbeteiligung langfristig dazu beiträgt, negative Beschäftigungseffekte der Digitalisierung 
abzumildern.

\subsection{Einordnung der Ergebnisse in den theoretischen Kontext}

Die Theorie des \ac{SBTC} besagt, dass technologische Innovationen die Nachfrage nach 
hochqualifizierten Arbeitskräften steigern, während gering Qualifizierte durch 
Automatisierung verdrängt werden \parencite[vgl.][S. 7]{acemoglu2002technical}. Die Ergebnisse 
dieser Arbeit bestätigen diese Annahme jedoch nur teilweise. Während erwartet wurde, dass 
\ac{ICT}-Investitionen primär die Arbeitslosenquote von gering Qualifizierten erhöhen, 
zeigen die Modelle einen durchweg signifikanten positiven Effekt für alle Bildungsgruppen in 
den Basis-Modellen. Das bedeutet, dass die negativen Arbeitsmarkteffekte der Digitalisierung 
nicht nur auf niedrigqualifizierte Arbeitnehmer beschränkt sind, sondern auch mittel- und 
hochqualifizierte Arbeitskräfte betreffen können.

Die Interaktionseffekte legen jedoch nahe, dass diese Effekte durch institutionelle Faktoren 
modifiziert werden. Die Beobachtung, dass in postsozialistischen und südeuropäischen Ländern 
die negativen Arbeitsmarkteffekte abgemildert werden, spricht dafür, dass arbeitsmarktpolitische 
Maßnahmen und Anpassungsprozesse eine entscheidende Rolle spielen. Dies könnte im Einklang mit 
der Theorie der "schöpferischen Zerstörung" nach Schumpeter stehen, die besagt, dass wirtschaftlicher 
Wandel und technologische Disruptionen langfristig zu Wachstum führen, kurzfristig jedoch strukturelle 
Umwälzungen verursachen \parencite[vgl.][S. 103-105]{schumpeter1976capitalism}.

Eine alternative Erklärung könnte die Verzögerungseffekte der Digitalisierung sein. Während in 
hoch digitalisierten liberalen Staaten wie den USA oder Großbritannien die Arbeitsmärkte bereits 
stark von technologischen Umstellungen betroffen sind, könnten in postsozialistischen Staaten 
die Digitalisierungseffekte erst mit zeitlicher Verzögerung eintreten. Dies könnte erklären, warum 
die negativen Beschäftigungseffekte dort (noch) nicht sichtbar sind.

\subsection{Limitationen und zukünftige Forschung}

Trotz der wertvollen Erkenntnisse dieser Untersuchung sind einige Limitationen zu 
berücksichtigen. Erstens basiert die Analyse auf aggregierten \ac{OECD}-Daten für den 
Zeitraum 2005–2022, wodurch Unterschiede in der Erhebungsmethodik zwischen den 
Ländern die Ergebnisse potenziell verzerren könnten. Zweitens liegt der Fokus der 
Untersuchung auf makroökonomischen Zusammenhängen. Individuelle Anpassungsstrategien 
von Arbeitnehmer*innen oder Unternehmen an die Digitalisierung konnten in dieser 
Analyse nicht berücksichtigt werden. Künftige Studien sollten daher verstärkt auf 
Umfragedaten oder firmenspezifische Datenquellen zurückgreifen, um differenziertere 
Erkenntnisse zu gewinnen.

Darüber hinaus besteht die Möglichkeit einer umgekehrten Kausalität zwischen \ac{ICT}-Investitionen 
und Arbeitslosigkeit. Während in dieser Analyse angenommen wurde, dass \ac{ICT}-Investitionen 
die Arbeitslosenquote beeinflussen, könnte es auch sein, dass hohe Arbeitslosigkeit 
Regierungen oder Unternehmen zu verstärkten Investitionen in Digitalisierung veranlasst. 
Diese Hypothese könnte mit Methoden wie Granger-Kausalitätstests oder Instrumentalvariablen 
weiter geprüft werden.

\subsection{Gesamtfazit}

Zusammenfassend zeigen die Ergebnisse, dass \ac{ICT}-Investitionen keine universelle 
Lösung für Arbeitsmarktprobleme darstellen, sondern dass ihre Wirkungen maßgeblich von 
institutionellen Rahmenbedingungen abhängen. Während in Ländern mit gut entwickelten 
Arbeitsmarkt- und Bildungssystemen die negativen Effekte begrenzt sind, weisen insbesondere 
südeuropäische und postsozialistische Wohlfahrtsstaaten signifikante Herausforderungen auf.
Die Ergebnisse legen nahe, dass Digitalisierung auch in wissensintensiven Sektoren Arbeitsplätze 
substituieren kann. Gleichzeitig deuten die Interaktionseffekte darauf hin, dass 
institutionelle Anpassungsmechanismen wie Weiterbildungsprogramme oder soziale Sicherungssysteme 
eine zentrale Rolle bei der Milderung der Digitalisierungseffekte spielen. Eine erfolgreiche 
digitale Transformation erfordert daher nicht nur technologische Investitionen, sondern auch 
begleitende arbeitsmarktpolitische, wirtschaftliche und bildungspolitische Maßnahmen.
