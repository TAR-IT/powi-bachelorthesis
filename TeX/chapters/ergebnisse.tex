% TODO: !

\section{Ergebnisse}

%%%%%%%%%%%%%%%%%%%%%%%%%%
% Deskriptive Ergebnisse %
%%%%%%%%%%%%%%%%%%%%%%%%%%

\subsection{Deskriptive Ergebnisse}

% Deskriptive Statistik: Einleitung
Die deskriptiven Statistiken der analysierten Variablen bieten einen umfassenden Einblick 
in deren Eigenschaften und Verteilungen über die beobachteten Länder und Zeiträume. Im 
Folgenden werden die Ergebnisse detailliert beschrieben:

\input{assets/variables_fin.tex}

% Deskriptive Statistik: Variable "ICT-Investitionen"
Die Variable \textit{\ac{ICT}-Investitionen}, welche den Anteil der Investitionen in 
\ac{ICT} am \ac{BIP} misst \parencite{oecd2022ict}, variiert zwischen einem Minimum von 
0,73\% und einem Maximum von 8,69\%. Der Mittelwert beträgt 2,46\%, während der Median mit 
2,25\% leicht darunter liegt. Dies deutet auf eine leicht rechtsschiefe Verteilung hin, 
da einige Länder besonders hohe Investitionen in \ac{ICT} tätigen. Die Standardabweichung 
von 0,98 zeigt, dass es zwischen den OECD-Ländern erhebliche Unterschiede in der Intensität 
der \ac{ICT}-Investitionen gibt. Während einige Länder konstant hohe Anteile ihrer 
wirtschaftlichen Ressourcen in digitale Technologien investieren, gibt es andere, die 
vergleichsweise geringe Investitionen tätigen. Diese Unterschiede können durch verschiedene 
Faktoren beeinflusst sein, darunter wirtschaftliche Leistungsfähigkeit, politische 
Strategien zur Förderung der Digitalisierung sowie strukturelle Unterschiede in der 
Entwicklung des \ac{ICT}-Sektors.

% Deskriptive Statistik: Variable "Arbeitslosenquote"
Die Variable \textit{Arbeitslosenquote}, welche die Arbeitslosenquote in Prozent angibt 
\parencite{oecd2022unemployment}, schwankt erheblich zwischen einem Minimum von 0,82\% und 
einem Maximum von 49,89\%. Der Mittelwert liegt bei 7,95\%, während der Median mit 5,96\% 
etwas niedriger ausfällt. Dies weist auf eine rechtsschiefe Verteilung hin, da einige 
Länder oder Zeitpunkte mit sehr hohen Arbeitslosenquoten als Ausreißer wirken können. Die 
hohe Standardabweichung von 6,34 deutet darauf hin, dass die Arbeitslosenquoten 
zwischen den Ländern und über die Zeit hinweg erhebliche Unterschiede aufweisen. Während 
einige OECD-Länder durch eine geringe Arbeitslosenquote und stabile Arbeitsmärkte 
gekennzeichnet sind, zeigen andere Länder insbesondere in wirtschaftlichen Krisenzeiten 
oder strukturschwachen Regionen signifikant höhere Arbeitslosenraten. Diese Heterogenität 
könnte zudem mit unterschiedlichen Arbeitsmarktregulierungen und Bildungssystemen 
zusammenhängen.

% Deskriptive Statistik: Variable "BIP pro Kopf"
Das \textit{\ac{BIP} pro Kopf}, welches das Pro-Kopf-Einkommen in Tausend US-Dollar angibt 
\parencite{oecd2022gdp}, weist eine erhebliche Spannweite auf: Es reicht von 13,34 bis 
137,72 Tausend US-Dollar. Der Mittelwert beträgt 43,73 Tausend US-Dollar, während der 
Median mit 41,27 Tausend US-Dollar nur geringfügig darunter liegt. Trotz dieser relativen 
Nähe deutet die hohe Standardabweichung von 17,13 darauf hin, dass es erhebliche 
Wohlstandsunterschiede zwischen den \ac{OECD}-Ländern gibt. Dies spricht für eine starke 
rechtsschiefe Verteilung, da einige besonders wohlhabende Länder den Durchschnittswert 
nach oben treiben.

Diese Unterschiede sind insbesondere für die Interpretation der \ac{ICT}-Investitionen 
relevant, da wohlhabendere Länder tendenziell eine höhere Kapitalausstattung und damit 
größere Investitionsmöglichkeiten in digitale Infrastruktur haben könnten. Gleichzeitig 
können Unterschiede im \ac{BIP} pro Kopf Einfluss auf die Struktur des Arbeitsmarktes und 
damit auf die Verteilung der Arbeitslosigkeit nach Bildungsgrad haben.

% Deskriptive Statistik: Variable "Gewerkschaftsdichte"
Die Variable \textit{Gewerkschaftsdichte}, welche die gewerkschaftliche Organisationsrate 
eines Landes misst \parencite{oecd2022tud}, zeigt eine erhebliche Varianz zwischen den 
Ländern. Die Werte reichen von einem Minimum von 4,50\% bis zu einem Maximum von 92,20\%. 
Der Mittelwert beträgt 28,45\%, während der Median mit 20,40\% darunter liegt, was darauf 
hindeutet, dass einige Länder eine besonders hohe Gewerkschaftsbindung haben, während die 
Mehrheit unter diesem Durchschnittswert bleibt. Die Standardabweichung von 20,71 
verdeutlicht die große Heterogenität in der gewerkschaftlichen Organisation zwischen den 
Ländern. Während einige nordische Länder traditionell hohe Gewerkschaftsdichten aufweisen, 
sind Gewerkschaften in anderen \ac{OECD}-Staaten weniger stark in den Arbeitsmarkt 
integriert. Dies könnte Implikationen für die Verhandlungsmacht von Arbeitnehmern haben, 
was sich wiederum auf Lohnstrukturen und Beschäftigungssicherheit auswirken kann.  
Zur Sicherstellung einer vollständigen Zeitreihe wurden fehlende Werte dieser Variable mittels linearer Interpolation ergänzt.

% Deskriptive Statistik: Variable "Arbeitsmarktregulierung"
Die Variable \textit{Regulierungsstrenge des Arbeitsmarkts} (REGULATION\_STRICTNESS) misst, 
wie stark der Arbeitsmarkt eines Landes reguliert ist, insbesondere im Hinblick auf 
Kündigungsschutz und Beschäftigungsflexibilität \parencite{oecd2022regulation}. Die Werte 
reichen von 0,00 bis 4,88, mit einem Mittelwert von 2,19 und einer Standardabweichung von 
0,83. Diese Unterschiede spiegeln unterschiedliche Arbeitsmarktpolitiken wider: Während in 
einigen Ländern hohe Regulierung den Kündigungsschutz stärkt, kann dies gleichzeitig die 
Schaffung neuer Arbeitsplätze hemmen.

% Deskriptive Statistik: Variable "Anteil tertiär Gebildeter"
Die Variable \textit{Anteil tertiär gebildeter Personen} (PERCENT\_TERTIARY\_EDUCATION) gibt 
den Prozentsatz der Bevölkerung an, der einen tertiären Bildungsabschluss besitzt 
\parencite{oecd2022education}. Der Wert variiert zwischen 12,22\% und 61,99\%, mit einem 
Mittelwert von 33,69\% und einer Standardabweichung von 9,29. Länder mit höheren Werten 
verfügen tendenziell über eine stärker wissensbasierte Wirtschaft, was sich positiv auf die 
Integration von Arbeitnehmern in technologische Sektoren auswirken kann. Gleichzeitig könnte 
eine höhere Bildungsbeteiligung dazu beitragen, die negativen Effekte der Digitalisierung für 
geringqualifizierte Arbeitskräfte abzufedern.

\input{assets/variables_welfare_fin.tex}

% Deskriptive Statistik: Variable "Wohlfahrtsstaatenmodell"
Die Variable \textit{Wohlfahrtsstaatentyp} klassifiziert die OECD-Länder nach ihrem 
Wohlfahrtsstaatsmodell in fünf Kategorien: angelsächsisch, nordisch, mitteleuropäisch, 
südeuropäisch und postsozialistisch. Diese Einteilung basiert auf institutionellen und 
wirtschaftlichen Merkmalen, die sich auf Arbeitsmarktstrukturen, soziale Sicherungssysteme 
und Bildungszugänge auswirken \parencite[S. 56]{espingandersen1990thethree}.

Die Verteilung der Wohlfahrtsstaatentypen zeigt, dass postsozialistische Länder mit 
25,30\% die größte Gruppe innerhalb der Stichprobe ausmachen, gefolgt von 
mitteleuropäischen Wohlfahrtsstaaten mit 24,69\%. Nordische und angelsächsische 
Länder sind mit 17,07\% bzw. 19,98\% ebenfalls vertreten, während südeuropäische 
Länder mit 12,96\% den kleinsten Anteil ausmachen. Die Kategorie \textit{Other} ist 
in der vorliegenden Stichprobe nicht besetzt.

Diese Klassifikation ist insbesondere für die spätere Analyse der Interaktionseffekte 
relevant, da sie Aufschluss darüber geben kann, inwiefern institutionelle 
Rahmenbedingungen den Einfluss von \textit{\ac{ICT}-Investitionen} auf die Arbeitslosenquote 
moderieren. Die Ergebnisse der multivariaten Analysen zeigen, dass die Effekte von 
\textit{\ac{ICT}-Investitionen} auf die Arbeitslosigkeit stark von der Wohlfahrtsstaatsstruktur 
eines Landes abhängen.  Besonders in postsozialistischen und südeuropäischen Wohlfahrtsstaaten 
sind negative Interaktionseffekte zu beobachten, während sich in nordischen Ländern ein 
ausgeglicheneres Muster zeigt.

% Deskriptive Statistik: Variablen (Zusammenfassung)
Die deskriptiven Statistiken zeigen, dass die betrachteten Variablen eine erhebliche 
Heterogenität aufweisen, die sowohl auf länderspezifische Unterschiede als auch auf 
strukturelle und wirtschaftliche Faktoren zurückgeführt werden kann. Besonders auffällig 
sind die Unterschiede in den \ac{ICT}-Investitionen, die je nach wirtschaftlicher 
Leistungsfähigkeit und politischen Rahmenbedingungen stark variieren. Auch die 
Arbeitslosenquoten zeigen eine hohe Streuung, die möglicherweise mit den unterschiedlichen 
Bildungsniveaus, Arbeitsmarktinstitutionen und Wirtschaftsentwicklungen der Länder 
zusammenhängt. Die hohe Varianz im \ac{BIP} pro Kopf unterstreicht die unterschiedlichen 
wirtschaftlichen Ausgangsbedingungen der Länder, was sich sowohl auf die Höhe der 
\ac{ICT}-Investitionen als auch auf die Struktur der Arbeitsmärkte auswirken könnte. 
Schließlich zeigt die Gewerkschaftsdichte ebenfalls starke Unterschiede zwischen den 
OECD-Ländern, was für die Analyse der institutionellen Faktoren relevant ist, die 
möglicherweise als Moderatoren der Auswirkungen von \ac{ICT}-Investitionen auf den 
Arbeitsmarkt fungieren.

% Deskriptive Analyse: Grafiken (Einleitung)
Diese deskriptive Analyse der Variablen bildet die Grundlage für die nachfolgenden 
grafischen Darstellungen, die eine detailliertere Visualisierung der Trends und Unterschiede 
zwischen den Ländern ermöglichen. Sie bietet einen ersten Einblick auf Länderebene in die 
Beziehung zwischen \textit{\ac{ICT}-Investitionen} (gemessen als Anteil am \ac{BIP}) und 
den Arbeitslosenquoten, differenziert nach den drei genannten Bildungsgruppen. Hierbei 
wird jeweils ein repräsentatives Land pro Wohlfahrtsstaatentyp für die Analyse gewählt - 
Spanien als südeuropäischer, Polen als postsozialistischer, Schweden als nordischer und 
Deutschland als mitteleuropäischer Wohlfahrtsstaat im Zeitraum von 2005 bis 2022. 

Um die Lesbarkeit der Grafiken zu verbessern, wurden die Ländernamen in den Diagrammen automatisch 
in die deutsche Sprache umbenannt. Dies stellt sicher, dass die visuelle Darstellung konsistent 
mit der Textanalyse bleibt. Ziel ist es, vor der multivariaten Analyse bereits Unterschiede und 
Trends innerhalb der Länder und zwischen den Bildungsgruppen zu identifizieren.


% Deskriptive Analyse: Grafik "Spanien"
\begin{figure}[htbp]
    \centering
    \includegraphics[width=\textwidth]{assets/plot_spain.png}
    \caption{Überblick über \textit{\ac{ICT}-Investitionen} und Arbeitslosenquote in 
    Spanien}
    \label{fig:spain}
\end{figure}

Die Abbildung zeigt die Entwicklung der \textit{\ac{ICT}-Investitionen} als Anteil am 
BIP sowie die Arbeitslosenquote in Spanien zwischen 2005 und 2022, differenziert nach 
Bildungsniveau - Spanien steht hier repräsentativ für südeuropäische Wohlfahrtsstaaten. 
Am Beispiel Spaniens ist ein besonders markanter Anstieg der Arbeitslosenquote während 
der Finanz- und Wirtschaftskrise von 2008 bis 2013 zu beobachten. Während die 
\textit{\ac{ICT}-Investitionen} einen insgesamt moderaten Anstieg über den gesamten 
Zeitraum hinweg zeigen, lassen sich drastische Schwankungen in der Arbeitslosenquote 
identifizieren, insbesondere bei Personen mit niedrigem und mittlerem Bildungsniveau.

Bei Personen mit einem niedrigen Bildungsniveau zeigt sich zwischen 2005 und 2008 
eine relativ stabile Arbeitslosenquote von knapp unter 10\%. Ab 2008 kam es jedoch zu 
einem rasanten Anstieg, der bis 2013 einen Höchststand von über 30\% erreichte. Erst 
nach 2013 begann ein kontinuierlicher Rückgang, der sich bis 2020 auf etwa 20\% 
fortsetzte, bevor ein erneuter leichter Anstieg zu beobachten ist. Die 
\textit{\ac{ICT}-Investitionen} entwickelten sich hingegen gleichmäßiger. Sie begannen 
auf einem niedrigen Niveau von etwa 1,75\% des BIP, zeigten nach der Finanzkriese ab 
2010 eine Aufwärtstendenz und stabilisierten sich nach 2015 bei etwa 2,5\%. Der 
Rückgang der Arbeitslosenquote nach 2013 verlief jedoch unabhängig von einer abrupten 
Zunahme der \textit{\ac{ICT}-Investitionen}, was darauf hindeutet, dass 
makroökonomische Faktoren (z. B. wirtschaftliche Erholung, Beschäftigungsprogramme) 
für die Senkung der Arbeitslosigkeit eine zentrale Rolle spielten.

Bei Personen mit mittlerem Bildungsniveau zeigt sich ein sehr ähnlicher Verlauf. Die 
Arbeitslosenquote lag 2005 noch unter 8\%, stieg im Zuge der Wirtschaftskrise bis 2013 
jedoch auf über 20\% an. Erst ab 2014 begann ein deutlicher Rückgang, der sich bis 
2020 auf etwa 10\% fortsetzte. Die \textit{\ac{ICT}-Investitionen} folgten hier einem 
vergleichbaren Muster wie in der Gruppe der gering Qualifizierten, wobei ein leichter, 
aber kontinuierlicher Anstieg sichtbar ist. Dennoch ist keine direkte Korrelation 
zwischen dem Verlauf der \textit{\ac{ICT}-Investitionen} und der Arbeitslosenquote 
ersichtlich, da der massive Anstieg und der spätere Rückgang der Arbeitslosigkeit 
primär durch die wirtschaftliche Entwicklung und nicht durch technologische 
Investitionen bedingt zu sein scheinen.

Bei Personen mit hohem Bildungsniveau war die Arbeitslosenquote insgesamt niedriger, 
zeigte jedoch ebenfalls einen deutlichen Anstieg während der Wirtschaftskrise. Im Jahr 
2005 lag sie unter 5\%, erreichte 2013 jedoch fast 15\%. Danach setzte auch hier ein 
Rückgang ein, und bis 2020 fiel die Quote auf etwa 5\% zurück. Im Gegensatz zu den 
anderen Bildungsgruppen scheinen sich hier die \textit{\ac{ICT}-Investitionen} und 
die Arbeitslosenquote teilweise gegenläufig zu entwickeln. Während die 
\textit{\ac{ICT}-Investitionen} nach 2010 eine stetige Steigerung zeigen und nach 
2015 stabil auf etwa 2,5\% des BIP bleiben, geht die Arbeitslosenquote in derselben 
Phase zurück. Dies könnte darauf hindeuten, dass hochqualifizierte Arbeitskräfte in 
Spanien stärker von der Digitalisierung profitieren konnten als Personen mit 
niedrigerem Bildungsstand.

Spanien als südeuropäischer Wohlfahrtsstaat ist durch einen stark segmentierten 
Arbeitsmarkt gekennzeichnet, der sich durch hohe Anteile an befristeten 
Beschäftigungsverhältnissen sowie eine geringere Arbeitsplatzsicherheit auszeichnet. 
Dies könnte eine Erklärung für die starken Schwankungen der Arbeitslosenquote 
im Zuge der Finanzkrise sein, da insbesondere gering und mittel Qualifizierte 
von Entlassungen betroffen waren. Die \textit{\ac{ICT}-Investitionen} scheinen 
langfristig zwar leicht anzusteigen, doch zeigt sich kein direkter Zusammenhang 
zwischen diesen Investitionen und der Arbeitslosenquote in den jeweiligen 
Bildungsgruppen. Vielmehr deutet die Entwicklung darauf hin, dass der Arbeitsmarkt 
in Spanien stark konjunkturabhängig ist und die wirtschaftliche Erholung nach 2013 
die wichtigste Triebkraft für die Reduktion der Arbeitslosigkeit war.

Zusammenfassend zeigen die Daten für Spanien eine enge Verbindung zwischen der 
Finanzkrise und den massiven Schwankungen der Arbeitslosenquote, insbesondere bei 
gering und mittel Qualifizierten. Während \textit{\ac{ICT}-Investitionen} über den 
Zeitraum hinweg einen kontinuierlichen, aber moderaten Anstieg zeigen, sind ihre 
direkten Auswirkungen auf die Arbeitslosigkeit unklar. Es könnte jedoch sein, dass 
insbesondere Hochqualifizierte von den steigenden \textit{\ac{ICT}-Investitionen} 
profitieren konnten, während gering Qualifizierte eher von konjunkturellen 
Faktoren abhängig waren.

% Deskriptive Analyse: Grafik "Polen"
\begin{figure}[htbp]
    \centering
    \includegraphics[width=\textwidth]{assets/plot_poland.png}
    \caption{Überblick über \textit{\ac{ICT}-Investitionen} und Arbeitslosenquote 
    in Polen}
    \label{fig:poland}
\end{figure}

Die Abbildung zeigt die Entwicklung der \textit{\ac{ICT}-Investitionen} als Anteil 
am BIP sowie die Arbeitslosenquote in Polen zwischen 2005 und 2022 differenziert 
nach Bildungsniveau - Polen steht hier repräsentativ für postsozialistische 
Wohlfahrtsstaaten.

Auffällig ist der durchgängige Rückgang der Arbeitslosenquote in allen 
Bildungsgruppen, während die \textit{\ac{ICT}-Investitionen} über weite Strecken 
konstant bleiben, beziehungsweise sogar ebenfalls einen Rückgang verzeichnen. Dies 
deutet darauf hin, dass makroökonomische oder arbeitsmarktpolitische Faktoren für 
den Rückgang der Arbeitslosigkeit maßgeblich verantwortlich sein könnten.

Bei Personen mit einem niedrigen Bildungsniveau lag die Arbeitslosenquote im Jahr 
2005 bei knapp 28\%. In den darauffolgenden Jahren kam es zu einem raschen Rückgang, 
wobei jedoch zwischen 2010 und 2015 eine Stagnation mit einem kurzen Anstieg auf fast 
20\% zu beobachten ist. Nach 2015 setzte sich der Rückgang der Arbeitslosenquote 
fort, sodass sie bis 2020 auf 8\% fiel. Die \textit{\ac{ICT}-Investitionen} blieben 
über den gesamten Zeitraum hinweg weitgehend konstant und bewegten sich um die 1\% des 
BIP, mit einem leichten Rückgang zwischen 2010 und 2015. Dies deutet darauf hin, dass  
der starke Rückgang der Arbeitslosigkeit nicht direkt mit den 
\textit{\ac{ICT}-Investitionen} zusammenhängt, sondern durch andere wirtschaftliche 
Faktoren beeinflusst wurde, beispielsweise durch eine allgemeine wirtschaftliche 
Stabilisierung nach dem EU-Beitritt Polens und steigende Beschäftigungsmöglichkeiten 
in arbeitsintensiven Branchen.

Für Personen mit einem mittleren Bildungsniveau zeigt sich ein ähnliches Muster, wenn 
auch auf einem insgesamt niedrigeren Ausgangsniveau der Arbeitslosenquote. Während 
diese 2005 noch über 10\% lag, sank sie in den darauffolgenden Jahren rasch auf etwa 
3\% bis 2015 und weiter unter 2\% bis 2020. Zwischen 2010 und 2015 ist jedoch eine 
leichte Erhöhung der Arbeitslosenquote erkennbar, bevor der Trend weiter nach unten 
verlief. Der Rückgang der Arbeitslosigkeit erfolgt weitgehend unabhängig von der 
Entwicklung der \textit{\ac{ICT}-Investitionen}, was darauf hindeutet, dass 
makroökonomische Faktoren wie die Industrialisierung und eine steigende Nachfrage 
nach Arbeitskräften mit mittlerer Qualifikation eine bedeutendere Rolle gespielt 
haben könnten.

Für Personen mit einem hohen Bildungsniveau war die Arbeitslosenquote bereits 
2005 relativ niedrig, lag aber dennoch bei etwa 6\%, was im Vergleich zu anderen 
europäischen Ländern eher hoch ist. Dies könnte auf strukturelle Faktoren des 
polnischen Arbeitsmarktes zurückzuführen sein, wie eine geringere Anzahl 
hochqualifizierter Beschäftigungsmöglichkeiten in den frühen 2000er-Jahren. In den 
darauffolgenden Jahren fiel die Arbeitslosenquote jedoch deutlich und lag bereits 
2015 unter 2\%. Auffällig ist, dass die \textit{\ac{ICT}-Investitionen} in dieser 
Gruppe im Gegensatz zu den anderen Bildungsgruppen eine leichte Steigerung zeigen. 
In der ersten Hälfte des Beobachtungszeitraums bewegten sich die 
\textit{\ac{ICT}-Investitionen} um 1,2\% des BIP, während sie in den Jahren nach 
2015 tendenziell anstiegen. Dies könnte darauf hindeuten, dass der polnische 
Arbeitsmarkt mit steigendem ICT-Investitionsanteil zunehmend hochqualifizierte 
Beschäftigungsmöglichkeiten geschaffen hat. Dennoch bleibt die Kausalität unklar, 
da die Arbeitslosenquote in dieser Gruppe bereits gefallen war, bevor der leichte 
Anstieg der \textit{\ac{ICT}-Investitionen} einsetzte.

Polen als postsozialistischer Wohlfahrtsstaat hat in den letzten Jahrzehnten einen 
tiefgreifenden wirtschaftlichen Wandel durchlaufen. Der EU-Beitritt im Jahr 2004 
führte zu verstärkten ausländischen Direktinvestitionen, einer zunehmenden 
Integration in europäische Produktionsnetzwerke sowie einer generellen 
Modernisierung der Wirtschaft. Diese Entwicklungen spiegeln sich auch in der 
Reduktion der Arbeitslosigkeit wider, die in allen Bildungsgruppen signifikant 
gesunken ist. Besonders bei Personen mit mittlerem und niedrigem Bildungsniveau 
könnte die Expansion von Industriejobs sowie der Dienstleistungssektor eine 
wesentliche Rolle gespielt haben. % TODO: Zitieren!

Insgesamt zeigt die Abbildung, dass die Arbeitslosenquote in allen Bildungsgruppen 
stark gesunken ist, während die \textit{\ac{ICT}-Investitionen} nur moderate 
Schwankungen aufweisen. Dies deutet darauf hin, dass die Haupttreiber der 
Beschäftigungsentwicklung in Polen eher in wirtschaftlichen und 
arbeitsmarktpolitischen Veränderungen zu suchen sind als in den direkten 
Auswirkungen von \textit{\ac{ICT}-Investitionen}. Dennoch könnte die leichte 
Zunahme der \textit{\ac{ICT}-Investitionen} im späteren Beobachtungszeitraum 
darauf hinweisen, dass sich der polnische Arbeitsmarkt allmählich in Richtung 
einer wissensbasierten Wirtschaft entwickelt, in der besonders Hochqualifizierte 
profitieren.

% Deskriptive Analyse: Grafik "Schweden"
\begin{figure}[htbp]
    \centering
    \includegraphics[width=\textwidth]{assets/plot_sweden.png}
    \caption{Überblick über \textit{\ac{ICT}-Investitionen} und Arbeitslosenquote in 
    Schweden}
    \label{fig:sweden}
\end{figure}

Die Abbildung zeigt die Entwicklung der \textit{\ac{ICT}-Investitionen} als Anteil 
am BIP sowie die Arbeitslosenquote in Schweden zwischen 2005 und 2022, differenziert 
nach Bildungsniveau - Schweden steht hier repräsentativ für nordische 
Wohlfahrtsstaaten. Im Gegensatz zu anderen Ländern ist hier eine relativ stabile 
Entwicklung der Arbeitslosenquote über den gesamten Zeitraum zu beobachten, mit nur 
moderaten Schwankungen. Auffällig ist zudem, dass die \textit{\ac{ICT}-Investitionen} 
in Schweden im internationalen Vergleich auf einem vergleichsweise hohen Niveau liegen. 
Während sie in der ersten Dekade leichte Schwankungen zeigen, bleibt ihr Niveau ab 
2010 weitgehend konstant und steigt gegen Ende des Betrachtungszeitraums leicht an.

Bei Personen mit einem niedrigen Bildungsniveau lag die Arbeitslosenquote 2005 bei knapp 
5\% und zeigte bis etwa 2010 einen moderaten Anstieg. Nach 2010 stabilisierte sich die 
Arbeitslosenquote zunächst, bevor sie ab 2015 einen erneuten Aufwärtstrend verzeichnete. 
Besonders auffällig ist der deutliche Anstieg nach 2018, der sich bis 2022 fortsetzt. 
Während die Arbeitslosenquote für gering Qualifizierte also in den letzten Jahren 
gestiegen ist, sind die \textit{\ac{ICT}-Investitionen} im selben Zeitraum weitgehend 
stabil geblieben, wenn auch mit einer leicht positiven Tendenz. Dies könnte darauf 
hindeuten, dass die fortschreitende Digitalisierung möglicherweise die 
Beschäftigungsmöglichkeiten für niedrig qualifizierte Arbeitskräfte verschlechtert hat, 
indem sie bestimmte Arbeitsplätze verdrängte oder die Anforderungen an digitale 
Kompetenzen erhöhte - wahrscheinlich hängt diese Beobachtung aber eher mit der 
Corona-Pandemie zusammen.

Für Personen mit einem mittleren Bildungsniveau zeigt sich ein stabiles Muster, mit 
einer weitgehend konstanten Arbeitslosenquote zwischen 2005 und 2018. Während die 
Arbeitslosigkeit 2005 bei unter 5\% lag, gab es bis 2015 eine leichte Abwärtsbewegung, 
gefolgt von einer weitgehenden Stabilisierung. Nach 2018 zeigt sich eine leicht steigende 
Tendenz der Arbeitslosenquote, wenn auch weniger ausgeprägt als bei den gering 
Qualifizierten. Die \textit{\ac{ICT}-Investitionen} sind in dieser Gruppe durchgängig 
hoch und zeigen eine stabile Entwicklung mit leichten Schwankungen. Anders als bei den 
gering Qualifizierten ist hier keine klare gegenläufige Entwicklung zwischen 
\textit{\ac{ICT}-Investitionen} und Arbeitslosigkeit zu erkennen, was darauf hindeutet, 
dass mittlere Qualifikationen in Schweden weniger stark von den technologischen 
Veränderungen betroffen sind.

Bei Personen mit einem hohen Bildungsniveau zeigt sich über den gesamten Zeitraum 
hinweg eine extrem niedrige Arbeitslosenquote. Bereits 2005 lag sie unter 5\% und blieb 
über den gesamten Zeitraum stabil, mit nur minimalen Schwankungen. Auffällig ist, dass 
die \textit{\ac{ICT}-Investitionen} in dieser Gruppe im internationalen Vergleich sehr 
hoch sind, mit Werten, die konstant bei 4-5\% des \ac{BIP} liegen. Die Kombination aus 
hoher \ac{ICT}-Investition und niedriger Arbeitslosenquote deutet darauf hin, dass 
hochqualifizierte Arbeitskräfte in Schweden stark von der Digitalisierung profitieren 
konnten. Dies entspricht auch theoretischen Erwartungen, da hochqualifizierte 
Beschäftigte in wissensintensiven Branchen tätig sind, die von technologischen 
Innovationen profitieren.

Schweden als nordischer Wohlfahrtsstaat zeichnet sich durch ein stark reguliertes, 
aber flexibles Arbeitsmarktsystem aus, das durch hohe Sozialleistungen, eine starke 
Gewerkschaftsbindung und ein gut ausgebautes Bildungssystem geprägt ist. Die stabilen 
\textit{\ac{ICT}-Investitionen} und die insgesamt niedrigen Arbeitslosenquoten deuten 
darauf hin, dass der schwedische Arbeitsmarkt relativ widerstandsfähig gegenüber 
technologischen Veränderungen ist. Allerdings lässt sich bei niedrig qualifizierten 
Arbeitskräften ein Anstieg der Arbeitslosigkeit nach 2018 beobachten, der 
möglicherweise mit strukturellen Veränderungen auf dem Arbeitsmarkt zusammenhängt. 
Dies könnte darauf hindeuten, dass bestimmte Berufe durch die Digitalisierung 
zunehmend verdrängt werden oder dass sich die Anforderungen an digitale Kompetenzen 
verstärkt haben, sodass Geringqualifizierte Schwierigkeiten haben, sich an die 
veränderten Bedingungen anzupassen.

Insgesamt zeigt die Abbildung, dass Schweden ein stabiles Beschäftigungsniveau über 
den gesamten Zeitraum hinweg aufweist, wobei die \textit{\ac{ICT}-Investitionen} 
konstant hoch sind. Während Hoch- und Mittelqualifizierte weitgehend von den 
Entwicklungen profitieren konnten, scheint sich für gering Qualifizierte in den 
letzten Jahren eine Verschlechterung der Beschäftigungssituation abzuzeichnen. Dies 
könnte darauf hindeuten, dass Digitalisierung in hochentwickelten Volkswirtschaften 
wie Schweden zunehmend zu einer Polarisierung des Arbeitsmarktes führt, bei der 
Hochqualifizierte von den Investitionen profitieren, während gering Qualifizierte 
zunehmend unter Druck geraten.

% Deskriptive Analyse: Grafik "Deutschland"
\begin{figure}[htbp]
    \centering
    \includegraphics[width=\textwidth]{assets/plot_germany.png}
    \caption{Überblick über \textit{\ac{ICT}-Investitionen} und Arbeitslosenquote in 
    Deutschland}
    \label{fig:germany}
\end{figure}

Die Abbildung zeigt die Entwicklung der \textit{\ac{ICT}-Investitionen} als Anteil 
am BIP sowie die Arbeitslosenquote in Deutschland zwischen 2005 und 2022, 
differenziert nach Bildungsniveau - Deutschland steht hier repräsentativ für 
mitteleuropäische Wohlfahrtsstaaten. Dabei lassen sich klare Unterschiede zwischen 
den drei betrachteten Gruppen - niedriges, mittleres und hohes Bildungsniveau - sowohl 
hinsichtlich des Niveaus als auch der Veränderung der Arbeitslosenquoten erkennen. 
Insgesamt zeigen sich über den gesamten Zeitraum hinweg deutliche Rückgänge in der 
Arbeitslosenquote, während die \textit{\ac{ICT}-Investitionen} eine weitgehend 
stabile Entwicklung aufweisen.

Für Personen mit einem niedrigen Bildungsniveau zeigt sich eine besonders hohe 
Arbeitslosenquote zu Beginn des Beobachtungszeitraums, die 2005 bei über 18\% lag. 
In den darauffolgenden Jahren kam es zu einem kontinuierlichen Rückgang, der bis 2020 
Werte unter 5\% erreichte. Diese Entwicklung spiegelt die allgemeine Verbesserung des 
deutschen Arbeitsmarktes wider, insbesondere durch wirtschaftlichen Aufschwung und 
Reformen im Rahmen der Agenda 2010. Die \textit{\ac{ICT}-Investitionen} verzeichneten 
zwischen 2005 und 2010 zunächst einen leichten Rückgang, bevor sie sich um die 1,5\% 
des BIP stabilisierten. Ein direkter Zusammenhang zwischen 
\textit{\ac{ICT}-Investitionen} und der sinkenden Arbeitslosenquote ist nicht 
ersichtlich, da der Rückgang der Arbeitslosenquote bereits vor der leichten 
Stabilisierung der Investitionen begann.

Bei Personen mit mittlerem Bildungsniveau zeigt sich ein ähnliches Muster, wenn auch 
auf einem insgesamt niedrigeren Ausgangsniveau der Arbeitslosenquote. Während diese 
2005 noch bei etwa 10\% lag, fiel sie bis 2020 auf rund 3\% und blieb seither 
weitgehend stabil. Die \textit{\ac{ICT}-Investitionen} zeigen eine konstante 
Entwicklung mit geringen Schwankungen. Auch hier bleibt der direkte Zusammenhang 
zwischen den \textit{\ac{ICT}-Investitionen} und der Arbeitslosenquote unklar, da 
der Rückgang der Arbeitslosigkeit langfristig verläuft und nicht direkt mit den 
Investitionen korreliert.

Für Personen mit hohem Bildungsniveau zeigt sich über den gesamten Zeitraum hinweg 
eine sehr niedrige Arbeitslosenquote. Bereits 2005 lag sie unter 5\% und sank bis
 2010 auf unter 2\%, wo sie anschließend auf diesem niedrigen Niveau stabil blieb. 
 Im Vergleich zu den anderen Bildungsgruppen weist diese Gruppe somit die geringsten 
 Schwankungen auf. Die \textit{\ac{ICT}-Investitionen} zeigen auch hier eine 
 weitgehend stabile Entwicklung. Dies könnte darauf hindeuten, dass Hochqualifizierte 
 vermehrt in Berufen tätig sind, die von steigenden \textit{\ac{ICT}-Investitionen} 
 profitieren, jedoch bleibt auch hier die Kausalität unklar.

Deutschland als konservativer Wohlfahrtsstaat zeichnet sich durch eine enge Verzahnung 
von Bildungssystem und Arbeitsmarkt aus. Insbesondere das duale Ausbildungssystem 
und gezielte arbeitsmarktpolitische Maßnahmen könnten eine Rolle beim Rückgang der 
Arbeitslosenquoten in den niedrigen und mittleren Bildungsgruppen gespielt haben. 
Die \textit{\ac{ICT}-Investitionen} zeigen über den Beobachtungszeitraum hinweg keine 
drastischen Veränderungen, was darauf hindeutet, dass technologische Entwicklungen 
schrittweise in den Arbeitsmarkt integriert wurden. Besonders für Hochqualifizierte 
könnte eine steigende Nachfrage nach digitalen Fähigkeiten eine Rolle gespielt 
haben, während bei den niedrigen und mittleren Bildungsniveaus der 
Arbeitsmarktrückgang vermutlich durch andere makroökonomische Faktoren beeinflusst 
wurde.

Die Abbildung verdeutlicht insgesamt, dass die Arbeitslosenquoten in allen 
Bildungsgruppen über die Jahre hinweg gesunken sind, während die 
\textit{\ac{ICT}-Investitionen} vergleichsweise stabil geblieben sind. Dies lässt darauf 
schließen, dass der Rückgang der Arbeitslosigkeit nicht direkt durch 
\textit{\ac{ICT}-Investitionen} getrieben wurde, sondern eher mit makroökonomischen 
Entwicklungen und strukturellen Veränderungen auf dem deutschen Arbeitsmarkt 
zusammenhängt.

% Deskriptive Analyse: Grafik "Großbritannien"
\begin{figure}[htbp]
    \centering
    \includegraphics[width=\textwidth]{assets/plot_uk.png}
    \caption{Überblick über \textit{\ac{ICT}-Investitionen} und Arbeitslosenquote in 
    Großbritannien}
    \label{fig:uk}
\end{figure}

Die Abbildung zeigt die Entwicklung der \textit{\ac{ICT}-Investitionen} als Anteil am BIP 
sowie die Arbeitslosenquote in Großbritannien zwischen 2005 und 2022, differenziert nach 
Bildungsniveau - Großbritannien steht hier repräsentativ für angelsächsische 
Wohlfahrtsstaaten. Im Vergleich zu anderen Wohlfahrtsstaatentypen weist Großbritannien eine 
relativ konstante Arbeitslosenquote auf, die über den Zeitraum hinweg nur leichte 
Rückgänge zeigt. Auffällig ist, dass die \textit{\ac{ICT}-Investitionen} in Großbritannien 
zwar einen moderaten Anstieg aufweisen, sich jedoch auf einem relativ niedrigen Niveau 
bewegen.

Bei Personen mit einem niedrigen Bildungsniveau lag die Arbeitslosenquote im Jahr 2005 
bei etwa 8\% und sank bis 2020 auf unter 4\%. Anders als in Ländern mit stärker 
regulierten Arbeitsmärkten zeigt sich hier kein abrupter Rückgang, sondern eine 
schrittweise Anpassung über den Zeitraum hinweg. Gleichzeitig zeigen die 
\textit{\ac{ICT}-Investitionen} einen leicht steigenden Trend, bleiben jedoch im Bereich 
von etwa 1,5\% des BIP. Ein klarer Zusammenhang zwischen \textit{\ac{ICT}-Investitionen} 
und der Arbeitslosenquote lässt sich nicht unmittelbar erkennen, was darauf hindeuten 
könnte, dass andere arbeitsmarktpolitische oder wirtschaftliche Faktoren maßgeblicher für 
die Reduktion der Arbeitslosigkeit sind.

Bei Personen mit mittlerem Bildungsniveau zeigt sich ein ähnliches Muster. Die 
Arbeitslosenquote lag 2005 bei etwa 6\% und fiel bis 2015 auf rund 3\%, wo sie sich 
anschließend stabilisierte. Die \textit{\ac{ICT}-Investitionen} zeigen hier eine geringe 
Zunahme, bleiben jedoch weitgehend konstant im Bereich von 1,5\% bis 2\% des BIP. Auch in 
dieser Gruppe scheint der Rückgang der Arbeitslosenquote eher mit marktwirtschaftlichen 
Anpassungen als mit direkten Effekten der \textit{\ac{ICT}-Investitionen} zusammenzuhängen. 
Der relativ geringe Anstieg der Investitionen deutet darauf hin, dass die britische 
Wirtschaft zwar technologische Entwicklungen integriert, jedoch nicht in dem Ausmaß wie 
andere hochdigitalisierte Volkswirtschaften.

Für Personen mit hohem Bildungsniveau zeigt sich über den gesamten Zeitraum hinweg eine 
sehr niedrige Arbeitslosenquote. Bereits 2005 lag sie unter 3\% und blieb über den 
gesamten Zeitraum weitgehend stabil, mit nur minimalen Schwankungen. Die 
\textit{\ac{ICT}-Investitionen} zeigen auch hier eine relativ konstante Entwicklung, 
liegen jedoch ebenfalls im Bereich von 1,5\% bis 2\% des BIP. Dies deutet darauf hin, 
dass Hochqualifizierte kaum von negativen Beschäftigungseffekten durch Digitalisierung 
betroffen sind. Vielmehr könnte der flexible britische Arbeitsmarkt es dieser Gruppe 
erleichtert haben, sich an technologische Veränderungen anzupassen.

Großbritannien als anglo-sächsischer Wohlfahrtsstaat zeichnet sich durch einen 
weniger regulierten Arbeitsmarkt aus, der sich durch eine hohe Flexibilität und eine 
geringere staatliche Intervention auszeichnet. Diese Charakteristik könnte erklären, 
warum die Arbeitslosenquoten über den Zeitraum hinweg relativ stabil bleiben und 
gleichzeitig keine drastischen Veränderungen im Bereich der \textit{\ac{ICT}-Investitionen} 
feststellbar sind. Der moderate Rückgang der Arbeitslosigkeit deutet darauf hin, dass sich 
der britische Arbeitsmarkt schrittweise an Digitalisierung angepasst hat, ohne dass 
bestimmte Gruppen massiv benachteiligt wurden.

Zusammenfassend zeigt die Abbildung, dass sich die britische Arbeitslosenquote über die 
Jahre hinweg in allen Bildungsgruppen verringert hat, wenn auch nicht so drastisch wie in 
anderen Ländern. Gleichzeitig bleiben die \textit{\ac{ICT}-Investitionen} auf einem 
relativ niedrigen Niveau und zeigen keine unmittelbare Korrelation mit den Veränderungen 
der Arbeitslosenquote. Dies deutet darauf hin, dass makroökonomische Faktoren wie die 
Arbeitsmarktflexibilität und allgemeine wirtschaftliche Entwicklung eine wichtigere Rolle 
für die Beschäftigungsdynamik spielen als allein die Höhe der \textit{\ac{ICT}-Investitionen}.


%%%%%%%%%%%%%%%%%%%%%%%%%
% Multivariate Analysen %
%%%%%%%%%%%%%%%%%%%%%%%%%

\subsection{Multivariate Analysen}

% Multivariate Analyse: Einleitung
Die Zusammenfassung der Ergebnisse aus den Modellen mit Kontrollvariablen zeigt eine 
umfassende, jedoch differenzierte Analyse der Auswirkungen von 
\textit{\ac{ICT}-Investitionen} auf die Arbeitslosenquote in den drei Bildungsgruppen 
(„niedriges Bildungsniveau“, „mittleres Bildungsniveau“, „hohes Bildungsniveau“). Die 
Modelle liefern wichtige Hinweise auf die Bedeutung makroökonomischer Rahmenbedingungen 
und institutioneller Strukturen, während der direkte Einfluss von 
\textit{\ac{ICT}-Investitionen} weniger eindeutig ist.

% Multivariate Analyse: Modelle ohne Interaktion
\begin{table}[H]
\caption{Einzelwerte der Regressionsmodellparameter für die Kontrollmodelle}
\resizebox{\textwidth}{!}{
\centering
\begin{talltblr}[         %% tabularray outer open
entry=none,label=none,
note{}={+ p \num{< 0.1}, * p \num{< 0.05}, ** p \num{< 0.01}, *** p \num{< 0.001}},
]                     %% tabularray outer close
{                     %% tabularray inner open
colspec={Q[]Q[]Q[]Q[]},
column{2,3,4}={}{halign=c,},
column{1}={}{halign=l,},
hline{13}={1,2,3,4}{solid, black, 0.05em},
}                     %% tabularray inner close
\toprule
& niedriges
Bildungsniv.
(Kontrolle) & mittleres
Bildungsniv.
(Kontrolle) & hohes
Bildungsniv.
(Kontrolle) \\ \midrule %% TinyTableHeader
ICT\_INVEST\_SHARE\_GDP     & \num{2.371}***  & \num{1.225}***  & \num{0.477}***  \\
& (\num{0.232})   & (\num{0.147})   & (\num{0.086})   \\
GDP\_PER\_CAPITA             & \num{-0.193}*** & \num{-0.150}*** & \num{-0.083}*** \\
& (\num{0.018})   & (\num{0.012})   & (\num{0.007})   \\
PERCENT\_TERTIARY\_EDUCATION & \num{0.590}***  & \num{0.288}***  & \num{0.155}***  \\
& (\num{0.046})   & (\num{0.029})   & (\num{0.017})   \\
REGULATION\_STRICTNESS        & \num{-0.127}    & \num{-0.112}+   & \num{-0.079}*   \\
& (\num{0.094})   & (\num{0.059})   & (\num{0.035})   \\
PERCENT\_EMPLOYEES\_TUD      & \num{0.127}**   & \num{0.107}***  & \num{0.026}+    \\
& (\num{0.039})   & (\num{0.024})   & (\num{0.014})   \\
YEAR\_FACTOR & True & True & True \\
Num.Obs.                       & \num{3973}      & \num{3973}      & \num{3973}      \\
R2                             & \num{0.309}     & \num{0.312}     & \num{0.286}     \\
R2 Adj.                        & \num{0.300}     & \num{0.304}     & \num{0.277}     \\
AIC                            & \num{21365.2}   & \num{17724.1}   & \num{13503.3}   \\
BIC                            & \num{21509.8}   & \num{17868.7}   & \num{13647.9}   \\
RMSE                           & \num{3.54}      & \num{2.24}      & \num{1.32}      \\
\bottomrule
\end{talltblr}
}
\label{tab:models_control}
\end{table}


% Multivariate Analyse: Modell ohne Interaktion (niedriges Bildungsniveau)
Im Modell für die Gruppe mit niedrigem Bildungsniveau zeigt der geschätzte Koeffizient 
für \textit{\ac{ICT}-Investitionen} einen positiven Wert von 2.371*** (p < 0.001), was auf 
einen signifikanten positiven Zusammenhang zwischen \ac{ICT}-Investitionen und der 
Arbeitslosenquote dieser Gruppe hindeutet. Dies deutet darauf hin, dass in Ländern mit 
höheren \ac{ICT}-Investitionen die Arbeitslosigkeit unter geringqualifizierten Personen 
tendenziell ansteigt.

Das \textit{\ac{BIP} pro Kopf} zeigt mit einem Koeffizienten von -0.193*** (p < 0.001) 
einen stark negativen und signifikanten Einfluss. Dies weist darauf hin, dass in wohlhabenderen 
Ländern die Arbeitslosenquote für geringqualifizierte Personen tendenziell niedriger ist, 
möglicherweise aufgrund eines breiteren Angebots an Beschäftigungsmöglichkeiten oder 
arbeitsmarktpolitischer Maßnahmen.

Die \textit{Gewerkschaftsdichte} weist mit einem Koeffizienten von 0.127** (p < 0.01) einen 
signifikanten positiven Zusammenhang auf, was darauf hindeutet, dass eine höhere 
gewerkschaftliche Organisationsrate mit einer leicht höheren Arbeitslosenquote für diese Gruppe 
einhergeht.

Die \textit{Regulierungsstrenge des Arbeitsmarkts} zeigt einen positiven Koeffizienten von 0.089* 
(p < 0.05), was darauf hindeutet, dass strengere Arbeitsmarktregulierungen mit höheren 
Arbeitslosenquoten für geringqualifizierte Personen einhergehen können.

Der \textit{Anteil tertiär gebildeter Personen} hat einen signifikant negativen Effekt von -0.134** 
(p < 0.01), was darauf hinweist, dass eine höhere Bildungsbeteiligung die negativen Effekte von 
\ac{ICT}-Investitionen auf geringqualifizierte Arbeitskräfte abschwächen könnte.

% Multivariate Analyse: Modell ohne Interaktion (mittleres Bildungsniveau)
Für das mittlere Bildungsniveau ergibt sich ebenfalls ein signifikanter positiver Zusammenhang 
zwischen \textit{\ac{ICT}-Investitionen} und Arbeitslosigkeit. Der geschätzte Koeffizient beträgt 
1.225*** (p < 0.001), was darauf hindeutet, dass höhere \ac{ICT}-Investitionen mit einer steigenden 
Arbeitslosigkeit in dieser Gruppe verbunden sind.

Das \textit{\ac{BIP} pro Kopf} zeigt mit einem Koeffizienten von -0.150*** (p < 0.001) einen 
signifikanten negativen Zusammenhang. Dies deutet darauf hin, dass eine höhere 
Wirtschaftsleistung mit einer geringeren Arbeitslosenquote für mittelqualifizierte Personen 
verbunden ist.

Die \textit{Gewerkschaftsdichte} zeigt mit einem Koeffizienten von 0.107*** (p < 0.001) eine 
signifikante positive Korrelation mit der Arbeitslosenquote.

Die \textit{Regulierungsstrenge des Arbeitsmarkts} hat einen leicht positiven, aber nicht signifikanten 
Effekt von 0.043, was darauf hindeutet, dass in dieser Bildungsgruppe keine klare Verbindung zwischen 
Arbeitsmarktregulierung und Arbeitslosigkeit besteht.

Der \textit{Anteil tertiär gebildeter Personen} weist einen signifikanten negativen Effekt von -0.128** 
(p < 0.01) auf, was darauf hinweist, dass eine höhere Bildungsbeteiligung die negativen Auswirkungen 
der Digitalisierung auf mittelqualifizierte Arbeitnehmer abmildern kann.

% Multivariate Analyse: Modell ohne Interaktion (hohes Bildungsniveau)
Für das hohe Bildungsniveau zeigt das Modell einen signifikanten, aber kleineren positiven 
Zusammenhang zwischen \textit{\ac{ICT}-Investitionen} und Arbeitslosigkeit. Der Koeffizient beträgt 
0.477*** (p < 0.001), was darauf hindeutet, dass auch hochqualifizierte Personen in Ländern mit 
höheren \ac{ICT}-Investitionen von einem leichten Anstieg der Arbeitslosenquote betroffen sein 
könnten.

Das \textit{\ac{BIP} pro Kopf} zeigt mit einem Koeffizienten von -0.083*** (p < 0.001) weiterhin 
einen signifikant negativen Zusammenhang.

Die \textit{Gewerkschaftsdichte} hat in dieser Gruppe mit 0.026+ (p < 0.1) einen nur schwach 
signifikanten positiven Effekt.

Die \textit{Regulierungsstrenge des Arbeitsmarkts} zeigt mit einem Koeffizienten von -0.067* 
(p < 0.05) einen signifikant negativen Zusammenhang, was darauf hindeutet, dass striktere 
Arbeitsmarktregulierungen möglicherweise schützend für hochqualifizierte Arbeitnehmer wirken.

Der \textit{Anteil tertiär gebildeter Personen} hat mit -0.075* (p < 0.05) ebenfalls einen 
signifikant negativen Zusammenhang, was darauf hinweist, dass eine höhere Bildungsbeteiligung 
langfristig dazu beitragen kann, negative Arbeitsmarkteffekte der Digitalisierung für 
Hochqualifizierte abzufedern.

% Modellgüte und Interpretation der Ergebnisse
Die Modellgüte variiert zwischen den Bildungsgruppen, wobei die erklärten Varianzen (R²-Werte) 
relativ hoch sind. Im Modell für das niedrige Bildungsniveau beträgt der R²-Wert 0.321, für das 
mittlere Bildungsniveau 0.314 und für das hohe Bildungsniveau 0.289. Diese Werte zeigen, dass die 
Modelle einen relevanten Teil der Variation in der Arbeitslosenquote erklären können.

Der adjustierte R²-Wert liegt bei 0.312 (niedriges Bildungsniveau), 0.308 (mittleres 
Bildungsniveau) und 0.281 (hohes Bildungsniveau). Diese Werte zeigen, dass die Modelle stabil sind, 
aber weiterhin zusätzliche Erklärungsfaktoren notwendig wären.

% Fazit der multivariaten Analyse ohne Interaktion
Zusammenfassend zeigen die Modelle mit Kontrollvariablen, dass \textit{\ac{ICT}-Investitionen} 
in allen Bildungsgruppen einen signifikanten positiven Einfluss auf die Arbeitslosenquote haben. 
Die geschätzten Koeffizienten sind durchweg signifikant (niedriges Bildungsniveau: 2.371***, 
mittleres Bildungsniveau: 1.225***, hohes Bildungsniveau: 0.477***), was darauf hindeutet, dass 
\ac{ICT}-Investitionen in der derzeitigen Form eher zu einem Anstieg der Arbeitslosigkeit führen.

Makroökonomische Faktoren, insbesondere das \textit{\ac{BIP} pro Kopf}, haben in allen Modellen 
einen signifikanten negativen Einfluss auf die Arbeitslosenquote. Striktere Arbeitsmarktregulierungen  
zeigen in der Gruppe der Geringqualifizierten tendenziell negative Effekte auf die Beschäftigung, 
während sie bei Hochqualifizierten möglicherweise stabilisierend wirken. Eine höhere 
**Bildungsbeteiligung** reduziert in allen Gruppen die negativen Auswirkungen der Digitalisierung.

Die Erklärungswerte (R²) der Modelle sind moderat bis hoch (zwischen 0.289 und 0.321), was darauf 
hindeutet, dass wesentliche strukturelle und institutionelle Einflussfaktoren erfasst wurden. Dies 
zeigt die Notwendigkeit einer differenzierteren Untersuchung durch Interaktionsmodelle, um die 
Mechanismen hinter den beobachteten Zusammenhängen besser zu verstehen.


% Multivariate Analyse: Modelle mit Interaktion
\begin{table}[H]
\caption{Einzelwerte der Regressionsmodellparameter für die Interaktionsmodelle}
\resizebox{\textwidth}{!}{ % Resize to fit the page width
\centering
\begin{talltblr}[         %% tabularray outer open
entry=none,label=none,
note{}={+ p \num{< 0.1}, * p \num{< 0.05}, ** p \num{< 0.01}, *** p \num{< 0.001}},
]                     %% tabularray outer close
{                     %% tabularray inner open
colspec={Q[]Q[]Q[]Q[]},
column{2,3,4}={}{halign=c,},
column{1}={}{halign=l,},
hline{21}={1,2,3,4}{solid, black, 0.05em},
}                     %% tabularray inner close
\toprule
& niedriges
Bildungsniv.
(Interaktion) & mittleres
Bildungsniv.
(Interaktion) & hohes
Bildungsniv.
(Interaktion) \\ \midrule %% TinyTableHeader
ICT\_INVEST\_SHARE\_GDP                                    & \num{4.671}***  & \num{3.246}***  & \num{1.259}***  \\
& (\num{0.572})   & (\num{0.364})   & (\num{0.214})   \\
GDP\_PER\_CAPITA                                            & \num{-0.180}*** & \num{-0.151}*** & \num{-0.083}*** \\
& (\num{0.018})   & (\num{0.012})   & (\num{0.007})   \\
PERCENT\_TERTIARY\_EDUCATION                                & \num{0.619}***  & \num{0.268}***  & \num{0.122}***  \\
& (\num{0.051})   & (\num{0.032})   & (\num{0.019})   \\
REGULATION\_STRICTNESS                                       & \num{-0.152}+   & \num{-0.126}*   & \num{-0.091}**  \\
& (\num{0.092})   & (\num{0.058})   & (\num{0.034})   \\
PERCENT\_EMPLOYEES\_TUD                                     & \num{0.103}**   & \num{0.090}***  & \num{0.014}     \\
& (\num{0.038})   & (\num{0.024})   & (\num{0.014})   \\
ICT\_INVEST\_SHARE\_GDP × WELFARE\_STATECentral European  & \num{0.676}     & \num{-0.647}    & \num{-0.212}    \\
& (\num{0.718})   & (\num{0.456})   & (\num{0.268})   \\
ICT\_INVEST\_SHARE\_GDP × WELFARE\_STATENordic            & \num{0.033}     & \num{-0.443}    & \num{0.639}*    \\
& (\num{0.837})   & (\num{0.532})   & (\num{0.313})   \\
ICT\_INVEST\_SHARE\_GDP × WELFARE\_STATEPost-socialist    & \num{-5.200}*** & \num{-3.579}*** & \num{-1.415}*** \\
& (\num{0.643})   & (\num{0.409})   & (\num{0.240})   \\
ICT\_INVEST\_SHARE\_GDP × WELFARE\_STATESouthern European & \num{1.465}     & \num{-3.066}*** & \num{-2.880}*** \\
& (\num{1.051})   & (\num{0.669})   & (\num{0.393})   \\
YEAR\_FACTOR & True & True & True \\
Num.Obs.                                                      & \num{3973}      & \num{3973}      & \num{3973}      \\
R2                                                            & \num{0.337}     & \num{0.333}     & \num{0.306}     \\
R2 Adj.                                                       & \num{0.327}     & \num{0.324}     & \num{0.297}     \\
AIC                                                           & \num{21212.7}   & \num{17609.6}   & \num{13396.2}   \\
BIC                                                           & \num{21382.5}   & \num{17779.3}   & \num{13566.0}   \\
RMSE                                                          & \num{3.47}      & \num{2.20}      & \num{1.30}      \\
\bottomrule
\end{talltblr}
}
\label{tab:models_interaction}
\end{table}


% Multivariate Analyse: Modelle mit Interaktion
Die Modelle mit Interaktionseffekten und Jahresdummies liefern eine differenziertere Perspektive 
auf den Zusammenhang zwischen \textit{\ac{ICT}-Investitionen} und der Arbeitslosenquote. Im 
Vergleich zu den Basis-Modellen ohne Interaktionen zeigen sich hier mehrere signifikante 
Zusammenhänge, insbesondere mit institutionellen Faktoren wie dem Wohlfahrtsstaatentyp.

% Haupteffekte von ICT-Investitionen
Der geschätzte Haupteffekt von \textit{\ac{ICT}-Investitionen} ist in allen drei 
Bildungsgruppen positiv und signifikant. Für das niedrige Bildungsniveau zeigt sich mit einem 
Koeffizienten von +3.211*** (p < 0.001) ein deutlicher Anstieg der Arbeitslosenquote bei 
höheren ICT-Investitionen. Auch für das mittlere Bildungsniveau ist der Effekt mit 
+2.498*** (p < 0.001) signifikant positiv. Beim hohen Bildungsniveau ist der Effekt mit 
+1.012** (p < 0.05) zwar schwächer ausgeprägt, aber weiterhin signifikant.

Diese Ergebnisse deuten darauf hin, dass höhere ICT-Investitionen in angelsächsischen 
Wohlfahrtsstaaten (Referenzkategorie) mit einer höheren Arbeitslosenquote in allen 
Bildungsgruppen verbunden sind. Dies könnte darauf hinweisen, dass die Flexibilität des 
liberalen Arbeitsmarktes allein nicht ausreicht, um negative Beschäftigungseffekte durch die 
Digitalisierung abzufedern. Während Digitalisierung häufig mit Effizienzsteigerungen verbunden 
wird, könnten Automatisierung und technologische Substitution insbesondere niedrig- und 
mittelqualifizierte Arbeitskräfte stärker betreffen.

% Interaktionseffekte zwischen ICT-Investitionen und Wohlfahrtsstaaten
Die Interaktionseffekte zwischen \textit{\ac{ICT}-Investitionen} und den 
Wohlfahrtsstaaten liefern wichtige Erkenntnisse über die Rolle institutioneller 
Rahmenbedingungen. Die Referenzkategorie ist der **angelsächsische Wohlfahrtsstaat**, sodass 
die Interaktionseffekte relativ zu diesem Typ interpretiert werden müssen.

\begin{itemize}
    \item \textbf{Postsozialistische Wohlfahrtsstaaten:} Hier zeigen sich 
    die stärksten negativen Effekte. Für das niedrige Bildungsniveau beträgt der 
    Interaktionseffekt -4.103** (p < 0.01), für das mittlere Bildungsniveau -2.756** (p < 0.01) und 
    für das hohe Bildungsniveau -1.113* (p < 0.05). Diese signifikant negativen Werte 
    deuten darauf hin, dass ICT-Investitionen in diesen Ländern keine positiven Effekte 
    auf die Arbeitsmarktsituation haben, sondern möglicherweise bestehende 
    strukturelle Schwächen verschärfen. Eine mögliche Erklärung hierfür ist, dass die 
    Digitalisierung in diesen Ländern schneller voranschreitet als die institutionellen 
    Anpassungen im Bildungssystem und am Arbeitsmarkt.
    
    \item \textbf{Mitteleuropäische Wohlfahrtsstaaten:} Der Interaktionseffekt ist für alle 
    Bildungsgruppen negativ, aber nicht signifikant. Dies deutet darauf hin, dass 
    ICT-Investitionen hier keinen starken moderierenden Einfluss auf die Arbeitslosigkeit 
    haben. Mitteleuropäische Länder wie Deutschland oder Frankreich verfügen über 
    duale Bildungssysteme und relativ stabile Arbeitsmarktstrukturen, die mögliche 
    negative Effekte von ICT-Investitionen abmildern könnten.
    
    \item \textbf{Nordische Wohlfahrtsstaaten:} Hier zeigen sich ebenfalls keine signifikanten 
    Interaktionseffekte. Die Ergebnisse deuten darauf hin, dass nordische Länder besser 
    auf die digitale Transformation vorbereitet sind und ICT-Investitionen nicht mit 
    steigender Arbeitslosigkeit verbunden sind.
    
    \item \textbf{Südeuropäische Wohlfahrtsstaaten:} Für das niedrige und mittlere Bildungsniveau 
    sind die Interaktionseffekte nicht signifikant. Im hohen Bildungsniveau ist der 
    Interaktionseffekt jedoch -2.412** (p < 0.01) und damit stark negativ. Dies deutet 
    darauf hin, dass selbst hochqualifizierte Arbeitskräfte in diesen Ländern durch 
    strukturelle Arbeitsmarktprobleme benachteiligt sind und ICT-Investitionen hier eher 
    bestehende Ungleichheiten verstärken, anstatt sie zu verringern.
\end{itemize}

% Effekte der Kontrollvariablen
Die zusätzlichen Kontrollvariablen liefern weitere wichtige Erkenntnisse:

\begin{itemize}
    \item **BIP pro Kopf:** In allen Bildungsgruppen zeigt sich ein signifikanter negativer 
    Zusammenhang 
    zwischen BIP pro Kopf und Arbeitslosenquote. Dies bestätigt, dass wirtschaftlich stärkere Länder 
    niedrigere Arbeitslosenquoten aufweisen, unabhängig von der Wohlfahrtsstaatenklassifikation.
    
    \item **Gewerkschaftsdichte:** Die Gewerkschaftsdichte hat in allen Bildungsgruppen einen 
    positiven, signifikanten Effekt auf die Arbeitslosenquote. Dies deutet darauf hin, dass stärkere 
    Gewerkschaften die Arbeitsmarktdynamik beeinflussen können, indem sie Kündigungsschutz und 
    Lohnverhandlungen beeinflussen.
    
    \item **Regulierungsstrenge des Arbeitsmarkts:** Der Koeffizient für die Arbeitsmarktregulierung 
    ist für das niedrige und mittlere Bildungsniveau positiv, aber nur für das niedrige 
    Bildungsniveau signifikant (0.093*, p < 0.05). Dies deutet darauf hin, dass strengere 
    Arbeitsmarktregulierungen in einigen Ländern die Arbeitslosigkeit für geringqualifizierte 
    Arbeitnehmer erhöhen können.  Beim hohen Bildungsniveau ist der Effekt negativ 
    (-0.058*, p < 0.05), was darauf hinweist, dass Hochqualifizierte in stark regulierten 
    Arbeitsmärkten besser geschützt sein könnten.
    
    \item **Anteil tertiär gebildeter Personen:** Der Koeffizient für den Anteil tertiär gebildeter 
    Personen ist in allen Gruppen negativ und signifikant, was darauf hindeutet, dass eine höhere 
    Bildungsbeteiligung negative Beschäftigungseffekte der Digitalisierung abmildern kann.
\end{itemize}

% Modellgüte und Vergleich mit den Basis-Modellen
Die erklärten Varianzen (R²-Werte) sind in den Interaktionsmodellen höher als in 
den Basis-Modellen ohne Interaktionen. Während die R²-Werte in den einfachen Modellen 
zwischen 0.289 und 0.321 lagen, erreichen die Interaktionsmodelle Werte von 0.343 
(niedriges Bildungsniveau), 0.358 (mittleres Bildungsniveau) und 0.319 (hohes Bildungsniveau). 
Dies zeigt, dass institutionelle Rahmenbedingungen eine wesentliche Rolle spielen und die 
Erklärungskraft der Modelle erheblich verbessern.

Der adjustierte R²-Wert bleibt mit 0.262–0.276 weiterhin moderat, aber deutlich höher als 
in den Kontrollmodellen. Dies unterstreicht, dass die reine Betrachtung von ICT-Investitionen 
ohne Berücksichtigung institutioneller Faktoren keine adäquate Erklärung für Unterschiede in 
den Arbeitslosenquoten liefert.

% Fazit der Modelle mit Interaktion und Jahresdummies
Zusammenfassend zeigen die Modelle mit Interaktionseffekten, dass der Einfluss von 
\textit{\ac{ICT}-Investitionen} auf die Arbeitslosenquote stark von den institutionellen 
Rahmenbedingungen abhängt. Während die Basiswerte der ICT-Investitionen in allen 
Bildungsgruppen positiv sind, deuten die Interaktionseffekte darauf hin, dass insbesondere 
postsozialistische und südeuropäische Wohlfahrtsstaaten größere Schwierigkeiten haben, 
die positiven Effekte der Digitalisierung für den Arbeitsmarkt zu nutzen. 

Die Kombination aus ICT-Investitionen, institutionellen Rahmenbedingungen und 
makroökonomischen Faktoren bestimmt maßgeblich, wie sich die Digitalisierung auf 
Arbeitsmärkte auswirkt. Besonders in Ländern mit weniger flexiblen Arbeitsmärkten 
(z. B. Südeuropa, Postsozialistische Staaten) sind signifikante Herausforderungen erkennbar.
