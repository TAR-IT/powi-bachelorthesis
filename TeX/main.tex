%%%%%%%%%%%%%%%%%%%%%%%
% Dokument definieren %
%%%%%%%%%%%%%%%%%%%%%%%

\documentclass[a4paper, 12pt]{article}

%%%%%%%%%%%%%%%%%%%%%%%%%%%%%%%%%%
% Pakete laden und konfigurieren %
%%%%%%%%%%%%%%%%%%%%%%%%%%%%%%%%%%

% deutsche Spracheinstellungen und -kodierung
\usepackage[utf8]{inputenc}                                                 % Umlaute korrekt verwenden
\usepackage[T1]{fontenc}                                                    % Schriftenkodierung
\usepackage[ngerman]{babel}                                                 % Deutsche Spracheinstellungen
\usepackage{csquotes}                                                       % Für korrekte Anführungszeichen
\usepackage{hyphenat}                                                       % Silbentrennung

% Pakete für R
\usepackage{tabularray}                                                     % Für erweiterte Tabellen
\usepackage{float}                                                          % Fixiert Objekte mit "H"
\usepackage[normalem]{ulem}                                                 % Für Unter- und Durchstreichungen
\UseTblrLibrary{booktabs}                                                   % Unterstützt Booktabs in Tabellen
\UseTblrLibrary{siunitx}                                                    % Unterstützt Zahlen und Einheiten
\newcommand{\tinytableTabularrayUnderline}[1]{\underline{#1}}               % Unterstreichung in Tabellen
\newcommand{\tinytableTabularrayStrikeout}[1]{\sout{#1}}                    % Durchstreichung in Tabellen
\NewTableCommand{\tinytableDefineColor}[3]{\definecolor{#1}{#2}{#3}}        % Farben in Tabellen
\usepackage{adjustbox}                                                      % Skaliert Tabellen/Objekte
\usepackage{multicol}                                                       % Mehrspaltiger Text
\usepackage{listings}                                                       % Für Code-Listings
\lstset{
    language=R,                                                             % Specify the language
    basicstyle=\ttfamily\scriptsize,                                        % Font style for code
    numbers=left,                                                           % Line numbers on the left
    numberstyle=\tiny,                                                      % Font size for line numbers
    stepnumber=1,                                                           % Number every line
    breaklines=true,                                                        % Automatically break long lines
    frame=single,                                                           % Add a border around the code
    captionpos=b,                                                           % Caption position (bottom)
    showstringspaces=false                                                  % Do not show spaces in strings
}

% Layout und Seitenränder
\usepackage{geometry}
\geometry{a4paper, top=3cm, bottom=3cm, left=2.5cm, right=2.5cm}

% Literaturverzeichnis mit APA-Zitierstil
\usepackage[style=apa, backend=biber, language=ngerman]{biblatex}
\addbibresource{assets/literature.bib}                                      % Bibliographiedatei einbinden

% Pakete für Abkürzungsverzeichnis und Glossar
\usepackage[printonlyused]{acronym}

% mathematische Symbole (optional)
\usepackage{amsmath, amssymb}

% Für Tabellen und Grafiken
\usepackage{graphicx}                                                       % Für Bilder
\usepackage{booktabs}                                                       % Für schönere Tabellen

% Hyperlinks
\usepackage[hidelinks]{hyperref}

% Codebook einbinden
\usepackage{verbatim}                                                       % Für Einbinden als plain text

% Zeilenabstand
\usepackage{setspace}
\onehalfspacing                                                             % 1,5-facher Zeilenabstand

% Kopf- und Fußzeilen
\usepackage{fancyhdr}
\pagestyle{fancy}
\fancyhf{}                                                                  % Setzt Kopf- und Fußzeile zurück
\fancyhead[R]{T. A. Rau}                                                    % Name links
\fancyhead[L]{ICT-Investitionen und Arbeitslosigkeit in Wohlfahrtsstaaten}  % Titel rechts
\setlength{\headheight}{14.5pt}                                             % Setzt die Höhe der Kopfzeile
\fancyfoot[C]{\thepage}                                                     % Setzt die Seitenzahl mittig unten

%%%%%%%%%%%%%%%%%%%%%
% Dokument beginnen %
%%%%%%%%%%%%%%%%%%%%%

% Titel und Autoreninformationen
\title{ICT-Investitionen und Arbeitslosigkeit in Wohlfahrtsstaaten\\ 
\large Eine Paneldatenanalyse nach Bildungsniveau in OECD-Ländern}
\author{Tobias Achim Rau}
\date{\today}

\begin{document}

% Titelseite
\maketitle
\thispagestyle{empty}
\vspace{2cm}

\begin{center}
  \textbf{Bachelorarbeit} \\
  vorgelegt im Studiengang Politikwissenschaften \\
  am Fachbereich 03 an der \textbf{Goethe-Universität Frankfurt} \\
  \vspace{2cm}
  \textbf{Verfasser:} Tobias Achim Rau \\
  \textbf{Matrikelnummer:} 6619097 \\
  \textbf{E-Mail:} s3045892@stud.uni-frankfurt.de \\
  \vspace{2cm}
  \textbf{Betreuerin:} Anna Gerlach \\
  \vspace{2cm}
  \textbf{Abgabedatum:} 25.03.2025
\end{center}

\newpage

% Leerseite nach Titelblatt
\newpage
\thispagestyle{empty}
\null
\newpage

% Inhaltsverzeichnis
\tableofcontents

\newpage

% Abkürzungsverzeichnis
\section*{Abkürzungsverzeichnis}
\begin{acronym}[AAAAA] % Breite der längsten Abkürzung

  \acro{AI}{Künstliche Intelligenz (aus d. Engl.: \textit{Artificial Intelligence})}

  \acro{BIP}{Bruttoinlandsprodukt}

  \acro{FE}{Fixed Effects (z. Dt.: \textit{feste Effekte})}
  
  \acro{ICT}{Informations- und Kommunikationstechnologien (aus d. Engl.: 
  \textit{Information and Communication Technologies})}

  \acro{IoT}{Internet der Dinge (aus d. Engl.: \textit{Internet of Things})}

  \acro{IT}{Informationstechnologie/-n (aus d. Engl.: \textit{Information Technology})}
  
  \acro{OECD}{Organisation für wirtschaftliche Zusammenarbeit und Entwicklung 
  (aus d. Engl.: \textit{Organisation for Economic Co-operation and Development})}

  \acro{RBTC}{Routine-Biased Technological Change (z. Dt.: \textit{Technologie mit 
  routinespezifischen Effekten})}

  \acro{RE}{Random Effects (z. Dt.: \textit{zufällige Effekte})}

  \acro{SBTC}{Skill-Biased Technological Change (z. Dt.: \textit{Technologie mit 
  qualifikationsspezifischen Effekten})}
 
  \acro{SNA08}{System of National Accounts 2008}

\end{acronym}

\newpage

% Glossar
\section*{Glossar}
\begin{description}

  \item[Big Data] bezeichnet große, komplexe und schnell wachsende Datenmengen, die mit 
  herkömmlichen Datenverarbeitungssystemen nur schwer analysierbar sind. Die Analyse 
  von Big Data erfolgt häufig mit Verfahren des maschinellen Lernens, Data Mining und 
  verteilten Datenbanksystemen \parencite[vgl.][S. 138–139]{gandomi2014beyond}.

  \item[Cloud Computing] beschreibt die Bereitstellung von Ressourcen für \ac{IT} wie Speicherplatz, 
  Rechenleistung und Anwendungen über das Internet („die Cloud“) anstelle lokaler Server 
  oder Computer. Dieses Modell ermöglicht den flexiblen Zugriff auf Daten und Anwendungen 
  unabhängig vom Standort und fördert Skalierbarkeit sowie Kosteneffizienz 
  \parencite[vgl.][S. 50–52]{armbrust2010aview}. Die wichtigsten Servicemodelle sind 
  Infrastructure-as-a-Service, Platform-as-a-Service und 
  Software-as-a-Service.

  \item[Digitalisierung] bezeichnet den Prozess der Umwandlung von analogen Informationen, 
  Prozessen und Geschäftsmodellen in digitale Formate. Sie umfasst den Einsatz digitaler 
  Technologien zur Automatisierung und Optimierung von Abläufen sowie zur Schaffung neuer 
  Wertschöpfungspotenziale \parencite[vgl.][S. 6]{brennen2016theinternational}. Im 
  wirtschaftlichen Kontext wird Digitalisierung oft im Zusammenhang mit der vierten 
  industriellen Revolution (Industrie 4.0) gesehen, die durch die Integration digitaler 
  und physischer Systeme gekennzeichnet ist \parencite[vgl.][S. 114]{hofman2018arbeit}.

  \item[Künstliche Intelligenz] bezeichnet die Fähigkeit von Maschinen oder 
  Computersystemen, menschenähnliche kognitive Funktionen wie Lernen, Problemlösung und 
  Entscheidungsfindung auszuführen. Sie basiert auf Algorithmen, die Muster in Daten 
  erkennen und selbstständig aus Erfahrungen lernen können 
  \parencite[vgl.][S. 10–13]{russell2020artificial}. \ac{AI} umfasst verschiedene Teilbereiche 
  wie maschinelles Lernen, neuronale Netzwerke und natürliche Sprachverarbeitung.

  \item[Reversed Causality] beschreibt eine umgekehrte Kausalbeziehung zwischen zwei 
  Variablen, bei der die vermeintliche unabhängige Variable tatsächlich durch die 
  abhängige Variable beeinflusst wird \parencite[vgl.][S. 29-33]{pearl2009causality}. Dies kann zu 
  Fehlinterpretationen in statistischen Analysen führen, insbesondere in nicht-experimentellen 
  Studien. Reversed Causality tritt häufig in wirtschaftlichen und sozialwissenschaftlichen 
  Untersuchungen auf.

\end{description}

\newpage

% Kapitel einbinden
%%%%%%%%%%%%%%
% Einleitung %
% TODO:      %
%%%%%%%%%%%%%%

\section{Einleitung}

Mit der zunehmenden Digitalisierung und Automatisierung der Arbeitswelt erleben viele Länder 
tiefgreifende strukturelle Veränderungen ihrer Arbeitsmärkte. Eine zentrale Rolle spielen dabei 
\ac{ICT}, deren Einsatz weltweit zu erheblichen Effizienzsteigerungen und Innovationsprozessen 
führt \parencite[vgl.][S. 49]{oecd2019measuring}.

Während technologische Fortschritte typischerweise die Nachfrage nach hochqualifizierten 
Arbeitskräften erhöhen, bleibt die Rolle von geringqualifizierten Arbeitskräften im 
technologischen Wandel unklar. Besonders die Differenzierung zwischen „Skills“ und „Tasks“ spielt 
eine entscheidende Rolle bei der Analyse dieser Veränderungen, da technologische Entwicklungen 
bestimmte Aufgaben automatisieren oder auslagern können, während sie gleichzeitig die 
Anforderungen an die verbleibenden Tätigkeiten verändern 
\parencite[vgl.][S. 1045]{acemoglu2011skills}. Die zunehmende Digitalisierung und Automatisierung 
geht mit einer Polarisierung des Arbeitsmarktes einher 
\parencite[vgl.][S. 1070]{acemoglu2011skills}. Während der Anteil hochqualifizierter Tätigkeiten 
wächst, steigt gleichzeitig die Beschäftigung in geringqualifizierten, niedrig entlohnten Berufen 
- hierbei geraten mittlere Qualifikationsniveaus unter Druck. In der Forschung wird daher 
diskutiert, inwiefern schon Investitionen in \ac{ICT} und dazugehörige Ressourcen zu einer 
Polarisierung des Arbeitsmarktes beiträgt, indem Sie die Nachfrage nach hochqualifizierten 
Arbeitskräften erhöhen, während gleichzeitig Tätigkeiten von geringer Qualifizierten durch neue 
Technologien und Automatisierung ersetzt werden \parencite[vgl.][S. 2–4]{balsmeier2019isthis}. 

Die Auswirkungen von \ac{ICT}-Investitionen auf den Arbeitsmarkt sind nicht nur ökonomisch, 
sondern auch sozial von Bedeutung. Technologischer Fortschritt führt dazu, dass manche Aufgaben 
zunehmend automatisierbar werden, wodurch insbesondere geringqualifizierte Arbeitskräfte einem 
höheren Substitutionsrisiko ausgesetzt sind, während hochqualifizierte Fachkräfte tendenziell von 
diesen Entwicklungen profitieren \parencite[vgl.][S. 1073]{acemoglu2011skills}. Der 
technologische Wandel führt nicht nur zu Arbeitsplatzverlusten durch Automatisierung, sondern 
schafft auch neue Berufsfelder, die insbesondere auf die Zusammenarbeit zwischen menschlichen 
kognitiven Fähigkeiten und digitalen Technologien angewiesen sind 
\parencite[vgl.][Kap. 12]{brynjolfsson2014thesecond}. Welche Auswirkungen diese Prozesse letzlich 
wirklich auf die Verteilung von Arbeitsplätzen haben und ob sie aktiv zur Polarisierung des 
Arbeitsmarktes beitragen, bleibt eine zentrale Frage der aktuellen Forschung. 

Ziel dieser Arbeit ist es daher, den Einfluss von \ac{ICT}-Investitionen auf die Arbeitslosigkeit 
in verschiedenen Bildungsniveaus zu untersuchen. Es wird angenommen, dass technologischer 
Fortschritt die Nachfrage nach hochqualifizierten Arbeitskräften steigert, während gleichzeitig 
die Beschäftigungsmöglichkeiten für geringqualifizierte Personen durch Automatisierung reduzieren 
werden. Dies kann zu einer Polarisierung des Arbeitsmarktes führen 
\parencite[vgl.][S. 1070]{acemoglu2011skills}. Diese Analyse soll zur Debatte über die 
Auswirkungen der Digitalisierung auf den Arbeitsmarkt beitragen und empirisch untersuchen, in 
welchem Maße technologische Investitionen mit der Arbeitslosenquote nach Bildungsniveau 
korrelieren. 

Aus der zuvor dargelegten Argumentation ergibt sich die zentrale Forschungsfrage dieser Arbeit: 

\begin{quote} 
    \textbf{„Wie beeinflussen nationale Investitionen in Informations- und 
    Kommunikationstechnologien die Arbeitslosenquoten verschiedener Bildungsniveaus in 
    Wohlfahrtsstaaten?“}
\end{quote}

Die Analyse der Auswirkungen von Digitalisierung und \ac{ICT}-Investitionen auf die 
Beschäftigungsstruktur in Ländern der \ac{OECD} ist von hoher Relevanz, da sie Aufschluss über 
die Anpassungsfähigkeit verschiedener Wirtschaftssysteme an technologische Umbrüche gibt. Zudem 
bietet sie eine Grundlage für politische Entscheidungen im Bereich Arbeitsmarktregulierung und 
Bildungsinvestitionen. Angesichts der zunehmenden Bedeutung digitaler Technologien für 
wirtschaftliches Wachstum und soziale Gerechtigkeit ist es essenziell, die damit verbundenen 
Herausforderungen und Chancen für verschiedene gesellschaftliche Gruppen besser zu verstehen. 

%%%%%%%%%%%%%%%%%%%%%%%%
% Forschungsgegenstand %
%%%%%%%%%%%%%%%%%%%%%%%%

\section{Forschungsgegenstand}

Der Forschungsgegenstand dieser Arbeit umfasst die Untersuchung der Auswirkungen von 
Investitionen in \ac{ICT} auf den Arbeitsmarkt in \ac{OECD}-Ländern. Im Zentrum steht dabei die 
Frage, wie sich diese Investitionen auf die Beschäftigungslage, insbesondere die 
Arbeitslosenquote, auf verschiedenen Bildungsniveaus auswirken. Dabei werden sowohl die 
nationalen Investitionen in digitale Technologien als auch die Auswirkungen auf Arbeitsmärkte und 
Beschäftigungsstrukturen betrachtet. 

%%%%%%%%%%%%%%%%%%%%%%%%%%%%%%%%%%%%%%%%%%%%%%%%%%%%%%%%%%%
% Forschungsgegenstand: Digitalisierung und Industrie 4.0 %
%%%%%%%%%%%%%%%%%%%%%%%%%%%%%%%%%%%%%%%%%%%%%%%%%%%%%%%%%%%

\subsection{Digitalisierung und Industrie 4.0}

Der Begriff der Digitalisierung beschreibt den zunehmenden Einsatz digitaler Technologien zur 
Automatisierung, Optimierung und Schaffung neuer Wertschöpfungspotenziale 
\parencite[vgl.][S. 6]{brennen2016theinternational}. Im wirtschaftlichen Kontext wird dies mit 
der vierten industriellen Revolution (Industrie 4.0) in Verbindung gebracht, die durch die 
Integration von \ac{ICT}, \ac{AI}, Big Data, Cloud Computing und cyber-physischen 
Systemengekennzeichnet ist \parencite[vgl.][S. 13–14]{kagermann2013recommendations}. Diese 
Entwicklungen führen zu einer zunehmenden Automatisierung von Produktionsprozessen und 
ermöglichen eine vernetzte Wertschöpfung entlang der gesamten Lieferkette. 

Während Unternehmen durch den Einsatz digitaler Technologien Effizienzsteigerungen erzielen 
können, ergeben sich für den Arbeitsmarkt erhebliche Herausforderungen. Hochqualifizierte 
Arbeitsplätze entstehen in den Bereichen Softwareentwicklung, Datenanalyse und 
Automatisierungstechnik, während Routineaufgaben in der industriellen Fertigung, im 
Transportwesen und in administrativen Berufen zunehmend automatisiert werden 
\parencite[vgl.][S. 36–38]{frey2013thefuture}. Dies kann zur Verdrängung mittlerer 
Qualifikationsniveaus führen, was in der Literatur als „Jobpolarisierung“ beschrieben wird 
\parencite[vgl.][S. 1070]{acemoglu2011skills}. 

Die Digitalisierung bringt zudem tiefgreifende Veränderungen in der Arbeitsorganisation mit sich. 
Neue Arbeitsformen wie Plattformarbeit, Remote Work und flexible Arbeitszeitmodelle gewinnen an 
Bedeutung \parencite[vgl.][S. 58–60]{schwab2016thefourth}.Dies erfordert nicht nur neue 
Kompetenzen, sondern stellt auch Arbeitnehmer*innen vor Herausforderungen in Bezug auf 
Arbeitsplatzsicherheit, Datenschutz und \ac{IT}-Sicherheit. 

%%%%%%%%%%%%%%%%%%%%%%%%%%%%%%%%%%%%%%%%%%%
% Forschungsgegenstand: ICT-Investitionen %
%%%%%%%%%%%%%%%%%%%%%%%%%%%%%%%%%%%%%%%%%%%

\subsection{ICT-Investitionen}

Laut der OECD umfassen Investitionen in \ac{ICT} materielle Ressourcen wie digitale 
Infrastrukturen (z. B. Glasfasernetze, Rechenzentren) und immaterielle Ressourcen wie 
Softwarelösungen, Cloud-Dienste und Plattformtechnologien \parencite{oecd2022ict}. Sie spielen 
eine zentrale Rolle für die digitale Transformation und haben weitreichende Implikationen für 
Wirtschaftswachstum und Arbeitsmärkte \parencite[vgl.][S. 50]{oecd2019measuring}. 

Fortschritte in \ac{AI} und Big Data erlauben eine effizientere Ressourcennutzung, optimierte 
Entscheidungsprozesse und die Automatisierung komplexer Abläufe 
\parencite[vgl.][S. 120]{oecd2019measuring}. Dies kann die Wettbewerbsfähigkeit von Unternehmen 
erhöhen, bringt jedoch gleichzeitig Veränderungen in der Beschäftigungsstruktur mit sich. 

Die Auswirkungen von \ac{ICT}-Investitionen auf den Arbeitsmarkt sind ambivalent. Einerseits 
entstehen neue, zukunftsorientierte Berufe, andererseits werden zahlreiche Routineaufgaben 
automatisiert, was insbesondere niedrigqualifizierte und  niedrigvergütete Arbeitsplätze einem 
Automatisierungsrisiko aussetzt \parencite[vgl.][S. 42]{frey2013thefuture}. Diese Entwicklung 
verstärkt bestehende Ungleichheiten auf dem Arbeitsmarkt und führt zu Herausforderungen für 
Weiterbildungs- und Umschulungsmaßnahmen. 

Ein weiterer Aspekt von \ac{ICT}-Investitionen ist ihre Rolle in der Verringerung regionaler 
Disparitäten. Während digitale Infrastrukturen in urbanen Zentren stark vorangetrieben werden, 
bleiben ländliche Regionen oft zurück, was den Zugang zu Technologien und digitalen 
Arbeitsmärkten erschwert \parencite[vgl.][S. 106]{oecd2019measuring}. Die politischen 
Rahmenbedingungen können hier eine entscheidende Rolle spielen, um digitale Ungleichheiten zu 
vermeiden. 

%%%%%%%%%%%%%%%%%%%%%%%%%%%%%%%%%%%%%%%%%%%%%%%%%%%%%%%%%
% Forschungsgegenstand: Arbeitsmarkt und Bildungsniveau %
%%%%%%%%%%%%%%%%%%%%%%%%%%%%%%%%%%%%%%%%%%%%%%%%%%%%%%%%%

\subsection{Arbeitsmarkt und Bildungsniveau}

Der Arbeitsmarkt wird maßgeblich durch technologische Entwicklungen beeinflusst, wobei 
insbesondere die Digitalisierung bestehende Strukturen verändert 
\parencite[vgl.][S. 50-51]{oecd2019measuring}. In dieser Arbeit liegt der Fokus auf der 
Arbeitslosigkeit nach Bildungsniveau, das üblicherweise in niedrig, mittel und hoch eingeteilt 
wird. Diese Differenzierung ermöglicht eine gezielte Analyse der Betroffenheit unterschiedlicher 
Gruppen.

Die Auswirkungen der Digitalisierung auf den Arbeitsmarkt werden häufig mit dem Konzept der 
„Jobpolarisierung“ beschrieben \parencite[vgl.][S. 12]{autor2015whyare}. Hochqualifizierte 
Fachkräfte profitieren von der steigenden Nachfrage nach digitalen Kompetenzen, während 
Arbeitsplätze mit mittleren Qualifikationsanforderungen verstärkt unter Automatisierungsdruck 
geraten. Geringqualifizierte sind besonders anfällig für Arbeitsplatzverdrängung, da ihre 
Tätigkeiten häufig leichter durch technologische Systeme ersetzt werden können 
\parencite[vgl.][S. 4]{balsmeier2019isthis}. Um den negativen Effekten entgegenzuwirken, werden 
gezielte bildungs- und arbeitsmarktpolitische Maßnahmen als notwendig erachtet. Besonders 
Weiterbildungsprogramme für digitale Kompetenzen sind zentral, um den Strukturwandel am 
Arbeitsmarkt abzufedern \parencite[vgl.][Kap. 13]{brynjolfsson2014thesecond}. Länder mit einem 
gut ausgebauten Weiterbildungssystem könnten die negativen Folgen der Arbeitsmarktpolarisierung 
besser kompensieren. 

%%%%%%%%%%%%%%%%%%%
% Forschungsstand %
%%%%%%%%%%%%%%%%%%%

\section{Forschungsstand}

Die Auswirkungen von Digitalisierung und \ac{ICT}-Investitionen auf den Arbeitsmarkt sind
ein zentrales Thema der arbeitsmarktökonomischen Forschung. Während einige Studien den
Fokus auf die technologische Verdrängung bestimmter Berufsgruppen legen, untersuchen
andere, inwieweit institutionelle Rahmenbedingungen wie Wohlfahrtsstaaten die Effekte von
Digitalisierung abmildern oder verstärken. In diesem Kapitel werden zunächst die
allgemeinen Auswirkungen der Digitalisierung auf Arbeitsmärkte analysiert, bevor der Fokus
auf die Rolle von \ac{ICT}-Investitionen und die Unterschiede zwischen verschiedenen
Wohlfahrtsstaaten gelegt wird. Abschließend werden bestehende Forschungslücken aufgezeigt,
die eine weiterführende Analyse notwendig machen.

%%%%%%%%%%%%%%%%%%%%%%%%%%%%%%%%%%%%%%%%%%%%%%%%%%%%%%%%%%%%%%%%%%%%%%%
% Forschungsstand: Auswirkungen der Digitalisierung auf Arbeitsmärkte %
%%%%%%%%%%%%%%%%%%%%%%%%%%%%%%%%%%%%%%%%%%%%%%%%%%%%%%%%%%%%%%%%%%%%%%%

\subsection{Auswirkungen der Digitalisierung auf Arbeitsmärkte}

Die Digitalisierung hat die Arbeitsmärkte weltweit grundlegend verändert. Empirische 
Studien zeigen, dass diese Entwicklungen die Beschäftigungsstrukturen in verschiedenen 
Bildungsgruppen unterschiedlich beeinflussen \parencite[vgl.][S. 1589]{autor2013thegrowth}. 
Die Automatisierung betrifft besonders routinisierbare und standardisierbare Tätigkeiten 
\parencite[vgl.][S. 20]{frey2013thefuture}. Diese Entwicklungen führen zu einer 
Polarisierung des Arbeitsmarktes: Hochqualifizierte profitieren von einer steigenden 
Nachfrage nach digitalen Kompetenzen, während Arbeitsplätze mit mittlerem 
Qualifikationsniveau zunehmend unter Automatisierungsdruck geraten
\parencite[vgl.][S. 2509–2510]{goos2014explaining}.

Autor, Levy und Murnane zeigen, dass Tätigkeiten mit hohem Anteil an routinemäßigen
kognitiven und manuellen Aufgaben besonders anfällig für Automatisierung sind. Die Theorie
des \ac{RBTC} besagt, dass insbesondere klar definierte, sich wiederholende Tätigkeiten 
durch Maschinen ersetzt werden können \parencite[vgl.][S. 1281]{autor2003theskill}. Frey 
und Osborne erweiterten diese Analyse und schätzten, dass bis zu 47\% der 
Arbeitsplätze in den USA potenziell automatisierbar sind, wobei Berufe mit niedrigem 
Qualifikationsniveau besonders betroffen sind \parencite[S. 36–38]{frey2013thefuture}. 
Diese Erkenntnisse  werfen die Frage auf, inwiefern sich diese Tendenz auf andere Länder 
übertragen lässt und welche Faktoren zur Abmilderung der negativen Effekte beitragen 
können.

Parallel zur Automatisierung zeigt sich eine Polarisierung der Arbeitsmärkte. Während 
mittlere Qualifikationsgruppen unter Druck geraten, profitieren insbesondere 
hochqualifizierte Beschäftigte, die über spezialisierte technologische Kenntnisse 
verfügen, von der steigenden Nachfrage nach digitalen und analytischen Fähigkeiten 
\parencite[vgl.][S. 2509]{goos2014explaining}. Diese Entwicklung führt dazu, dass gut 
ausgebildete Arbeitskräfte mit hohen Qualifikationen von der Digitalisierung profitieren, 
während gering Qualifizierte in wachsendem Maße von Arbeitsplatzverlusten betroffen sind. 
Dies verstärkt das Risiko sozialer Ungleichheit, da Beschäftigungschancen zunehmend 
ungleich verteilt sind. Diese Divergenz wird häufig als „Digital Divide“ bezeichnet, da 
sie die Kluft zwischen hoch- und niedrigqualifizierten Arbeitskräften weiter vertieft 
\parencite[vgl.][S. 3]{oecd2019measuring}.

Die Auswirkungen der Digitalisierung variieren zudem stark nach Branche und
Wirtschaftssektor. Während einige Sektoren wie die Industrieproduktion oder der
Einzelhandel durch die Einführung automatisierter Systeme massiv verändert wurden,
profitieren wissensintensive Dienstleistungsbranchen von den neuen technologischen 
Möglichkeiten \parencite[vgl.][S. 1555]{autor2013thegrowth}. Besonders betroffen sind 
manuelle Tätigkeiten in der Fertigungsindustrie sowie administrative Büroarbeiten, die 
zunehmend durch algorithmische Prozesse ersetzt werden 
\parencite[vgl.][S. 36–38]{frey2013thefuture}. Die zunehmende Kluft zwischen verschiedenen 
Qualifikationsgruppen hat tiefgreifende Auswirkungen auf die Einkommensverteilung, soziale 
Mobilität und die Notwendigkeit gezielter arbeitsmarktpolitischer Maßnahmen 
\parencite[S. 1589–1591]{autor2013thegrowth}.

%%%%%%%%%%%%%%%%%%%%%%%%%%%%%%%%%%%%%%%%%%%%%%%%%%%%%%%%%%%%%%%%%%%%%
% Forschungsstand: ICT-Investitionen als Treiber der Transformation %
%%%%%%%%%%%%%%%%%%%%%%%%%%%%%%%%%%%%%%%%%%%%%%%%%%%%%%%%%%%%%%%%%%%%%

\subsection{ICT-Investitionen als Treiber der Transformation}

Investitionen in \ac{ICT} gelten als zentraler Indikator für den Digitalisierungsgrad eines 
Landes und spielen eine Schlüsselrolle bei der Transformation moderner Arbeitsmärkte. 
Empirische Studien zeigen, dass Unternehmen, die verstärkt in \ac{ICT} investieren, 
effizientere Abläufe entwickeln, ihre Wettbewerbsfähigkeit steigern 
und tendenziell eine höhere Nachfrage nach qualifizierten Arbeitskräften verzeichnen 
\parencite[vgl.][S. 30–31]{corrado2018intangible}.

Der Einfluss von \ac{ICT}-Investitionen auf den Arbeitsmarkt ist dabei vielschichtig. Laut 
der \ac{OECD} ermöglichen diese Investitionen nicht nur eine zunehmende Automatisierung, 
sondern tragen auch zur Integration globaler Wertschöpfungsketten bei und treiben das 
wirtschaftliche Wachstum voran \parencite[vgl.][S. 144]{oecd2019measuring}. Besonders 
in wissensintensiven Sektoren wie Finanzdienstleistungen, \ac{IT}-gestützte 
Geschäftsprozesse, E-Commerce oder digitale Plattformarbeit entstehen neue 
Geschäftsmodelle, die verstärkt auf Automatisierung und datenbasierte Entscheidungsprozesse 
setzen.

Während \ac{ICT}-Investitionen neue Arbeitsplätze schaffen können, zeigen zahlreiche 
Studien, dass diese Transformation auch polarisierende Effekte mit sich bringt. 
Hochqualifizierte Arbeitskräfte profitieren von der steigenden Nachfrage nach digitalen 
und analytischen Fähigkeiten, während geringqualifizierte Beschäftigte einem zunehmenden 
Risiko der Arbeitsplatzverdrängung ausgesetzt sind 
\parencite[vgl.][Kap. 2]{brynjolfsson2014thesecond}.

Gleichzeitig zeigt sich, dass \ac{ICT}-Investitionen nicht in allen Ländern und Branchen 
gleichermaßen produktivitätssteigernd wirken. Ihre Effekte hängen stark von begleitenden 
wirtschaftspolitischen Maßnahmen ab, darunter Investitionen in digitale Infrastruktur, 
die Förderung digitaler Kompetenzen und die Anpassung von Bildungsprogrammen an die 
veränderten Anforderungen des Arbeitsmarktes 
\parencite[vgl.][Kap. 13]{brynjolfsson2014thesecond}. Empirische 
Studien zeigen, dass digitale Infrastruktur und eine strategische Förderung von digitaler 
Bildung eine zentrale Rolle dabei spielen, die wirtschaftlichen Vorteile von 
\ac{ICT}-Investitionen vollständig auszuschöpfen \parencite[vgl.][S. 357–358]{vu2011ict}. 
Länder mit einem schwächeren Fokus auf digitale Bildung haben größere Schwierigkeiten, 
von diesen Entwicklungen zu profitieren, da der Mangel an digitalen Kompetenzen die 
Innovationskraft und Produktivität hemmt \parencite[vgl.][S. 19]{oecd2020digital}.

Zusammenfassend zeigen \ac{ICT}-Investitionen sowohl wachstumsfördernde als auch 
polarisierende Effekte auf den Arbeitsmarkt. Während sie die Produktivität und 
Wettbewerbsfähigkeit von Unternehmen steigern und neue Beschäftigungsmöglichkeiten für 
hochqualifizierte Arbeitskräfte schaffen \parencite[vgl.][S. 19–20]{oecd2020digital}, 
verstärken sie gleichzeitig das Risiko der Arbeitsplatzverdrängung für geringqualifizierte 
Arbeitskräfte. Die Digitalisierung des Dienstleistungssektors, die zunehmende Automatisierung 
administrativer Prozesse und die Integration neuer Technologien in industrielle 
Produktionsabläufe führen dazu, dass traditionelle Berufsbilder zunehmend hinterfragt und 
an neue Anforderungen angepasst werden müssen \parencite[vgl.][S. 20]{oecd2020digital}. Diese 
Entwicklungen unterstreichen die Notwendigkeit arbeitsmarktpolitischer Maßnahmen, um den 
Wandel sozial abzufedern und die Vorteile der Digitalisierung möglichst breit in der 
Gesellschaft zu verteilen.

%%%%%%%%%%%%%%%%%%%%%%%%%%%%%%%%%%%%%%%%%%%%%%%%%%%%%%%%%%%%
% Forschungsstand: Unterschiede zwischen Wohlfahrtsstaaten %
%%%%%%%%%%%%%%%%%%%%%%%%%%%%%%%%%%%%%%%%%%%%%%%%%%%%%%%%%%%%

\subsection{Unterschiede zwischen Wohlfahrtsstaaten}

Empirische Studien zeigen, dass die Auswirkungen von Digitalisierung und Investitionen in
\ac{ICT} auf Arbeitsmärkte stark von den institutionellen Rahmenbedingungen eines Landes 
abhängen. Regierungen spielen eine zentrale Rolle bei der Förderung digitaler Infrastruktur, 
der Implementierung von Bildungspolitik und der Regulierung des 
Arbeitsmarktes \parencite[vgl.][S. 4–5]{hall2001varieties}. Studien haben gezeigt, dass 
Länder mit hohen Investitionen in digitale Bildung und Infrastruktur tendenziell bessere 
Anpassungsprozesse an den technologischen Wandel durchlaufen 
\parencite[vgl.][S. 22–23]{oecd2020digital}.

Laut der \ac{OECD} variieren die Effekte der Digitalisierung erheblich zwischen 
Ländern. Skandinavische Staaten und die Niederlande investieren überdurchschnittlich in 
digitale Bildung und Innovationen, während süd- und osteuropäische Länder vergleichsweise 
niedrigere Investitionen tätigen \parencite[vgl.][S. 24]{oecd2020digital}. Diese 
Unterschiede spiegeln sich auch in der Entwicklung der Arbeitsmärkte wider.

Neben Investitionen in Digitalisierung spielen staatliche Bildungs- und 
Arbeitsmarktpolitiken eine entscheidende Rolle für die Fähigkeit eines Landes, sich an 
technologische Veränderungen anzupassen. Länder mit umfassenden Umschulungs- und 
Weiterbildungsprogrammen könnten bessere Voraussetzungen, um durch lebenslanges Lernen 
den digitalen Wandel sozialverträglich zu gestalten \parencite[vgl.][S. 370–371]{vu2011ict}. 
Staaten mit weniger regulierten Arbeitsmärkten haben eine schnellere, aber oft ungleichere 
Anpassung an technologische Innovationen, was zu verstärkter Arbeitsplatzpolarisierung führen 
kann \parencite[vgl.][S. 27–28]{hall2001varieties}.

Studien zeigen, dass das Automatisierungsrisiko je nach Land und Bildungsstruktur stark 
variiert. Laut einer OECD-Analyse von Arntz, Gregory \& Zierahn sind in Deutschland und 
Österreich etwa 12\% der Arbeitsplätze einem hohen Automatisierungsrisiko ausgesetzt, 
während der Anteil in Südkorea und Estland nur bei 6\% liegt. In vielen Ländern hängt 
das Risiko stark mit dem Bildungsniveau zusammen, da höher qualifizierte Arbeitskräfte 
seltener automatisierbare Tätigkeiten ausüben \parencite[vgl.][S. 15–16]{arntz2016therisk}. 

Zusammenfassend zeigen die Unterschiede zwischen Wohlfahrtsstaaten, dass Digitalisierung 
und \ac{ICT}-Investitionen stark von den institutionellen Rahmenbedingungen abhängen. 
Länder mit gezielter Förderung digitaler Infrastruktur und Bildung können die 
Herausforderungen des technologischen Wandels besser bewältigen, während Länder mit 
geringeren Investitionen und restriktiveren Arbeitsmärkten stärkeren Risiken durch 
Automatisierung ausgesetzt sind. Dies verdeutlicht die Bedeutung staatlicher Steuerung 
bei der Gestaltung von Transformationsprozessen im digitalen Zeitalter.

%%%%%%%%%%%%%%%%%%%%%%%%%%%%%%%%%%%%%
% Forschungsstand: Forschungslücken %
%%%%%%%%%%%%%%%%%%%%%%%%%%%%%%%%%%%%%

\subsection{Forschungslücken}

Obwohl zahlreiche Studien die Auswirkungen von Digitalisierung und \ac{ICT}-Investitionen 
untersuchen, bestehen weiterhin relevante Forschungslücken. Die Mehrheit der bisherigen 
Studien konzentriert sich auf die allgemeinen Effekte von Digitalisierung auf den 
Arbeitsmarkt, ohne spezifisch zwischen verschiedenen Wohlfahrtsstaatentypen zu 
unterscheiden. Es fehlen systematische Vergleiche, die institutionelle Faktoren wie 
Bildungssysteme und Arbeitsmarktregulierungen einbeziehen. Viele empirische Studien zur 
Automatisierung betrachten vorwiegend die Situation in den USA, wohingegen umfassende 
Analysen für \ac{OECD}-Länder mit unterschiedlichen Wohlfahrtsmodellen begrenzt sind. Der 
langfristige Einfluss von \ac{ICT}-Investitionen auf die Arbeitslosigkeit verschiedener 
Bildungsgruppen wurde bisher nicht ausreichend mit einer quantitativen, 
länderübergreifenden Panelanalyse untersucht.

Um diese Forschungslücken zu schließen, wird im empirischen Teil dieser Arbeit eine 
Paneldatenanalyse über \ac{OECD}-Länder durchgeführt. Dadurch sollen systematische 
Unterschiede in den Auswirkungen von Digitalisierung auf die Arbeitslosigkeit nach 
Bildungsniveaus erfasst werden.

\section{Theorie und Hypothesen}

Der technologische Wandel und die damit einhergehende Digitalisierung haben tiefgreifende 
Auswirkungen auf die Arbeitswelt. Die sogenannten „disruptiven Technologien“, darunter auch 
\ac{ICT}, verändern nicht nur die Art und Weise, wie Unternehmen arbeiten und Märkte 
funktionieren, sondern auch die gesamte Struktur der Arbeitsmärkte 
\parencite[vgl.][S. 27]{brynjolfsson2014thesecond}. Besonders durch den Fortschritt in der 
Automatisierung und Computergestützten Arbeitsprozessen sind zahlreiche Berufe einem 
fundamentalen Wandel unterworfen \parencite[vgl.][S. 256]{frey2013thefuture}.

Diese Veränderungen werden durch tiefgreifende Innovationsprozesse vorangetrieben, die 
bestehende Praktiken und Strukturen aufbrechen und neue Möglichkeiten schaffen. Ökonomische 
Theorien wie die der „kreativen Zerstörung“ von Joseph Schumpeter 
\parencite[vgl.][S. 81]{schumpeter1976capitalism} bieten in Verbindung mit dem \ac{SBTC} 
wertvolle Einsichten in diesen Wandel und beschreiben, wie technologische Innovationen 
bestehende Strukturen destabilisieren und dabei Platz für neue schaffen.

%%%%%%%%%%%%%%%%%%%%%%%%%%%%%%%%%%%%%
% Schumpeters „kreative Zerstörung“ %
%%%%%%%%%%%%%%%%%%%%%%%%%%%%%%%%%%%%%

\subsection{Schumpeters „kreative Zerstörung“}

Schumpeter beschreibt den Kapitalismus nicht als ein statisches, sondern als ein 
dynamisches, sich ständig veränderndes System 
\parencite[vgl.][S. 82]{schumpeter1976capitalism}. 
Seine Theorie geht davon aus, dass der Kapitalismus durch einen kontinuierlichen 
Innovationsprozess geprägt ist, der ständig neue Produkte, Prozesse und Märkte hervorbringt. 
Innovationen, vorangetrieben von Unternehmer*innen, sind die treibende Kraft hinter diesem 
Prozess, der sowohl technologischen Fortschritt als auch die Veränderung gesellschaftlicher 
und wirtschaftlicher Strukturen umfasst \parencite[vgl.][S. 82]{schumpeter1976capitalism}. 
Gleichzeitig führen diese Innovationen nicht nur zur Entstehung neuer Märkte und 
Unternehmen, sondern zerstören auch bestehende Strukturen. Schumpeter bezeichnet diesen 
Prozess als „schöpferische Zerstörung“ 
\parencite[vgl.][S. 103–105]{schumpeter1976capitalism}.

Dieser Prozess der „schöpferischen Zerstörung“ ist nicht nur unvermeidbar, sondern laut 
Schumpeter auch notwendig, um Raum für Innovation und langfristiges Wirtschaftswachstum zu 
schaffen \parencite[vgl.][S. 87]{schumpeter1976capitalism}. Der Kapitalismus lebt von dieser 
kontinuierlichen Erneuerung, wobei neue Technologien und Geschäftsmodelle ältere verdrängen. 
Die Digitalisierung und insbesondere Investitionen in \ac{ICT} sind moderne Beispiele für 
diesen Prozess. Neue technologische Entwicklungen, wie Cloud Computing, \ac{AI} und 
Automatisierung haben dazu geführt, dass bestehende Arbeitsmethoden und Geschäftsprozesse 
obsolet werden, was zu einer Umstrukturierung ganzer Branchen führt 
\parencite[vgl.][S. 110]{schumpeter1976capitalism}.

Im Zusammenhang mit der Digitalisierung wird die „schöpferische Zerstörung“ zu einem 
zentralen Konzept. Während durch neue Technologien neue Arbeitsplätze entstehen, 
verschwinden gleichzeitig traditionelle Tätigkeiten. So werden bestehende Berufe und 
Qualifikationen durch neue Anforderungen ersetzt. Diese Entwicklung prägt den Arbeitsmarkt 
tiefgreifend und trägt zur Polarisierung zwischen hoch- und niedrigqualifizierten 
Arbeitskräften bei. Schumpeter zeigt, dass dieser Prozess kurzfristig zu Marktunruhe und 
Arbeitsplatzverlusten führen kann, aber langfristig eine Voraussetzung für wirtschaftliche 
Erneuerung und Wachstum ist \parencite[vgl.][S. 110–111]{schumpeter1976capitalism}.

Ein entscheidender Faktor bei der „schöpferischen Zerstörung“ ist die Fähigkeit von 
Unternehmen und Arbeitskräften, sich an neue technologische Anforderungen anzupassen. Diese 
Anpassungsfähigkeit erfordert nicht nur den Einsatz neuer Technologien, sondern auch eine 
schnelle Entwicklung neuer Fähigkeiten und Kenntnisse bei den Arbeitskräften. Schumpeter 
betont, dass nicht alle gleichermaßen von den neuen Entwicklungen profitieren. Der Übergang 
zu neuen Technologien bringt oft schmerzhafte Anpassungsprozesse mit sich, die besonders 
für weniger anpassungsfähige Arbeitskräfte herausfordernd sind 
\parencite[vgl.][S. 110–111]{schumpeter1976capitalism}.

Im Kontext der Digitalisierung gewinnen daher nicht nur Investitionen in technologische 
Infrastruktur, sondern auch in die Bildung und Ausbildung von Arbeitskräften zunehmend an 
Bedeutung. Schumpeter selbst betont, dass Innovation und die Anpassung an neue Technologien 
nicht nur von den Unternehmern, sondern auch von der gesamten Gesellschaft getragen werden 
müssen \parencite[vgl.][S. 132]{schumpeter1976capitalism}. Es geht darum, die Arbeitskräfte 
mit den notwendigen Fähigkeiten auszustatten, damit sie den Wandel aktiv mitgestalten 
können. 

Studien zeigen, dass Länder, die verstärkt in digitale Kompetenzen investieren, besser auf 
die technologischen Herausforderungen reagieren können 
\parencite[S. 15–17]{oecd2019measuring}. Investitionen in \ac{ICT} durch Unternehmen führen 
zudem zu Effizienzgewinnen und Produktivitätssteigerungen, die sich positiv auf das 
Wirtschaftswachstum und die Beschäftigung auswirken können. Schumpeter weist jedoch auch 
darauf hin, dass diese Effekte nicht automatisch und gleichmäßig auf alle Teile der 
Gesellschaft verteilt sind. Die Auswirkungen hängen stark von der Anpassungsfähigkeit der 
Arbeitskräfte und der Struktur des Bildungssystems ab \parencite[S. 48]{oecd2019measuring}.

Im Kontext von Wohlfahrtsstaaten wird Schumpeters Konzept besonders relevant. So könnten 
Institutionen, die Weiterbildung und Umschulungen fördern, den Übergang erleichtern und die 
negativen Effekte der „schöpferischen Zerstörung“ abmildern. Länder mit aktiven 
Bildungssystemen und sozialpolitischen Maßnahmen könnten besser in der Lage sein, den 
Wandel positiv zu gestalten, während weniger flexible Systeme größere Schwierigkeiten haben 
könnten, die Herausforderungen der Digitalisierung zu bewältigen 
\parencite[vgl.][S. 29–31]{espingandersen1990thethree}.

%%%%%%%%%%%%%%%%%%%%%%%%%%%%%%%%%%%%%
% Skill-biased technological change %
%%%%%%%%%%%%%%%%%%%%%%%%%%%%%%%%%%%%%

\subsection{Skill-biased technological change}

Ein zentraler Ansatz in der Debatte über die Folgen der Digitalisierung auf den 
Arbeitsmarkt ist die Theorie des \ac{SBTC}. Diese Theorie besagt, dass technologische 
Innovationen, insbesondere im Bereich der digitalen Technologien, eine 
Nachfrageverschiebung zugunsten von hochqualifizierten Arbeitskräften verursachen 
\parencite[vgl.][S. 25–26]{acemoglu2002technical}. Der Grundmechanismus hinter \ac{SBTC} 
liegt in der Tatsache, dass digitale Technologien und Automatisierungsprozesse bestimmte 
Aufgaben in Unternehmen effizienter und kostengünstiger machen. Dabei werden insbesondere 
Routineaufgaben, die vorher von mittleren Qualifikationsniveaus übernommen wurden, 
zunehmend durch Algorithmen, Maschinen oder Outsourcing ersetzt, während gleichzeitig die 
Nachfrage nach hochqualifizierten Arbeitskräften steigt 
\parencite[vgl.][S. 1282]{autor2003theskill}.

Diese Nachfrageverschiebung hat tiefgreifende Konsequenzen für den Arbeitsmarkt. 
Hochqualifizierte Arbeitnehmer*innen, die über die notwendigen digitalen und technischen 
Fähigkeiten verfügen, profitieren in der digitalen Transformation von der wachsenden 
Nachfrage. Sie können neue Technologien effektiv nutzen und sind daher nicht nur besser in 
der Lage, mit der „schöpferischen Zerstörung“ 
\parencite[vgl.][S. 81]{schumpeter1976capitalism} umzugehen, 
sondern auch von ihr zu profitieren. Gleichzeitig steigen in den Bereichen der \ac{ICT}, 
des Ingenieurwesens sowie in datenintensiven Berufen die Löhne, was den bereits bestehenden 
Einkommensunterschied zwischen Hoch- und Geringqualifizierten weiter verstärkt 
\parencite[vgl.][S. 2511]{goos2014explaining}.

Im Gegensatz dazu stehen die niedrigqualifizierten Arbeitskräfte, die mit der 
fortschreitenden Automatisierung und der Verlagerung von Arbeitsplätzen ins Ausland stärker 
gefährdet sind, durch Maschinen oder Outsourcing ersetzt zu werden. Während einige manuelle 
Tätigkeiten in Bereichen wie dem Dienstleistungssektor (z. B. Gastronomie, Pflege) 
weiterhin Bestand haben, sind vor allem industrielle Produktionsprozesse und administrative 
Büroarbeiten stark von Automatisierung betroffen \parencite[vgl.][S. 260]{frey2013thefuture}. 
Dies führt nicht nur zu Arbeitsplatzverlusten in bestimmten Sektoren, sondern verstärkt auch 
die Polarisierung des Arbeitsmarktes, indem vor allem Berufe mit mittlerem 
Qualifikationsniveau wegfallen und der Markt sich zunehmend in hochqualifizierte (besser 
bezahlte) und geringqualifizierte (schlechter bezahlte) Jobs aufteilt 
\parencite[vgl.][S. 1283]{autor2003theskill}.

Diese Polarisierung der Arbeitsmärkte, die durch \ac{SBTC} verursacht wird, hat bedeutende 
soziale und wirtschaftliche Folgen. Der technologische Fortschritt führt dazu, dass immer 
mehr einfache und routinemäßige Aufgaben durch Algorithmen und Maschinen übernommen werden, 
was zu einer strukturellen Arbeitslosigkeit in bestimmten Qualifikationsgruppen führt 
\parencite[vgl.][S. 2512]{goos2014explaining}. Gleichzeitig verstärkt sich die soziale und 
wirtschaftliche Ungleichheit, da hochqualifizierte Arbeitskräfte zunehmend gefragt sind und 
von steigenden Löhnen profitieren, während niedrigqualifizierte Arbeitnehmer*innen in 
prekären Beschäftigungsverhältnissen oder in struktureller Arbeitslosigkeit verbleiben 
\parencite[vgl.][S. 12]{arntz2016therisk}.

%%%%%%%%%%%%%%%%%%%%%
% Wohlfahrtsstaaten %
%%%%%%%%%%%%%%%%%%%%%

\subsection{Wohlfahrtsstaaten} 

Esping-Andersen (1990) beschreibt in seiner klassischen Typologie der Wohlfahrtsstaaten, 
wie institutionelle Rahmenbedingungen die Struktur und Dynamik von Arbeitsmärkten und 
Bildungssystemen prägen. Diese Typologie unterscheidet drei grundlegende Regimetypen, die 
unterschiedliche Ansätze verfolgen, um soziale Sicherheit, Beschäftigung und 
wirtschaftliche Entwicklung zu fördern \parencite[vgl.][S. 27–28]{espingandersen1990thethree}.

\begin{enumerate}

    \item \textbf{Sozialdemokratisch} (im Folgenden „nordisch“, z. B. Schweden): Diese 
    Staaten zeichnen sich durch umfassende soziale Absicherung, hohe Gewerkschaftsdichte 
    und aktive Arbeitsmarktpolitiken aus \parencite[vgl.][S. 26]{espingandersen1990thethree}. 
    Das Konzept der „Flexicurity“ kombiniert flexible Arbeitsmärkte mit starken sozialen 
    Sicherungssystemen, um Arbeitskräfte auf technologische Veränderungen vorzubereiten. 
    Zudem investieren sozialdemokratische Regime stark in Bildung und lebenslanges Lernen, 
    was die Anpassungsfähigkeit der Arbeitskräfte an die Digitalisierung fördert 
    \parencite[vgl.][S. 56]{espingandersen1990thethree}. In diesen Ländern sind digitale 
    Technologien oft tief in wirtschaftliche und administrative Prozesse integriert, während 
    gleichzeitig umfangreiche Umschulungsprogramme sicherstellen, dass Arbeitskräfte neue 
    digitale Kompetenzen erwerben können. Daher könnten nordische Staaten in der Regel 
    besser auf digitale Transformationen vorbereitet sein.

    \item \textbf{Konservativ} (im Folgenden „mitteleuropäisch“, z. B. Deutschland): Diese 
    Regime sind durch stark regulierte Arbeitsmärkte und umfassende Tarifvereinbarungen 
    gekennzeichnet, die auf Arbeitsplatzsicherheit abzielen 
    \parencite[vgl.][S. 27]{espingandersen1990thethree}. Gleichzeitig verfügen sie über 
    gut ausgebaute duale Ausbildungssysteme, die Arbeitskräfte mit praktischen und 
    technischen Fähigkeiten ausstatten und ihre Wettbewerbsfähigkeit in technologieintensiven 
    Sektoren stärken \parencite[vgl.][S. 78]{hall2001varieties}. Während die starke 
    Regulierung kurzfristig Schutz bietet, kann sie langfristig zu starren Strukturen 
    führen, die Anpassungen an digitale Disruptionen erschweren 
    \parencite[vgl.][S. 20–21]{hall2001varieties}. Dennoch ermöglicht das duale 
    Ausbildungssystem eine enge Verzahnung zwischen Bildung und Wirtschaft, was sich 
    positiv auf die Resilienz des Arbeitsmarktes gegenüber Automatisierung und 
    Digitalisierung auswirken kann \parencite[vgl.][S. 25–27]{hall2001varieties}.

    \item \textbf{Liberal} (im Folgenden „angelsächsisch“, z. B. USA): In liberalen Regimen 
    stehen marktorientierte Mechanismen im Vordergrund, während staatliche Eingriffe auf 
    ein Minimum beschränkt sind \parencite[vgl.][S. 27]{espingandersen1990thethree}. Diese 
    Systeme fördern individuelle Verantwortung und weisen geringe Arbeitsmarktregulierungen 
    auf. Während hochqualifizierte Arbeitskräfte in diesen Systemen von Digitalisierung und 
    \ac{ICT}-Investitionen profitieren, sind geringqualifizierte Arbeitskräfte aufgrund 
    fehlender Schutzmechanismen und Weiterbildungsmöglichkeiten stärker gefährdet 
    \parencite[vgl.][S. 12–13]{goodin1999thereal}. Da der technologische Fortschritt 
    weitgehend von privaten Unternehmen vorangetrieben wird, können digitale Innovationen 
    in diesen Ländern zwar schneller implementiert werden, doch die ungleiche Verteilung 
    der Anpassungsfähigkeit führt oft zu einer stärkeren Polarisierung des Arbeitsmarktes. 

\end{enumerate}

Die ursprüngliche Typologie von Esping-Andersen wurde von verschiedenen Forschern 
erweitert, um regionale Besonderheiten und neue Entwicklungen zu berücksichtigen. Zwei 
zusätzliche Wohlfahrtsstaatentypen sind besonders relevant:

\begin{enumerate}

    \item \textbf{Südeuropäisch} (z. B. Spanien): Diese Regime zeichnen sich durch 
    segmentierte Arbeitsmärkte, starre Arbeitsmarktregulierungen und eine schwache 
    Verknüpfung zwischen Bildungssystemen und Arbeitsmarkt aus 
    \parencite[S. 19]{ferrera1996thesouthern}. Schwächen in der Weiterbildung 
    und starke Ungleichheiten zwischen regulären und prekären Beschäftigungsverhältnissen 
    machen diese Länder anfälliger für negative Effekte der Digitalisierung 
    \parencite[S. 19-20]{ferrera1996thesouthern}. Da unbefristete Stellen oft starken 
    Kündigungsschutz genießen, führt dies zu einer Zweiteilung des Arbeitsmarktes, in dem 
    junge und geringqualifizierte Arbeitskräfte besonders stark von Arbeitslosigkeit 
    betroffen sind \parencite[S. 19-21]{ferrera1996thesouthern}.
    
    \item \textbf{Postsozialistisch} (z. B. Polen): Postsozialistische Länder befinden sich 
    in einem Übergang von zentral geplanten zu marktwirtschaftlichen Systemen. Sie sind 
    häufig durch geringe Regulierung und unzureichend entwickelte Bildungs- und 
    Weiterbildungsstrukturen gekennzeichnet \parencite[S. 88–93]{cerami2006socialpolicy}. 
    Diese Defizite erschweren die Integration von Arbeitskräften in digitale Sektoren und 
    verstärken die regionale Ungleichheit \parencite[S. 88–93]{cerami2006socialpolicy}. 
    Zudem sind postsozialistische Staaten oft durch eine hohe wirtschaftliche Dynamik 
    geprägt, doch der technologische Wandel verläuft nicht überall gleichmäßig. Während 
    große Städte und wirtschaftliche Zentren stark in digitale Technologien investieren, 
    bleiben ländliche Regionen oft zurück, was zu einer wachsenden Kluft zwischen digital 
    integrierten und traditionell geprägten Wirtschaftssektoren führt 
    \parencite[S. 90]{cerami2006socialpolicy}.
    
\end{enumerate}

Die institutionellen Rahmenbedingungen der verschiedenen Wohlfahrtsstaatentypen beeinflussen 
maßgeblich, wie Arbeitsmärkte auf technologische Veränderungen reagieren. Nordische Systeme 
mit umfassenden Bildungs- und Arbeitsmarktprogrammen können die negativen Effekte der 
Digitalisierung abmildern und die Integration sowohl hoch- als auch geringqualifizierter 
Arbeitskräfte fördern \parencite[S. 27–30]{espingandersen1990thethree}. Im Gegensatz dazu 
können starre Regulierungen in mittel- und südeuropäischen Regimen die Anpassung an 
digitale Transformationen verlangsamen \parencite[S. 155]{ferrera1996thesouthern}, während 
angelsächsische Regime oft eine stärkere Polarisierung zwischen Qualifikationsgruppen 
erleben \parencite[S. 3–5]{hall2001varieties}. Postsozialistische Wohlfahrtsstaaten 
kämpfen hingegen mit strukturellen Schwächen, die ihre Fähigkeit zur erfolgreichen 
Integration digitaler Technologien behindern \parencite[S. 88–93]{cerami2006socialpolicy}. 
Die digitale Transformation bringt somit unterschiedliche Herausforderungen für die 
jeweiligen Wohlfahrtsstaaten mit sich, deren Bewältigung maßgeblich von staatlicher 
Steuerung, Bildungs- und Arbeitsmarktpolitiken sowie Investitionen in digitale 
Infrastruktur abhängt \parencite[S. 23]{oecd2020digital}.

%%%%%%%%%%%%%%
% Hypothesen %
%%%%%%%%%%%%%%

\subsection{Hypothesen}

Basierend auf den theoretischen Überlegungen, insbesondere der Theorie der „schöpferischen 
Zerstörung“ von Schumpeter, sowie auf dem aktuellen Forschungsstand lassen sich im Rahmen 
dieser Arbeit mehrere Hypothesen formulieren, die empirisch überprüft werden sollen. Diese 
Hypothesen zielen darauf ab, die Auswirkungen der Digitalisierung und der Investitionen in 
\ac{ICT} auf den Arbeitsmarkt zu analysieren, insbesondere in Bezug auf unterschiedliche 
Bildungsniveaus. Dabei wird berücksichtigt, dass technologische Innovationen nicht nur 
bestehende Arbeitsplätze verdrängen, sondern auch neue berufliche Chancen eröffnen - 
abhängig von den Fähigkeiten und Qualifikationen der Arbeitskräfte.

\textbf{H1:} \textit{ Länder, in denen verstärkt in Informations- und Kommunikationstechnologien 
investiert wird, weisen eine geringere Arbeitslosenquote unter hochqualifizierten 
Arbeitskräften auf.}

Dieser Effekt ist auf die gesteigerte Produktivität und den Bedarf an komplexen, kreativen 
Fähigkeiten zurückzuführen, die durch digitale Technologien gefördert werden 
\parencite[vgl.][S. 5–8]{acemoglu2002technical}. Im Zuge der Digitalisierung entstehen neue 
Arbeitsplätze, die spezifische technologische Fähigkeiten und Kompetenzen erfordern. 
Hochqualifizierte Arbeitskräfte, die über die notwendigen digitalen und technischen 
Fähigkeiten verfügen, sind in der Lage, diese neuen Arbeitsmärkte zu bedienen. Gleichzeitig 
können sie die mit digitalen Technologien verbundenen Produktivitätssteigerungen und 
Effizienzgewinne optimal nutzen \parencite[vgl.][Kap. 2]{brynjolfsson2014thesecond}. In 
Ländern, die verstärkt in \ac{ICT} investieren, ist zu erwarten, dass diese Investitionen 
eine verstärkte Nachfrage nach hochqualifizierten Arbeitskräften erzeugen. Da die Nachfrage 
nach qualifizierten Arbeitskräften wächst und gleichzeitig das Angebot dieser Arbeitskräfte 
nicht in gleichem Maße wächst, wird die Arbeitslosenquote unter hochqualifizierten 
Arbeitskräften tendenziell sinken. Diese Entwicklung spiegelt die These der „schöpferischen 
Zerstörung“ von Schumpeter wider, nach der durch Innovationen neue Märkte entstehen, die 
speziell hochqualifizierte Fachkräfte anziehen und somit Arbeitslosigkeit in dieser Gruppe 
verringern können \parencite[vgl.][S. 103–106]{schumpeter1976capitalism}.

\textbf{H2:} \textit{In Ländern mit hohen \ac{ICT}-Investitionen verlagert sich die Arbeitslosigkeit 
auf niedrigqualifizierte Arbeitskräfte.}

Der technologische Fortschritt im Bereich der Digitalisierung führt zu einer zunehmenden 
Automatisierung und der Nutzung von Algorithmen und Maschinen in Arbeitsprozessen, die 
früher manuelle oder einfache Aufgaben erforderten. Niedrigqualifizierte Arbeitskräfte, die 
auf diese Art von Tätigkeiten angewiesen sind, sehen sich einem höheren Risiko ausgesetzt, 
durch Maschinen ersetzt zu werden \parencite[vgl.][S. 5–10]{autor2015whyare}. Gleichzeitig 
steigt die Nachfrage nach hochqualifizierten Arbeitskräften, die in der Lage sind, mit den 
neuen Technologien zu arbeiten und sie zu steuern. In Ländern, die hohe Investitionen in 
\ac{ICT} tätigen, werden diese Trends noch verstärkt, da die Automatisierung in den 
Sektoren, die viele niedrigqualifizierte Arbeitskräfte beschäftigen, schneller 
voranschreitet \parencite[vgl.][S. 254]{frey2013thefuture}. Diese Entwicklung kann dazu 
führen, dass sich die Arbeitslosigkeit stärker auf niedrigqualifizierte Gruppen verlagert, 
während hochqualifizierte Arbeitskräfte von der Digitalisierung profitieren. Im Einklang 
mit der Theorie der \ac{SBTC} wird angenommen, dass diese Verschiebung der Arbeitslosigkeit 
von niedrigqualifizierten hin zu hochqualifizierten Arbeitskräften in Ländern mit 
intensiven \ac{ICT}-Investitionen besonders ausgeprägt ist 
\parencite[vgl.][S. 3]{acemoglu2019robots}.

\textbf{H3:} \textit{Der Typ des Wohlfahrtsstaates hat Einfluss auf die Polarisierung des 
Arbeitsmarktes. Länder mit stark entwickelten wohlfahrtsstaatlichen Systemen und flexiblen 
Arbeitsmarktstrukturen zeigen eine geringere Polarisierung auf.}

Die institutionellen Rahmenbedingungen eines Landes beeinflussen maßgeblich, wie stark die 
Polarisierung des Arbeitsmarktes durch Digitalisierung ausgeprägt ist. Nordische 
Wohlfahrtsstaaten mit robusten sozialen Sicherungssystemen und umfassenden Bildungs- und 
Weiterbildungsprogrammen können die negativen Effekte der Digitalisierung abmildern 
\parencite[vgl.][S. 27–28]{espingandersen1990thethree}. Im Gegensatz dazu fördern 
angelsächsische Arbeitsmärkte, wie sie in den USA und Großbritannien vorherrschen, häufig 
eine stärkere Spaltung zwischen hoch- und niedrigqualifizierten Arbeitskräften 
\parencite[vgl.][12–13]{goodin1999thereal}. mitteleuropäische Wohlfahrtsstaaten mit stark 
regulierten Arbeitsmärkten (z. B. Deutschland, Frankreich) bieten zwar Schutzmechanismen, 
können aber die Integration von geringqualifizierten Arbeitskräften erschweren 
\parencite[vgl.][S. 78]{hall2001varieties}. Südeuropäische Wohlfahrtsstaaten hingegen 
verstärken durch starre Regulierungen und segmentierte Arbeitsmärkte die Polarisierung 
\parencite[vgl.][S. 17–37]{ferrera1996thesouthern}.

% TODO: - (Stand 08.03.2025)

\section{Daten und Methodik}

%%%%%%%%%%%%%%
% Datensätze %
%%%%%%%%%%%%%%

\subsection{Datensätze}

Die vorliegenden Daten stammen aus den umfangreichen Datensätzen der \ac{OECD}, einer 
internationalen Organisation, die vergleichbare Wirtschafts- und Sozialstatistiken für ihre 
Mitgliedsländer bereitstellt. Die \ac{OECD} sammelt und veröffentlicht regelmäßig Daten zu 
wirtschaftlichen, sozialen und technologischen Entwicklungen, die es ermöglichen, langfristige 
Trends und länderspezifische Unterschiede zu analysieren \parencite{oecd2022ict}.

Für diese Untersuchung werden insbesondere die Datensätze zu \ac{ICT}-Investitionen 
\parencite{oecd2022ict} sowie zu den Arbeitslosenquoten nach Bildungsniveau 
\parencite{oecd2022unemployment} verwendet. Zusätzlich wurden weitere ökonomische und 
institutionelle Indikatoren als Kontrollvariablen integriert, um die Robustheit der Analyse 
zu erhöhen. Dazu gehören das \ac{BIP} pro Kopf \parencite{oecd2022gdp}, die 
Gewerkschaftsdichte \parencite{oecd2022tud}, der Anteil der Bevölkerung mit tertiärem 
Bildungsabschluss \parencite{oecd2022education} sowie der Grad der Regulierung des 
Arbeitnehmerschutzes \parencite{oecd2022regulation}. Zudem wird die Wohlfahrtsstaatentypologie 
nach Esping-Andersen \parencite{espingandersen1990thethree} genutzt, um institutionelle 
Unterschiede zwischen den Ländern zu erfassen. Die Wohlfahrtsstaaten-Variable wird separat 
betrachtet, da sie nichtnur eine Kontrollvariable darstellt, sondern auch eine eigene 
Hypothese in der Analyse testet. Die finalen Daten umfassen insgesamt 35 OECD- und 
ausgewählte Nicht-OECD-Länder
\footnote{
    Untersuchte Länder: Australien, Österreich, Belgien, Bulgarien, Brasilien, Kanada, 
    Kroatien, Tschechien, Dänemark, Estland, Finnland, Frankreich, Deutschland, Griechenland, 
    Ungarn, Island, Italien, Irland, Lettland, Litauen, Luxemburg, Niederlande, Neuseeland, 
    Norwegen, Polen, Portugal, Rumänien, Spanien, Schweden, Schweiz, Türkei, Slowakei, 
    Slowenien, Vereinigtes Königreich, USA.
} und decken den Zeitraum von 2005 bis 2022 ab. Nach 
der Bereinigung und Zusammenführung der relevanten Variablen verbleiben 3973 Beobachtungen 
für die Paneldatenanalyse.

Die \ac{ICT}-Investitionen messen die Bruttoanlageinvestitionen in digitale Infrastrukturen 
und Technologien \parencite{oecd2022ict}. Die Arbeitsmarktstatistiken bieten detaillierte 
Informationen über die Arbeitslosenquoten in verschiedenen Bildungsgruppen 
\parencite{oecd2022unemployment}. Durch die Ergänzung um Kontrollvariablen wie den Anteil 
tertiär gebildeter Personen und die \textit{Regulierungsstrenge des Arbeitsmarktes} wird 
sichergestellt, dass sowohl wirtschaftliche als auch institutionelle Unterschiede in den 
Ländern angemessen berücksichtigt werden. Zur Sicherstellung einer vollständigen Zeitreihe 
wurden fehlende Werte der Variablen \textit{Gewerkschaftsdichte}, 
\textit{Tertiärer Bildungsanteil} und \textit{Arbeitsmarktregulierung} mittels linearer 
Inter- und Extrapolation ergänzt. Zudem wurde die Variable \textit{\ac{BIP} pro Kopf} zur 
besseren Interpretierbarkeit durch 1000 geteilt.

%%%%%%%%%%%%%%%%%%%%%%%
% Operationalisierung %
%%%%%%%%%%%%%%%%%%%%%%%

\subsection{Operationalisierung}

Zur Beantwortung der Forschungsfrage - wie Investitionen in \ac{ICT} die Arbeitslosenquoten 
in unterschiedlichen Bildungsniveaus beeinflussen - ist eine präzise und konsistente 
Operationalisierung der zentralen Konzepte notwendig. Dies gewährleistet, dass die 
Untersuchung die beabsichtigten Zusammenhänge abbildet und die Daten sinnvoll ausgewertet 
werden können.

Die abhängige Variable dieser Untersuchung ist die \textit{Arbeitslosenquote} 
(UNEMPLOYMENT\_RATE\_PERCENT), die nach dem Bildungsniveau der Bevölkerung differenziert 
wird. Der \ac{OECD}-Datensatz unterteilt das Bildungsniveau in drei Hauptkategorien:

\begin{enumerate}

    \item \textbf{niedriges Bildungsniveau} (Low education): Personen ohne abgeschlossene 
    Schulbildung oder mit einem maximalen Hauptschulabschluss \parencite{oecd2022unemployment}.

    \item \textbf{mittleres Bildungsniveau} (Medium education): Personen mit 
    Sekundarschulabschluss oder einer abgeschlossenen Berufsausbildung 
    \parencite{oecd2022unemployment}.

    \item \textbf{hohes Bildungsniveau} (High education): Personen mit Hochschulabschluss, wie 
    einem Bachelor, Master oder Doktortitel \parencite{oecd2022unemployment}.

\end{enumerate}

Arbeitslose sind nach der Definition der \ac{OECD} Personen im erwerbsfähigen Alter, die keine 
Arbeit haben, für eine Arbeit zur Verfügung stehen und in den letzten vier Wochen konkrete 
Schritte unternommen haben, um eine Arbeit zu finden \parencite{oecd2022unemployment}. Dieser 
Indikator wird als Prozentsatz der Erwerbsbevölkerung gemessen und ist saisonbereinigt.

Die zentrale unabhängige Variable \textit{\ac{ICT}-Investitionen} (ICT\_INVEST\_SHARE\_GDP) 
misst Investitionen in digitale Infrastruktur, Software, Hardware und Technologien, die zur 
Verbesserung betrieblicher Effizienz und Produktivität beitragen \parencite{oecd2022ict}. Die 
Daten basieren auf den Definitionen des \ac{SNA08} und werden als Anteil am \ac{BIP} in 
Prozent angegeben.

Um sicherzustellen, dass der Effekt der \textit{\ac{ICT}-Investitionen} auf die 
\textit{Arbeitslosenquote} nicht durch andere Faktoren verzerrt wird, werden mehrere 
Kontrollvariablen in die Analyse aufgenommen:

\begin{itemize}
    \item \textbf{\ac{BIP} pro Kopf} (GDP\_PER\_CAPITA): Diese Variable misst den 
    wirtschaftlichen Wohlstand eines Landes in tausend US-Dollar pro Jahr und kontrolliert 
    den Entwicklungsstand eines Landes, da wirtschaftlich wohlhabendere Länder tendenziell 
    niedrigere Arbeitslosenquoten aufweisen \parencite{oecd2022gdp}.

    \item \textbf{Gewerkschaftsdichte} (PERCENT\_EMPLOYEES\_TUD): Der Anteil der in 
    Gewerkschaften organisierten Arbeitnehmer wird berücksichtigt, da Gewerkschaften eine 
    wichtige Rolle bei der Aushandlung von Arbeitsbedingungen und Arbeitsplatzsicherheit 
    spielen \parencite{oecd2022tud}. Frühere Studien zeigen, dass eine hohe Gewerkschaftsdichte 
    oft mit niedrigeren Arbeitslosenquoten für geringqualifizierte Arbeitnehmer verbunden ist, 
    da Gewerkschaften Mindestlöhne sichern und Beschäftigungsschutzmaßnahmen verstärken 
    \parencite[S. 61]{nickell1997unemployment}.

    \item \textbf{Tertiärer Bildungsanteil} (PERCENT\_TERTIARY\_EDUCATION): Diese 
    Variable gibt den Prozentsatz der Bevölkerung an, der einen tertiären Bildungsabschluss 
    besitzt. Eine höhere Bildungsbeteiligung könnte den negativen Einfluss von 
    \ac{ICT}-Investitionen auf geringqualifizierte Arbeitnehmer abschwächen, da mehr Menschen 
    für technologische Berufe qualifiziert sind \parencite{oecd2022education}.

    \item \textbf{Arbeitsmarktregulierung} (REGULATION\_STRICTNESS): Diese 
    Variable misst, wie stark der Arbeitsmarkt eines Landes reguliert ist, insbesondere im 
    Hinblick auf Kündigungsschutz und Beschäftigungsflexibilität. Striktere Regulierung kann 
    die Arbeitslosenquote erhöhen, da Unternehmen zögerlicher bei Neueinstellungen sind 
    \parencite{oecd2022regulation}.

    \item \textbf{Wohlfahrtsstaatentyp} (WELFARE\_STATE): Die Typologie nach Esping-Andersen 
    \parencite{espingandersen1990thethree} wird in die Analyse integriert, um zu untersuchen, 
    inwiefern institutionelle Unterschiede den Effekt von \textit{\ac{ICT}-Investitionen} auf 
    die \textit{Arbeitslosenquote} beeinflussen. Die Länder werden in der Analyse in fünf 
    Kategorien unterteilt
    \footnote{
        Die Zuordnung erfolgt wie folgt: 
        \textit{nordisch} (Dänemark, Schweden, Norwegen, Finnland, Island); 
        \textit{mitteleuropäisch} (Deutschland, Frankreich, Österreich, Belgien, Niederlande, 
        Luxemburg, Schweiz); 
        \textit{angelsächsisch} 
        (Vereinigte Staaten, Vereinigtes Königreich, Kanada, Australien, Neuseeland, Irland); 
        \textit{südeuropäisch} (Italien, Spanien, Portugal, Griechenland); 
        \textit{postsozialistisch} (Polen, Tschechien, Ungarn, Slowakei, Slowenien, Estland, 
        Lettland, Litauen, Rumänien, Bulgarien).
    }: \textit{nordisch}, \textit{mitteleuropäisch}, \textit{angelsächsisch}, 
    \textit{südeuropäisch} und \textit{postsozialistisch}. Diese Kontrollvariable wird 
    gesondert als eigene Hypothese getestet.

\end{itemize}

Die Kombination dieser Daten ermöglicht es, länderspezifische Unterschiede in der 
Wirtschaftskraft, den regulatorischen Rahmenbedingungen und der Bildungsstruktur zu 
kontrollieren, um Zusammenhänge zwischen \textit{\ac{ICT}-Investitionen} und 
\textit{Arbeitslosenquote} differenziert zu analysieren. Zudem erlauben die aufgenommenen 
Kontrollvariablen eine differenziertere Betrachtung institutioneller Faktoren, die den 
Arbeitsmarkt beeinflussen. 

%%%%%%%%%%%%%%%%%%%%%%%
% Analytische Methode %
%%%%%%%%%%%%%%%%%%%%%%%

\subsection{Analytische Methode}

Die Analyse dieser Arbeit basiert auf einer Paneldatenanalyse, um die Auswirkungen von 
\textit{\ac{ICT}-Investitionen} auf die \textit{Arbeitslosenquote} nach Bildungsniveau zu 
untersuchen. Die Wahl einer Paneldatenmethode ermöglicht es, sowohl individuelle Heterogenität 
zwischen Ländern als auch dynamische Entwicklungen über die Zeit zu erfassen 
\parencite{wooldridge2010econometric}. 

Für diese Analyse wurden \ac{FE}-Modelle gewählt, da sie eine robustere Schätzung der 
Zusammenhänge zwischen \textit{\ac{ICT}-Investitionen} und der \textit{Arbeitslosenquote} 
ermöglichen. Dies ist besonders relevant, da die Untersuchung auf Veränderungen innerhalb 
eines Landes über die Zeit fokussiert und länderspezifische Eigenschaften nicht als erklärende 
Variablen modelliert werden. \ac{RE}-Modelle werden aufgrund der potenziellen Korrelation 
zwischen länderspezifischen Effekten und den unabhängigen Variablen nicht verwendet, da es in 
diesem Fall zu verzerrten Schätzungen führen könnte 
\parencite[S. 251–256]{wooldridge2010econometric}. 

Darüber hinaus wird die Analyse durch Interaktionseffekte ergänzt, die es ermöglichen, 
institutionelle Unterschiede zwischen den Ländern zu berücksichtigen. Die Modelle beinhalten 
eine Variable für den \textit{Wohlfahrtsstaatentyp}, um systematische Unterschiede zwischen den 
Regimetypen in ihrer Reaktion auf \textit{\ac{ICT}-Investitionen} zu identifizieren 
\parencite{espingandersen1990thethree}. Zusätzlich werden Jahres-Fixed Effects integriert, um 
allgemeine makroökonomische Entwicklungen (z. B. Finanzkrisen oder technologische Schübe) aus 
den Modellen zu kontrollieren. Die Interaktion zwischen \textit{\ac{ICT}-Investitionen} und 
\textit{Wohlfahrtsstaatentyp} ermöglicht eine differenzierte Analyse der moderierenden Effekte 
institutioneller Rahmenbedingungen.

Durch diese Kombination aus \ac{FE}-Modellen, Interaktionseffekten und Zeitdummies bietet die
Paneldatenanalyse eine solide Grundlage für die Untersuchung der Auswirkungen von 
\textit{\ac{ICT}-Investitionen} auf die \textit{Arbeitslosenquote}. Die longitudinale Struktur 
der Daten erlaubt eine differenzierte Betrachtung, die sowohl kurzfristige als auch 
langfristige Beschäftigungseffekte berücksichtigt. 

\section{Ergebnisse}

%%%%%%%%%%%%%%%%%%%%%%%%%%%%%%%%%%%%%%
% Ergebnisse: Deskriptive Ergebnisse %
% TODO:                              %
%%%%%%%%%%%%%%%%%%%%%%%%%%%%%%%%%%%%%%

\subsection{Deskriptive Ergebnisse}

% Deskriptive Statistik: Einleitung
Die deskriptiven Statistiken der analysierten Variablen bieten einen umfassenden Einblick in 
deren Eigenschaften und Verteilungen über die beobachteten Länder und Zeiträume. Im Folgenden 
werden die Ergebnisse detailliert beschrieben: 

% Variablenbeschreibung
% TODO: - (Stand 04.03.2025)

\begin{table}[H]
\caption{Übersicht über die Variablen}
\resizebox{\textwidth}{!}{
\centering
\begin{tabular}[t]{lrrrrrr}
\toprule
Variable & Min & Max & Mean & Median & SD & N\\
\midrule
UNEMPLOYMENT\_RATE\_PERCENT & 0.82 & 49.89 & 7.95 & 5.96 & 6.34 & 11919\\
ICT\_INVEST\_SHARE\_GDP & 0.73 & 8.69 & 2.46 & 2.25 & 0.98 & 11919\\
GDP\_PER\_CAPITA & 13.34 & 137.72 & 43.73 & 41.27 & 17.13 & 11919\\
PERCENT\_EMPLOYEES\_TUD & 4.50 & 92.20 & 28.45 & 20.40 & 20.71 & 11919\\
PERCENT\_TERTIARY\_EDUCATION & 12.87 & 59.96 & 33.65 & 34.56 & 9.27 & 11919\\
\addlinespace
REGULATION\_STRICTNESS & 0.00 & 4.88 & 2.19 & 2.26 & 0.83 & 11919\\
\bottomrule
\end{tabular}
}
\label{tab:variables_codebook}
\end{table}


% Variable "Arbeitslosenquote"
Die Variable \textit{Arbeitslosenquote} schwankt erheblich zwischen einem Minimum von 0,82\% und 
einem Maximum von 49,89\%. Der Mittelwert liegt bei 7,95\%, während der Median mit 5,96\% etwas 
niedriger ausfällt. Dies weist auf eine rechtsschiefe Verteilung hin, da einige Länder oder 
Zeitpunkte mit sehr hohen Arbeitslosenquoten als Ausreißer wirken können. Die hohe 
Standardabweichung von 6,34 deutet darauf hin, dass die Arbeitslosenquoten zwischen den Ländern 
und über die Zeit hinweg erhebliche Unterschiede aufweisen. Während einige \ac{OECD}-Länder durch 
eine geringe Arbeitslosenquote und stabile Arbeitsmärkte gekennzeichnet sind, zeigen andere 
Länder insbesondere in wirtschaftlichen Krisenzeiten oder strukturschwachen Regionen signifikant 
höhere Arbeitslosenraten. Diese Heterogenität könnte zudem mit unterschiedlichen 
Arbeitsmarktregulierungen und Bildungssystemen zusammenhängen. 

% Variable "ICT-Investitionen"
Die Variable \textit{\ac{ICT}-Investitionen} variiert zwischen einem Minimum von 0,73\% und einem 
Maximum von 8,69\%. Der Mittelwert beträgt 2,46\%, während der Median mit 2,25\% leicht darunter 
liegt. Dies deutet auf eine leicht rechtsschiefe Verteilung hin, da einige Länder besonders hohe 
Investitionen in \ac{ICT} tätigen. Die Standardabweichung von 0,98 zeigt, dass es zwischen den 
\ac{OECD}-Ländern erhebliche Unterschiede in der Intensität der \textit{\ac{ICT}-Investitionen} 
gibt. Während einige Länder konstant hohe Anteile ihrer wirtschaftlichen Ressourcen in digitale 
Technologien investieren, gibt es andere, die vergleichsweise geringe Investitionen tätigen. 
Diese Unterschiede können durch verschiedene Faktoren beeinflusst sein, darunter wirtschaftliche 
Leistungsfähigkeit, politische Strategien zur Förderung der Digitalisierung sowie strukturelle 
Unterschiede in der Entwicklung des \ac{ICT}-Sektors. 

% Variable "BIP pro Kopf"
Das \textit{\ac{BIP} pro Kopf} weist eine erhebliche Spannweite auf: Es reicht von 13,34 bis 
137,72 Tausend US-Dollar. Der Mittelwert beträgt 43,73 Tausend US-Dollar, während der Median mit 
41,27 Tausend US-Dollar nur geringfügig darunter liegt. Trotz dieser relativen Nähe deutet die 
hohe Standardabweichung von 17,13 darauf hin, dass es erhebliche Wohlstandsunterschiede zwischen 
den \ac{OECD}-Ländern gibt. Dies spricht für eine starke rechtsschiefe Verteilung, da einige 
besonders wohlhabende Länder den Durchschnittswert nach oben treiben. Diese Unterschiede sind 
insbesondere für die Interpretation der \textit{\ac{ICT}-Investitionen} relevant, da 
wohlhabendere Länder tendenziell eine höhere Kapitalausstattung und damit größere 
Investitionsmöglichkeiten in digitale Infrastruktur haben könnten. Gleichzeitig könnten 
Unterschiede im \textit{\ac{BIP} pro Kopf} Einfluss auf die Struktur des Arbeitsmarktes und 
damit auf die Verteilung der \textit{Arbeitslosenquote} nach Bildungsgrad haben. 

% Variable "Gewerkschaftsdichte"
Die Variable \textit{Gewerkschaftsdichte} zeigt eine erhebliche Varianz zwischen den Ländern. Die 
Werte reichen von einem Minimum von 4,50\% bis zu einem Maximum von 92,20\%. Der Mittelwert 
beträgt 28,45\%, während der Median mit 20,40\% darunter liegt, was darauf hindeutet, dass einige 
Länder eine besonders hohe Gewerkschaftsbindung haben, während die Mehrheit unter diesem 
Durchschnittswert bleibt. Die Standardabweichung von 20,71 verdeutlicht die große Heterogenität 
in der gewerkschaftlichen Organisation zwischen den Ländern. 

% Variable "Anteil tertiär Gebildeter"
Die Variable \textit{Tertiärer Bildungsanteil} (PERCENT\_TERTIARY\_EDUCATION) variiert zwischen 
12,87\% und 59,96\%, mit einem Mittelwert von 33,65\% und einer Standardabweichung von 9,27. 
Länder mit höheren Werten verfügen tendenziell über eine stärker wissensbasierte Wirtschaft, was 
sich positiv auf die Integration von Arbeitnehmern in technologische Sektoren auswirken kann. 
Gleichzeitig könnte eine höhere Bildungsbeteiligung dazu beitragen, die negativen Effekte der 
Digitalisierung für geringqualifizierte Arbeitskräfte abzufedern. 

% Variable "Arbeitsmarktregulierung"
Die Variable \textit{Arbeitsmarktregulierung} (REGULATION\_STRICTNESS) reicht von 0,00 bis 4,88, 
mit einem Mittelwert von 2,19 und einer Standardabweichung von 0,83. Diese Unterschiede spiegeln 
unterschiedliche Arbeitsmarktpolitiken wider: Während in einigen Ländern hohe Regulierung den 
Kündigungsschutz stärkt, kann dies gleichzeitig die Schaffung neuer Arbeitsplätze hemmen. 

% Variable "Wohlfahrtsstaatentyp"
\begin{table}[H]
\centering
\caption{Übersicht über die Verteilung der Wohlfahrtsstaatentypen}
\centering
\begin{tabular}[t]{lrr}
\toprule
Kategorie & Anzahl & Prozent\\
\midrule
Anglo-Saxon & 2382 & 19.98\\
Central European & 2943 & 24.69\\
Nordic & 2034 & 17.07\\
Other & 0 & 0.00\\
Post-socialist & 3015 & 25.30\\
\addlinespace
Southern European & 1545 & 12.96\\
\bottomrule
\end{tabular}
\end{table}


Die Verteilung der Variable \textit{Wohlfahrtsstaatentyp} zeigt, dass postsozialistische Länder 
mit 25,30\% die größte Gruppe innerhalb der Stichprobe ausmachen, gefolgt von mitteleuropäischen 
Wohlfahrtsstaaten mit 24,69\%. Nordische und angelsächsische Länder sind mit 17,07\% bzw. 19,98\% 
ebenfalls vertreten, während südeuropäische Länder mit 12,96\% den kleinsten Anteil ausmachen. 
Die Kategorie \textit{Other} ist in der vorliegenden Stichprobe nicht besetzt und dient lediglich 
als Indikator für übrige Länder, welche keiner Ausprägung der Variable zugeteilt wurden.

Diese Klassifikation ist insbesondere für die spätere Analyse der Interaktionseffekte relevant, 
da sie Aufschluss darüber geben kann, inwiefern institutionelle Rahmenbedingungen den Einfluss 
von \textit{\ac{ICT}-Investitionen} auf die \textit{Arbeitslosenquote} moderieren.

% Variablen (Zusammenfassung)
Die deskriptiven Statistiken zeigen, dass die betrachteten Variablen eine erhebliche 
Heterogenität aufweisen, die sowohl auf länderspezifische Unterschiede als auch auf strukturelle 
und wirtschaftliche Faktoren zurückgeführt werden kann. Besonders auffällig sind die Unterschiede 
in den \textit{\ac{ICT}-Investitionen}, die je nach wirtschaftlicher Leistungsfähigkeit und 
politischen Rahmenbedingungen stark variieren. Auch die \textit{Arbeitslosenquote} zeigt eine 
hohe Streuung, die möglicherweise mit den unterschiedlichen Bildungsniveaus, 
Arbeitsmarktinstitutionen und Wirtschaftsentwicklungen der Länder zusammenhängt. Die hohe Varianz 
im \textit{\ac{BIP} pro Kopf} unterstreicht die unterschiedlichen wirtschaftlichen 
Ausgangsbedingungen der Länder, was sich sowohl auf die Höhe der \textit{\ac{ICT}-Investitionen} 
als auch auf die Struktur der Arbeitsmärkte auswirken könnte. Schließlich zeigt die 
\textit{Gewerkschaftsdichte} ebenfalls starke Unterschiede zwischen den \textit{OECD}-Ländern, 
was für die Analyse der institutionellen Faktoren relevant ist, die möglicherweise als 
Moderatoren der Auswirkungen von \ac{ICT}-Investitionen auf den Arbeitsmarkt fungieren.

% Grafiken
Diese deskriptive Analyse der Variablen bildet die Grundlage für die nachfolgenden grafischen 
Darstellungen, die eine detailliertere Visualisierung der Trends und Unterschiede zwischen den 
Ländern ermöglichen. Sie bietet einen ersten Einblick auf Länderebene in die Beziehung zwischen 
\textit{\ac{ICT}-Investitionen} (gemessen als Anteil am \textit{\ac{BIP} pro Kopf}) und den 
\textit{Arbeitslosenquote}, differenziert nach den drei genannten Bildungsgruppen. Hierbei wird 
jeweils ein repräsentatives Land pro \textit{Wohlfahrtsstaatentyp} für die Analyse gewählt - 
Spanien als südeuropäischer, Polen als postsozialistischer, Schweden als nordischer und 
Deutschland als mitteleuropäischer Wohlfahrtsstaat im Zeitraum von 2005 bis 2022. 

Um die Lesbarkeit der Grafiken zu verbessern, wurden die Ländernamen in den Diagrammen 
automatisch in die deutsche Sprache übersetzt. Dies stellt sicher, dass die visuelle Darstellung 
konsistent mit der Textanalyse bleibt. Ziel ist es, vor der multivariaten Analyse bereits 
Unterschiede und Trends innerhalb der Länder und zwischen den Bildungsgruppen zu identifizieren.

% Grafik "Spanien"
\begin{figure}[htbp]
    \centering
    \caption{Überblick über \textit{\ac{ICT}-Investitionen} und \textit{Arbeitslosenquote} in 
    Spanien}
    \includegraphics[width=\textwidth]{assets/plot_spain_final.png}
    \label{fig:spain}
\end{figure}

Die Abbildung zeigt die Entwicklung der \textit{\ac{ICT}-Investitionen} als Anteil am \ac{BIP} 
pro Kopf sowie die \textit{Arbeitslosenquote} in Spanien zwischen 2005 und 2022, differenziert 
nach Bildungsniveau - Spanien steht hier repräsentativ für südeuropäische Wohlfahrtsstaaten. Am 
Beispiel Spaniens ist ein besonders markanter Anstieg der \textit{Arbeitslosenquote} während der 
Finanz- und Wirtschaftskrise von 2008 bis 2013 zu beobachten. Während die 
\textit{\ac{ICT}-Investitionen} einen insgesamt moderaten Anstieg über den gesamten Zeitraum 
hinweg zeigen, lassen sich drastische Schwankungen in der \textit{Arbeitslosenquote} 
identifizieren, insbesondere bei Personen mit niedrigem und mittlerem Bildungsniveau.

Bei Personen mit einem niedrigen Bildungsniveau zeigt sich zwischen 2005 und 2008 eine relativ 
stabile \textit{Arbeitslosenquote} von knapp unter 10\%. Ab 2008 kam es jedoch zu einem rasanten 
Anstieg, der bis 2013 einen Höchststand von über 30\% erreichte. Erst nach 2013 begann ein 
kontinuierlicher Rückgang, der sich bis 2020 auf etwa 20\% fortsetzte, bevor ein erneuter 
leichter Anstieg zu beobachten ist. Die \textit{\ac{ICT}-Investitionen} entwickelten sich 
hingegen gleichmäßiger. Sie begannen auf einem niedrigen Niveau von etwa 1,75\% des BIP, zeigten 
nach der Finanzkriese ab 2010 eine Aufwärtstendenz und stabilisierten sich nach 2015 bei etwa 
2,5\%. Der Rückgang der \textit{Arbeitslosenquote} nach 2013 verlief jedoch unabhängig von einer 
abrupten Zunahme der \textit{\ac{ICT}-Investitionen}, was darauf hindeutet, dass makroökonomische 
Faktoren (z. B. wirtschaftliche Erholung, Beschäftigungsprogramme) für die Senkung der 
Arbeitslosigkeit eine zentrale Rolle spielten.

Bei Personen mit mittlerem Bildungsniveau zeigt sich ein sehr ähnlicher Verlauf. Die 
\textit{Arbeitslosenquote} lag 2005 noch unter 8\%, stieg im Zuge der Wirtschaftskrise bis 2013 
jedoch auf über 20\% an. Erst ab 2014 begann ein deutlicher Rückgang, der sich bis 2020 auf etwa 
10\% fortsetzte. Die \textit{\ac{ICT}-Investitionen} folgten hier einem vergleichbaren Muster wie 
in der Gruppe der gering Qualifizierten, wobei ein leichter, aber kontinuierlicher Anstieg 
sichtbar ist. Dennoch ist keine direkte Korrelation zwischen dem Verlauf der 
\textit{\ac{ICT}-Investitionen} und der \textit{Arbeitslosenquote} ersichtlich, da der massive 
Anstieg und der spätere Rückgang der Arbeitslosigkeit primär durch die wirtschaftliche 
Entwicklung und nicht durch technologische Investitionen bedingt zu sein scheinen.

Bei Personen mit hohem Bildungsniveau war die \textit{Arbeitslosenquote} insgesamt niedriger, 
zeigte jedoch ebenfalls einen deutlichen Anstieg während der Wirtschaftskrise. Im Jahr 2005 lag 
sie unter 5\%, erreichte 2013 jedoch fast 15\%. Danach setzte auch hier ein Rückgang ein, und bis 
2020 fiel die \textit{Arbeitslosenquote} auf etwa 5\% zurück. Im Gegensatz zu den anderen 
Bildungsgruppen scheinen sich hier die \textit{\ac{ICT}-Investitionen} und die 
\textit{Arbeitslosenquote} teilweise gegenläufig zu entwickeln. Während die 
\textit{\ac{ICT}-Investitionen} nach 2010 eine stetige Steigerung zeigen und nach 2015 stabil auf 
etwa 2,5\% des BIP bleiben, geht die \textit{Arbeitslosenquote} in derselben Phase zurück. Dies 
könnte darauf hindeuten, dass hochqualifizierte Arbeitskräfte in Spanien stärker von der 
Digitalisierung profitieren konnten als Personen mit niedrigerem Bildungsstand.

Spanien als südeuropäischer Wohlfahrtsstaat ist durch einen stark segmentierten Arbeitsmarkt 
gekennzeichnet, der sich durch hohe Anteile an befristeten Beschäftigungsverhältnissen sowie eine 
geringere Arbeitsplatzsicherheit auszeichnet \parencite[vgl.][S. F160–F163]{bentolila2012two}. 
Dies könnte eine Erklärung für die starken Schwankungen der \textit{Arbeitslosenquote} im Zuge 
der Finanzkrise sein, da insbesondere gering und mittel Qualifizierte von Entlassungen betroffen 
waren. Die \textit{\ac{ICT}-Investitionen} scheinen langfristig zwar leicht anzusteigen, doch 
zeigt sich kein direkter Zusammenhang zwischen diesen Investitionen und der 
\textit{Arbeitslosenquote} in den jeweiligen Bildungsgruppen. Vielmehr deutet die Entwicklung 
darauf hin, dass der Arbeitsmarkt in Spanien stark konjunkturabhängig ist und die wirtschaftliche 
Erholung nach 2013 die wichtigste Triebkraft für die Reduktion der Arbeitslosigkeit war.

Zusammenfassend zeigen die Daten für Spanien eine enge Verbindung zwischen der Finanzkrise und 
den massiven Schwankungen der \textit{Arbeitslosenquote}, insbesondere bei gering und mittel 
Qualifizierten. Während \textit{\ac{ICT}-Investitionen} über den Zeitraum hinweg einen 
kontinuierlichen, aber moderaten Anstieg zeigen, sind ihre direkten Auswirkungen auf die 
Arbeitslosigkeit unklar. Es könnte jedoch sein, dass insbesondere Hochqualifizierte von den 
steigenden \textit{\ac{ICT}-Investitionen} profitieren konnten, während gering Qualifizierte eher 
von konjunkturellen Faktoren abhängig waren.

% Grafik "Polen"
\begin{figure}[htbp]
    \centering
    \caption{Überblick über \textit{\ac{ICT}-Investitionen} und \textit{Arbeitslosenquote} in 
    Polen}
    \includegraphics[width=\textwidth]{assets/plot_poland_final.png}
    \label{fig:poland}
\end{figure}

Die Abbildung zeigt die Entwicklung der \textit{\ac{ICT}-Investitionen} als Anteil am 
\ac{BIP} sowie die \textit{Arbeitslosenquote} in Polen zwischen 2005 und 2022 differenziert nach 
Bildungsniveau - Polen steht hier repräsentativ für postsozialistische Wohlfahrtsstaaten.

Auffällig ist der durchgängige Rückgang der \textit{Arbeitslosenquote} in allen Bildungsgruppen, 
während die \textit{\ac{ICT}-Investitionen} über weite Strecken konstant bleiben, beziehungsweise 
sogar ebenfalls einen Rückgang verzeichnen. Dies deutet darauf hin, dass makroökonomische oder 
arbeitsmarktpolitische Faktoren für den Rückgang der Arbeitslosigkeit maßgeblich verantwortlich 
sein könnten.

Bei Personen mit einem niedrigen Bildungsniveau lag die \textit{Arbeitslosenquote} im Jahr 2005 
bei knapp 28\%. In den darauffolgenden Jahren kam es zu einem raschen Rückgang, wobei jedoch 
zwischen 2010 und 2015 eine Stagnation mit einem kurzen Anstieg auf fast 20\% zu beobachten ist. 
Nach 2015 setzte sich der Rückgang der \textit{Arbeitslosenquote}  fort, sodass sie bis 2020 auf 
8\% fiel. Die \textit{\ac{ICT}-Investitionen} blieben über den gesamten Zeitraum hinweg 
weitgehend konstant und bewegten sich um die 1\% des BIP, mit einem leichten Rückgang zwischen 
2010 und 2015. Dies deutet darauf hin, dass  der starke Rückgang der Arbeitslosigkeit nicht 
direkt mit den \textit{\ac{ICT}-Investitionen} zusammenhängt, sondern durch andere 
wirtschaftliche Faktoren beeinflusst wurde, beispielsweise durch eine allgemeine 
wirtschaftliche Stabilisierung nach dem EU-Beitritt Polens und steigende 
Beschäftigungsmöglichkeiten in arbeitsintensiven Branchen.

Für Personen mit einem mittleren Bildungsniveau zeigt sich ein ähnliches Muster, wenn auch auf 
einem insgesamt niedrigeren Ausgangsniveau der \textit{Arbeitslosenquote}. Während diese 2005 
noch über 10\% lag, sank sie in den darauffolgenden Jahren rasch auf etwa 3\% bis 2015 und weiter 
unter 2\% bis 2020. Zwischen 2010 und 2015 ist jedoch eine leichte Erhöhung der 
\textit{Arbeitslosenquote} erkennbar, bevor der Trend weiter nach unten verlief. Der Rückgang der 
Arbeitslosigkeit erfolgt weitgehend unabhängig von der Entwicklung der 
\textit{\ac{ICT}-Investitionen}, was darauf hindeutet, dass makroökonomische Faktoren wie die 
Industrialisierung und eine steigende Nachfrage nach Arbeitskräften mit mittlerer Qualifikation 
eine bedeutendere Rolle gespielt haben könnten.

Für Personen mit einem hohen Bildungsniveau war die \textit{Arbeitslosenquote} bereits 2005 
relativ niedrig, lag aber dennoch bei etwa 6\%, was im Vergleich zu anderen europäischen Ländern 
eher hoch ist. Dies könnte auf strukturelle Faktoren des polnischen Arbeitsmarktes zurückzuführen 
sein, wie eine geringere Anzahl hochqualifizierter Beschäftigungsmöglichkeiten in den frühen 
2000er-Jahren. In den darauffolgenden Jahren fiel die \textit{Arbeitslosenquote} jedoch deutlich 
und lag bereits 2015 unter 2\%. Auffällig ist, dass die \textit{\ac{ICT}-Investitionen} in dieser 
Gruppe im Gegensatz zu den anderen Bildungsgruppen eine leichte Steigerung zeigen. In der ersten 
Hälfte des Beobachtungszeitraums bewegten sich die \textit{\ac{ICT}-Investitionen} um 1,2\% des 
BIP, während sie in den Jahren nach 2015 tendenziell anstiegen. Dies könnte darauf hindeuten, 
dass der polnische Arbeitsmarkt mit steigenden \textit{\ac{ICT}-Investitionen} zunehmend 
hochqualifizierte Beschäftigungsmöglichkeiten geschaffen hat. Dennoch bleibt die Kausalität 
unklar, da die \textit{Arbeitslosenquote} in dieser Gruppe bereits gefallen war, bevor der 
leichte Anstieg der \textit{\ac{ICT}-Investitionen} einsetzte.

Polen als postsozialistischer Wohlfahrtsstaat hat in den letzten Jahrzehnten einen tiefgreifenden 
wirtschaftlichen Wandel durchlaufen. Der EU-Beitritt im Jahr 2004 führte zu verstärkten 
ausländischen Direktinvestitionen, einer zunehmenden Integration in europäische 
Produktionsnetzwerke sowie einer generellen Modernisierung der Wirtschaft 
\parencite[vgl.][S. 186–189]{myant2013transition}. Diese Entwicklungen spiegeln sich auch in der 
Reduktion der Arbeitslosigkeit wider, die in allen Bildungsgruppen signifikant gesunken ist. 
Besonders bei Personen mit mittlerem und niedrigem Bildungsniveau könnte die Expansion von 
Industriejobs sowie der Dienstleistungssektor eine wesentliche Rolle gespielt haben.

Insgesamt zeigt die Abbildung, dass die \textit{Arbeitslosenquote} in allen Bildungsgruppen stark 
gesunken ist, während die \textit{\ac{ICT}-Investitionen} nur moderate Schwankungen aufweisen. 
Dies deutet darauf hin, dass die Haupttreiber der Beschäftigungsentwicklung in Polen eher in 
wirtschaftlichen und arbeitsmarktpolitischen Veränderungen zu suchen sind als in den direkten 
Auswirkungen von \textit{\ac{ICT}-Investitionen}. Dennoch könnte die leichte Zunahme der 
\textit{\ac{ICT}-Investitionen} im späteren Beobachtungszeitraum darauf hinweisen, dass sich der 
polnische Arbeitsmarkt allmählich in Richtung einer wissensbasierten Wirtschaft entwickelt, in 
der besonders Hochqualifizierte profitieren.

% Grafik "Schweden"
\begin{figure}[htbp]
    \centering
    \caption{Überblick über \textit{\ac{ICT}-Investitionen} und \textit{Arbeitslosenquote} in 
    Schweden} 
    \includegraphics[width=\textwidth]{assets/plot_sweden_final.png}
    \label{fig:sweden}
\end{figure}

Die Abbildung zeigt die Entwicklung der \textit{\ac{ICT}-Investitionen} als Anteil am BIP sowie 
die \textit{Arbeitslosenquote} in Schweden zwischen 2005 und 2022, differenziert nach 
Bildungsniveau - Schweden steht hier repräsentativ für nordische Wohlfahrtsstaaten. Im Gegensatz 
zu anderen Ländern ist hier eine relativ stabile Entwicklung der \textit{Arbeitslosenquote} über 
den gesamten Zeitraum zu beobachten, mit nur moderaten Schwankungen. Auffällig ist zudem, dass 
die \textit{\ac{ICT}-Investitionen} in Schweden im internationalen Vergleich auf einem 
vergleichsweise hohen Niveau liegen. Während sie in der ersten Dekade leichte Schwankungen 
zeigen, bleibt ihr Niveau ab 2010 weitgehend konstant und steigt gegen Ende des 
Betrachtungszeitraums leicht an.

Bei Personen mit einem niedrigen Bildungsniveau lag die \textit{Arbeitslosenquote} 2005 bei knapp 
5\% und zeigte bis etwa 2010 einen moderaten Anstieg. Nach 2010 stabilisierte sich die 
\textit{Arbeitslosenquote} zunächst, bevor sie ab 2015 einen erneuten Aufwärtstrend verzeichnete. 
Besonders auffällig ist der deutliche Anstieg nach 2018, der sich bis 2022 fortsetzt. Während die 
\textit{Arbeitslosenquote} für gering Qualifizierte also in den letzten Jahren gestiegen ist, 
sind die \textit{\ac{ICT}-Investitionen} im selben Zeitraum weitgehend stabil geblieben, wenn 
auch mit einer leicht positiven Tendenz. Dies könnte darauf hindeuten, dass die fortschreitende 
Digitalisierung möglicherweise die Beschäftigungsmöglichkeiten für niedrig qualifizierte 
Arbeitskräfte verschlechtert hat, indem sie bestimmte Arbeitsplätze verdrängte oder die 
Anforderungen an digitale Kompetenzen erhöhte - wahrscheinlich hängt diese Beobachtung aber eher 
mit der Corona-Pandemie zusammen.

Für Personen mit einem mittleren Bildungsniveau zeigt sich ein stabiles Muster, mit einer 
weitgehend konstanten \textit{Arbeitslosenquote} zwischen 2005 und 2018. Während die 
Arbeitslosigkeit 2005 bei unter 5\% lag, gab es bis 2015 eine leichte Abwärtsbewegung, gefolgt 
von einer weitgehenden Stabilisierung. Nach 2018 zeigt sich eine leicht steigende Tendenz der 
\textit{Arbeitslosenquote}, wenn auch weniger ausgeprägt als bei den gering Qualifizierten. Die 
\textit{\ac{ICT}-Investitionen} sind in dieser Gruppe durchgängig hoch und zeigen eine stabile 
Entwicklung mit leichten Schwankungen. Anders als bei den gering Qualifizierten ist hier keine 
klare gegenläufige Entwicklung zwischen \textit{\ac{ICT}-Investitionen} und Arbeitslosigkeit zu 
erkennen, was darauf hindeutet, dass mittlere Qualifikationen in Schweden weniger stark von den 
technologischen Veränderungen betroffen sind.

Bei Personen mit einem hohen Bildungsniveau zeigt sich über den gesamten Zeitraum hinweg eine 
extrem niedrige \textit{Arbeitslosenquote}. Bereits 2005 lag sie unter 5\% und blieb über den 
gesamten Zeitraum stabil, mit nur minimalen Schwankungen. Auffällig ist, dass die 
\textit{\ac{ICT}-Investitionen} in dieser Gruppe im internationalen Vergleich sehr hoch sind, mit 
Werten, die konstant bei 4-5\% des \ac{BIP} liegen. Die Kombination aus hoher 
\ac{ICT}-Investition und niedriger \textit{Arbeitslosenquote} deutet darauf hin, dass 
hochqualifizierte Arbeitskräfte in Schweden stark von der Digitalisierung profitieren konnten. 
Dies entspricht auch theoretischen Erwartungen, da hochqualifizierte Beschäftigte in 
wissensintensiven Branchen tätig sind, die von technologischen Innovationen profitieren.

Die stabilen \textit{\ac{ICT}-Investitionen} und die insgesamt niedrige 
\textit{Arbeitslosenquote} deuten darauf hin, dass der schwedische Arbeitsmarkt relativ 
widerstandsfähig gegenüber technologischen Veränderungen ist. Allerdings lässt sich bei niedrig 
qualifizierten Arbeitskräften ein Anstieg der Arbeitslosigkeit nach 2018 beobachten, der 
möglicherweise mit strukturellen Veränderungen auf dem Arbeitsmarkt zusammenhängt. Dies könnte 
darauf hindeuten, dass bestimmte Berufe durch die Digitalisierung zunehmend verdrängt werden oder 
dass sich die Anforderungen an digitale Kompetenzen verstärkt haben, sodass Geringqualifizierte 
Schwierigkeiten haben, sich an die veränderten Bedingungen anzupassen.

Insgesamt zeigt die Abbildung, dass Schweden ein stabiles Beschäftigungsniveau über den gesamten 
Zeitraum hinweg aufweist, wobei die \textit{\ac{ICT}-Investitionen} konstant hoch sind. Während 
Hoch- und Mittelqualifizierte weitgehend von den Entwicklungen profitieren konnten, scheint sich 
für gering Qualifizierte in den letzten Jahren eine Verschlechterung der Beschäftigungssituation 
abzuzeichnen. Dies könnte darauf hindeuten, dass Digitalisierung in hochentwickelten 
Volkswirtschaften wie Schweden zunehmend zu einer Polarisierung des Arbeitsmarktes führt, bei der 
Hochqualifizierte von den Investitionen profitieren, während gering Qualifizierte zunehmend unter 
Druck geraten.

% Grafik "Deutschland"
\begin{figure}[htbp]
    \centering
    \caption{Überblick über \textit{\ac{ICT}-Investitionen} und \textit{Arbeitslosenquote} in 
    Deutschland}
    \includegraphics[width=\textwidth]{assets/plot_germany_final.png}
    \label{fig:germany}
\end{figure}

Die Abbildung zeigt die Entwicklung der \textit{\ac{ICT}-Investitionen} als Anteil am BIP sowie 
die \textit{Arbeitslosenquote} in Deutschland zwischen 2005 und 2022, differenziert nach 
Bildungsniveau - Deutschland steht hier repräsentativ für mitteleuropäische Wohlfahrtsstaaten. 
Dabei lassen sich klare Unterschiede zwischen den drei betrachteten Gruppen - niedriges, 
mittleres und hohes Bildungsniveau - sowohl hinsichtlich des Niveaus als auch der Veränderung 
der \textit{Arbeitslosenquote} erkennen. Insgesamt zeigen sich über den gesamten Zeitraum hinweg 
deutliche Rückgänge in der \textit{Arbeitslosenquote}, während die 
\textit{\ac{ICT}-Investitionen} eine weitgehend stabile Entwicklung aufweisen.

Für Personen mit einem niedrigen Bildungsniveau zeigt sich eine besonders hohe 
\textit{Arbeitslosenquote} zu Beginn des Beobachtungszeitraums, die 2005 bei über 18\% lag. In 
den darauffolgenden Jahren kam es zu einem kontinuierlichen Rückgang, der bis 2020 Werte unter 
5\% erreichte. Diese Entwicklung spiegelt die allgemeine Verbesserung des deutschen 
Arbeitsmarktes wider, insbesondere durch wirtschaftlichen Aufschwung und Reformen im Rahmen der 
Agenda 2010. Die \textit{\ac{ICT}-Investitionen} verzeichneten zwischen 2005 und 2010 zunächst 
einen leichten Rückgang, bevor sie sich um die 1,5\% des BIP stabilisierten. Ein direkter 
Zusammenhang zwischen \textit{\ac{ICT}-Investitionen} und der sinkenden 
\textit{Arbeitslosenquote} ist nicht ersichtlich, da der Rückgang der \textit{Arbeitslosenquote} 
bereits vor der leichten Stabilisierung der Investitionen begann.

Bei Personen mit mittlerem Bildungsniveau zeigt sich ein ähnliches Muster, wenn auch auf einem 
insgesamt niedrigeren Ausgangsniveau der \textit{Arbeitslosenquote}. Während diese 2005 noch bei 
etwa 10\% lag, fiel sie bis 2020 auf rund 3\% und blieb seither weitgehend stabil. Die 
\textit{\ac{ICT}-Investitionen} zeigen eine konstante Entwicklung mit geringen Schwankungen. Auch 
hier bleibt der direkte Zusammenhang zwischen den \textit{\ac{ICT}-Investitionen} und der 
\textit{Arbeitslosenquote} unklar, da der Rückgang der Arbeitslosigkeit langfristig verläuft und 
nicht direkt mit den Investitionen korreliert.

Für Personen mit hohem Bildungsniveau zeigt sich über den gesamten Zeitraum hinweg eine sehr 
niedrige \textit{Arbeitslosenquote}. Bereits 2005 lag sie unter 5\% und sank bis 2010 auf unter 
2\%, wo sie anschließend auf diesem niedrigen Niveau stabil blieb. Im Vergleich zu den anderen 
Bildungsgruppen weist diese Gruppe somit die geringsten Schwankungen auf. Die 
\textit{\ac{ICT}-Investitionen} zeigen auch hier eine weitgehend stabile Entwicklung. Dies könnte 
darauf hindeuten, dass Hochqualifizierte vermehrt in Berufen tätig sind, die von steigenden 
\textit{\ac{ICT}-Investitionen} profitieren, jedoch bleibt auch hier die Kausalität unklar.

Deutschland als mitteleuropäischer Wohlfahrtsstaat zeichnet sich durch eine enge Verzahnung von 
Bildungssystem und Arbeitsmarkt aus \parencite[vgl.][S. 21–27]{hall2001varieties}. Insbesondere 
das duale Ausbildungssystem und gezielte arbeitsmarktpolitische Maßnahmen könnten eine Rolle beim 
Rückgang der \textit{Arbeitslosenquote} in den niedrigen und mittleren Bildungsgruppen gespielt 
haben. Die \textit{\ac{ICT}-Investitionen} zeigen über den Beobachtungszeitraum hinweg keine 
drastischen Veränderungen, was darauf hindeutet, dass technologische Entwicklungen schrittweise 
in den Arbeitsmarkt integriert wurden. Besonders für Hochqualifizierte könnte eine steigende 
Nachfrage nach digitalen Fähigkeiten eine Rolle gespielt haben, während bei den niedrigen und 
mittleren Bildungsniveaus der Arbeitsmarktrückgang vermutlich durch andere makroökonomische 
Faktoren beeinflusst wurde.

Die Abbildung verdeutlicht insgesamt, dass die \textit{Arbeitslosenquote} in allen 
Bildungsgruppen über die Jahre hinweg gesunken sind, während die \textit{\ac{ICT}-Investitionen} 
vergleichsweise stabil geblieben sind. Dies lässt darauf schließen, dass der Rückgang der 
Arbeitslosigkeit nicht direkt durch \textit{\ac{ICT}-Investitionen} getrieben wurde, sondern eher 
mit makroökonomischen Entwicklungen und strukturellen Veränderungen auf dem deutschen 
Arbeitsmarkt zusammenhängt.

% Grafik "Großbritannien"
\begin{figure}[htbp]
    \centering
    \caption{Überblick über \textit{\ac{ICT}-Investitionen} und Arbeitslosenquote in 
    Großbritannien}
    \includegraphics[width=\textwidth]{assets/plot_uk_final.png}
    \label{fig:uk}
\end{figure}

Die Abbildung zeigt die Entwicklung der \textit{\ac{ICT}-Investitionen} als Anteil am BIP sowie 
die \textit{Arbeitslosenquote} in Großbritannien zwischen 2005 und 2022, differenziert nach 
Bildungsniveau - Großbritannien steht hier repräsentativ für angelsächsische Wohlfahrtsstaaten. 
Im Vergleich zu anderen Wohlfahrtsstaatentypen weist Großbritannien eine relativ konstante 
\textit{Arbeitslosenquote} auf, die über den Zeitraum hinweg nur leichte Rückgänge zeigt. 
Auffällig ist, dass die \textit{\ac{ICT}-Investitionen} in Großbritannien zwar einen moderaten 
Anstieg aufweisen, sich jedoch auf einem relativ niedrigen Niveau bewegen.

Bei Personen mit einem niedrigen Bildungsniveau lag die \textit{Arbeitslosenquote} im Jahr 2005 
bei etwa 8\% und sank bis 2020 auf unter 4\%. Anders als in Ländern mit stärker regulierten 
Arbeitsmärkten zeigt sich hier kein abrupter Rückgang, sondern eine schrittweise Anpassung über 
den Zeitraum hinweg. Gleichzeitig zeigen die \textit{\ac{ICT}-Investitionen} einen leicht 
steigenden Trend, bleiben jedoch im Bereich von etwa 1,5\% des BIP. Ein klarer Zusammenhang 
zwischen \textit{\ac{ICT}-Investitionen} und der \textit{Arbeitslosenquote} lässt sich nicht 
unmittelbar erkennen, was darauf hindeuten könnte, dass andere arbeitsmarktpolitische oder 
wirtschaftliche Faktoren maßgeblicher für die Reduktion der Arbeitslosigkeit sind.

Bei Personen mit mittlerem Bildungsniveau zeigt sich ein ähnliches Muster. Die 
\textit{Arbeitslosenquote} lag 2005 bei etwa 6\% und fiel bis 2015 auf rund 3\%, wo sie sich 
anschließend stabilisierte. Die \textit{\ac{ICT}-Investitionen} zeigen hier eine geringe Zunahme, 
bleiben jedoch weitgehend konstant im Bereich von 1,5\% bis 2\% des BIP. Auch in dieser Gruppe 
scheint der Rückgang der \textit{Arbeitslosenquote} eher mit marktwirtschaftlichen Anpassungen 
als mit direkten Effekten der \textit{\ac{ICT}-Investitionen} zusammenzuhängen. Der relativ 
geringe Anstieg der Investitionen deutet darauf hin, dass die britische Wirtschaft zwar 
technologische Entwicklungen integriert, jedoch nicht in dem Ausmaß wie andere hochdigitalisierte 
Volkswirtschaften.

Für Personen mit hohem Bildungsniveau zeigt sich über den gesamten Zeitraum hinweg eine sehr 
niedrige \textit{Arbeitslosenquote}. Bereits 2005 lag sie unter 3\% und blieb über den gesamten 
Zeitraum weitgehend stabil, mit nur minimalen Schwankungen. Die \textit{\ac{ICT}-Investitionen} 
zeigen auch hier eine relativ konstante Entwicklung, liegen jedoch ebenfalls im Bereich von 1,5\% 
bis 2\% des BIP. Dies deutet darauf hin, dass Hochqualifizierte kaum von negativen 
Beschäftigungseffekten durch Digitalisierung betroffen sind. Vielmehr könnte der flexible 
britische Arbeitsmarkt es dieser Gruppe erleichtert haben, sich an technologische Veränderungen 
anzupassen.

Großbritannien als anglo-sächsischer Wohlfahrtsstaat zeichnet sich durch einen weniger 
regulierten Arbeitsmarkt aus, der sich durch eine hohe Flexibilität und eine geringere staatliche 
Intervention auszeichnet \parencite[vgl.][S. 21]{trabert1997entwicklung}. Diese Charakteristik 
könnte erklären, warum die \textit{Arbeitslosenquote} über den Zeitraum hinweg relativ stabil 
bleiben und gleichzeitig keine drastischen Veränderungen im Bereich der 
\textit{\ac{ICT}-Investitionen} feststellbar sind. Der moderate Rückgang der Arbeitslosigkeit 
deutet darauf hin, dass sich der britische Arbeitsmarkt schrittweise an Digitalisierung angepasst 
hat, ohne dass bestimmte Gruppen massiv benachteiligt wurden.

Zusammenfassend zeigt die Abbildung, dass sich die britische \textit{Arbeitslosenquote} über die 
Jahre hinweg in allen Bildungsgruppen verringert hat, wenn auch nicht so drastisch wie in anderen 
Ländern. Gleichzeitig bleiben die \textit{\ac{ICT}-Investitionen} auf einem relativ niedrigen 
Niveau und zeigen keine unmittelbare Korrelation mit den Veränderungen der 
\textit{Arbeitslosenquote}. Dies deutet darauf hin, dass makroökonomische Faktoren wie die 
Arbeitsmarktflexibilität und allgemeine wirtschaftliche Entwicklung eine wichtigere Rolle für die 
Beschäftigungsdynamik spielen als allein die Höhe der \textit{\ac{ICT}-Investitionen}.

%%%%%%%%%%%%%%%%%%%%%%%%%%%%%%%%%%%%%
% Ergebnisse: Multivariate Analysen %
% TODO:                             %
%%%%%%%%%%%%%%%%%%%%%%%%%%%%%%%%%%%%%

\subsection{Multivariate Analysen}

Die Zusammenfassung der Ergebnisse aus den Modellen mit Kontrollvariablen zeigt eine umfassende, 
jedoch differenzierte Analyse der Auswirkungen von \textit{\ac{ICT}-Investitionen} auf die 
\textit{Arbeitslosenquote} in den drei Bildungsgruppen („niedriges Bildungsniveau“, „mittleres 
Bildungsniveau“, „hohes Bildungsniveau“). Die Modelle liefern wichtige Hinweise auf die Bedeutung 
makroökonomischer Rahmenbedingungenund institutioneller Strukturen, während der direkte Einfluss 
von\textit{\ac{ICT}-Investitionen} signifikant, aber unterschiedlich stark ausfällt.

% Modelle ohne Interaktion
\input{assets/models_control_final.tex}

% Modell ohne Interaktion (niedriges Bildungsniveau)
Im Modell für die Gruppe mit niedrigem Bildungsniveau zeigt der geschätzte Koeffizient für 
\textit{\ac{ICT}-Investitionen} einen positiven Wert von 2,302*** (p < 0,001), was auf einen 
signifikanten Anstieg der \textit{Arbeitslosenquote} in dieser Gruppe hinweist. Dies unterstützt 
die Hypothese, dass gering Qualifizierte besonders von den negativen Effekten der Digitalisierung 
betroffen sind, da einfache Tätigkeiten stärker automatisiert werden.

Das \textit{\ac{BIP} pro Kopf} weist mit -0,194*** (p < 0,001) einen signifikanten negativen 
Einfluss auf, was darauf hindeutet, dass wirtschaftlich stärkere Länder tendenziell geringere 
\textit{Arbeitslosenquoten} für das niedrige Bildungsniveau aufweisen. Der 
\textit{Tertiärer Bildungsanteil} zeigt mit 0,606*** (p < 0,001) einen positiven Zusammenhang mit 
der \textit{Arbeitslosenquote}, was auf mögliche Verdrängungseffekte im Arbeitsmarkt hindeuten 
könnte.

Die \textit{Gewerkschaftsdichte} hat mit 0,128*** (p < 0,001) ebenfalls einen signifikant 
positiven Effekt, was darauf hindeuten könnte, dass starke Gewerkschaften zwar 
Arbeitsplatzsicherheit für bestehende Arbeitnehmer gewährleisten, aber möglicherweise den 
Marktzugang für neue gering qualifizierte Arbeitnehmer erschweren. Die 
\textit{Arbeitsmarktregulierung} zeigt mit -0,147 keinen signifikanten Effekt auf die 
\textit{Arbeitslosenquote} dieser Gruppe.

Das Modell erklärt mit einem R²-Wert von 0,304 etwa 30,4\% der Varianz in der 
\textit{Arbeitslosenquote}, wobei der adjustierte R²-Wert bei 0,295 liegt. Dies zeigt, dass 
relevante Einflussfaktoren erfasst wurden, aber weitere Determinanten für die 
Arbeitsmarktentwicklung von Geringqualifizierten berücksichtigt werden sollten.

% Modell ohne Interaktion (mittleres Bildungsniveau)
Für das mittlere Bildungsniveau zeigt sich ebenfalls ein signifikanter positiver Zusammenhang 
zwischen \textit{\ac{ICT}-Investitionen} und \textit{Arbeitslosenquote}. Der geschätzte 
Koeffizient beträgt 1,157*** (p < 0,001), was darauf hindeutet, dass höhere 
\ac{ICT}-Investitionen mit einem Anstieg der Arbeitslosigkeit in dieser Gruppe verbunden sind. 
Dies bestätigt die Annahme, dass auch mittelqualifizierte Arbeitskräfte durch technologische 
Veränderungen betroffen sein können, wenn auch weniger stark als gering Qualifizierte.

Das \textit{\ac{BIP} pro Kopf} zeigt mit -0,153*** (p < 0,001) einen signifikanten negativen 
Zusammenhang. Dies legt nahe, dass eine höhere wirtschaftliche Leistungsfähigkeit mit einer 
geringeren \textit{Arbeitslosenquote} für mittelqualifizierte Personen einhergeht, möglicherweise 
aufgrund größerer Weiterbildungsmöglichkeiten und besserer Arbeitsmarktanpassung.

Der \textit{Tertiärer Bildungsanteil} weist mit 0,282*** (p < 0,001) einen signifikanten 
positiven Effekt auf. Dies könnte darauf hinweisen, dass eine wachsende akademische Bildung den 
Wettbewerb auf dem Arbeitsmarkt verschärft und dadurch mittelqualifizierte Arbeitskräfte 
zunehmend unter Druck setzt. Die \textit{Gewerkschaftsdichte} zeigt mit 0,106*** (p < 0,001) 
ebenfalls einen signifikanten positiven Zusammenhang mit der \textit{Arbeitslosenquote}, was 
darauf hindeutet, dass Gewerkschaften möglicherweise den Kündigungsschutz stärken, aber 
gleichzeitig den Eintritt neuer Arbeitskräfte erschweren könnten.

Die \textit{Arbeitsmarktregulierung} zeigt mit -0,118* (p < 0,05) einen signifikanten negativen 
Effekt, was darauf hindeutet, dass strengere Regulierung einen stabilisierenden Einfluss auf die 
Beschäftigungssituation von Mittelqualifizierten haben könnte. 

Das Modell erklärt mit einem R²-Wert von 0,308 etwa 30,8\% der Varianz in der 
\textit{Arbeitslosenquote}, wobei der adjustierte R²-Wert bei 0,299 liegt. Dies zeigt, dass die 
erfassten Variablen relevante Einflussfaktoren für die Arbeitsmarktentwicklung dieser Gruppe 
darstellen, jedoch weitere strukturelle Mechanismen berücksichtigt werden sollten.

% Modell ohne Interaktion (hohes Bildungsniveau)
Für das hohe Bildungsniveau zeigt das Modell einen signifikanten, aber geringeren positiven 
Zusammenhang zwischen \textit{\ac{ICT}-Investitionen} und \textit{Arbeitslosenquote}. Der 
geschätzte Koeffizient beträgt 0,455*** (p < 0,001), was darauf hindeutet, dass auch 
hochqualifizierte Personen in Ländern mit höheren \ac{ICT}-Investitionen einen moderaten Anstieg 
der Arbeitslosenquote erfahren. Dies widerspricht der Erwartung, dass Hochqualifizierte weniger 
von negativen Arbeitsmarkteffekten der Digitalisierung betroffen sind. Eine mögliche Erklärung 
könnte sein, dass technologische Entwicklungen auch wissensintensive Tätigkeiten transformieren 
oder durch Automatisierung teilweise ersetzen.

Das \textit{\ac{BIP} pro Kopf} zeigt mit -0,083*** (p < 0,001) weiterhin einen signifikant 
negativen Zusammenhang, was bestätigt, dass wirtschaftlich stärkere Länder tendenziell eine 
niedrigere \textit{Arbeitslosenquote} für Hochqualifizierte aufweisen. Der 
\textit{Tertiärer Bildungsanteil} hat mit 0,140*** (p < 0,001) einen signifikant positiven 
Effekt, was darauf hindeutet, dass ein wachsender Anteil an akademisch ausgebildeten Personen den 
Wettbewerb innerhalb dieser Gruppe verstärken könnte.

Die \textit{Gewerkschaftsdichte} zeigt mit 0,025+ (p < 0,1) einen nur schwach signifikanten 
positiven Effekt. Dies könnte darauf hinweisen, dass Gewerkschaften für Hochqualifizierte eine 
geringere Rolle spielen als für andere Bildungsgruppen. Die \textit{Arbeitsmarktregulierung} 
weist mit -0,085* (p < 0,05) einen signifikant negativen Zusammenhang auf, was darauf hindeutet, 
dass striktere arbeitsrechtliche Regelungen möglicherweise stabilisierend wirken, indem sie 
Arbeitsplatzsicherheit erhöhen oder den Zugang zum Arbeitsmarkt regulieren.

Die Modellgüte zeigt mit einem R²-Wert von 0,272, dass etwa 27,2\% der Variation der 
\textit{Arbeitslosenquote} für diese Gruppe durch das Modell erklärt werden. Der adjustierte 
R²-Wert liegt bei 0,262, was darauf hindeutet, dass weitere unberücksichtigte Faktoren für die 
Beschäftigungssituation Hochqualifizierter eine Rolle spielen.

% Modell ohne Interaktion (Fazit)
Zusammenfassend zeigen die Modelle mit Kontrollvariablen, dass \textit{\ac{ICT}-Investitionen} in 
allen Bildungsgruppen einen signifikanten positiven Einfluss auf die \textit{Arbeitslosenquote} 
haben. Die geschätzten Koeffizienten sind durchweg signifikant (niedriges Bildungsniveau: 
2,302***, mittleres Bildungsniveau: 1,157***, hohes Bildungsniveau: 0,455***), was darauf 
hindeutet, dass \ac{ICT}-Investitionen in der derzeitigen Form eher mit einem Anstieg der 
Arbeitslosigkeit einhergehen. Dies bestätigt die Hypothese, dass die Digitalisierung nicht nur 
gering Qualifizierte, sondern auch Mittel- und Hochqualifizierte betrifft.

Besonders für Geringqualifizierte zeigt sich der stärkste Zusammenhang, was mit bisherigen 
Annahmen über die Verwundbarkeit dieser Gruppe auf digitalen Arbeitsmärkten übereinstimmt. 
Gleichzeitig widersprechen die Ergebnisse der Erwartung, dass Hochqualifizierte von steigenden 
\textit{\ac{ICT}-Investitionen} automatisch profitieren. Die Tatsache, dass die Arbeitslosigkeit 
auch in dieser Gruppe zunimmt, deutet darauf hin, dass technologische Innovationen zunehmend auch 
wissensintensive Tätigkeiten transformieren oder teilweise ersetzen.

Makroökonomische Faktoren, insbesondere das \textit{\ac{BIP} pro Kopf}, haben in allen Modellen 
einen signifikanten negativen Einfluss auf die \textit{Arbeitslosenquote}. Dies bestätigt, dass 
wirtschaftlich stärkere Länder tendenziell niedrigere Arbeitslosenquoten aufweisen. Striktere 
\textit{Arbeitsmarktregulierungen} zeigen in der Gruppe der Mittel- und Hochqualifizierten 
negative Effekte auf die \textit{Arbeitslosenquote}, was darauf hindeutet, dass Regulierungen 
möglicherweise stabilisierend wirken, indem sie Arbeitsplatzsicherheit erhöhen. Der 
\textit{Tertiäre Bildungsanteil} ist in allen Gruppen positiv mit der Arbeitslosenquote 
assoziiert, was darauf hindeutet, dass eine höhere Bildungsbeteiligung allein nicht ausreicht, 
um Digitalisierungseffekte auf dem Arbeitsmarkt auszugleichen.

Die Modellgüte (R²-Werte zwischen 0,272 und 0,308) zeigt, dass wesentliche strukturelle und 
institutionelle Einflussfaktoren erfasst wurden, jedoch nicht die gesamte Variation in den 
Arbeitslosenquoten erklären. Dies unterstreicht die Notwendigkeit einer differenzierteren Analyse 
durch Interaktionsmodelle, um die Mechanismen hinter den beobachteten Zusammenhängen besser zu 
verstehen.

% Modelle mit Interaktion
\input{assets/models_interaction_final.tex}

% Modell mit Interaktion (Einführung)
Die Modelle mit Interaktionseffekten liefern eine differenziertere Perspektive auf den 
Zusammenhang zwischen \textit{\ac{ICT}-Investitionen} und der \textit{Arbeitslosenquote}. Im 
Vergleich zu den Basis-Modellen ohne Interaktionen zeigen sich hier mehrere signifikante 
Zusammenhänge, insbesondere mit institutionellen Faktoren wie den \textit{Wohlfahrtsstaatentypen}.

Die Modelle mit Interaktionseffekten liefern eine differenziertere Perspektive auf den 
Zusammenhang zwischen \textit{\ac{ICT}-Investitionen} und der \textit{Arbeitslosenquote}. Im 
Vergleich zu den Basis-Modellen ohne Interaktionen zeigen sich hier mehrere signifikante 
Zusammenhänge, insbesondere mit institutionellen Faktoren wie den \textit{Wohlfahrtsstaatentypen}. 

% Modell mit Interaktion (Referenzkategorie)
Zunächst ist jedoch der Haupteffekt von \textit{\ac{ICT}-Investitionen} zu betrachten, der für 
die Referenzkategorie, die angelsächsischen Wohlfahrtsstaaten, gilt. Die Ergebnisse zeigen, dass 
höhere \textit{\ac{ICT}-Investitionen} in liberalen Staaten in allen Bildungsgruppen mit einer 
steigenden \textit{Arbeitslosenquote} assoziiert sind. Der geschätzte Effekt beträgt 4,671*** 
(p < 0,001) für das niedrige, 3,246*** (p < 0,001) für mittlere und 1,259*** (p < 0,001) für hohe 
Bildungsniveau. Diese Werte deuten darauf hin, dass der Zusammenhang zwischen Digitalisierung und 
Arbeitslosigkeit in angelsächsischen Arbeitsmärkten besonders stark ausgeprägt ist. 

Eine mögliche Erklärung hierfür ist die schwache arbeitsmarktpolitische Regulierung in 
angelsächsischen Staaten. Diese Märkte sind durch eine hohe Flexibilität, geringe staatliche 
Eingriffe und schwache gewerkschaftliche Strukturen geprägt, was bedeutet, dass Arbeitnehmer sich 
weitgehend selbst an technologische Veränderungen anpassen müssen 
\parencite[vgl.][S. 30]{hall2001varieties}. Während in anderen wohlfahrtsstaatlichen 
Modellen Umschulungsprogramme oder Arbeitsmarktregulierungen negative Beschäftigungseffekte der 
Digitalisierung abfedern können, gibt es in angelsächsischen Märkten weniger Schutzmechanismen 
für Arbeitnehmer. Dies könnte dazu führen, dass Arbeitsplatzverluste durch Automatisierung und 
Digitalisierung direkt in steigender Arbeitslosigkeit resultieren, insbesondere für gering- und 
mittelqualifizierte Arbeitnehmer.

Die Interaktionseffekte zwischen \textit{\ac{ICT}-Investitionen} und den 
\textit{Wohlfahrtsstaatentypen} liefern nun wichtige Erkenntnisse darüber, wie institutionelle 
Rahmenbedingungen diesen Zusammenhang beeinflussen. Da die liberalen Wohlfahrtsstaaten als 
Referenzkategorie dienen, sind die Interaktionseffekte relativ zu diesen Staaten zu 
interpretieren.

% Modell mit Interaktion (Interaktionen)
Für postsozialistische Wohlfahrtsstaaten zeigen sich die stärksten negativen Interaktionseffekte. 
Für das niedrige Bildungsniveau beträgt der Interaktionseffekt -5,200*** (p < 0,001), für das 
mittlere Bildungsniveau -3,579*** (p < 0,001) und für das hohe Bildungsniveau -1,415*** 
(p < 0,001). Diese signifikant negativen Werte bedeuten, dass die negativen Auswirkungen von 
\ac{ICT}-Investitionen auf die \textit{Arbeitslosenquote} in diesen Ländern geringer ausfallen 
als in den liberalen Wohlfahrtsstaaten. Da die Interaktionseffekte in ihrer absoluten Höhe sogar 
größer sind als die Haupteffekte der \ac{ICT}-Investitionen, deutet dies darauf hin, dass 
\ac{ICT}-Investitionen in postsozialistischen Staaten nicht zu einer höheren Arbeitslosigkeit 
führen, sondern möglicherweise stabilisierend wirken. Eine mögliche Erklärung hierfür ist, dass 
diese Länder gezielte wirtschaftspolitische Maßnahmen implementiert haben oder strukturelle 
Besonderheiten aufweisen, die negative Effekte der Digitalisierung auf den Arbeitsmarkt abmildern.

Für mitteleuropäische Wohlfahrtsstaaten sind die Interaktionseffekte negativ, aber nicht 
signifikant. Dies bedeutet, dass sich diese Länder nicht signifikant von der Referenzkategorie 
unterscheiden. Mit anderen Worten gibt es keine eindeutigen Belege dafür, dass 
\ac{ICT}-Investitionen in diesen Ländern systematisch andere Auswirkungen auf die 
Arbeitslosigkeit haben als in den liberalen Wohlfahrtsstaaten. Mitteleuropäische Länder wie 
Deutschland oder Frankreich verfügen über relativ stabile Arbeitsmarktstrukturen und ein starkes 
duales Bildungssystem, das möglicherweise negative Effekte abfedert. Allerdings zeigen die 
Ergebnisse keine signifikanten Vorteile dieser Struktur im Vergleich zu liberalen Arbeitsmärkten.

Für nordische Wohlfahrtsstaaten sind die Interaktionseffekte für das niedrige und mittlere 
Bildungsniveau nicht signifikant, während für das hohe Bildungsniveau ein positiver Effekt von 
0,639* (p < 0,05) beobachtet wird. Dies bedeutet, dass in nordischen Ländern Hochqualifizierte 
im Vergleich zu liberalen Wohlfahrtsstaaten tendenziell stärker von steigender Arbeitslosigkeit 
betroffen sind. Eine mögliche Erklärung könnte sein, dass nordische Staaten stark in 
Digitalisierung investieren und dadurch Arbeitsplätze in wissensintensiven Bereichen 
umstrukturieren. Dies könnte hochqualifizierte Arbeitskräfte vor neue Herausforderungen stellen, 
insbesondere wenn technologische Entwicklungen schneller voranschreiten als Bildungs- und 
Umschulungsmaßnahmen.

Für südeuropäische Wohlfahrtsstaaten zeigen sich signifikant negative Interaktionseffekte für das 
mittlere und hohe Bildungsniveau. Der Effekt für das mittlere Bildungsniveau beträgt -3,066*** 
(p < 0,001), für das hohe Bildungsniveau -2,880*** (p < 0,001). Dies deutet darauf hin, dass 
\ac{ICT}-Investitionen in diesen Ländern mit einer geringeren Arbeitslosenquote für Mittel- und 
Hochqualifizierte im Vergleich zu liberalen Wohlfahrtsstaaten verbunden sind. Dies könnte 
bedeuten, dass technologische Investitionen gezielt in qualifizierte Arbeitsplätze fließen oder 
dass strukturelle Gegebenheiten, wie Arbeitsplatzsicherheit durch Regulierungen, digitale 
Umbrüche abfedern. Gleichzeitig könnte diese geringe Dynamik jedoch auch bedeuten, dass 
technologische Transformationen langsamer verlaufen, was langfristig negative Folgen für die 
Wettbewerbsfähigkeit haben könnte.

% Modell mit Interaktion (Modelgüte)
Die erklärten Varianzen (R²-Werte) sind in den Interaktionsmodellen höher als in den 
Basis-Modellen ohne Interaktionen. Während die R²-Werte in den einfachen Modellen zwischen 
0,272 und 0,308 lagen, erreichen die Interaktionsmodelle Werte von 0,337 
(niedriges Bildungsniveau), 0,333 (mittleres Bildungsniveau) und 0,306 (hohes Bildungsniveau). 
Dies zeigt, dass institutionelle Rahmenbedingungen eine wesentliche Rolle spielen und die 
Erklärungskraft der Modelle verbessern. Trotzdem bleibt ein relevanter Teil der Variation in der 
Arbeitslosenquote unaufgeklärt, was darauf hindeutet, dass weitere Faktoren eine Rolle spielen.

% Modell mit Interaktion (Kontrollvariablen)
Das \textit{\ac{BIP} pro Kopf} bleibt weiterhin negativ und signifikant für alle Bildungsgruppen, 
zeigt jedoch einen leicht abgeschwächten Effekt im Vergleich zum Modell ohne Interaktion. Dies 
könnte darauf hindeuten, dass institutionelle Unterschiede die Rolle des 
\ac{BIP} für die Arbeitsmarktentwicklung teilweise moderieren. Die \textit{Gewerkschaftsdichte} 
bleibt für niedrig- und mittelqualifizierte Personen positiv signifikant, während sie für 
Hochqualifizierte keine signifikanten Effekte zeigt. Dies deutet darauf hin, dass Gewerkschaften 
für geringer qualifizierte Beschäftigte eine stärkere Schutzfunktion haben, während sie bei 
Hochqualifizierten eine untergeordnete Rolle spielen. Die 
\textit{Regulierungsstrenge des Arbeitsmarktes} zeigt für Hochqualifizierte weiterhin eine 
signifikant negative Wirkung (-0,091**), was darauf hindeutet, dass strengere 
Arbeitsmarktregulierungen in dieser Gruppe schützend wirken können. Für Mittelqualifizierte ist 
der Effekt hingegen nicht mehr signifikant, was darauf hindeutet, dass institutionelle 
Unterschiede den Einfluss der Regulierung auf diese Gruppe abschwächen. Der 
\textit{Anteil tertiär gebildeter Personen} zeigt ebenfalls durchweg signifikante Effekte, bleibt 
jedoch in der Richtung konsistent mit den Basis-Modellen. Die positiven Werte deuten darauf hin, 
dass eine höhere Bildungsbeteiligung weiterhin mit einer steigenden Arbeitslosenquote korreliert, 
was möglicherweise auf steigende Konkurrenz innerhalb der jeweiligen Qualifikationsgruppe 
zurückzuführen ist.

% Modell mit Interaktion (Fazit)
Insgesamt zeigen die Interaktionsmodelle, dass die Auswirkungen von \ac{ICT}-Investitionen stark 
vom Wohlfahrtsstaatentyp abhängen. In postsozialistischen Ländern scheinen Investitionen nicht 
mit einer steigenden Arbeitslosigkeit verbunden zu sein, während sich in mitteleuropäischen 
Staaten keine signifikanten Unterschiede zu liberalen Wohlfahrtsstaaten zeigen. In nordischen 
Ländern scheint es eine leicht negative Wirkung auf hochqualifizierte Arbeitskräfte zu geben, 
während in südeuropäischen Ländern \ac{ICT}-Investitionen mit einer geringeren Arbeitslosigkeit 
für Mittel- und Hochqualifizierte verbunden sind. Die insgesamt verbesserten R²-Werte zeigen, 
dass institutionelle Faktoren eine wichtige Rolle in der Erklärung der Beschäftigungseffekte der 
Digitalisierung spielen, dennoch bleibt ein relevanter Teil der Varianz unerklärt. Dies 
verdeutlicht, dass Digitalisierung und technologische Transformation nicht in allen Ländern 
dieselben Beschäftigungseffekte haben und dass wirtschaftspolitische sowie institutionelle 
Rahmenbedingungen maßgeblich beeinflussen, wie Arbeitsmärkte auf technologische Investitionen 
reagieren.

% TODO: Ausführen (Stand 09.03.2025)

\section{Diskussion und Fazit}

Die Ergebnisse dieser Arbeit bieten wertvolle Einblicke in die Beziehung zwischen 
\textit{\ac{ICT}-Investitionen} und der \textit{Arbeitslosenquote} in verschiedenen 
Bildungsniveaus. Sie zeigen signifikante Zusammenhänge und verdeutlichen die Rolle 
institutioneller Rahmenbedingungen für die Beschäftigungswirkungen der Digitalisierung. 
Die Untersuchung trägt zur wissenschaftlichen Debatte über die Wechselwirkungen 
zwischen technologischer Entwicklung, Arbeitsmarktstrukturen und politischen 
Institutionen bei und liefert praktische Implikationen für Politik, Unternehmen 
und Bildungssysteme.

\subsection{Zentrale Ergebnisse der Analyse}

Die erste Hypothese (\textbf{H1}), dass \textit{\ac{ICT}-Investitionen} mit einer 
niedrigeren \textit{Arbeitslosenquote} unter Hochqualifizierten verbunden sind, wird 
durch die Ergebnisse nicht unterstützt. Die Basis-Modelle zeigen, dass höhere 
\textit{\ac{ICT}-Investitionen} über alle Bildungsgruppen hinweg mit steigender 
\textit{Arbeitslosenquote} korrelieren, auch für Hochqualifizierte. Dies steht im 
Widerspruch zur weit verbreiteten Annahme, dass Hochqualifizierte primär von 
technologischen Fortschritten profitieren.

Die zweite Hypothese (\textbf{H2}), dass \textit{\ac{ICT}-Investitionen} die 
\textit{Arbeitslosenquote} unter gering Qualifizierten erhöhen, wird hingegen durch 
die Ergebnisse gestützt. Besonders im Basis-Modell ohne Interaktion zeigt sich für 
Geringqualifizierte der stärkste positive Effekt, was darauf hindeutet, dass einfache 
Tätigkeiten besonders stark von Automatisierung betroffen sind.

Die dritte Hypothese (\textbf{H3}), dass institutionelle Faktoren wie 
Wohlfahrtsstaatentypen die negativen Effekte von \textit{\ac{ICT}-Investitionen} 
abmildern können, wird teilweise durch die Ergebnisse unterstützt. Die Interaktionsmodelle 
zeigen, dass postsozialistische und südeuropäische Wohlfahrtsstaaten in der Lage sind, die 
negativen Beschäftigungseffekte der Digitalisierung abzuschwächen, während in 
liberalen Staaten ein deutlicher Zusammenhang zwischen \textit{\ac{ICT}-Investitionen} 
und steigender \textit{Arbeitslosenquote} besteht. Dies deutet darauf hin, dass 
institutionelle Strukturen eine wesentliche Rolle für die Auswirkungen der 
Digitalisierung auf den Arbeitsmarkt spielen.

Die Kontrollvariablen liefern weitere relevante Erkenntnisse. Das 
\textit{\ac{BIP} pro Kopf} bleibt durch alle Bildungsniveaus hinweg ein signifikanter 
Faktor mit einem negativen Effekt auf die \textit{Arbeitslosenquote}. Die 
\textit{Regulierungsstrenge des Arbeitsmarktes} hat für Hochqualifizierte einen 
stabilisierenden Einfluss, während sie für Geringqualifizierte tendenziell mit einer 
höheren \textit{Arbeitslosenquote} korreliert. Der \textit{tertiäre Bildungsanteil} zeigt 
in allen Gruppen einen positiven Effekt auf die \textit{Arbeitslosenquote}, was darauf 
hindeutet, dass eine größere Bildungsbeteiligung allein nicht ausreicht, um die negativen 
Effekte der Digitalisierung auszugleichen.

\subsection{Einordnung der Ergebnisse in den theoretischen Kontext}

Die Theorie des \ac{SBTC} besagt, dass technologische Innovationen die Nachfrage nach 
hochqualifizierten Arbeitskräften steigern, während gering Qualifizierte durch 
Automatisierung verdrängt werden \parencite[vgl.][S. 7]{acemoglu2002technical}. Die 
Ergebnisse dieser Arbeit bestätigen diese Annahme nur teilweise. Während erwartet wurde, 
dass \ac{ICT}-Investitionen primär die Arbeitslosigkeit von gering Qualifizierten erhöhen 
(was durch die Basis-Modelle bestätigt wird), zeigen sich für Hochqualifizierte ebenfalls 
steigende Arbeitslosenquoten. Dies widerspricht der Annahme, dass Hochqualifizierte 
generell von der Digitalisierung profitieren.

Die Interaktionsmodelle legen nahe, dass institutionelle Faktoren diese Effekte 
modifizieren. Besonders in postsozialistischen und südeuropäischen Ländern fällt der 
Zusammenhang zwischen \ac{ICT}-Investitionen und Arbeitslosigkeit schwächer aus. Dies 
könnte im Einklang mit Schumpeters Konzept der „schöpferischen Zerstörung“ stehen, wonach 
wirtschaftlicher Wandel langfristig zu Wachstum führt, kurzfristig aber strukturelle 
Umbrüche verursacht \parencite[vgl.][S. 103-105]{schumpeter1976capitalism}. Während in 
hoch digitalisierten liberalen Staaten wie den USA oder Großbritannien Arbeitsmärkte 
bereits stark durch digitale Umstellungen transformiert wurden, könnte es in 
postsozialistischen Staaten Verzögerungseffekte geben, die erklären, warum dort die 
negativen Beschäftigungseffekte weniger ausgeprägt sind.

\subsection{Limitationen und zukünftige Forschung}

Trotz der wertvollen Erkenntnisse dieser Untersuchung sind einige Limitationen zu 
berücksichtigen. Erstens basiert die Analyse auf aggregierten \ac{OECD}-Daten für den 
Zeitraum 2005–2022, wodurch Unterschiede in der Erhebungsmethodik zwischen den Ländern 
die Ergebnisse beeinflussen könnten. Zweitens liegt der Fokus auf makroökonomischen 
Zusammenhängen, sodass individuelle Anpassungsstrategien von Arbeitnehmer*innen oder 
Unternehmen an die Digitalisierung nicht berücksichtigt werden konnten. Künftige Studien 
sollten verstärkt auf Umfragedaten oder firmenspezifische Datenquellen zurückgreifen, um 
differenziertere Erkenntnisse zu gewinnen.

Darüber hinaus besteht die Möglichkeit einer umgekehrten Kausalität („Reversed Causality“) 
zwischen \textit{\ac{ICT}-Investitionen} und der \textit{Arbeitslosenquote} 
\parencite[S. 134]{pearl2009causality}. Während in dieser Analyse angenommen wurde, dass 
\textit{\ac{ICT}-Investitionen} die \textit{Arbeitslosenquote} beeinflussen, könnte es 
auch sein, dass hohe Arbeitslosigkeit Regierungen oder Unternehmen zu verstärkten 
Investitionen in Digitalisierung veranlasst. Diese Hypothese könnte mit Methoden wie 
Granger-Kausalitätstests oder Instrumentalvariablen weiter geprüft werden.

\subsection{Gesamtfazit}

Zusammenfassend zeigen die Ergebnisse, dass \ac{ICT}-Investitionen keine universelle 
Lösung für Arbeitsmarktprobleme darstellen, sondern dass ihre Wirkungen maßgeblich von 
institutionellen Rahmenbedingungen abhängen. Während in liberalen Wohlfahrtsstaaten ein 
deutlicher positiver Zusammenhang zwischen \textit{\ac{ICT}-Investitionen} und 
\textit{Arbeitslosenquote} besteht, zeigen sich in postsozialistischen und 
südeuropäischen Ländern stabilisierende Effekte. Dies deutet darauf hin, dass 
wirtschafts- und arbeitsmarktpolitische Maßnahmen entscheidend sind, um die negativen 
Effekte der Digitalisierung abzufedern.

Die Ergebnisse bestätigen die Hypothese, dass gering Qualifizierte besonders stark von 
negativen Digitalisierungseffekten betroffen sind. Gleichzeitig zeigen sich auch für 
Hochqualifizierte unerwartet steigende Arbeitslosenquoten, was auf eine zunehmende 
Automatisierung und Umstrukturierung wissensintensiver Tätigkeiten hinweisen könnte. Die 
institutionelle Einbettung digitaler Transformationen erweist sich als zentraler Faktor: 
Während in manchen Ländern technologische Innovationen mit steigender Arbeitslosigkeit 
verbunden sind, können wohlfahrtsstaatliche Strukturen in anderen Ländern diese Effekte 
teilweise kompensieren.

Daraus ergibt sich eine klare politische Implikation: Eine erfolgreiche digitale 
Transformation erfordert nicht nur technologische Investitionen, sondern auch begleitende 
arbeitsmarktpolitische, wirtschaftliche und bildungspolitische Maßnahmen, um den 
Strukturwandel sozialverträglich zu gestalten.


% Literaturverzeichnis (kleiner Font)
\renewcommand{\bibfont}{\small}
\printbibliography

\newpage

% Anhang
\appendix
\include{chapters/A}

\end{document}
