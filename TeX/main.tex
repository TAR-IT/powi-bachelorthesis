% TODO: Abgabedatum einfügen

%%%%%%%%%%%%%%%%%%%%%%%
% Dokument definieren %
%%%%%%%%%%%%%%%%%%%%%%%

\documentclass[a4paper, 12pt]{article}

%%%%%%%%%%%%%%%%%%%%%%%%%%%%%%%%%%
% Pakete laden und konfigurieren %
%%%%%%%%%%%%%%%%%%%%%%%%%%%%%%%%%%

% deutsche Spracheinstellungen und -kodierung
\usepackage[utf8]{inputenc}                                                 % Umlaute korrekt verwenden
\usepackage[T1]{fontenc}                                                    % Schriftenkodierung
\usepackage[ngerman]{babel}                                                 % Deutsche Spracheinstellungen
\usepackage{csquotes}                                                       % Für korrekte Anführungszeichen
\usepackage{hyphenat}                                                       % Silbentrennung

% Pakete für R
\usepackage{tabularray}                                                     % Für erweiterte Tabellen
\usepackage{float}                                                          % Fixiert Objekte mit "H"
\usepackage[normalem]{ulem}                                                 % Für Unter- und Durchstreichungen
\UseTblrLibrary{booktabs}                                                   % Unterstützt Booktabs in Tabellen
\UseTblrLibrary{siunitx}                                                    % Unterstützt Zahlen und Einheiten
\newcommand{\tinytableTabularrayUnderline}[1]{\underline{#1}}               % Unterstreichung in Tabellen
\newcommand{\tinytableTabularrayStrikeout}[1]{\sout{#1}}                    % Durchstreichung in Tabellen
\NewTableCommand{\tinytableDefineColor}[3]{\definecolor{#1}{#2}{#3}}        % Farben in Tabellen
\usepackage{adjustbox}                                                      % Skaliert Tabellen/Objekte
\usepackage{multicol}                                                       % Mehrspaltiger Text
\usepackage{listings}                                                       % Für Code-Listings
\lstset{
    language=R,                                                             % Specify the language
    basicstyle=\ttfamily\scriptsize,                                        % Font style for code
    numbers=left,                                                           % Line numbers on the left
    numberstyle=\tiny,                                                      % Font size for line numbers
    stepnumber=1,                                                           % Number every line
    breaklines=true,                                                        % Automatically break long lines
    frame=single,                                                           % Add a border around the code
    captionpos=b,                                                           % Caption position (bottom)
    showstringspaces=false                                                  % Do not show spaces in strings
}

% Layout und Seitenränder
\usepackage{geometry}
\geometry{a4paper, top=3cm, bottom=3cm, left=2.5cm, right=2.5cm}

% Literaturverzeichnis mit APA-Zitierstil
\usepackage[style=apa, backend=biber, language=ngerman]{biblatex}
\addbibresource{assets/literature.bib}                                      % Bibliographiedatei einbinden

% Pakete für Abkürzungsverzeichnis und Glossar
\usepackage[printonlyused]{acronym}

% mathematische Symbole (optional)
\usepackage{amsmath, amssymb}

% Für Tabellen und Grafiken
\usepackage{graphicx}                                                       % Für Bilder
\usepackage{booktabs}                                                       % Für schönere Tabellen

% Hyperlinks
\usepackage[hidelinks]{hyperref}

% Codebook einbinden
\usepackage{verbatim}                                                       % Für Einbinden als plain text

% Zeilenabstand
\usepackage{setspace}
\onehalfspacing                                                             % 1,5-facher Zeilenabstand

% Kopf- und Fußzeilen
\usepackage{fancyhdr}
\pagestyle{fancy}
\fancyhf{}                                                                  % Setzt Kopf- und Fußzeile zurück
\fancyhead[R]{T. A. Rau}                                                    % Name links
\fancyhead[L]{ICT-Investitionen und Arbeitslosigkeit in Wohlfahrtsstaaten}  % Titel rechts
\setlength{\headheight}{14.5pt}                                             % Setzt die Höhe der Kopfzeile
\fancyfoot[C]{\thepage}                                                     % Setzt die Seitenzahl mittig unten

%%%%%%%%%%%%%%%%%%%%%
% Dokument beginnen %
%%%%%%%%%%%%%%%%%%%%%

% Titel und Autoreninformationen
\title{ICT-Investitionen und Arbeitslosigkeit in Wohlfahrtsstaaten\\ 
\large Eine Paneldatenanalyse nach Bildungsniveau in OECD-Ländern}
\author{Tobias Achim Rau}
\date{\today}

\begin{document}

% Titelseite
\maketitle
\thispagestyle{empty}
\vspace{2cm}

\begin{center}
  \textbf{Bachelorarbeit} \\
  vorgelegt im Studiengang Politikwissenschaften \\
  am Fachbereich 03 an der \textbf{Goethe-Universität Frankfurt} \\
  \vspace{2cm}
  \textbf{Verfasser:} Tobias Achim Rau \\
  \textbf{Matrikelnummer:} 6619097 \\
  \textbf{E-Mail:} s3045892@stud.uni-frankfurt.de \\
  \vspace{2cm}
  \textbf{Betreuerin:} Anna Gerlach \\
  \vspace{2cm}
  \textbf{Abgabedatum:} 25.03.2025
\end{center}

\newpage

% Inhaltsverzeichnis
\tableofcontents

\newpage

% Abkürzungsverzeichnis
\section*{Abkürzungsverzeichnis}
\begin{acronym}[OECD]                                                       % Breite der längsten Abkürzung für Ausrichtung
  \acro{ICT}{Informations- und Kommunikationstechnologien (aus d. Engl.: 
  \textit{Information and Communication Technologies})}
  
  \acro{OECD}{Organisation für wirtschaftliche Zusammenarbeit und Entwicklung 
  (aus d. Engl.: \textit{Organisation for Economic Co-operation and Development})}
 
  \acro{SNA08}{System of National Accounts 2008}

  \acro{SBTC}{Skill-Biased Technological Change (z. Dt.: \textit{Technologie mit 
  qualifikationsspezifischen Effekten})}

  \acro{RBTC}{Routine-Biased Technological Change (z. Dt.: \textit{Technologie mit 
  routinespezifischen Effekten})}
  
  \acro{BIP}{Bruttoinlandsprodukt}
  
  \acro{RE}{Fixed Effects (z. Dt.: \textit{feste Effekte})}
  
  \acro{FE}{Random Effects (z. Dt.: \textit{zufällige Effekte})}

  \acro{AI}{Künstliche Intelligenz (aus d. Engl.: \textit{Artificial Intelligence})}

  \acro{IoT}{Internet der Dinge (aus d. Engl.: \textit{Internet of Things})}
\end{acronym}

\newpage

% Glossar
\section*{Glossar}
\begin{description}
  \item[Digitalisierung] bezeichnet den Prozess der Umwandlung von analogen Informationen, 
  Prozessen und Geschäftsmodellen in digitale Formate. Sie umfasst den Einsatz digitaler 
  Technologien zur Automatisierung und Optimierung von Abläufen sowie zur Schaffung neuer 
  Wertschöpfungspotenziale \parencite[S. 6]{brennen2016theinternational}. Im 
  wirtschaftlichen Kontext wird Digitalisierung oft im Zusammenhang mit der vierten 
  industriellen Revolution (Industrie 4.0) gesehen, die durch die Integration digitaler 
  und physischer Systeme gekennzeichnet ist \parencite[S. 114]{hofman2018arbeit}.

  \item[Cloud Computing] beschreibt die Bereitstellung von IT-Ressourcen wie Speicherplatz, 
  Rechenleistung und Anwendungen über das Internet („die Cloud“) anstelle lokaler Server 
  oder Computer. Dieses Modell ermöglicht den flexiblen Zugriff auf Daten und Anwendungen 
  unabhängig vom Standort und fördert Skalierbarkeit sowie Kosteneffizienz 
  \parencite[S. 52]{armbrust2010aview}. Die wichtigsten Servicemodelle sind 
  Infrastructure-as-a-Service, Platform-as-a-Service und 
  Software-as-a-Service.

  \item[Künstliche Intelligenz] bezeichnet die Fähigkeit von Maschinen oder 
  Computersystemen, menschenähnliche kognitive Funktionen wie Lernen, Problemlösung und 
  Entscheidungsfindung auszuführen. Sie basiert auf Algorithmen, die Muster in Daten 
  erkennen und selbstständig aus Erfahrungen lernen können 
  \parencite[S. 28]{russell2020artificial}. \ac{AI} umfasst verschiedene Teilbereiche wie 
  maschinelles Lernen, neuronale Netzwerke und natürliche Sprachverarbeitung.

  \item[Big Data] bezeichnet große, komplexe und schnell wachsende Datenmengen, die mit 
  herkömmlichen Datenverarbeitungssystemen nur schwer analysierbar sind. Die Analyse 
  von Big Data erfolgt häufig mit Verfahren des maschinellen Lernens, Data Mining und 
  verteilten Datenbanksystemen \parencite[S. 92]{gandomi2014beyond}.
\end{description}

\newpage

% Kapitel einbinden
% TODO: Keine TODOs.

\section{Einleitung}

Mit der zunehmenden Digitalisierung und Automatisierung der Arbeitswelt erleben viele 
Länder tiefgreifende strukturelle Veränderungen ihrer Arbeitsmärkte. Eine zentrale Rolle 
spielen dabei \ac{ICT}, deren Einsatz weltweit zu erheblichen Effizienzsteigerungen und 
Innovationsprozessen führt \parencite[vgl.][Kap. 6]{brynjolfsson2014thesecond}.

Während technologische Fortschritte die Nachfrage nach hochqualifizierten Arbeitskräften 
erhöhen können, bleibt unklar, welche Auswirkungen diese Entwicklung auf 
geringqualifizierte Arbeitskräfte hat und inwiefern sie sich messen lässt 
\parencite[vgl.][S. 1045]{acemoglu2011skills}. Einerseits eröffnen die gestiegenen 
Anforderungen an technologische Kompetenzen neue Chancen für hochqualifizierte 
Fachkräfte \parencite[vgl.][S. 1070]{acemoglu2011skills}, andererseits wächst die 
Befürchtung, dass rasante Investitionen in \ac{ICT} insbesondere für geringqualifizierte 
Personen zum Verlust von Arbeitsplätzen führen, da viele ihrer Tätigkeiten routinisiert 
und damit leichter durch Maschinen ersetzbar sind. Frey und Osborne (2013) schätzen, 
dass rund 47\% der Arbeitsplätze in den USA potenziell automatisierbar sind, wobei vor 
allem Berufe mit niedriger Qualifikation betroffen sind 
\parencite[vgl.][S. 14–15]{frey2013thefuture}. In der Forschung wird daher diskutiert, 
ob der technologische Fortschritt die Kluft zwischen den Bildungsniveaus weiter vertieft 
und die Arbeitslosigkeit unter geringer Qualifizierten verstärkt 
\parencite[vgl.][S. 2–4]{balsmeier2019isthis}.

Die Frage nach dem Zusammenhang zwischen \ac{ICT}-Investitionen und Arbeitslosigkeit 
nach Bildungsniveau ist nicht nur ökonomisch, sondern auch sozial von Bedeutung. 
Investitionen in digitale Technologien können Polarisierungseffekte hervorrufen, 
indem sie hochqualifizierte Arbeitskräfte begünstigen, während geringer Qualifizierte 
durch Automatisierung verdrängt werden \parencite[vgl.][S. 14–15]{frey2013thefuture}. 
Gleichzeitig entstehen durch den technologischen Wandel neue Berufsfelder, die 
innovative Fähigkeiten erfordern \parencite[vgl.][Kap. 12]{brynjolfsson2014thesecond}. 
Welche Auswirkungen diese Prozesse auf die Verteilung von Arbeitsplätzen haben und 
ob sie zur Polarisierung des Arbeitsmarktes beitragen, bleibt eine zentrale Frage der 
aktuellen Forschung \parencite[vgl.][S. 2–4]{balsmeier2019isthis}.

Ziel dieser Arbeit ist es daher, den Einfluss von \ac{ICT}-Investitionen auf die 
Arbeitslosigkeit in verschiedenen Bildungsgruppen zu untersuchen. Es wird angenommen, 
dass hohe Investitionen in digitale Technologien die Arbeitslosigkeit unter 
Hochqualifizierten senken, während sie bei geringer Qualifizierten steigen könnte 
\parencite[vgl.][S. 1045]{acemoglu2011skills}. Diese Analyse soll zur Debatte über die 
Folgen der Digitalisierung auf den Arbeitsmarkt beitragen und empirisch untersuchen, 
in welchem Maße technologische Investitionen mit der Arbeitslosenquote nach 
Bildungsniveau korrelieren.

Aus der zuvor dargelegten Argumentation ergibt sich die zentrale Forschungsfrage 
dieser Arbeit:

\begin{quote} \textbf{„Wie beeinflussen nationale Investitionen in Informations- und 
    Kommunikationstechnologien die Arbeitslosenquoten verschiedener Bildungsniveaus 
    in Wohlfahrtsstaaten?“} \end{quote}

Die Analyse der Auswirkungen von Digitalisierung und \ac{ICT}-Investitionen auf die 
Beschäftigungsstruktur in \ac{OECD}-Ländern ist von hoher Relevanz, da sie Aufschluss 
über die Anpassungsfähigkeit verschiedener Wirtschaftssysteme an technologische 
Umbrüche gibt. Zudem bietet sie eine Grundlage für politische Entscheidungen im 
Bereich Arbeitsmarktregulierung und Bildungsinvestitionen. Angesichts der zunehmenden 
Bedeutung digitaler Technologien für wirtschaftliches Wachstum und soziale 
Gerechtigkeit ist es essenziell, die damit verbundenen Herausforderungen und Chancen 
für verschiedene gesellschaftliche Gruppen besser zu verstehen.

% TODO: Durchlesen!

\section{Forschungsgegenstand}

Der Forschungsgegenstand dieser Arbeit umfasst die Untersuchung der Auswirkungen von 
Investitionen in \ac{ICT} auf den Arbeitsmarkt in \ac{OECD}-Ländern. Im Zentrum steht 
dabei die Frage, wie sich diese Investitionen auf die Beschäftigungslage, insbesondere die 
Arbeitslosenquote, auf verschiedene Bildungsniveaus in einer Gesellschaft auswirken. Die 
Untersuchung konzentriert sich dabei auf zwei wesentliche Dimensionen: die nationalen 
Investitionen in digitale Technologien sowie die Auswirkungen auf Arbeitsmärkte und 
Beschäftigungsstrukturen.

%%%%%%%%%%%%%%%%%%%%%%%%%%%%%%%%%%%%%
% Digitalisierung und Industrie 4.0 %
%%%%%%%%%%%%%%%%%%%%%%%%%%%%%%%%%%%%%

\subsection{Digitalisierung und Industrie 4.0}

Der Begriff der Digitalisierung beschreibt den Prozess, bei dem digitale Technologien 
zunehmend in allen gesellschaftlichen und wirtschaftlichen Bereichen eingesetzt werden, um 
Prozesse zu automatisieren, zu optimieren und neue Wertschöpfungspotenziale zu erschließen 
\parencite[S. 6]{brennen2016theinternational}. Im wirtschaftlichen Kontext wird dieser 
Begriff häufig mit der vierten industriellen Revolution (Industrie 4.0) in Verbindung 
gebracht. Diese ist gekennzeichnet durch die Verschmelzung von physischen und digitalen 
Systemen und wirft zentrale Fragen zur politischen und wirtschaftlichen Umsetzbarkeit des 
digitalen Wandels auf \parencite[S. 114]{hofman2018arbeit}. Die Digitalisierung verändert 
Produktionsprozesse, die Art und Weise, wie Dienstleistungen erbracht werden, sowie die 
Anforderungen an Arbeitnehmer*innen in einem globalisierten Arbeitsmarkt.

Industrie 4.0 bezeichnet die fortschreitende Integration von \ac{ICT} in industrielle 
Produktionsprozesse. Sie ist geprägt durch den Einsatz von \ac{AI}, maschinellem Lernen, 
Big Data, Cloud Computing, Cyber-Physischen Systemen (CPS) sowie dem \ac{IoT} 
\parencite[S. 22]{kagermann2013recommendations}. Diese Technologien ermöglichen die 
weitreichende Automatisierung von Produktionsabläufen und die Vernetzung einzelner 
Prozessschritte, was in einer höheren Effizienz, geringeren Produktionskosten und einer 
flexibleren Fertigung resultiert. Die Vernetzung intelligenter Maschinen und Systeme 
erlaubt eine selbstorganisierte Produktion, bei der Maschinen in Echtzeit miteinander 
kommunizieren und Entscheidungen auf Basis umfangreicher Datensätze treffen können.

Ein zentraler Treiber der Industrie 4.0 ist die wachsende Fähigkeit, Daten in Echtzeit zu 
erfassen, zu analysieren und zur Prozessoptimierung zu nutzen. Dadurch können Unternehmen 
"vorausschauende Wartungen"(aus d. Eng.: \textit{predictive maintenance}) implementieren, 
wodurch Maschinenausfälle minimiert und Produktionsprozesse optimiert werden. Ebenso 
spielt die Individualisierung von Produkten eine zunehmend wichtige Rolle, da moderne 
Fertigungssysteme auf individuelle Kundenwünsche eingehen können, ohne signifikante 
Effizienzverluste zu erleiden \parencite[S. 85]{bartodziej2016theconcept}.

Der Wandel durch Industrie 4.0 hat weitreichende Implikationen für den Arbeitsmarkt. 
Während einerseits neue, hochqualifizierte Arbeitsplätze entstehen, insbesondere im Bereich 
der Softwareentwicklung, Datenanalyse und Automatisierungstechnik, besteht andererseits 
die Gefahr, dass traditionelle Berufe - insbesondere in der industriellen Fertigung, im 
Transportwesen und in administrativen Tätigkeiten - durch digitale Prozesse ersetzt oder 
stark verändert werden \parencite[S. 40]{frey2016thefuture}. Dies führt zu einer 
zunehmenden Polarisierung des Arbeitsmarktes: Hochqualifizierte Fachkräfte mit digitalen 
Kompetenzen profitieren von der digitalen Transformation, während Geringqualifizierte 
einem steigenden Risiko der Arbeitsplatzverlagerung oder -substitution ausgesetzt sind.

Zudem hat Digitalisierung tiefgreifende Auswirkungen auf die Arbeitsorganisation und 
Beschäftigungsformen. Flexible Arbeitszeitmodelle, Remote Work und Plattformarbeit werden 
durch digitale Technologien gefördert, wodurch sich klassische Beschäftigungsstrukturen 
verändern \parencite[S. 112]{schwab2016thefourth}. Unternehmen setzen verstärkt auf agile 
Arbeitsmethoden, um sich den schnell wandelnden Marktanforderungen anzupassen. 
Gleichzeitig stellen die digitale Vernetzung und Automatisierung neue Herausforderungen 
an den Datenschutz, die IT-Sicherheit und die Regulierung von algorithmengesteuerten 
Entscheidungsprozessen.

Die Implementierung von Industrie 4.0 variiert stark zwischen verschiedenen Branchen und 
Ländern. Während hochtechnologisierte Industrien wie der Maschinenbau, die 
Automobilbranche oder die Elektronikfertigung bereits stark digitalisiert sind, zeigen 
sich in anderen Sektoren wie dem Handwerk, der Bauwirtschaft oder dem Einzelhandel noch 
deutliche Unterschiede im Digitalisierungsgrad 
\parencite[S. 77]{brennen2016theinternational}. Ebenso beeinflussen wirtschaftspolitische 
Rahmenbedingungen die Geschwindigkeit und Richtung des digitalen Wandels, insbesondere in 
Bezug auf Investitionen in digitale Infrastruktur, Weiterbildungsprogramme für 
Arbeitskräfte und Regularien zum Schutz von Beschäftigten im digitalen Zeitalter.

Zusammenfassend stellt Digitalisierung und Industrie 4.0 eine der zentralen 
Transformationen der modernen Wirtschaft dar. Die zunehmende Automatisierung, Vernetzung 
und datenbasierte Steuerung von Produktionsprozessen führt zu signifikanten 
Effizienzgewinnen, birgt jedoch zugleich Herausforderungen für den Arbeitsmarkt, 
insbesondere in Bezug auf Jobpolarisation, Qualifikationsanforderungen und soziale 
Ungleichheit. Die Frage, inwieweit \ac{ICT}-Investitionen tatsächlich zu einer Reduzierung 
oder Verschärfung von Arbeitslosigkeit führen, hängt daher maßgeblich von der Gestaltung 
wirtschafts- und bildungspolitischer Maßnahmen sowie von den Anpassungsfähigkeiten 
nationaler Arbeitsmärkte ab.

%%%%%%%%%%%%%%%%%%%%%
% ICT-Investitionen %
%%%%%%%%%%%%%%%%%%%%%

\subsection{ICT-Investitionen}

Investitionen in \ac{ICT} umfassen eine Vielzahl an materiellen und immateriellen 
Ressourcen, die zur Digitalisierung von Wirtschaft und Gesellschaft beitragen. Dazu gehören 
physische Infrastrukturen wie Glasfasernetze, Rechenzentren und Netzwerkausrüstungen sowie 
immaterielle Investitionen in Software, Cloud-Dienste, Datenmanagementsysteme und digitale 
Plattformen \parencite[S. 15ff]{oecd2019measuring}. \ac{ICT}-Investitionen gelten als 
wesentlicher Indikator für die digitale Transformation eines Landes, da sie maßgeblich 
beeinflussen, in welchem Umfang neue Technologien in Produktions- und 
Dienstleistungsprozesse integriert werden.

Der zunehmende Einsatz digitaler Technologien ermöglicht nicht nur Effizienzsteigerungen, 
sondern auch die Entwicklung neuer Geschäftsmodelle und Dienstleistungen. Insbesondere 
durch Fortschritte in Cloud Computing, Big Data, \ac{AI} und Automatisierung können 
Unternehmen ihre Produktionsabläufe flexibler gestalten, Kosten reduzieren und innovative 
Produkte auf den Markt bringen \parencite[S. 15ff]{oecd2019measuring}. Cloud-Technologien 
erleichtern beispielsweise den Zugriff auf skalierbare Rechenleistung und 
Speicherressourcen, was insbesondere kleinen und mittelständischen Unternehmen den 
Einstieg in digitale Geschäftsmodelle erleichtert. Gleichzeitig erlaubt der Einsatz von 
\ac{AI} eine effizientere Datennutzung, wodurch Entscheidungsprozesse optimiert und 
Produktionsabläufe automatisiert werden können.

Ein Vorteil von Investitionen in \ac{ICT} liegt in der Vernetzung und Integration globaler 
Wertschöpfungsketten. Durch digitale Plattformen und Echtzeit-Datenverarbeitung können 
Unternehmen Produktions- und Logistikprozesse über Ländergrenzen hinweg koordinieren, 
wodurch Lieferketten optimiert und wirtschaftliche Abhängigkeiten reduziert werden 
\parencite[S. 48]{oecd2019measuring}. Die zunehmende Automatisierung von Fertigungs- und 
Verwaltungsprozessen führt dazu, dass Unternehmen produktiver arbeiten und gleichzeitig 
flexibler auf Marktentwicklungen reagieren können.

\ac{ICT}-Investitionen sind zudem ein Schlüsselfaktor für wirtschaftliches Wachstum. 
Zahlreiche Studien zeigen, dass ein hoher Digitalisierungsgrad mit einer gesteigerten 
Innovationsfähigkeit sowie einer erhöhten Wettbewerbsfähigkeit von Unternehmen und 
Volkswirtschaften korreliert \parencite[S. 22]{brynjolfsson2015thesecond}. In digital 
führenden Ländern wie den USA, Deutschland oder Südkorea sind \ac{ICT}-Investitionen 
ein wesentlicher Treiber von Produktivitätssteigerungen, da sie nicht nur bestehende 
Arbeitsprozesse effizienter machen, sondern auch neue Arbeitsfelder und Industrien 
hervorbringen. Gleichzeitig können gezielte Investitionen in digitale Infrastrukturen 
dazu beitragen, strukturelle Ungleichheiten zwischen urbanen und ländlichen Regionen 
zu verringern, indem sie einen besseren Zugang zu digitalen Dienstleistungen und Fernarbeit 
ermöglichen.

Allerdings sind die Auswirkungen von \ac{ICT}-Investitionen auf den Arbeitsmarkt 
ambivalent. Einerseits entstehen durch die Digitalisierung neue Arbeitsplätze, 
insbesondere in den Bereichen Softwareentwicklung, Datenanalyse, Automatisierungstechnik 
und digitale Dienstleistungen. Andererseits führt die Automatisierung in vielen Sektoren 
zur Verdrängung traditioneller Tätigkeiten, insbesondere bei Routinetätigkeiten mit 
mittlerem Qualifikationsniveau \parencite[S. 40]{frey2016thefuture}. Dies verstärkt 
bestehende Tendenzen der Arbeitsmarktpolarisierung, bei der sowohl hochqualifizierte als 
auch gering qualifizierte Arbeitskräfte von technologischen Veränderungen 
unterschiedlichbetroffen sind.

Darüber hinaus sind Investitionen in \ac{ICT} eng mit politischen Rahmenbedingungen 
verknüpft. Regierungen spielen eine zentrale Rolle bei der Förderung von Digitalisierung 
durch gezielte Subventionen, Investitionen in digitale Bildung und die Bereitstellung 
leistungsfähiger Infrastrukturen wie Glasfasernetze und Standards wie 5G-Mobilfunk 
\parencite[S. 45]{oecd2020digital}. Auch Fragen der Cybersicherheit, des Datenschutzes und 
der ethischen Regulierung von \ac{AI}-Technologien beeinflussen maßgeblich, wie effektiv 
und nachhaltig \ac{ICT}-Investitionen zur wirtschaftlichen Entwicklung eines Landes 
beitragen können \parencite[S. 45]{oecd2020digital}. 

Zusammenfassend sind \ac{ICT}-Investitionen ein zentraler Motor der digitalen 
Transformation und spielen eine entscheidende Rolle in der Modernisierung von Wirtschaft 
und Gesellschaft \parencite[S. 112]{brynjolfsson2015thesecond}. Ihre Auswirkungen auf 
Beschäftigung, wirtschaftliches Wachstum und globale Wertschöpfungsketten sind 
vielschichtig und hängen sowohl von technologischen Fortschritten als auch von politischen 
und wirtschaftlichen Rahmenbedingungen ab \parencite[S. 112]{brynjolfsson2015thesecond}. 
In der vorliegenden Untersuchung wird analysiert, inwieweit \ac{ICT}-Investitionen mit der 
Entwicklung der Arbeitslosenquote in verschiedenen Bildungsgruppen zusammenhängen und 
welche Rolle sie für den Strukturwandel auf dem Arbeitsmarkt spielen.

%%%%%%%%%%%%%%%%%%%%%%%%%%%%%%%%%%%%
% Arbeitsmarkt und Bildungsgruppen %
%%%%%%%%%%%%%%%%%%%%%%%%%%%%%%%%%%%%

\subsection{Arbeitsmarkt und Bildungsgruppen}

Der Arbeitsmarkt beschreibt die Gesamtheit der wirtschaftlichen Beziehungen zwischen 
Arbeitsangebot und Arbeitsnachfrage. Beschäftigung und Arbeitslosigkeit gelten dabei als 
zentrale Indikatoren für die Bewertung von Arbeitsmarktstrukturen 
\parencite[S. 10ff]{acemoglu2002technical}. In dieser Arbeit liegt der Fokus auf der 
Arbeitslosigkeit nach Bildungsniveau, das in der Regel in die Kategorien niedrig, mittel 
und hoch eingeteilt wird \parencite[S. 35–37]{frey2016thefuture}. Diese Differenzierung 
ermöglicht es, Unterschiede in der Betroffenheit von Arbeitslosigkeit im Zuge des 
digitalen Wandels besser zu analysieren und daraus wirtschafts- und bildungspolitische 
Implikationen abzuleiten.

Die Auswirkungen der Digitalisierung auf den Arbeitsmarkt sind vielschichtig und führen zu 
einer strukturellen Veränderung der Beschäftigungsmöglichkeiten. In vielen Bereichen 
ersetzen automatisierte Prozesse und digitale Technologien bisherige Routinetätigkeiten, 
wodurch insbesondere Berufe mit einem mittleren Qualifikationsniveau unter Druck geraten 
\parencite[S. 44f]{frey2016thefuture}.

Diese Entwicklung wird oft als Jobpolarisierung beschrieben: Einerseits entstehen neue 
hochqualifizierte Tätigkeiten in Bereichen wie Softwareentwicklung, \ac{AI}, Datenanalyse 
und Automatisierungstechnik, andererseits sind Arbeitsplätze mit mittleren 
Qualifikationsanforderungen - insbesondere in der Fertigung, im administrativen Bereich 
und im Dienstleistungssektor - einem verstärkten Automatisierungsdruck ausgesetzt 
\parencite[S. 40]{autor2015whyare}.

Diese technologische Polarisierung führt zu einer zunehmenden Einkommensungleichheit, da 
Hochqualifizierte von der Digitalisierung profitieren, während gering Qualifizierte häufig 
in niedrig entlohnten, weniger stabilen Beschäftigungsverhältnissen verbleiben oder 
verstärkt von Arbeitslosigkeit betroffen sind \parencite[S. 10]{acemoglu2002technical}. 
Ein weiteres Risiko dieser Entwicklung ist die sogenannte Skill-Bias-Hypothese, wonach 
technologischer Fortschritt vor allem die Nachfrage nach hochqualifizierten Arbeitskräften 
erhöht, während einfache Tätigkeiten zunehmend ersetzt werden 
\parencite[S. 25]{goos2014explaining}.

Die strukturellen Verschiebungen auf dem Arbeitsmarkt stellen auch neue Herausforderungen 
an die Bildungs- und Arbeitsmarktpolitik. Insbesondere Maßnahmen zur Förderung digitaler 
Kompetenzen, Weiterbildungsprogramme für Berufstätige sowie Investitionen in lebenslanges 
Lernen werden zunehmend als notwendig erachtet, um die negativen Auswirkungen der 
Digitalisierung auf gering und mittel Qualifizierte abzufedern 
\parencite[S. 75]{brynjolfsson2015thesecond}. Länder mit einem gut ausgebauten 
Weiterbildungssystem und gezielten Maßnahmen zur digitalen Qualifizierung können die 
negativen Folgen der Jobpolarisierung besser abfedern als Länder mit einer geringeren 
Bildungsdurchlässigkeit.

Insgesamt zeigt sich, dass der Einfluss von \ac{ICT}-Investitionen auf den Arbeitsmarkt 
stark vom Bildungsniveau der Erwerbsbevölkerung abhängt. Während Hochqualifizierte von den 
neuen technologischen Anforderungen profitieren, müssen niedrig und mittel qualifizierte 
Arbeitskräfte mit einem erhöhten Risiko der Arbeitsplatzverdrängung rechnen. Diese 
Entwicklung unterstreicht die Notwendigkeit einer gezielten arbeitsmarkt- und 
bildungspolitischen Anpassung, um negative soziale und wirtschaftliche Effekte der 
Digitalisierung zu minimieren. In der vorliegenden Arbeit wird analysiert, inwieweit 
\ac{ICT}-Investitionen in \ac{OECD}-Ländern mit der Arbeitslosigkeit in verschiedenen 
Bildungsgruppen korrelieren und ob sich bestimmte Muster in verschiedenen Wohlfahrtsstaaten 
identifizieren lassen.

% TODO: Durchlesen (Stand 04.03.2025)

\section{Forschungsstand}

Die Auswirkungen von Digitalisierung und \ac{ICT}-Investitionen auf den Arbeitsmarkt sind
ein zentrales Thema der arbeitsmarktökonomischen Forschung. Während einige Studien den
Fokus auf die technologische Verdrängung bestimmter Berufsgruppen legen, untersuchen
andere, inwieweit institutionelle Rahmenbedingungen wie Wohlfahrtsstaaten die Effekte von
Digitalisierung abmildern oder verstärken. In diesem Kapitel werden zunächst die
allgemeinen Auswirkungen der Digitalisierung auf Arbeitsmärkte analysiert, bevor der Fokus
auf die Rolle von \ac{ICT}-Investitionen und die Unterschiede zwischen verschiedenen
Wohlfahrtsstaaten gelegt wird. Abschließend werden bestehende Forschungslücken aufgezeigt,
die eine weiterführende Analyse notwendig machen.

%%%%%%%%%%%%%%%%%%%%%%%%%%%%%%%%%%%%%%%%%%%%%%%%%%%%%%
% Auswirkungen der Digitalisierung auf Arbeitsmärkte %
%%%%%%%%%%%%%%%%%%%%%%%%%%%%%%%%%%%%%%%%%%%%%%%%%%%%%%

\subsection{Auswirkungen der Digitalisierung auf Arbeitsmärkte}

Die Digitalisierung und insbesondere Investitionen in \ac{ICT} haben die Arbeitsmärkte
weltweit grundlegend verändert. Empirische Studien zeigen, dass diese Entwicklungen die
Beschäftigungsstrukturen in verschiedenen Bildungsgruppen unterschiedlich beeinflussen
\parencite[vgl.][S. 7]{autor2013thegrowth}. Die Automatisierung betrifft besonders
routinisierbare und standardisierbare Tätigkeiten \parencite[vgl.][S. 20]{frey2013thefuture}.
Diese Entwicklungen führen zu einer Polarisierung des Arbeitsmarktes: Hochqualifizierte
profitieren von einer steigenden Nachfrage nach digitalen Kompetenzen, während Arbeitsplätze
mit mittlerem Qualifikationsniveau zunehmend unter Automatisierungsdruck geraten
\parencite[vgl.][S. 2509]{goos2014explaining}.

Autor, Levy und Murnane (2003) zeigen, dass Tätigkeiten mit hohem Anteil an routinemäßigen
kognitiven und manuellen Aufgaben besonders anfällig für Automatisierung sind. Die Theorie
des \ac{RBTC} besagt, dass insbesondere klar definierte, sich wiederholende Tätigkeiten durch
Maschinen ersetzt werden können \parencite[vgl.][S. 1281]{autor2003theskill}. Frey und Osborne
(2017) erweiterten diese Analyse und schätzten, dass bis zu 47\% der Arbeitsplätze in den USA
potenziell automatisierbar sind, wobei Berufe mit niedrigem Qualifikationsniveau besonders
betroffen sind \parencite[vgl.][S. 254]{frey2013thefuture}. Diese Erkenntnisse lassen sich auch
auf andere \ac{OECD}-Länder übertragen.

Parallel zur Automatisierung zeigt sich eine Polarisierung der Arbeitsmärkte. Während mittlere
Qualifikationsgruppen unter Druck geraten, profitieren insbesondere hochqualifizierte
Beschäftigte, die über spezialisierte technologische Kenntnisse verfügen, von der
steigenden Nachfrage nach digitalen und analytischen Fähigkeiten 
\parencite[vgl.][S. 2510]{goos2014explaining}.
Diese Entwicklung führt dazu, dass gut ausgebildete Arbeitskräfte mit hohen Qualifikationen
von der Digitalisierung profitieren, während gering Qualifizierte in wachsendem Maße von
Arbeitsplatzverlusten betroffen sind. Dies verstärkt das Risiko sozialer Ungleichheit, da
Beschäftigungschancen zunehmend ungleich verteilt sind. Diese Divergenz wird häufig als
„Digital Divide“ bezeichnet, da sie die Kluft zwischen hoch- und niedrigqualifizierten
Arbeitskräften weiter vertieft \parencite[vgl.][S. 10]{acemoglu2002technical}.

Die Auswirkungen der Digitalisierung variieren zudem stark nach Branche und
Wirtschaftssektor. Während einige Sektoren wie die Industrieproduktion oder der
Einzelhandel durch die Einführung automatisierter Systeme massiv verändert wurden,
profitieren wissensintensive Dienstleistungsbranchen von den neuen technologischen Möglichkeiten
\parencite[vgl.][S. 1555]{autor2013thegrowth}. Besonders betroffen sind manuelle Tätigkeiten
in der Fertigungsindustrie sowie administrative Büroarbeiten, die zunehmend durch
algorithmische Prozesse ersetzt werden \parencite[vgl.][S. 260]{frey2013thefuture}.
Gleichzeitig entstehen neue Arbeitsplätze in Bereichen wie IT, Datenwissenschaft, Robotik
und \ac{AI}, wodurch sich auch die Qualifikationsanforderungen auf dem Arbeitsmarkt
verändern \parencite[vgl.][S. 2510]{goos2014explaining}.

Zusammenfassend verändert die Digitalisierung Arbeitsmärkte auf mehreren Ebenen: Einerseits
verstärkt sie das Risiko der Automatisierung insbesondere für Berufe mit mittlerem und niedrigem
Qualifikationsniveau, andererseits eröffnet sie neue Beschäftigungsmöglichkeiten für
Hochqualifizierte \parencite[vgl.][S. 1555]{autor2013thegrowth}. Die zunehmende Kluft zwischen
verschiedenen Qualifikationsgruppen hat tiefgreifende Auswirkungen auf die
Einkommensverteilung, soziale Mobilität und die Notwendigkeit gezielter
arbeitsmarktpolitischer Maßnahmen \parencite[S. 2510]{goos2014explaining}.

%%%%%%%%%%%%%%%%%%%%%%%%%%%%%%%%%%%%%%%%%%%%%%%%%%%%
% ICT-Investitionen als Treiber der Transformation %
%%%%%%%%%%%%%%%%%%%%%%%%%%%%%%%%%%%%%%%%%%%%%%%%%%%%

\subsection{ICT-Investitionen als Treiber der Transformation}

Investitionen in \ac{ICT} gelten als zentraler Indikator für den Digitalisierungsgrad eines 
Landes und spielen eine Schlüsselrolle bei der Transformation moderner Arbeitsmärkte. Der 
verstärkte Einsatz digitaler Technologien verändert Produktions- und Geschäftsprozesse 
grundlegend und beeinflusst die Nachfrage nach Arbeitskräften in verschiedenen 
Qualifikationsgruppen. Empirische Studien zeigen, dass Unternehmen, die verstärkt in 
\ac{ICT} investieren, effizientere Abläufe entwickeln, ihre Wettbewerbsfähigkeit steigern 
und tendenziell eine höhere Nachfrage nach qualifizierten Arbeitskräften verzeichnen 
\parencite[vgl.][S. 12]{corrado2018intangible}.

Der Einfluss von \ac{ICT}-Investitionen auf den Arbeitsmarkt ist dabei vielschichtig. Laut 
der \ac{OECD} (2019) ermöglichen diese Investitionen nicht nur eine zunehmende Automatisierung, 
sondern tragen auch zur Integration globaler Wertschöpfungsketten bei und treiben das 
wirtschaftliche Wachstum voran \parencite[vgl.][S. 15ff]{oecd2019measuring}. Besonders in 
wissensintensiven Sektoren wie Finanzdienstleistungen, IT-gestützte Geschäftsprozesse, 
E-Commerce oder digitale Plattformarbeit entstehen neue Geschäftsmodelle, die verstärkt 
auf Automatisierung und datenbasierte Entscheidungsprozesse setzen.

Während \ac{ICT}-Investitionen neue Arbeitsplätze schaffen können, zeigen zahlreiche 
Studien, dass diese Transformation auch polarisierende Effekte mit sich bringt. 
Hochqualifizierte Arbeitskräfte profitieren von der steigenden Nachfrage nach digitalen 
und analytischen Fähigkeiten, während geringqualifizierte Beschäftigte einem zunehmenden 
Risiko der Arbeitsplatzverdrängung ausgesetzt sind 
\parencite[vgl.][Kap. 2]{brynjolfsson2014thesecond}. Besonders betroffen sind Tätigkeiten 
mit einem hohen Anteil an repetitiven, standardisierten Prozessen, die sich leicht durch 
digitale Technologien oder \ac{AI} ersetzen lassen. Dazu zählen nicht nur 
manuelle Produktionsprozesse, sondern auch administrative Tätigkeiten im Bürobereich, die 
zunehmend durch automatisierte Softwarelösungen abgelöst werden.

Die Polarisierung des Arbeitsmarktes ist eng mit der Theorie des \ac{SBTC} verbunden, 
wonach technologischer Fortschritt die Nachfrage nach hochqualifizierten Arbeitskräften 
erhöht, während mittlere Qualifikationsniveaus unter Druck geraten 
\parencite[vgl.][S. 22]{acemoglu2002technical}. Diese Entwicklung führt zu einer Verschiebung 
in der Beschäftigungsstruktur, da insbesondere wissensintensive Berufe von 
\ac{ICT}-Investitionen profitieren, während traditionelle Berufe in der industriellen 
Fertigung oder im einfachen Dienstleistungsbereich zunehmend verdrängt werden.

Gleichzeitig zeigt sich, dass \ac{ICT}-Investitionen nicht in allen Ländern und Branchen 
gleichermaßen produktivitätssteigernd wirken. Ihre Effekte hängen stark von begleitenden 
wirtschaftspolitischen Maßnahmen ab, darunter Investitionen in digitale Infrastruktur, 
die Förderung digitaler Kompetenzen und die Anpassung von Bildungsprogrammen an die 
veränderten Anforderungen des Arbeitsmarktes \parencite[vgl.][S. 77]{brynjolfsson2014thesecond}. 
Länder mit einer gezielten digitalen Transformationsstrategie, wie etwa Südkorea oder die 
skandinavischen Staaten, konnten in den letzten Jahrzehnten eine positive Korrelation 
zwischen \ac{ICT}-Investitionen und Wirtschaftswachstum feststellen 
\parencite[vgl.][S. 34]{oecd2020digital}. Empirische Studien zeigen, dass digitale Infrastruktur 
und eine strategische Förderung von digitaler Bildung eine zentrale Rolle dabei spielen, 
die wirtschaftlichen Vorteile von ICT-Investitionen vollständig auszuschöpfen 
\parencite[vgl.][S. 360]{vu2011ict}. Länder mit einem schwächeren Fokus auf digitale Bildung 
haben größere Schwierigkeiten, von diesen Entwicklungen zu profitieren, da der Mangel an 
digitalen Kompetenzen die Innovationskraft und Produktivität hemmt 
\parencite[vgl.][S. 34]{oecd2020digital}.

Zusammenfassend zeigen \ac{ICT}-Investitionen sowohl wachstumsfördernde als auch polarisierende 
Effekte auf den Arbeitsmarkt. Während sie die Produktivität und Wettbewerbsfähigkeit 
von Unternehmen steigern und neue Beschäftigungsmöglichkeiten für hochqualifizierte 
Arbeitskräfte schaffen, verstärken sie gleichzeitig das Risiko der Arbeitsplatzverdrängung 
für geringqualifizierte Arbeitskräfte. Die Digitalisierung des Dienstleistungssektors, 
die zunehmende Automatisierung administrativer Prozesse und die Integration neuer Technologien 
in industrielle Produktionsabläufe führen dazu, dass traditionelle Berufsbilder zunehmend 
hinterfragt und an neue Anforderungen angepasst werden müssen. Diese Entwicklungen 
unterstreichen die Notwendigkeit arbeitsmarktpolitischer Maßnahmen, um den Wandel sozial 
abzufedern und die Vorteile der Digitalisierung möglichst breit in der Gesellschaft zu verteilen.

%%%%%%%%%%%%%%%%%%%%%%%%%%%%%%%%%%%%%%%%%%%
% Unterschiede zwischen Wohlfahrtsstaaten %
%%%%%%%%%%%%%%%%%%%%%%%%%%%%%%%%%%%%%%%%%%%

\subsection{Unterschiede zwischen Wohlfahrtsstaaten}

Empirische Studien zeigen, dass die Auswirkungen von Digitalisierung und \ac{ICT}-Investitionen 
auf Arbeitsmärkte stark von den institutionellen Rahmenbedingungen eines Landes abhängen. 
Regierungen spielen eine zentrale Rolle bei der Förderung digitaler Infrastruktur, der 
Implementierung von Bildungspolitik und der Regulierung des Arbeitsmarktes 
\parencite[vgl.][S. 1–5]{hall2001varieties}. Studien haben gezeigt, dass Länder mit hohen 
Investitionen in digitale Bildung und Infrastruktur tendenziell bessere Anpassungsprozesse 
an den technologischen Wandel durchlaufen \parencite[vgl.][S. 23]{oecd2020digital}.

Laut der \ac{OECD} (2019) variieren die Investitionen in Informations- und 
Kommunikationstechnologie erheblich zwischen Ländern. Skandinavische Staaten und die Niederlande 
investieren überdurchschnittlich in digitale Bildung und Innovationen, während süd- und 
osteuropäische Länder vergleichsweise niedrigere Investitionen tätigen 
\parencite[vgl.][S. 45]{oecd2020digital}. Diese Unterschiede spiegeln sich auch in der 
Entwicklung der Arbeitsmärkte wider.

Länder mit hoher \ac{ICT}-Investitionsquote (z. B. Schweden, Niederlande) zeigen niedrigere 
Arbeitslosenquoten unter Hochqualifizierten und profitieren von einer stärkeren Nachfrage 
nach digitalen Kompetenzen \parencite[vgl.][S. 78]{brynjolfsson2014thesecond}. In Ländern mit 
geringeren \ac{ICT}-Investitionen (z. B. Spanien, Ungarn) sind hingegen vor allem Beschäftigte 
mit mittlerem Qualifikationsniveau einem höheren Automatisierungsrisiko ausgesetzt 
\parencite[vgl.][S. 12]{frey2013thefuture}.

Neben Investitionen in Digitalisierung spielen staatliche Bildungs- und Arbeitsmarktpolitiken 
eine entscheidende Rolle für die Fähigkeit eines Landes, sich an technologische Veränderungen 
anzupassen. Länder mit umfassenden Umschulungs- und Weiterbildungsprogrammen (z. B. Dänemark, 
Deutschland) haben bessere Voraussetzungen, um durch lebenslanges Lernen den digitalen Wandel 
sozialverträglich zu gestalten \parencite[vgl.][S. 361]{vu2011ict}. Staaten mit weniger 
regulierten Arbeitsmärkten (z. B. USA, Großbritannien) haben eine schnellere, aber oft 
ungleichere Anpassung an technologische Innovationen, was zu verstärkter 
Arbeitsplatzpolarisierung führen kann \parencite[vgl.][S. 172]{goos2014explaining}.

Studien zeigen, dass das Automatisierungsrisiko je nach Land und Wirtschaftsstruktur stark 
variiert. Laut einer \ac{OECD}-Analyse von Arntz, Gregory \& Zierahn (2016) sind in süd- und 
osteuropäischen Ländern bis zu 40\% der Arbeitsplätze einem hohen Automatisierungsrisiko 
ausgesetzt, während es in skandinavischen Ländern und Deutschland nur etwa 20–25\% sind 
\parencite[vgl.][S. 12]{arntz2016therisk}. Ein entscheidender Faktor für diese Unterschiede 
ist die Wirtschaftsstruktur: Länder mit einem hohen Anteil wissensintensiver Dienstleistungen 
(z. B. Schweden, Niederlande) sind weniger von Automatisierung betroffen. Industrie- und 
produktionslastige Länder (z. B. Spanien, Polen) weisen hingegen höhere Risiken für 
Arbeitsplatzverluste durch Automatisierung auf \parencite[vgl.][S. 260]{frey2013thefuture}.

Zusammenfassend zeigen die Unterschiede zwischen Wohlfahrtsstaaten, dass Digitalisierung und 
\ac{ICT}-Investitionen stark von den institutionellen Rahmenbedingungen abhängen. Länder mit 
gezielter Förderung digitaler Infrastruktur und Bildung können die Herausforderungen des 
technologischen Wandels besser bewältigen, während Länder mit geringeren Investitionen und 
restriktiveren Arbeitsmärkten stärkeren Risiken durch Automatisierung ausgesetzt sind. Dies 
verdeutlicht die Bedeutung staatlicher Steuerung bei der Gestaltung von Transformationsprozessen 
im digitalen Zeitalter.

%%%%%%%%%%%%%%%%%%%%
% Forschungslücken %
%%%%%%%%%%%%%%%%%%%%

\subsection{Forschungslücken}

Obwohl zahlreiche Studien die Auswirkungen von Digitalisierung und \ac{ICT}-Investitionen 
untersuchen, bestehen weiterhin relevante Forschungslücken. Die Mehrheit der bisherigen 
Studien konzentriert sich auf die allgemeinen Effekte von Digitalisierung auf den 
Arbeitsmarkt, ohne spezifisch zwischen verschiedenen Wohlfahrtsstaatentypen zu 
unterscheiden. Es fehlen systematische Vergleiche, die institutionelle Faktoren wie 
Bildungssysteme und Arbeitsmarktregulierungen einbeziehen. Viele empirische Studien zur 
Automatisierung betrachten vorwiegend die Situation in den USA, wohingegen umfassende 
Analysen für \ac{OECD}-Länder mit unterschiedlichen Wohlfahrtsmodellen begrenzt sind.

Der langfristige Einfluss von \ac{ICT}-Investitionen auf die Arbeitslosigkeit 
verschiedener Bildungsgruppen wurde bisher nicht ausreichend mit einer quantitativen, 
länderübergreifenden Panelanalyse untersucht.

Um diese Forschungslücken zu schließen, wird im empirischen Teil dieser Arbeit eine 
Paneldatenanalyse über \ac{OECD}-Länder durchgeführt. Dadurch sollen systematische 
Unterschiede in den Auswirkungen von Digitalisierung auf die Arbeitslosigkeit nach 
Bildungsniveaus erfasst werden.

\include{chapters/theorie_und_hypothesen}
% TODO: Durchlesen...

%%%%%%%%%%%%%%
% Datensätze %
%%%%%%%%%%%%%%

\section{Daten und Methodik}

\subsection{Datensätze}

Die vorliegenden Daten stammen aus den umfangreichen Datensätzen der \ac{OECD}, einer 
internationalen Organisation, die vergleichbare Wirtschafts- und Sozialstatistiken für ihre 
Mitgliedsländer bereitstellt. Die \ac{OECD} sammelt und veröffentlicht regelmäßig Daten zu 
wirtschaftlichen, sozialen und technologischen Entwicklungen, die es ermöglichen, langfristige 
Trends und länderspezifische Unterschiede zu analysieren \parencite{oecd2022ict}.

Für diese Untersuchung werden insbesondere die Datensätze zu \ac{ICT}-Investitionen 
\parencite{oecd2022ict} sowie zu den Arbeitslosenquoten nach Bildungsniveau 
\parencite{oecd2022unemployment} verwendet. Zusätzlich wurden weitere ökonomische und 
institutionelle Indikatoren als Kontrollvariablen integriert, um die Robustheit der Analyse 
zu erhöhen. Dazu gehören das \ac{BIP} pro Kopf \parencite{oecd2022gdp}, die Gewerkschaftsdichte 
\parencite{oecd2022tud}, der Anteil der Bevölkerung mit tertiärem Bildungsabschluss 
\parencite{oecd2022education} sowie der Grad der Regulierung des Arbeitnehmerschutzes 
\parencite{oecd2022regulation}. Zudem wird die Wohlfahrtsstaatentypologie nach Esping-Andersen 
\parencite{espingandersen1990thethree} genutzt, um institutionelle Unterschiede zwischen den 
Ländern zu erfassen.

Die finalen Daten umfassen insgesamt 35 OECD- und ausgewählte Nicht-OECD-Länder
\footnote{Untersuchte Länder: Australien, Österreich, Belgien, Bulgarien, Brasilien, 
Kanada, Kroatien, Tschechien, Dänemark, Estland, Finnland, Frankreich, Deutschland, 
Griechenland, Ungarn, Island, Italien, Irland, Lettland, Litauen, Luxemburg, Niederlande, 
Neuseeland, Norwegen, Polen, Portugal, Rumänien, Spanien, Schweden, Schweiz, Türkei, Slowakei, 
Slowenien, Vereinigtes Königreich, USA.} und decken den Zeitraum von 2005 bis 2022 ab. Nach 
der Bereinigung und Zusammenführung der relevanten Variablen verbleiben 3973 Beobachtungen 
für die Paneldatenanalyse.

Die \ac{ICT}-Investitionen messen die Bruttoanlageinvestitionen in digitale Infrastrukturen 
und Technologien \parencite{oecd2022ict}. Die Arbeitsmarktstatistiken bieten detaillierte 
Informationen über die Arbeitslosenquoten in verschiedenen Bildungsgruppen 
\parencite{oecd2022unemployment}. Durch die Ergänzung um Kontrollvariablen wie den Anteil 
tertiär gebildeter Personen und die Regulierungsstrenge des Arbeitsmarkts wird sichergestellt, 
dass sowohl wirtschaftliche als auch institutionelle Unterschiede in den Ländern angemessen 
berücksichtigt werden. Zur Sicherstellung einer vollständigen Zeitreihe wurden fehlende Werte 
der Variable `PERCENT\_EMPLOYEES\_TUD` mittels linearer Interpolation ergänzt.  

Die längsschnittliche Struktur der Daten ermöglicht es, sowohl kurzfristige als auch 
langfristige Entwicklungen zu analysieren und Zusammenhänge zwischen ICT-Investitionen und 
Arbeitslosigkeit differenziert zu betrachten.


%%%%%%%%%%%%%%%%%%%%%%%
% Operationalisierung %
%%%%%%%%%%%%%%%%%%%%%%%

\subsection{Operationalisierung}

Zur Beantwortung der Forschungsfrage - wie Investitionen in Informations- und 
Kommunikationstechnologien die Arbeitslosenquoten in unterschiedlichen Bildungsgruppen 
beeinflussen - ist eine präzise und konsistente Operationalisierung der zentralen Konzepte 
notwendig. Dies gewährleistet, dass die Untersuchung die beabsichtigten Zusammenhänge abbildet 
und die Daten sinnvoll ausgewertet werden können.

Die abhängige Variable dieser Untersuchung ist die \textit{Arbeitslosenquote} 
(UNEMPLOYMENT\_RATE\_PERCENT), die nach dem Bildungsniveau der Bevölkerung differenziert wird. 
Der \ac{OECD}-Datensatz unterteilt das Bildungsniveau in drei Hauptkategorien:

\begin{enumerate}
    \item \textbf{niedriges Bildungsniveau} (Low education): Personen ohne abgeschlossene 
    Schulbildung oder mit einem maximalen Hauptschulabschluss.
    \item \textbf{mittleres Bildungsniveau} (Medium education): Personen mit 
    Sekundarschulabschluss oder einer abgeschlossenen Berufsausbildung.
    \item \textbf{hohes Bildungsniveau} (High education): Personen mit Hochschulabschluss, wie 
    einem Bachelor, Master oder Doktortitel.
\end{enumerate}

Arbeitslose sind nach der Definition der \ac{OECD} Personen im erwerbsfähigen Alter, die keine 
Arbeit haben, für eine Arbeit zur Verfügung stehen und in den letzten vier Wochen konkrete 
Schritte unternommen haben, um eine Arbeit zu finden \parencite{oecd2022unemployment}. Dieser 
Indikator wird als Prozentsatz der Erwerbsbevölkerung gemessen und ist saisonbereinigt.

Die zentrale unabhängige Variable \textit{\ac{ICT}-Investitionen} (ICT\_INVEST\_SHARE\_GDP) 
misst Investitionen in digitale Infrastruktur, Software, Hardware und Technologien, die zur 
Verbesserung betrieblicher Effizienz und Produktivität beitragen \parencite{oecd2022ict}. Die 
Daten basieren auf den Definitionen des \ac{SNA08} und werden als Anteil am \ac{BIP} in Prozent 
angegeben.

Um sicherzustellen, dass der Effekt der \ac{ICT}-Investitionen auf die Arbeitslosenquote nicht 
durch andere Faktoren verzerrt wird, werden mehrere Kontrollvariablen in die Analyse aufgenommen:

\begin{itemize}
    \item \textbf{\ac{BIP} pro Kopf} (GDP\_PER\_CAPITA): Diese Variable misst den wirtschaftlichen 
    Wohlstand eines Landes in tausend US-Dollar pro Jahr und kontrolliert den Entwicklungsstand 
    eines Landes, da wirtschaftlich wohlhabendere Länder tendenziell niedrigere Arbeitslosenquoten 
    aufweisen \parencite{oecd2022gdp}.
    \item \textbf{Gewerkschaftsdichte} (PERCENT\_EMPLOYEES\_TUD): Der Anteil der in Gewerkschaften 
    organisierten Arbeitnehmer wird berücksichtigt, da Gewerkschaften eine wichtige Rolle bei der 
    Aushandlung von Arbeitsbedingungen und Arbeitsplatzsicherheit spielen \parencite{oecd2022tud}. 
    Frühere Studien zeigen, dass eine hohe Gewerkschaftsdichte oft mit niedrigeren 
    Arbeitslosenquoten für geringqualifizierte Arbeitnehmer verbunden ist, da Gewerkschaften 
    Mindestlöhne sichern und Beschäftigungsschutzmaßnahmen verstärken 
    \parencite[S. 61]{nickell1997unemployment}. Zur Sicherstellung einer vollständigen Zeitreihe 
    wurden fehlende Werte dieser Variable mittels linearer Interpolation ergänzt.
    \item \textbf{Wohlfahrtsstaatentyp} (WELFARE\_STATE): Die Wohlfahrtsstaatentypologie nach 
    Esping-Andersen \parencite{espingandersen1990thethree} wird in die Analyse integriert, 
    um zu untersuchen, inwiefern institutionelle Unterschiede den Effekt von \ac{ICT}-Investitionen 
    auf die Arbeitslosenquote beeinflussen. Die Länder werden in der Analyse in fünf Kategorien 
    unterteilt: \textit{sozialdemokratisch}, \textit{konservativ}, \textit{liberal}, 
    \textit{südeuropäisch} und \textit{postsozialistisch}.
    \item \textbf{Tertiärer Bildungsanteil} (PERCENT\_TERTIARY\_EDUCATION): Diese 
    Variable gibt den Prozentsatz der Bevölkerung an, der einen tertiären Bildungsabschluss 
    besitzt. Eine höhere Bildungsbeteiligung könnte den negativen Einfluss von 
    \ac{ICT}-Investitionen auf geringqualifizierte Arbeitnehmer abschwächen, da mehr Menschen für 
    technologische Berufe qualifiziert sind \parencite{oecd2022education}.
    \item \textbf{Regulierungsstrenge des Arbeitsmarkts} (REGULATION\_STRICTNESS): Diese Variable 
    misst, wie stark der Arbeitsmarkt eines Landes reguliert ist, insbesondere im Hinblick auf 
    Kündigungsschutz und Beschäftigungsflexibilität. Striktere Regulierung kann die 
    Arbeitslosenquote erhöhen, da Unternehmen zögerlicher bei Neueinstellungen sind 
    \parencite{oecd2022regulation}.
\end{itemize}

Die Kombination dieser Daten ermöglicht es, länderspezifische Unterschiede in der Wirtschaftskraft, 
den regulatorischen Rahmenbedingungen und der Bildungsstruktur zu kontrollieren, um Zusammenhänge 
zwischen ICT-Investitionen und Arbeitslosenquoten differenziert zu analysieren. Zudem erlauben die 
aufgenommenen Kontrollvariablen eine differenziertere Betrachtung institutioneller Faktoren, die 
den Arbeitsmarkt beeinflussen.

\subsection{Analytische Methode}

Die Analyse dieser Arbeit basiert auf einer Paneldatenanalyse, um die Auswirkungen von 
\ac{ICT}-Investitionen auf die Arbeitslosenquote nach Bildungsniveau zu untersuchen. Die Wahl 
einer Paneldatenmethode ermöglicht es, sowohl individuelle Heterogenität zwischen Ländern als auch 
dynamische Entwicklungen über die Zeit zu erfassen \parencite{wooldridge2010econometric}.

Zur Untersuchung der Effekte wurden zwei gängige Modelle der Paneldatenanalyse betrachtet: das 
\ac{FE}- und das \ac{RE}-Modell. Während das \ac{FE}-Modell explizit für zeitinvariante 
länderspezifische Eigenschaften kontrolliert \parencite[S. 251–256]{wooldridge2010econometric}, 
geht das \ac{RE}-Modell davon aus, dass länderspezifische Effekte zufällig verteilt und nicht mit 
den unabhängigen Variablen korreliert sind \parencite[S. 17–20]{baltagi2021econometric}. Falls 
diese Annahme verletzt ist, können die Schätzungen im \ac{RE}-Modell verzerrt sein.

Für diese Analyse wurde das \ac{FE}-Modell gewählt, da es eine robustere Schätzung der 
Zusammenhänge zwischen \ac{ICT}-Investitionen und der Arbeitslosigkeit ermöglicht. Dies ist 
besonders relevant, da die Untersuchung auf Veränderungen innerhalb eines Landes über die Zeit 
fokussiert und länderspezifische Eigenschaften nicht als erklärende Variablen modelliert werden. 
Die Modellwahl basiert auf einem Hausman-Test, dessen Ergebnisse für Fixed Effects sprechen.

Darüber hinaus wird die Analyse durch Interaktionseffekte ergänzt, die es ermöglichen, 
institutionelle Unterschiede zwischen den Ländern zu berücksichtigen. Die Modelle beinhalten 
Dummy-Variablen für die Wohlfahrtsstaatentypen, um systematische Unterschiede zwischen den 
Regimetypen in ihrer Reaktion auf \ac{ICT}-Investitionen zu identifizieren 
\parencite{espingandersen1990thethree}. Zusätzlich werden Jahres-Faktor-Dummies integriert, um 
allgemeine makroökonomische Entwicklungen (z. B. Finanzkrisen oder technologische Schübe) aus den 
Modellen zu kontrollieren. Die Interaktion zwischen ICT-Investitionen und Wohlfahrtsstaatentypen 
ermöglicht eine differenzierte Analyse der moderierenden Effekte institutioneller Rahmenbedingungen.

Durch diese Kombination aus \ac{FE}-Modellen, Interaktionseffekten und Zeitdummies bietet die 
Paneldatenanalyse eine solide Grundlage für die Untersuchung der Auswirkungen von 
\ac{ICT}-Investitionen auf die Arbeitslosigkeit. Die longitudinale Struktur der Daten erlaubt eine 
differenzierte Betrachtung, die sowohl kurzfristige als auch langfristige Beschäftigungseffekte 
berücksichtigt.

% TODO: Durchlesen! Zitation!
% TODO: Variablen nochmal checken!
% TODO: Grafiken nochmal checken!

\section{Ergebnisse}

%%%%%%%%%%%%%%%%%%%%%%%%%%
% Deskriptive Ergebnisse %
%%%%%%%%%%%%%%%%%%%%%%%%%%

\subsection{Deskriptive Ergebnisse}

% Deskriptive Statistik: Einleitung
Die deskriptiven Statistiken der analysierten Variablen bieten einen umfassenden Einblick 
in deren Eigenschaften und Verteilungen über die beobachteten Länder und Zeiträume. Im 
Folgenden werden die Ergebnisse detailliert beschrieben:

\input{assets/variables_fin.tex}

% Deskriptive Statistik: Variablen "ICT-Investitionen"
Die Variable \textit{\ac{ICT}-Investitionen}, welche den Anteil der Investitionen in 
\ac{ICT} am \ac{BIP} misst \parencite{oecd2022ict}, variiert zwischen einem Minimum von 
0,73\% und einem Maximum von 8,69\%. Der Mittelwert beträgt 2,49\%, während der Median mit 
2,30\% leicht darunter liegt. Dies deutet auf eine leicht rechtsschiefe Verteilung hin, 
da einige Länder besonders hohe Investitionen in \ac{ICT} tätigen. Die Standardabweichung 
von 1,03 zeigt, dass es zwischen den OECD-Ländern erhebliche Unterschiede in der Intensität 
der \ac{ICT}-Investitionen gibt. Während einige Länder konstant hohe Anteile ihrer 
wirtschaftlichen Ressourcen in digitale Technologien investieren, gibt es andere, die 
vergleichsweise geringe Investitionen tätigen. Diese Unterschiede können durch verschiedene 
Faktoren beeinflusst sein, darunter wirtschaftliche Leistungsfähigkeit, politische 
Strategien zur Förderung der Digitalisierung sowie strukturelle Unterschiede in der 
Entwicklung des \ac{ICT}-Sektors.

% Deskriptive Statistik: Variable "Arbeitslosenquote"
Die Variable \textit{Arbeitslosenquote}, welche die Arbeitslosenquote in Prozent angibt 
\parencite{oecd2022unemployment}, schwankt erheblich zwischen einem Minimum von 0,82\% und 
einem Maximum von 49,89\%. Der Mittelwert liegt bei 7,79\%, während der Median mit 5,83\% 
etwas niedriger ausfällt. Dies weist auf eine rechtsschiefe Verteilung hin, da einige 
Länder oder Zeitpunkte mit sehr hohen Arbeitslosenquoten als Ausreißer wirken können. Die 
hohe Standardabweichung von 6,36 deutet darauf hin, dass die Arbeitslosenquoten 
zwischen den Ländern und über die Zeit hinweg erhebliche Unterschiede aufweisen. Während 
einige OECD-Länder durch eine geringe Arbeitslosenquote und stabile Arbeitsmärkte 
gekennzeichnet sind, zeigen andere Länder insbesondere in wirtschaftlichen Krisenzeiten 
oder strukturschwachen Regionen signifikant höhere Arbeitslosenraten. Diese Heterogenität 
könnte zudem mit unterschiedlichen Arbeitsmarktregulierungen und Bildungssystemen 
zusammenhängen.

% Deskriptive Statistik: Variable "BIP pro Kopf"
Das \textit{\ac{BIP} pro Kopf}, welches das Pro-Kopf-Einkommen in US-Dollar angibt 
\parencite{oecd2022gdp}, weist eine erhebliche Spannweite auf: Es reicht von 13.344,18 
US-Dollar bis 137.716,45 US-Dollar. Der Mittelwert beträgt 43.344,89 US-Dollar, während 
der Median mit 40.682,97 US-Dollar nur geringfügig darunter liegt. Trotz dieser relativen 
Nähe deutet die hohe Standardabweichung von 18.142,71 darauf hin, dass es erhebliche 
Wohlstandsunterschiede zwischen den \ac{OECD}-Ländern gibt. Dies spricht für eine starke 
rechtsschiefe Verteilung, da einige besonders wohlhabende Länder den Durchschnittswert 
nach oben treiben. Diese Unterschiede sind insbesondere für die Interpretation der 
\ac{ICT}-Investitionen relevant, da wohlhabendere Länder tendenziell eine höhere 
Kapitalausstattung und damit größere Investitionsmöglichkeiten in digitale Infrastruktur 
haben könnten. Gleichzeitig können Unterschiede im \ac{BIP} pro Kopf Einfluss auf die 
Struktur des Arbeitsmarktes und damit auf die Verteilung der Arbeitslosigkeit nach 
Bildungsgrad haben.

% Deskriptive Statistik: Variable "Gewerkschaftsdichte"
Die Variable \textit{Gewerkschaftsdichte}, welche die gewerkschaftliche Organisationsrate 
eines Landes misst \parencite{oecd2022tud}, zeigt eine erhebliche Varianz zwischen den 
Ländern. Die Werte reichen von einem Minimum von 4,50\% bis zu einem Maximum von 92,20\%. 
Der Mittelwert beträgt 28,61\%, während der Median mit 20,45\% darunter liegt, was darauf 
hindeutet, dass einige Länder eine besonders hohe Gewerkschaftsbindung haben, während die 
Mehrheit unter diesem Durchschnittswert bleibt.Die Standardabweichung von 21,03 
verdeutlicht die große Heterogenität in der gewerkschaftlichen Organisation zwischen den 
Ländern. Während einige nordische Länder traditionell hohe Gewerkschaftsdichten aufweisen, 
sind Gewerkschaften in anderen \ac{OECD}-Staaten weniger stark in den Arbeitsmarkt 
integriert. Dies könnte Implikationen für die Verhandlungsmacht von Arbeitnehmern haben, 
was sich wiederum auf Lohnstrukturen und Beschäftigungssicherheit auswirken kann.

% Deskriptive Statistik: Variablen (Zusammenfassung)
Die deskriptiven Statistiken zeigen, dass die betrachteten Variablen eine erhebliche 
Heterogenität aufweisen, die sowohl auf länderspezifische Unterschiede als auch auf 
strukturelle und wirtschaftliche Faktoren zurückgeführt werden kann. Besonders auffällig 
sind die Unterschiede in den \ac{ICT}-Investitionen, die je nach wirtschaftlicher 
Leistungsfähigkeit und politischen Rahmenbedingungen stark variieren. Auch die 
Arbeitslosenquoten zeigen eine hohe Streuung, die möglicherweise mit den unterschiedlichen 
Bildungsniveaus, Arbeitsmarktinstitutionen und Wirtschaftsentwicklungen der Länder 
zusammenhängt.Die hohe Varianz im \ac{BIP} pro Kopf unterstreicht die unterschiedlichen 
wirtschaftlichen Ausgangsbedingungen der Länder, was sich sowohl auf die Höhe der 
\ac{ICT}-Investitionen als auch auf die Struktur der Arbeitsmärkte auswirken könnte. 
Schließlich zeigt die Gewerkschaftsdichte ebenfalls starke Unterschiede zwischen den 
OECD-Ländern, was für die Analyse der institutionellen Faktoren relevant ist, die 
möglicherweise als Moderatoren der Auswirkungen von \ac{ICT}-Investitionen auf den 
Arbeitsmarkt fungieren.

% Deskriptive Analyse: Grafiken (Einleitung)
Diese deskriptive Analyse der Variablen bildet die Grundlage für die nachfolgenden 
grafische Darstellungen, die eine detailliertere Visualisierung der Trends und Unterschiede 
zwischen den Ländern ermöglicht. Sie bietet einen ersten Einblick auf Länderebene in die 
Beziehung zwischen \textit{\ac{ICT}-Investitionen} (gemessen als Anteil am \ac{BIP}) und 
den Arbeitslosenquoten, differenziert nach den drei genannten Bildungsgruppen. Hierbei 
wird jeweils ein repräsentatives Land pro Wohlfahrtsstaatentyp für die Analyse gewählt - 
Spanien als südeuropäischer, Polen als post-sozialisitscher, Schweden als nordischer und 
Deutschland als mitteleuropäischer Wohlfahrtsstaat im Zeitraum von 2005 bis 2022. Ziel ist 
es, vor der multivariaten Analyse bereits Unterschiede und Trends innerhalb der Länder und 
zwischen den Bildungsgruppen zu identifizieren.

% Deskriptive Analyse: Grafik "Spanien"
\begin{figure}[htbp]
    \centering
    \includegraphics[width=\textwidth]{assets/plot_spain.png}
    \caption{Überblick über \textit{\ac{ICT}-Investitionen} und Arbeitslosenquote in 
    Spanien}
    \label{fig:spain}
\end{figure}

Die Abbildung zeigt die Entwicklung der \textit{\ac{ICT}-Investitionen} als Anteil am 
BIP sowie die Arbeitslosenquote in Spanien zwischen 2005 und 2022, differenziert nach 
Bildungsniveau - Spanien steht hier repräsentativ für südeuropäische Wohlfahrtsstaaten. 
Am Beispiel Spaniens ist ein besonders markanter Anstieg der Arbeitslosenquote während 
der Finanz- und Wirtschaftskrise von 2008 bis 2013 zu beobachten. Während die 
\textit{\ac{ICT}-Investitionen} einen insgesamt moderaten Anstieg über den gesamten 
Zeitraum hinweg zeigen, lassen sich drastische Schwankungen in der Arbeitslosenquote 
identifizieren, insbesondere bei Personen mit niedrigem und mittlerem Bildungsniveau.

Bei Personen mit einem niedrigen Bildungsniveau zeigt sich zwischen 2005 und 2008 
eine relativ stabile Arbeitslosenquote von knapp unter 10\%. Ab 2008 kam es jedoch zu 
einem rasanten Anstieg, der bis 2013 einen Höchststand von über 30\% erreichte. Erst 
nach 2013 begann ein kontinuierlicher Rückgang, der sich bis 2020 auf etwa 20\% 
fortsetzte, bevor ein erneuter leichter Anstieg zu beobachten ist. Die 
\textit{\ac{ICT}-Investitionen} entwickelten sich hingegen gleichmäßiger. Sie begannen 
auf einem niedrigen Niveau von etwa 1,75\% des BIP, zeigten nach der Finanzkriese ab 
2010 eine Aufwärtstendenz und stabilisierten sich nach 2015 bei etwa 2,5\%. Der 
Rückgang der Arbeitslosenquote nach 2013 verlief jedoch unabhängig von einer abrupten 
Zunahme der \textit{\ac{ICT}-Investitionen}, was darauf hindeutet, dass 
makroökonomische Faktoren (z. B. wirtschaftliche Erholung, Beschäftigungsprogramme) 
für die Senkung der Arbeitslosigkeit eine zentrale Rolle spielten.

Bei Personen mit mittlerem Bildungsniveau zeigt sich ein sehr ähnlicher Verlauf. Die 
Arbeitslosenquote lag 2005 noch unter 8\%, stieg im Zuge der Wirtschaftskrise bis 2013 
jedoch auf über 20\% an. Erst ab 2014 begann ein deutlicher Rückgang, der sich bis 
2020 auf etwa 10\% fortsetzte. Die \textit{\ac{ICT}-Investitionen} folgten hier einem 
vergleichbaren Muster wie in der Gruppe der gering Qualifizierten, wobei ein leichter, 
aber kontinuierlicher Anstieg sichtbar ist. Dennoch ist keine direkte Korrelation 
zwischen dem Verlauf der \textit{\ac{ICT}-Investitionen} und der Arbeitslosenquote 
ersichtlich, da der massive Anstieg und der spätere Rückgang der Arbeitslosigkeit 
primär durch die wirtschaftliche Entwicklung und nicht durch technologische 
Investitionen bedingt zu sein scheinen.

Bei Personen mit hohem Bildungsniveau war die Arbeitslosenquote insgesamt niedriger, 
zeigte jedoch ebenfalls einen deutlichen Anstieg während der Wirtschaftskrise. Im Jahr 
2005 lag sie unter 5\%, erreichte 2013 jedoch fast 15\%. Danach setzte auch hier ein 
Rückgang ein, und bis 2020 fiel die Quote auf etwa 5\% zurück. Im Gegensatz zu den 
anderen Bildungsgruppen scheinen sich hier die \textit{\ac{ICT}-Investitionen} und 
die Arbeitslosenquote teilweise gegenläufig zu entwickeln. Während die 
\textit{\ac{ICT}-Investitionen} nach 2010 eine stetige Steigerung zeigen und nach 
2015 stabil auf etwa 2,5\% des BIP bleiben, geht die Arbeitslosenquote in derselben 
Phase zurück. Dies könnte darauf hindeuten, dass hochqualifizierte Arbeitskräfte in 
Spanien stärker von der Digitalisierung profitieren konnten als Personen mit 
niedrigerem Bildungsstand.

Spanien als südeuropäischer Wohlfahrtsstaat ist durch einen stark segmentierten 
Arbeitsmarkt gekennzeichnet, der sich durch hohe Anteile an befristeten 
Beschäftigungsverhältnissen sowie eine geringere Arbeitsplatzsicherheit auszeichnet. 
Dies könnte eine Erklärung für die starken Schwankungen der Arbeitslosenquote 
im Zuge der Finanzkrise sein, da insbesondere gering und mittel Qualifizierte 
von Entlassungen betroffen waren. Die \textit{\ac{ICT}-Investitionen} scheinen 
langfristig zwar leicht anzusteigen, doch zeigt sich kein direkter Zusammenhang 
zwischen diesen Investitionen und der Arbeitslosenquote in den jeweiligen 
Bildungsgruppen. Vielmehr deutet die Entwicklung darauf hin, dass der Arbeitsmarkt 
in Spanien stark konjunkturabhängig ist und die wirtschaftliche Erholung nach 2013 
die wichtigste Triebkraft für die Reduktion der Arbeitslosigkeit war.

Zusammenfassend zeigen die Daten für Spanien eine enge Verbindung zwischen der 
Finanzkrise und den massiven Schwankungen der Arbeitslosenquote, insbesondere bei 
gering und mittel Qualifizierten. Während \textit{\ac{ICT}-Investitionen} über den 
Zeitraum hinweg einen kontinuierlichen, aber moderaten Anstieg zeigen, sind ihre 
direkten Auswirkungen auf die Arbeitslosigkeit unklar. Es könnte jedoch sein, dass 
insbesondere Hochqualifizierte von den steigenden \textit{\ac{ICT}-Investitionen} 
profitieren konnten, während gering Qualifizierte eher von konjunkturellen 
Faktoren abhängig waren.

% Deskriptive Analyse: Grafik "Polen"
\begin{figure}[htbp]
    \centering
    \includegraphics[width=\textwidth]{assets/plot_poland.png}
    \caption{Überblick über \textit{\ac{ICT}-Investitionen} und Arbeitslosenquote 
    in Polen}
    \label{fig:poland}
\end{figure}

Die Abbildung zeigt die Entwicklung der \textit{\ac{ICT}-Investitionen} als Anteil 
am BIP sowie die Arbeitslosenquote in Polen zwischen 2005 und 2022 differenziert 
nach Bildungsniveau - Polen steht hier repräsentativ für postsozialistische 
Wohlfahrtsstaaten.

Auffällig ist der durchgängige Rückgang der Arbeitslosenquote in allen 
Bildungsgruppen, während die \textit{\ac{ICT}-Investitionen} über weite Strecken 
konstant bleiben, beziehungsweise sogar ebenfalls einen Rückgang verzeichnen. Dies 
deutet darauf hin, dass makroökonomische oder arbeitsmarktpolitische Faktoren für 
den Rückgang der Arbeitslosigkeit maßgeblich verantwortlich sein könnten.

Bei Personen mit einem niedrigen Bildungsniveau lag die Arbeitslosenquote im Jahr 
2005 bei knapp 28\%. In den darauffolgenden Jahren kam es zu einem raschen Rückgang, 
wobei jedoch zwischen 2010 und 2015 eine Stagnation mit einem kurzen Anstieg auf fast 
20\% zu beobachten ist. Nach 2015 setzte sich der Rückgang der Arbeitslosenquote 
fort, sodass sie bis 2020 auf 8\% fiel. Die \textit{\ac{ICT}-Investitionen} blieben 
über den gesamten Zeitraum hinweg weitgehend konstant und bewegten sich um die 1\% des 
BIP, mit einem leichten Rückgang zwischen 2010 und 2015. Dies deutet darauf hin, dass  
der starke Rückgang der Arbeitslosigkeit nicht direkt mit den 
\textit{\ac{ICT}-Investitionen} zusammenhängt, sondern durch andere wirtschaftliche 
Faktoren beeinflusst wurde, beispielsweise durch eine allgemeine wirtschaftliche 
Stabilisierung nach dem EU-Beitritt Polens und steigende Beschäftigungsmöglichkeiten 
in arbeitsintensiven Branchen.

Für Personen mit einem mittleren Bildungsniveau zeigt sich ein ähnliches Muster, wenn 
auch auf einem insgesamt niedrigeren Ausgangsniveau der Arbeitslosenquote. Während 
diese 2005 noch über 10\% lag, sank sie in den darauffolgenden Jahren rasch auf etwa 
3\% bis 2015 und weiter unter 2\% bis 2020. Zwischen 2010 und 2015 ist jedoch eine 
leichte Erhöhung der Arbeitslosenquote erkennbar, bevor der Trend weiter nach unten 
verlief. Der Rückgang der Arbeitslosigkeit erfolgt weitgehend unabhängig von der 
Entwicklung der \textit{\ac{ICT}-Investitionen}, was darauf hindeutet, dass 
makroökonomische Faktoren wie die Industrialisierung und eine steigende Nachfrage 
nach Arbeitskräften mit mittlerer Qualifikation eine bedeutendere Rolle gespielt 
haben könnten.

Für Personen mit einem hohen Bildungsniveau war die Arbeitslosenquote bereits 
2005 relativ niedrig, lag aber dennoch bei etwa 6\%, was im Vergleich zu anderen 
europäischen Ländern eher hoch ist. Dies könnte auf strukturelle Faktoren des 
polnischen Arbeitsmarktes zurückzuführen sein, wie eine geringere Anzahl 
hochqualifizierter Beschäftigungsmöglichkeiten in den frühen 2000er-Jahren. In den 
darauffolgenden Jahren fiel die Arbeitslosenquote jedoch deutlich und lag bereits 
2015 unter 2\%. Auffällig ist, dass die \textit{\ac{ICT}-Investitionen} in dieser 
Gruppe im Gegensatz zu den anderen Bildungsgruppen eine leichte Steigerung zeigen. 
In der ersten Hälfte des Beobachtungszeitraums bewegten sich die 
\textit{\ac{ICT}-Investitionen} um 1,2\% des BIP, während sie in den Jahren nach 
2015 tendenziell anstiegen. Dies könnte darauf hindeuten, dass der polnische 
Arbeitsmarkt mit steigendem ICT-Investitionsanteil zunehmend hochqualifizierte 
Beschäftigungsmöglichkeiten geschaffen hat. Dennoch bleibt die Kausalität unklar, 
da die Arbeitslosenquote in dieser Gruppe bereits gefallen war, bevor der leichte 
Anstieg der \textit{\ac{ICT}-Investitionen} einsetzte.

Polen als postsozialistischer Wohlfahrtsstaat hat in den letzten Jahrzehnten einen 
tiefgreifenden wirtschaftlichen Wandel durchlaufen. Der EU-Beitritt im Jahr 2004 
führte zu verstärkten ausländischen Direktinvestitionen, einer zunehmenden 
Integration in europäische Produktionsnetzwerke sowie einer generellen 
Modernisierung der Wirtschaft. Diese Entwicklungen spiegeln sich auch in der 
Reduktion der Arbeitslosigkeit wider, die in allen Bildungsgruppen signifikant 
gesunken ist. Besonders bei Personen mit mittlerem und niedrigem Bildungsniveau 
könnte die Expansion von Industriejobs sowie der Dienstleistungssektor eine 
wesentliche Rolle gespielt haben. % TODO: Zitieren!

Insgesamt zeigt die Abbildung, dass die Arbeitslosenquote in allen Bildungsgruppen 
stark gesunken ist, während die \textit{\ac{ICT}-Investitionen} nur moderate 
Schwankungen aufweisen. Dies deutet darauf hin, dass die Haupttreiber der 
Beschäftigungsentwicklung in Polen eher in wirtschaftlichen und 
arbeitsmarktpolitischen Veränderungen zu suchen sind als in den direkten 
Auswirkungen von \textit{\ac{ICT}-Investitionen}. Dennoch könnte die leichte 
Zunahme der \textit{\ac{ICT}-Investitionen} im späteren Beobachtungszeitraum 
darauf hinweisen, dass sich der polnische Arbeitsmarkt allmählich in Richtung 
einer wissensbasierten Wirtschaft entwickelt, in der besonders Hochqualifizierte 
profitieren.

% Deskriptive Analyse: Grafik "Schweden"
\begin{figure}[htbp]
    \centering
    \includegraphics[width=\textwidth]{assets/plot_sweden.png}
    \caption{Überblick über \textit{\ac{ICT}-Investitionen} und Arbeitslosenquote in 
    Schweden}
    \label{fig:sweden}
\end{figure}

Die Abbildung zeigt die Entwicklung der \textit{\ac{ICT}-Investitionen} als Anteil 
am BIP sowie die Arbeitslosenquote in Schweden zwischen 2005 und 2022, differenziert 
nach Bildungsniveau - Schweden steht hier repräsentativ für nordische 
Wohlfahrtsstaaten. Im Gegensatz zu anderen Ländern ist hier eine relativ stabile 
Entwicklung der Arbeitslosenquote über den gesamten Zeitraum zu beobachten, mit nur 
moderaten Schwankungen. Auffällig ist zudem, dass die \textit{\ac{ICT}-Investitionen} 
in Schweden im internationalen Vergleich auf einem vergleichsweise hohen Niveau liegen. 
Während sie in der ersten Dekade leichte Schwankungen zeigen, bleibt ihr Niveau ab 
2010 weitgehend konstant und steigt gegen Ende des Betrachtungszeitraums leicht an.

Bei Personen mit einem niedrigen Bildungsniveau lag die Arbeitslosenquote 2005 bei knapp 
5\% und zeigte bis etwa 2010 einen moderaten Anstieg. Nach 2010 stabilisierte sich die 
Arbeitslosenquote zunächst, bevor sie ab 2015 einen erneuten Aufwärtstrend verzeichnete. 
Besonders auffällig ist der deutliche Anstieg nach 2018, der sich bis 2022 fortsetzt. 
Während die Arbeitslosenquote für gering Qualifizierte also in den letzten Jahren 
gestiegen ist, sind die \textit{\ac{ICT}-Investitionen} im selben Zeitraum weitgehend 
stabil geblieben, wenn auch mit einer leicht positiven Tendenz. Dies könnte darauf 
hindeuten, dass die fortschreitende Digitalisierung möglicherweise die 
Beschäftigungsmöglichkeiten für niedrig qualifizierte Arbeitskräfte verschlechtert hat, 
indem sie bestimmte Arbeitsplätze verdrängte oder die Anforderungen an digitale 
Kompetenzen erhöhte - wahrscheinlich hängt diese Beobachtung aber eher mit der 
Corona-Pandemie zusammen.

Für Personen mit einem mittleren Bildungsniveau zeigt sich ein stabiles Muster, mit 
einer weitgehend konstanten Arbeitslosenquote zwischen 2005 und 2018. Während die 
Arbeitslosigkeit 2005 bei unter 5\% lag, gab es bis 2015 eine leichte Abwärtsbewegung, 
gefolgt von einer weitgehenden Stabilisierung. Nach 2018 zeigt sich eine leicht steigende 
Tendenz der Arbeitslosenquote, wenn auch weniger ausgeprägt als bei den gering 
Qualifizierten. Die \textit{\ac{ICT}-Investitionen} sind in dieser Gruppe durchgängig 
hoch und zeigen eine stabile Entwicklung mit leichten Schwankungen. Anders als bei den 
gering Qualifizierten ist hier keine klare gegenläufige Entwicklung zwischen 
\textit{\ac{ICT}-Investitionen} und Arbeitslosigkeit zu erkennen, was darauf hindeutet, 
dass mittlere Qualifikationen in Schweden weniger stark von den technologischen 
Veränderungen betroffen sind.

Bei Personen mit einem hohen Bildungsniveau zeigt sich über den gesamten Zeitraum 
hinweg eine extrem niedrige Arbeitslosenquote. Bereits 2005 lag sie unter 5\% und blieb 
über den gesamten Zeitraum stabil, mit nur minimalen Schwankungen. Auffällig ist, dass 
die \textit{\ac{ICT}-Investitionen} in dieser Gruppe im internationalen Vergleich sehr 
hoch sind, mit Werten, die konstant bei 4-5\% des \ac{BIP} liegen. Die Kombination aus 
hoher \ac{ICT}-Investition und niedriger Arbeitslosenquote deutet darauf hin, dass 
hochqualifizierte Arbeitskräfte in Schweden stark von der Digitalisierung profitieren 
konnten. Dies entspricht auch theoretischen Erwartungen, da hochqualifizierte 
Beschäftigte in wissensintensiven Branchen tätig sind, die von technologischen 
Innovationen profitieren.

Schweden als nordischer Wohlfahrtsstaat zeichnet sich durch ein stark reguliertes, 
aber flexibles Arbeitsmarktsystem aus, das durch hohe Sozialleistungen, eine starke 
Gewerkschaftsbindung und ein gut ausgebautes Bildungssystem geprägt ist. Die stabilen 
\textit{\ac{ICT}-Investitionen} und die insgesamt niedrigen Arbeitslosenquoten deuten 
darauf hin, dass der schwedische Arbeitsmarkt relativ widerstandsfähig gegenüber 
technologischen Veränderungen ist. Allerdings lässt sich bei niedrig qualifizierten 
Arbeitskräften ein Anstieg der Arbeitslosigkeit nach 2018 beobachten, der 
möglicherweise mit strukturellen Veränderungen auf dem Arbeitsmarkt zusammenhängt. 
Dies könnte darauf hindeuten, dass bestimmte Berufe durch die Digitalisierung 
zunehmend verdrängt werden oder dass sich die Anforderungen an digitale Kompetenzen 
verstärkt haben, sodass Geringqualifizierte Schwierigkeiten haben, sich an die 
veränderten Bedingungen anzupassen.

Insgesamt zeigt die Abbildung, dass Schweden ein stabiles Beschäftigungsniveau über 
den gesamten Zeitraum hinweg aufweist, wobei die \textit{\ac{ICT}-Investitionen} 
konstant hoch sind. Während Hoch- und Mittelqualifizierte weitgehend von den 
Entwicklungen profitieren konnten, scheint sich für gering Qualifizierte in den 
letzten Jahren eine Verschlechterung der Beschäftigungssituation abzuzeichnen. Dies 
könnte darauf hindeuten, dass Digitalisierung in hochentwickelten Volkswirtschaften 
wie Schweden zunehmend zu einer Polarisierung des Arbeitsmarktes führt, bei der 
Hochqualifizierte von den Investitionen profitieren, während gering Qualifizierte 
zunehmend unter Druck geraten.

% Deskriptive Analyse: Grafik "Deutschland"
\begin{figure}[htbp]
    \centering
    \includegraphics[width=\textwidth]{assets/plot_germany.png}
    \caption{Überblick über \textit{\ac{ICT}-Investitionen} und Arbeitslosenquote in 
    Deutschland}
    \label{fig:germany}
\end{figure}

Die Abbildung zeigt die Entwicklung der \textit{\ac{ICT}-Investitionen} als Anteil 
am BIP sowie die Arbeitslosenquote in Deutschland zwischen 2005 und 2022, 
differenziert nach Bildungsniveau - Deutschland steht hier repräsentativ für 
mitteleuropäische Wohlfahrtsstaaten. Dabei lassen sich klare Unterschiede zwischen 
den drei betrachteten Gruppen - niedriges, mittleres und hohes Bildungsniveau - sowohl 
hinsichtlich des Niveaus als auch der Veränderung der Arbeitslosenquoten erkennen. 
Insgesamt zeigen sich über den gesamten Zeitraum hinweg deutliche Rückgänge in der 
Arbeitslosenquote, während die \textit{\ac{ICT}-Investitionen} eine weitgehend 
stabile Entwicklung aufweisen.

Für Personen mit einem niedrigen Bildungsniveau zeigt sich eine besonders hohe 
Arbeitslosenquote zu Beginn des Beobachtungszeitraums, die 2005 bei über 18\% lag. 
In den darauffolgenden Jahren kam es zu einem kontinuierlichen Rückgang, der bis 2020 
Werte unter 5\% erreichte. Diese Entwicklung spiegelt die allgemeine Verbesserung des 
deutschen Arbeitsmarktes wider, insbesondere durch wirtschaftlichen Aufschwung und 
Reformen im Rahmen der Agenda 2010. Die \textit{\ac{ICT}-Investitionen} verzeichneten 
zwischen 2005 und 2010 zunächst einen leichten Rückgang, bevor sie sich um die 1,5\% 
des BIP stabilisierten. Ein direkter Zusammenhang zwischen 
\textit{\ac{ICT}-Investitionen} und der sinkenden Arbeitslosenquote ist nicht 
ersichtlich, da der Rückgang der Arbeitslosenquote bereits vor der leichten 
Stabilisierung der Investitionen begann.

Bei Personen mit mittlerem Bildungsniveau zeigt sich ein ähnliches Muster, wenn auch 
auf einem insgesamt niedrigeren Ausgangsniveau der Arbeitslosenquote. Während diese 
2005 noch bei etwa 10\% lag, fiel sie bis 2020 auf rund 3\% und blieb seither 
weitgehend stabil. Die \textit{\ac{ICT}-Investitionen} zeigen eine konstante 
Entwicklung mit geringen Schwankungen. Auch hier bleibt der direkte Zusammenhang 
zwischen den \textit{\ac{ICT}-Investitionen} und der Arbeitslosenquote unklar, da 
der Rückgang der Arbeitslosigkeit langfristig verläuft und nicht direkt mit den 
Investitionen korreliert.

Für Personen mit hohem Bildungsniveau zeigt sich über den gesamten Zeitraum hinweg 
eine sehr niedrige Arbeitslosenquote. Bereits 2005 lag sie unter 5\% und sank bis
 2010 auf unter 2\%, wo sie anschließend auf diesem niedrigen Niveau stabil blieb. 
 Im Vergleich zu den anderen Bildungsgruppen weist diese Gruppe somit die geringsten 
 Schwankungen auf. Die \textit{\ac{ICT}-Investitionen} zeigen auch hier eine 
 weitgehend stabile Entwicklung. Dies könnte darauf hindeuten, dass Hochqualifizierte 
 vermehrt in Berufen tätig sind, die von steigenden \textit{\ac{ICT}-Investitionen} 
 profitieren, jedoch bleibt auch hier die Kausalität unklar.

Deutschland als konservativer Wohlfahrtsstaat zeichnet sich durch eine enge Verzahnung 
von Bildungssystem und Arbeitsmarkt aus. Insbesondere das duale Ausbildungssystem 
und gezielte arbeitsmarktpolitische Maßnahmen könnten eine Rolle beim Rückgang der 
Arbeitslosenquoten in den niedrigen und mittleren Bildungsgruppen gespielt haben. 
Die \textit{\ac{ICT}-Investitionen} zeigen über den Beobachtungszeitraum hinweg keine 
drastischen Veränderungen, was darauf hindeutet, dass technologische Entwicklungen 
schrittweise in den Arbeitsmarkt integriert wurden. Besonders für Hochqualifizierte 
könnte eine steigende Nachfrage nach digitalen Fähigkeiten eine Rolle gespielt 
haben, während bei den niedrigen und mittleren Bildungsniveaus der 
Arbeitsmarktrückgang vermutlich durch andere makroökonomische Faktoren beeinflusst 
wurde.

Die Abbildung verdeutlicht insgesamt, dass die Arbeitslosenquoten in allen 
Bildungsgruppen über die Jahre hinweg gesunken sind, während die 
\textit{\ac{ICT}-Investitionen} vergleichsweise stabil geblieben sind. Dies lässt darauf 
schließen, dass der Rückgang der Arbeitslosigkeit nicht direkt durch 
\textit{\ac{ICT}-Investitionen} getrieben wurde, sondern eher mit makroökonomischen 
Entwicklungen und strukturellen Veränderungen auf dem deutschen Arbeitsmarkt 
zusammenhängt.

% Deskriptive Analyse: Grafik "Großbritannien"
\begin{figure}[htbp]
    \centering
    \includegraphics[width=\textwidth]{assets/plot_uk.png}
    \caption{Überblick über \textit{\ac{ICT}-Investitionen} und Arbeitslosenquote in 
    Großbritannien}
    \label{fig:uk}
\end{figure}

Die Abbildung zeigt die Entwicklung der \textit{\ac{ICT}-Investitionen} als Anteil am BIP 
sowie die Arbeitslosenquote in Großbritannien zwischen 2005 und 2022, differenziert nach 
Bildungsniveau - Großbritannien steht hier repräsentativ für angelsächsische 
Wohlfahrtsstaaten. Im Vergleich zu anderen Wohlfahrtsstaatentypen weist Großbritannien eine 
relativ konstante Arbeitslosenquote auf, die über den Zeitraum hinweg nur leichte 
Rückgänge zeigt. Auffällig ist, dass die \textit{\ac{ICT}-Investitionen} in Großbritannien 
zwar einen moderaten Anstieg aufweisen, sich jedoch auf einem relativ niedrigen Niveau 
bewegen.

Bei Personen mit einem niedrigen Bildungsniveau lag die Arbeitslosenquote im Jahr 2005 
bei etwa 8\% und sank bis 2020 auf unter 4\%. Anders als in Ländern mit stärker 
regulierten Arbeitsmärkten zeigt sich hier kein abrupter Rückgang, sondern eine 
schrittweise Anpassung über den Zeitraum hinweg. Gleichzeitig zeigen die 
\textit{\ac{ICT}-Investitionen} einen leicht steigenden Trend, bleiben jedoch im Bereich 
von etwa 1,5\% des BIP. Ein klarer Zusammenhang zwischen \textit{\ac{ICT}-Investitionen} 
und der Arbeitslosenquote lässt sich nicht unmittelbar erkennen, was darauf hindeuten 
könnte, dass andere arbeitsmarktpolitische oder wirtschaftliche Faktoren maßgeblicher für 
die Reduktion der Arbeitslosigkeit sind.

Bei Personen mit mittlerem Bildungsniveau zeigt sich ein ähnliches Muster. Die 
Arbeitslosenquote lag 2005 bei etwa 6\% und fiel bis 2015 auf rund 3\%, wo sie sich 
anschließend stabilisierte. Die \textit{\ac{ICT}-Investitionen} zeigen hier eine geringe 
Zunahme, bleiben jedoch weitgehend konstant im Bereich von 1,5\% bis 2\% des BIP. Auch in 
dieser Gruppe scheint der Rückgang der Arbeitslosenquote eher mit marktwirtschaftlichen 
Anpassungen als mit direkten Effekten der \textit{\ac{ICT}-Investitionen} zusammenzuhängen. 
Der relativ geringe Anstieg der Investitionen deutet darauf hin, dass die britische 
Wirtschaft zwar technologische Entwicklungen integriert, jedoch nicht in dem Ausmaß wie 
andere hochdigitalisierte Volkswirtschaften.

Für Personen mit hohem Bildungsniveau zeigt sich über den gesamten Zeitraum hinweg eine 
sehr niedrige Arbeitslosenquote. Bereits 2005 lag sie unter 3\% und blieb über den 
gesamten Zeitraum weitgehend stabil, mit nur minimalen Schwankungen. Die 
\textit{\ac{ICT}-Investitionen} zeigen auch hier eine relativ konstante Entwicklung, 
liegen jedoch ebenfalls im Bereich von 1,5\% bis 2\% des BIP. Dies deutet darauf hin, 
dass Hochqualifizierte kaum von negativen Beschäftigungseffekten durch Digitalisierung 
betroffen sind. Vielmehr könnte der flexible britische Arbeitsmarkt es dieser Gruppe 
erleichtert haben, sich an technologische Veränderungen anzupassen.

Großbritannien als anglo-sächsischer Wohlfahrtsstaat zeichnet sich durch einen 
weniger regulierten Arbeitsmarkt aus, der sich durch eine hohe Flexibilität und eine 
geringere staatliche Intervention auszeichnet. Diese Charakteristik könnte erklären, 
warum die Arbeitslosenquoten über den Zeitraum hinweg relativ stabil bleiben und 
gleichzeitig keine drastischen Veränderungen im Bereich der \textit{\ac{ICT}-Investitionen} 
feststellbar sind. Der moderate Rückgang der Arbeitslosigkeit deutet darauf hin, dass sich 
der britische Arbeitsmarkt schrittweise an Digitalisierung angepasst hat, ohne dass 
bestimmte Gruppen massiv benachteiligt wurden.

Zusammenfassend zeigt die Abbildung, dass sich die britische Arbeitslosenquote über die 
Jahre hinweg in allen Bildungsgruppen verringert hat, wenn auch nicht so drastisch wie in 
anderen Ländern. Gleichzeitig bleiben die \textit{\ac{ICT}-Investitionen} auf einem 
relativ niedrigen Niveau und zeigen keine unmittelbare Korrelation mit den Veränderungen 
der Arbeitslosenquote. Dies deutet darauf hin, dass makroökonomische Faktoren wie die 
Arbeitsmarktflexibilität und allgemeine wirtschaftliche Entwicklung eine wichtigere Rolle 
für die Beschäftigungsdynamik spielen als allein die Höhe der \textit{\ac{ICT}-Investitionen}.


%%%%%%%%%%%%%%%%%%%%%%%%%
% Multivariate Analysen %
%%%%%%%%%%%%%%%%%%%%%%%%%

\subsection{Multivariate Analysen}

% Multivariate Analyse: Einleitung
Die Zusammenfassung der Ergebnisse aus den Modellen mit Kontrollvariablen zeigt eine 
umfassende, jedoch differenzierte Analyse der Auswirkungen von 
\textit{\ac{ICT}-Investitionen} auf die Arbeitslosenquote in den drei Bildungsgruppen 
(„niedriges Bildungsniveau“, „mittleres Bildungsniveau“, „hohes Bildungsniveau“). Die 
Modelle liefern wichtige Hinweise auf die Bedeutung makroökonomischer Rahmenbedingungen 
und institutioneller Strukturen, während der direkte Einfluss von 
\textit{\ac{ICT}-Investitionen} weniger eindeutig ist.

% Multivariate Analyse: Modelle ohne Interaktion/Jahres-Dummy
\begin{table}[H]
\caption{Einzelwerte der Regressionsmodellparameter für die Kontrollmodelle}
\resizebox{\textwidth}{!}{
\centering
\begin{talltblr}[         %% tabularray outer open
entry=none,label=none,
note{}={+ p \num{< 0.1}, * p \num{< 0.05}, ** p \num{< 0.01}, *** p \num{< 0.001}},
]                     %% tabularray outer close
{                     %% tabularray inner open
colspec={Q[]Q[]Q[]Q[]},
column{2,3,4}={}{halign=c,},
column{1}={}{halign=l,},
hline{13}={1,2,3,4}{solid, black, 0.05em},
}                     %% tabularray inner close
\toprule
& niedriges
Bildungsniv.
(Kontrolle) & mittleres
Bildungsniv.
(Kontrolle) & hohes
Bildungsniv.
(Kontrolle) \\ \midrule %% TinyTableHeader
ICT\_INVEST\_SHARE\_GDP     & \num{2.371}***  & \num{1.225}***  & \num{0.477}***  \\
& (\num{0.232})   & (\num{0.147})   & (\num{0.086})   \\
GDP\_PER\_CAPITA             & \num{-0.193}*** & \num{-0.150}*** & \num{-0.083}*** \\
& (\num{0.018})   & (\num{0.012})   & (\num{0.007})   \\
PERCENT\_TERTIARY\_EDUCATION & \num{0.590}***  & \num{0.288}***  & \num{0.155}***  \\
& (\num{0.046})   & (\num{0.029})   & (\num{0.017})   \\
REGULATION\_STRICTNESS        & \num{-0.127}    & \num{-0.112}+   & \num{-0.079}*   \\
& (\num{0.094})   & (\num{0.059})   & (\num{0.035})   \\
PERCENT\_EMPLOYEES\_TUD      & \num{0.127}**   & \num{0.107}***  & \num{0.026}+    \\
& (\num{0.039})   & (\num{0.024})   & (\num{0.014})   \\
YEAR\_FACTOR & True & True & True \\
Num.Obs.                       & \num{3973}      & \num{3973}      & \num{3973}      \\
R2                             & \num{0.309}     & \num{0.312}     & \num{0.286}     \\
R2 Adj.                        & \num{0.300}     & \num{0.304}     & \num{0.277}     \\
AIC                            & \num{21365.2}   & \num{17724.1}   & \num{13503.3}   \\
BIC                            & \num{21509.8}   & \num{17868.7}   & \num{13647.9}   \\
RMSE                           & \num{3.54}      & \num{2.24}      & \num{1.32}      \\
\bottomrule
\end{talltblr}
}
\label{tab:models_control}
\end{table}


% Multivariate Analyse: Modell ohne Interaktion/Jahres-Dummy (niedriges Bildungsniveau)
Im Modell für die Gruppe mit niedrigem Bildungsniveau zeigt der geschätzte Koeffizient für 
\textit{\ac{ICT}-Investitionen} einen leicht negativen Wert von -0.323, der jedoch 
statistisch nicht signifikant ist. Dies deutet darauf hin, dass \textit{\ac{ICT}-Investitionen} 
potenziell eine Reduktion der Arbeitslosenquote bewirken könnten, dieser Effekt jedoch weder 
systematisch noch stark genug ist, um als nachweisbar zu gelten. 

Das \textit{\ac{BIP} pro Kopf} zeigt mit einem Koeffizienten nahe Null (+0.000, p < 0.1) 
einen schwach positiven, aber nicht stark ausgeprägten Einfluss. Dies deutet darauf hin, dass 
wohlhabendere Länder höhere Anforderungen an Arbeitskräfte stellen, wodurch geringqualifizierte 
Personen möglicherweise stärker vom Arbeitsmarkt ausgeschlossen werden. 

Die \textit{Gewerkschaftsdichte} zeigt mit einem Koeffizienten von +0.011 den erwarteten 
positiven Zusammenhang, allerdings bleibt dieser statistisch insignifikant. Dies könnte 
darauf hindeuten, dass der gewerkschaftliche Einfluss auf die Arbeitslosigkeit in dieser Gruppe 
eher begrenzt ist.

% Multivariate Analyse: Modell ohne Interaktion/Jahres-Dummy (mittleres Bildungsniveau)
Für das mittlere Bildungsniveau ergibt sich ein ähnliches Bild. Der geschätzte 
\textit{\ac{ICT}-Investitionen}-Koeffizient von -0.116 deutet ebenfalls auf einen 
leicht negativen, jedoch nicht signifikanten Zusammenhang mit der Arbeitslosenquote hin. Dies 
lässt vermuten, dass \textit{\ac{ICT}-Investitionen} potenziell eine Rolle spielen könnten, 
andere, nicht modellierte Faktoren jedoch eine größere Bedeutung haben. 

Das \textit{\ac{BIP} pro Kopf} zeigt mit einem Koeffizienten von +0.000 einen signifikanten 
positiven Zusammenhang (p < 0.05), was darauf hinweist, dass eine höhere Wirtschaftsleistung 
eines Landes mit einer leicht steigenden Arbeitslosenquote für mittelqualifizierte Personen 
korrelieren könnte. Dies könnte mit strukturellen Anpassungsprozessen in der Wirtschaft 
zusammenhängen, bei denen traditionelle Berufe durch Automatisierung oder Digitalisierung 
ersetzt werden.

Die \textit{Gewerkschaftsdichte} weist einen Koeffizienten von +0.032 auf, bleibt jedoch auch 
in dieser Gruppe nicht signifikant. Dies könnte darauf hindeuten, dass Gewerkschaften 
allein keinen messbaren Einfluss auf die Arbeitslosenquote dieser Gruppe haben.

% Multivariate Analyse: Modell ohne Interaktion/Jahres-Dummy (hohes Bildungsniveau)
Für das hohe Bildungsniveau zeigt das Modell mit einem Koeffizienten von -0.057 den 
schwächsten Zusammenhang zwischen \textit{\ac{ICT}-Investitionen} und der Arbeitslosenquote. 
Dieser Wert deutet auf einen minimal negativen Effekt hin, der ebenfalls nicht signifikant 
ist. Dies könnte darauf hinweisen, dass hochqualifizierte Arbeitskräfte sich flexibler an 
die Anforderungen des digitalen Arbeitsmarkts anpassen können, sodass der direkte Einfluss 
von \textit{\ac{ICT}-Investitionen} in dieser Gruppe geringer ausfällt.

Das \textit{\ac{BIP} pro Kopf} zeigt in dieser Gruppe keine signifikante Wirkung, was darauf 
hindeutet, dass makroökonomische Unterschiede in Ländern mit höherem Bildungsniveau eine 
geringere Rolle für die Arbeitsmarktsituation von hochqualifizierten Arbeitskräften spielen.

Die \textit{Gewerkschaftsdichte} hat in dieser Gruppe einen leicht negativen Effekt (-0.026), 
bleibt jedoch ebenfalls insignifikant. Dies könnte darauf hindeuten, dass hochqualifizierte 
Arbeitskräfte weniger von institutionellen Faktoren wie gewerkschaftlicher Organisation 
abhängig sind.

% Modellgüte und Interpretation der Ergebnisse
Die Modellgüte variiert zwischen den Bildungsgruppen, wobei die erklärten Varianzen (R²-Werte) 
durchweg niedrig bleiben. Im Modell für das niedrige Bildungsniveau beträgt der R²-Wert 0.019, 
für das mittlere Bildungsniveau 0.025 und für das hohe Bildungsniveau 0.007. Dies deutet darauf 
hin, dass die Modelle nur einen sehr geringen Anteil der Variation in der Arbeitslosenquote 
erklären können und dass wesentliche Einflussfaktoren in dieser Analyse nicht berücksichtigt 
wurden.

Der adjustierte R²-Wert ist in allen drei Modellen sogar negativ, was darauf hindeutet, dass 
die eingefügten Prädiktoren keine Verbesserung gegenüber einem reinen Durchschnittsmodell bieten. 
Dies unterstreicht, dass weitere Faktoren eine Rolle spielen, die nicht in dieser Analyse 
erfasst wurden. Insbesondere branchenspezifische Entwicklungen, individuelle 
Qualifikationsmaßnahmen oder regionale Unterschiede könnten wichtige Erklärungsgrößen sein.

% Fazit der multivariaten Analyse ohne Interaktion
Zusammenfassend zeigen die Modelle mit Kontrollvariablen, dass \textit{\ac{ICT}-Investitionen} 
zwar potenziell eine Rolle bei der Beeinflussung der Arbeitslosenquote in allen 
Bildungsgruppen spielen könnten, diese Effekte jedoch weder stark ausgeprägt noch statistisch 
signifikant sind. Stattdessen zeigen makroökonomische Faktoren wie das \textit{\ac{BIP} pro Kopf} 
in bestimmten Gruppen einen signifikanten Einfluss auf die Arbeitslosenquote. Die schwachen 
Erklärungswerte (R²) legen nahe, dass weitere strukturelle und institutionelle Faktoren eine 
entscheidende Rolle spielen, die in dieser Modellierung nicht berücksichtigt wurden.

Die Ergebnisse verdeutlichen die Komplexität der Beziehungen zwischen Digitalisierung, 
Wirtschaftsentwicklung und Arbeitsmarktstrukturen. Dies legt nahe, dass die Effekte von 
\textit{\ac{ICT}-Investitionen} stark von weiteren, nicht erfassten Bedingungen und 
Kontextfaktoren abhängen. Eine differenziertere Untersuchung, insbesondere durch 
Interaktionsmodelle, könnte helfen, institutionelle Rahmenbedingungen besser zu verstehen 
und die Mechanismen hinter den beobachteten Zusammenhängen genauer zu erfassen.

% Multivariate Analyse: Modelle mit Interaktion & Jahres-Dummy
\begin{table}[H]
\caption{Einzelwerte der Regressionsmodellparameter für die Interaktionsmodelle}
\resizebox{\textwidth}{!}{ % Resize to fit the page width
\centering
\begin{talltblr}[         %% tabularray outer open
entry=none,label=none,
note{}={+ p \num{< 0.1}, * p \num{< 0.05}, ** p \num{< 0.01}, *** p \num{< 0.001}},
]                     %% tabularray outer close
{                     %% tabularray inner open
colspec={Q[]Q[]Q[]Q[]},
column{2,3,4}={}{halign=c,},
column{1}={}{halign=l,},
hline{21}={1,2,3,4}{solid, black, 0.05em},
}                     %% tabularray inner close
\toprule
& niedriges
Bildungsniv.
(Interaktion) & mittleres
Bildungsniv.
(Interaktion) & hohes
Bildungsniv.
(Interaktion) \\ \midrule %% TinyTableHeader
ICT\_INVEST\_SHARE\_GDP                                    & \num{4.671}***  & \num{3.246}***  & \num{1.259}***  \\
& (\num{0.572})   & (\num{0.364})   & (\num{0.214})   \\
GDP\_PER\_CAPITA                                            & \num{-0.180}*** & \num{-0.151}*** & \num{-0.083}*** \\
& (\num{0.018})   & (\num{0.012})   & (\num{0.007})   \\
PERCENT\_TERTIARY\_EDUCATION                                & \num{0.619}***  & \num{0.268}***  & \num{0.122}***  \\
& (\num{0.051})   & (\num{0.032})   & (\num{0.019})   \\
REGULATION\_STRICTNESS                                       & \num{-0.152}+   & \num{-0.126}*   & \num{-0.091}**  \\
& (\num{0.092})   & (\num{0.058})   & (\num{0.034})   \\
PERCENT\_EMPLOYEES\_TUD                                     & \num{0.103}**   & \num{0.090}***  & \num{0.014}     \\
& (\num{0.038})   & (\num{0.024})   & (\num{0.014})   \\
ICT\_INVEST\_SHARE\_GDP × WELFARE\_STATECentral European  & \num{0.676}     & \num{-0.647}    & \num{-0.212}    \\
& (\num{0.718})   & (\num{0.456})   & (\num{0.268})   \\
ICT\_INVEST\_SHARE\_GDP × WELFARE\_STATENordic            & \num{0.033}     & \num{-0.443}    & \num{0.639}*    \\
& (\num{0.837})   & (\num{0.532})   & (\num{0.313})   \\
ICT\_INVEST\_SHARE\_GDP × WELFARE\_STATEPost-socialist    & \num{-5.200}*** & \num{-3.579}*** & \num{-1.415}*** \\
& (\num{0.643})   & (\num{0.409})   & (\num{0.240})   \\
ICT\_INVEST\_SHARE\_GDP × WELFARE\_STATESouthern European & \num{1.465}     & \num{-3.066}*** & \num{-2.880}*** \\
& (\num{1.051})   & (\num{0.669})   & (\num{0.393})   \\
YEAR\_FACTOR & True & True & True \\
Num.Obs.                                                      & \num{3973}      & \num{3973}      & \num{3973}      \\
R2                                                            & \num{0.337}     & \num{0.333}     & \num{0.306}     \\
R2 Adj.                                                       & \num{0.327}     & \num{0.324}     & \num{0.297}     \\
AIC                                                           & \num{21212.7}   & \num{17609.6}   & \num{13396.2}   \\
BIC                                                           & \num{21382.5}   & \num{17779.3}   & \num{13566.0}   \\
RMSE                                                          & \num{3.47}      & \num{2.20}      & \num{1.30}      \\
\bottomrule
\end{talltblr}
}
\label{tab:models_interaction}
\end{table}


% Multivariate Analyse mit Interaktion und Jahres-Dummies
Die Modelle mit Interaktionseffekten und Jahresdummies liefern eine differenzierte Perspektive 
auf den Zusammenhang zwischen \textit{\ac{ICT}-Investitionen} und der Arbeitslosenquote. 
Im Gegensatz zu den Basis-Modellen ohne Interaktionen zeigen sich nun mehrere signifikante 
Zusammenhänge, die insbesondere die Bedeutung institutioneller Rahmenbedingungen betonen.

% Haupteffekte von ICT-Investitionen
Der geschätzte Haupteffekt von \textit{\ac{ICT}-Investitionen} ist in allen drei 
Bildungsgruppen positiv und signifikant. Für das niedrige Bildungsniveau zeigt sich mit einem 
Koeffizienten von +3.899 (p < 0.05) ein deutlicher Anstieg der Arbeitslosenquote bei 
höheren ICT-Investitionen. Auch für das mittlere Bildungsniveau ist der Effekt mit 
+2.817 (p < 0.01) signifikant positiv. Beim hohen Bildungsniveau ist der Effekt mit 
+1.132 (p < 0.05) zwar schwächer ausgeprägt, aber weiterhin signifikant.

Diese Ergebnisse stehen im Kontrast zu den nicht signifikanten Ergebnissen in den 
Kontrollmodellen und legen nahe, dass ICT-Investitionen allein keinen einheitlichen positiven 
Beschäftigungseffekt haben, sondern potenziell auch negative Konsequenzen für 
Arbeitsmarktsegmente mit geringerer Anpassungsfähigkeit haben können. Besonders in Gruppen mit 
niedrigem und mittlerem Bildungsniveau könnte der digitale Wandel Arbeitsplätze verdrängen, 
während sich hochqualifizierte Arbeitskräfte besser anpassen können.

% Einfluss der Jahresdummies
Die Jahresdummies zeigen eine klare zeitliche Entwicklung der Arbeitslosenquote. Während 
die Jahre 2006–2008 keine signifikanten Effekte aufweisen, steigt die Arbeitslosenquote ab 
2009 signifikant an, insbesondere in den Gruppen mit niedrigem und mittlerem 
Bildungsniveau. Dies spiegelt die Folgen der globalen Finanzkrise wider, deren Auswirkungen 
bis in die frühen 2010er Jahre anhielten.

Die höchsten positiven Effekte treten in den Jahren 2012–2015 auf, in denen die 
Arbeitslosenquote in allen Bildungsgruppen signifikant ansteigt. In den letzten Jahren des 
Beobachtungszeitraums (2017–2022) zeigen sich dagegen gemischte Effekte: Während die 
Arbeitslosenquote im niedrigen Bildungsniveau weiterhin schwankt, sinkt sie im mittleren 
und hohen Bildungsniveau leicht. Dies könnte darauf hindeuten, dass sich der Arbeitsmarkt 
nach der Finanzkrise stabilisiert hat und sich hochqualifizierte Arbeitskräfte zunehmend 
besser an den digitalen Wandel anpassen konnten.

% Interaktionseffekte zwischen ICT-Investitionen und Wohlfahrtsstaaten
Die Interaktionseffekte zwischen \textit{\ac{ICT}-Investitionen} und den 
Wohlfahrtsstaaten liefern wichtige Erkenntnisse über die Rolle institutioneller 
Rahmenbedingungen. Hierbei ist nochmals zu erwähnen, dass für die Analyse "anglosächsisch" 
als Referenzkategorie gewählt wurde - die Ergebnisse sind relativ zu dieser Kategorie zu 
interpretieren:

\begin{itemize}
    \item \textbf{Postsozialistische Wohlfahrtsstaaten:} Hier zeigen sich 
    die stärksten negativen Effekte. Für das niedrige Bildungsniveau beträgt der 
    Interaktionseffekt -4.817 (p < 0.01), für das mittlere Bildungsniveau -3.163 (p < 0.01) und 
    für das hohe Bildungsniveau -1.252 (p < 0.05). Diese signifikant negativen Werte 
    deuten darauf hin, dass ICT-Investitionen in diesen Ländern keine positiven Effekte 
    auf die Arbeitsmarktsituation haben, sondern möglicherweise bestehende 
    strukturelle Schwächen verschärfen. Eine mögliche Erklärung hierfür ist, dass die 
    Digitalisierung in diesen Ländern schneller voranschreitet als die institutionellen 
    Anpassungen im Bildungssystem und am Arbeitsmarkt.
    
    \item \textbf{Mitteleuropäische Wohlfahrtsstaaten:} Der Interaktionseffekt ist für alle 
    Bildungsgruppen negativ, aber nicht signifikant. Dies deutet darauf hin, dass 
    ICT-Investitionen hier keinen starken moderierenden Einfluss auf die Arbeitslosigkeit 
    haben. Mitteleuropäische Länder wie Deutschland oder Frankreich verfügen über 
    duale Bildungssysteme und relativ stabile Arbeitsmarktstrukturen, die mögliche 
    negative Effekte von ICT-Investitionen abmildern könnten.
    
    \item \textbf{Nordische Wohlfahrtsstaaten:} Hier zeigen sich ebenfalls keine signifikanten 
    Interaktionseffekte. Die Ergebnisse deuten darauf hin, dass nordische Länder besser 
    auf die digitale Transformation vorbereitet sind und ICT-Investitionen nicht mit 
    steigender Arbeitslosigkeit verbunden sind.
    
    \item \textbf{Südeuropäische Wohlfahrtsstaaten:} Für das niedrige und mittlere Bildungsniveau 
    sind die Interaktionseffekte nicht signifikant. Im hohen Bildungsniveau ist der 
    Interaktionseffekt jedoch -2.737 (p < 0.01) und damit stark negativ. Dies deutet 
    darauf hin, dass selbst hochqualifizierte Arbeitskräfte in diesen Ländern durch 
    strukturelle Arbeitsmarktprobleme benachteiligt sind und ICT-Investitionen hier eher 
    bestehende Ungleichheiten verstärken, anstatt sie zu verringern.
\end{itemize}

% Modellgüte und Vergleich mit den Basis-Modellen
Die erklärten Varianzen (R²-Werte) sind in den Interaktionsmodellen deutlich höher als in 
den Basis-Modellen ohne Interaktionen. Während die R²-Werte in den einfachen Modellen 
zwischen 0.019 und 0.025 lagen, erreichen die Interaktionsmodelle Werte von 0.312 
(niedriges Bildungsniveau), 0.328 (mittleres Bildungsniveau) und 0.303 (hohes Bildungsniveau). 
Dies zeigt, dass institutionelle Rahmenbedingungen eine wesentliche Rolle spielen und die 
Erklärungskraft der Modelle erheblich verbessern.

Der adjustierte R²-Wert bleibt zwar mit 0.231–0.248 weiterhin vergleichsweise niedrig, 
aber deutlich höher als in den Kontrollmodellen. Dies unterstreicht, dass die reine 
Betrachtung von ICT-Investitionen ohne Berücksichtigung institutioneller Faktoren keine 
adäquate Erklärung für Unterschiede in den Arbeitslosenquoten liefert.

% Fazit der Modelle mit Interaktion und Jahresdummies
Zusammenfassend zeigen die Modelle mit Interaktionseffekten, dass der Einfluss von 
\textit{\ac{ICT}-Investitionen} auf die Arbeitslosenquote stark von den institutionellen 
Rahmenbedingungen abhängt. Während die Basiswerte der ICT-Investitionen in allen 
Bildungsgruppen positiv sind, deuten die Interaktionseffekte darauf hin, dass insbesondere 
postsozialistische und südeuropäische Wohlfahrtsstaaten größere Schwierigkeiten haben, 
die positiven Effekte der Digitalisierung für den Arbeitsmarkt zu nutzen. 

Besonders in postsozialistischen Ländern könnte dies auf eine Diskrepanz zwischen 
Digitalisierungsfortschritt und institutioneller Anpassungsfähigkeit zurückzuführen sein. 
Die Ergebnisse zeigen, dass eine starke wirtschaftliche Basis allein nicht ausreicht, um 
die Herausforderungen der digitalen Transformation zu bewältigen. Vielmehr müssen 
gezielte politische Maßnahmen ergriffen werden, um Bildungs- und Arbeitsmarktsysteme auf 
die neuen Anforderungen anzupassen.

Die Kombination aus ICT-Investitionen, institutionellen Rahmenbedingungen und 
makroökonomischen Faktoren bestimmt maßgeblich, wie sich die Digitalisierung auf 
Arbeitsmärkte auswirkt. Während in gut funktionierenden Arbeitsmärkten mit starken 
Bildungssystemen (z. B. Nordische Länder) ICT-Investitionen kaum negative Effekte haben, 
sind in Ländern mit weniger flexiblen Arbeitsmärkten (z. B. Südeuropa, 
Postsozialistische Staaten) signifikante Herausforderungen erkennbar.

Die Modelle verdeutlichen somit, dass ICT-Investitionen alleine keine universelle Lösung 
für Arbeitsmarktprobleme darstellen, sondern dass ihre Wirksamkeit stark von den 
institutionellen Gegebenheiten abhängt. Besonders in Ländern mit ineffizienten 
Arbeitsmarktstrukturen und fehlenden Anpassungsmaßnahmen sind Reformen notwendig, um 
die Vorteile der Digitalisierung optimal zu nutzen.

% TODO: - (Stand 04.03.2025)

\section{Diskussion und Fazit}

Die Ergebnisse dieser Arbeit bieten wertvolle Einblicke in die Beziehung zwischen 
\ac{ICT}-Investitionen und der Arbeitslosenquote in verschiedenen Bildungsgruppen. Sie 
zeigen signifikante Zusammenhänge und verdeutlichen die Rolle institutioneller 
Rahmenbedingungen für die Beschäftigungswirkungen der Digitalisierung. Die Untersuchung 
trägt zur wissenschaftlichen Debatte über die Wechselwirkungen zwischen technologischer 
Entwicklung, Arbeitsmarktstrukturen und politischen Institutionen bei und liefert 
praktische Implikationen für Politik, Unternehmen und Bildungssysteme.

\subsection{Zentrale Ergebnisse der Analyse}

Die Analyse zeigt, dass \ac{ICT}-Investitionen in den Basis-Modellen ohne Interaktion 
über alle Bildungsgruppen hinweg einen signifikant positiven Effekt auf die Arbeitslosenquote 
haben. Dies bedeutet, dass höhere \ac{ICT}-Investitionen in diesen Modellen mit höheren 
Arbeitslosenquoten korrelieren. Besonders stark ausgeprägt ist dieser Effekt für 
gering- und mittelqualifizierte Personen, bleibt aber auch für hochqualifizierte 
Arbeitskräfte signifikant. Dies widerspricht der weit verbreiteten Annahme, dass 
Digitalisierung primär positive Effekte für Hochqualifizierte hat, während gering 
Qualifizierte besonders negativ betroffen sind. 

Allerdings zeigen die Interaktionsmodelle mit Wohlfahrtsstaaten, dass dieser Effekt nicht 
universell ist, sondern stark von institutionellen Rahmenbedingungen abhängt. Während in 
liberalen Wohlfahrtsstaaten (Referenzkategorie) der Haupteffekt positiv bleibt und höhere 
\ac{ICT}-Investitionen mit steigender Arbeitslosigkeit verbunden sind, zeigt sich in anderen 
Wohlfahrtsstaatentypen ein differenziertes Bild. In postsozialistischen Wohlfahrtsstaaten 
sind die Interaktionseffekte negativ und signifikant. Dies bedeutet, dass in diesen Ländern 
\ac{ICT}-Investitionen nicht zu einer höheren Arbeitslosigkeit führen - im Vergleich zu 
liberalen Staaten können diese Investitionen hier also besser in den Arbeitsmarkt integriert 
werden. Mitteleuropäische Wohlfahrtsstaaten zeigen keine signifikanten Unterschiede zu den 
liberalen Staaten, was darauf hindeutet, dass sich die Digitalisierungseffekte in diesen Ländern 
nicht signifikant von den Effekten in liberalen Märkten unterscheiden. In südeuropäischen 
Wohlfahrtsstaaten zeigen sich für Mittel- und Hochqualifizierte negative Interaktionseffekte. 
Dies bedeutet, dass diese Gruppen dort weniger von steigender Arbeitslosigkeit durch Digitalisierung 
betroffen sind als in liberalen Staaten. Nordische Wohlfahrtsstaaten zeigen für Hochqualifizierte 
positive Interaktionseffekte, was darauf hindeutet, dass in diesen Ländern hochqualifizierte 
Arbeitskräfte möglicherweise stärker von der Digitalisierung betroffen sind als in anderen 
Wohlfahrtsstaatentypen.

Diese Ergebnisse verdeutlichen, dass die Effekte von \ac{ICT}-Investitionen keineswegs linear oder 
eindeutig negativ sind, sondern stark von den institutionellen und wirtschaftlichen Rahmenbedingungen 
abhängen. Während in liberalen Wohlfahrtsstaaten der Anstieg der \ac{ICT}-Investitionen mit steigender 
Arbeitslosigkeit in allen Bildungsgruppen verbunden ist, scheinen postsozialistische und südeuropäische 
Länder in der Lage zu sein, die negativen Beschäftigungseffekte abzumildern.

Die Kontrollvariablen zeigen ebenfalls einige wichtige Zusammenhänge. Das \ac{BIP} pro Kopf 
hat in allen Bildungsgruppen einen signifikant negativen Einfluss auf die Arbeitslosenquote. 
Dies bestätigt, dass wirtschaftlich stärkere Länder tendenziell geringere Arbeitslosenquoten 
aufweisen. Die Regulierungsstrenge des Arbeitsmarkts wirkt sich für gering Qualifizierte eher 
negativ aus (höhere Regulierung erhöht die Arbeitslosenquote), während sie für Hochqualifizierte 
eher stabilisierend wirkt. Der tertiäre Bildungsanteil zeigt in allen Gruppen einen signifikant 
negativen Einfluss auf die Arbeitslosenquote, was darauf hindeutet, dass eine höhere 
Bildungsbeteiligung langfristig dazu beiträgt, negative Beschäftigungseffekte der Digitalisierung 
abzumildern.

\subsection{Einordnung der Ergebnisse in den theoretischen Kontext}

Die Theorie des \ac{SBTC} besagt, dass technologische Innovationen die Nachfrage nach 
hochqualifizierten Arbeitskräften steigern, während gering Qualifizierte durch 
Automatisierung verdrängt werden \parencite[vgl.][S. 7]{acemoglu2002technical}. Die Ergebnisse 
dieser Arbeit bestätigen diese Annahme jedoch nur teilweise. Während erwartet wurde, dass 
\ac{ICT}-Investitionen primär die Arbeitslosenquote von gering Qualifizierten erhöhen, 
zeigen die Modelle einen durchweg signifikanten positiven Effekt für alle Bildungsgruppen in 
den Basis-Modellen. Das bedeutet, dass die negativen Arbeitsmarkteffekte der Digitalisierung 
nicht nur auf niedrigqualifizierte Arbeitnehmer beschränkt sind, sondern auch mittel- und 
hochqualifizierte Arbeitskräfte betreffen können.

Die Interaktionseffekte legen jedoch nahe, dass diese Effekte durch institutionelle Faktoren 
modifiziert werden. Die Beobachtung, dass in postsozialistischen und südeuropäischen Ländern 
die negativen Arbeitsmarkteffekte abgemildert werden, spricht dafür, dass arbeitsmarktpolitische 
Maßnahmen und Anpassungsprozesse eine entscheidende Rolle spielen. Dies könnte im Einklang mit 
der Theorie der "schöpferischen Zerstörung" nach Schumpeter stehen, die besagt, dass wirtschaftlicher 
Wandel und technologische Disruptionen langfristig zu Wachstum führen, kurzfristig jedoch strukturelle 
Umwälzungen verursachen \parencite[vgl.][S. 103-105]{schumpeter1976capitalism}.

Eine alternative Erklärung könnte die Verzögerungseffekte der Digitalisierung sein. Während in 
hoch digitalisierten liberalen Staaten wie den USA oder Großbritannien die Arbeitsmärkte bereits 
stark von technologischen Umstellungen betroffen sind, könnten in postsozialistischen Staaten 
die Digitalisierungseffekte erst mit zeitlicher Verzögerung eintreten. Dies könnte erklären, warum 
die negativen Beschäftigungseffekte dort (noch) nicht sichtbar sind.

\subsection{Limitationen und zukünftige Forschung}

Trotz der wertvollen Erkenntnisse dieser Untersuchung sind einige Limitationen zu 
berücksichtigen. Erstens basiert die Analyse auf aggregierten \ac{OECD}-Daten für den 
Zeitraum 2005–2022, wodurch Unterschiede in der Erhebungsmethodik zwischen den 
Ländern die Ergebnisse potenziell verzerren könnten. Zweitens liegt der Fokus der 
Untersuchung auf makroökonomischen Zusammenhängen. Individuelle Anpassungsstrategien 
von Arbeitnehmer*innen oder Unternehmen an die Digitalisierung konnten in dieser 
Analyse nicht berücksichtigt werden. Künftige Studien sollten daher verstärkt auf 
Umfragedaten oder firmenspezifische Datenquellen zurückgreifen, um differenziertere 
Erkenntnisse zu gewinnen.

Darüber hinaus besteht die Möglichkeit einer umgekehrten Kausalität zwischen \ac{ICT}-Investitionen 
und Arbeitslosigkeit. Während in dieser Analyse angenommen wurde, dass \ac{ICT}-Investitionen 
die Arbeitslosenquote beeinflussen, könnte es auch sein, dass hohe Arbeitslosigkeit 
Regierungen oder Unternehmen zu verstärkten Investitionen in Digitalisierung veranlasst. 
Diese Hypothese könnte mit Methoden wie Granger-Kausalitätstests oder Instrumentalvariablen 
weiter geprüft werden.

\subsection{Gesamtfazit}

Zusammenfassend zeigen die Ergebnisse, dass \ac{ICT}-Investitionen keine universelle 
Lösung für Arbeitsmarktprobleme darstellen, sondern dass ihre Wirkungen maßgeblich von 
institutionellen Rahmenbedingungen abhängen. Während in Ländern mit gut entwickelten 
Arbeitsmarkt- und Bildungssystemen die negativen Effekte begrenzt sind, weisen insbesondere 
südeuropäische und postsozialistische Wohlfahrtsstaaten signifikante Herausforderungen auf.
Die Ergebnisse legen nahe, dass Digitalisierung auch in wissensintensiven Sektoren Arbeitsplätze 
substituieren kann. Gleichzeitig deuten die Interaktionseffekte darauf hin, dass 
institutionelle Anpassungsmechanismen wie Weiterbildungsprogramme oder soziale Sicherungssysteme 
eine zentrale Rolle bei der Milderung der Digitalisierungseffekte spielen. Eine erfolgreiche 
digitale Transformation erfordert daher nicht nur technologische Investitionen, sondern auch 
begleitende arbeitsmarktpolitische, wirtschaftliche und bildungspolitische Maßnahmen.


% Literaturverzeichnis
\printbibliography

\newpage

% Anhang
\appendix
% TODO: - (Passt so.)

\section{Anhang}

%%%%%%%%%%%%%%%%%%
% Projektdateien %
%%%%%%%%%%%%%%%%%%

\subsection{Projektdateien}

Alle Projektdateien (R-Code, TeX-Dateien, sowie alle Datensätze) welche für die Arbeit und die 
Analyse genutzt wurden, sind gebündelt im folgenden GitHub Repository zu finden (der erste Link 
führt zum Repository - der zweite direkt zum R-Codebook):

\vspace{2cm}

\begin{center}
    \includegraphics[width=0.3\textwidth]{assets/qrcode_repository.png}\\
    \small\url{https://github.com/TAR-IT/bachelorthesis}
\end{center}

\vspace{2cm}

\begin{center}
    \includegraphics[width=0.3\textwidth]{assets/qrcode_codebook.png}\\
    \small\url{https://github.com/TAR-IT/bachelorthesis/blob/main/R/codebook.R}
\end{center}
\newpage

%%%%%%%%%%%%%%%%%%%%%%%%%%%%%%%%%%
% Erklärung zur Prüfungsleistung %
%%%%%%%%%%%%%%%%%%%%%%%%%%%%%%%%%%

\subsection{Erklärung zur Prüfungsleistung}

Name, Vorname: Rau, Tobias Achim \\
Matrikelnummer: 6619097 \\
Studiengang: Politikwissenschaften BA

\vspace{0.5cm}

Die am FB03 gültige Definition von Plagiaten ist mir vertraut und verständlich: 

\begin{quote}
    „Eine am FB03 eingereichte Arbeit wird als Plagiat identifiziert, wenn in ihr nachweislich 
    fremdes geistiges Eigentum ohne Kennzeichnung verwendet wird und dadurch dessen Urheberschaft 
    suggeriert oder behauptet wird. Das geistige Eigentum kann ganze Texte, Textteile, 
    Formulierungen, Ideen, Argumente, Abbildungen, Tabellen oder Daten umfassen und muss als 
    geistiges Eigentum der Urheberin/des Urhebers gekennzeichnet sein. Sofern eingereichte 
    Arbeiten die Kennzeichnung vorsätzlich unterlassen, provozieren sie einen Irrtum bei 
    denjenigen, welche die Arbeit bewerten, und erfüllen somit den Tatbestand der Täuschung.“
\end{quote}

Ich versichere hiermit, dass ich die eingereichte Arbeit mit dem Titel 

% Titel der Arbeit
\begin{center}
    \textbf{"\ac{ICT}-Investitionen und Arbeitslosigkeit in Wohlfahrtsstaaten - \\ 
    eine Paneldatenanalyse nach Bildungsniveau in OECD-Ländern"} \\ 
\end{center}

nach den Regeln guter wissenschaftlicher Praxis angefertigt habe. Alle Stellen, die wörtlich oder 
sinngemäß aus Veröffentlichungen oder aus anderen fremden Mitteilungen entnommen wurden, sind als 
solche kenntlich gemacht. Die vorliegende Arbeit ist von mir selbständig und ohne Benutzung 
anderer als der angegebenen Quellen und Hilfsmittel verfasst worden. Ebenfalls versichere ich, 
dass diese Arbeit noch in keinem anderen Modul oder Studiengang als Prüfungsleistung vorgelegt 
wurde.

\vspace{0.5cm}

Mir ist bekannt, dass Plagiate auf Grundlage der Studien- und Prüfungsordnung im Prüfungsamt 
dokumentiert und vom Prüfungsausschuss sanktioniert werden. Diese Sanktionen können neben dem 
Nichtbestehen der Prüfungsleistung weitreichende Folgen bis hin zum Ausschluss von der Erbringung 
weiterer Prüfungsleistungen für mich haben. 

\vfill

\begin{flushleft}
    \small Rödermark, \today,
    % \raisebox{-0.65cm}{\includegraphics[width=0.3\textwidth]{assets/signature.pdf}}
    \\[-0.6cm]
    \_\_\_\_\_\_\_\_\_\_\_\_\_\_\_\_\_\_\_\_\_\_\_\_\_\_\_\_\_\_\_\_\_ \\
    \small\textit{(Ort, Datum, Unterschrift)}
\end{flushleft}



\end{document}